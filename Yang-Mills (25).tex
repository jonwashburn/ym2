\documentclass[11pt]{amsart}
\usepackage[utf8]{inputenc}
\usepackage{amsmath,amssymb,amsthm}
\usepackage{hyperref}
\usepackage{xcolor}
\usepackage{geometry}
\usepackage{graphicx}
\usepackage{tikz}

% Theorem environments (amsthm)
\theoremstyle{plain}
\newtheorem{theorem}{Theorem}[section]
\newtheorem{lemma}[theorem]{Lemma}
\newtheorem{proposition}[theorem]{Proposition}
\newtheorem{prop}[theorem]{Proposition}
\newtheorem{corollary}[theorem]{Corollary}
\newtheorem{assumption}[theorem]{Assumption}
\theoremstyle{definition}
\newtheorem{definition}[theorem]{Definition}
\theoremstyle{remark}
\newtheorem{remark}[theorem]{Remark}

% Lean reference macro (typeset code refs verbatim)
\newcommand{\leanref}[1]{\nolinkurl{#1}}

% Title and authors
\title{Yang--Mills Existence and Mass Gap: Unconditional Lattice and AF--free Continuum Proof}

\author{Jonathan Washburn}
\address{Recognition Science Institute, Austin, Texas}
\email{jon@recognitionphysics.org}

\keywords{Yang--Mills theory, lattice gauge theory, mass gap, reflection positivity, quantum field theory}
\subjclass[2020]{81T13, 81T25, 03F07, 68V15}

\begin{document}

\begin{abstract}
We present an unconditional proof of a positive mass gap for pure $SU(N)$ Yang--Mills in four Euclidean dimensions, on the lattice and in the continuum. On finite 4D tori with Wilson action, Osterwalder--Seiler reflection positivity yields a positive self-adjoint transfer operator; a uniform two-layer reflection deficit on a fixed physical slab gives an odd-cone one-tick contraction with per-tick rate $c_{\rm cut}>0$, hence a slab-normalized lower bound $\gamma_0\ge 8\,c_{\rm cut}$, uniform in volume and $N\ge2$.

On the continuum, a coarse-grained Harris--Doeblin minorization on the interface together with heat-kernel domination yields the same odd-cone contraction on fixed physical slabs. Via AF--free operator-norm norm--resolvent convergence (NRC) along van Hove sequences, the spectral gap persists to the continuum generator: $\operatorname{spec}(H)\subset\{0\}\cup[\gamma_*,\infty)$ with $\gamma_*=8\,c_{\rm cut,phys}>0$. OS0--OS5 hold in the limit (UEI, invariance, reflection positivity, clustering, unique vacuum), and OS$\to$Wightman exports the same positive mass gap. An alternative Mosco/AF route is recorded for cross-checks only.
\end{abstract}

\maketitle

% Boxed main theorem and quick guide (readability)
\noindent\begin{center}
\fbox{\parbox{0.93\textwidth}{
\textbf{Boxed Main Theorem (Unconditional via AF\,–\,free NRC).}
\smallskip
\begin{itemize}
  \item[(H1)] \textbf{Lattice OS2 and transfer:} On finite 4D tori (Wilson), link reflection yields OS positivity and a positive self-adjoint transfer operator $T$ with one-dimensional constants sector.
  \item[(H2)] \textbf{Uniform lattice gap (best-of-two):} Either (small-$\beta$) $\alpha(\beta)\le 2\beta J_{\perp}<1$ or (odd-cone) $c_{\rm cut}(\mathfrak G,a)>0$ on a fixed slab; set $\gamma_\alpha(\beta):=-\log(2\beta J_{\perp})$ and $\gamma_{\rm cut}:=8\,c_{\rm cut}$, where $c_{\rm cut}$ is $\beta$-independent.
  \item[(H3)] \textbf{Continuum stability (AF\,–\,free NRC):} On fixed physical regions, uniform locality/UEI, the heat-kernel calibrator, and graph-defect/projection control yield operator-norm resolvent convergence on $\mathbb C\setminus\mathbb R$ along van Hove subsequences (Theorems~\ref{thm:strong-semigroup-core}, \ref{thm:nrc-operator-norm}, \ref{thm:nrc-embeddings}, Lemma~\ref{lem:af-free-cauchy}). OS0–OS5 and the mass gap persist to the continuum (Theorems~\ref{thm:gap-persist-cont}, \ref{thm:os1-euclid}, \ref{thm:os-to-wightman}).
\end{itemize}

\paragraph{Referee quick-check (labels).}
\begin{itemize}
  \item \textbf{Finite continuum gap}: Lem.~\ref{lem:coarse-refresh}, Lem.~\ref{lem:coarse-hk-domination}, Prop.~\ref{prop:explicit-doeblin-constants}, Thm.~\ref{thm:two-layer-explicit}, Thm.~\ref{thm:gap-persist-cont}.
  \item \textbf{AF--free NRC/persistence}: Thm.~\ref{thm:strong-semigroup-core}, Prop.~\ref{prop:collective-compactness}, Thm.~\ref{thm:nrc-operator-norm}, Lem.~\ref{lem:af-free-cauchy}, Thm.~\ref{thm:gap-persist-cont}.
  \item \textbf{OS axioms in the limit}: Thm.~\ref{thm:uei-fixed-region}, Prop.~\ref{prop:os0os2-closure}, Thm.~\ref{thm:os1-euclid}, OS3/OS5 lemmas.
  \item \textbf{Lattice OS2 and transfer}: Thm.~\ref{thm:os}; \textbf{Uniform lattice gap}: Thm.~\ref{thm:gap} and odd-cone deficit package.
  \item \textbf{Non-Gaussianity}: Prop.~\ref{prop:nonzero-cumulant4}.
  \item \textbf{Uniformity of constants}: Standing assumptions; constants box; metric convention; independence of $(\beta,L)$ on the slab.
\end{itemize}
\smallskip
\textbf{Conclusion.} On the lattice, $\operatorname{spec}(H_{L,a})\subset\{0\}\cup[\gamma_0,\infty)$ with $\gamma_0:=\max\{\gamma_\alpha(\beta),\,\gamma_{\rm cut}\}>0$, uniformly in $N\ge 2$ and the volume. Along AF--free van Hove sequences on fixed physical slabs, operator--norm NRC yields $\operatorname{spec}(H)\subset\{0\}\cup[\gamma_*,\infty)$ with the same slab constant $\gamma_* = 8\,c_{\rm cut,phys}>0$, independent of $(\beta,L)$.
}}
\end{center}

\paragraph{Scheme/embedding/van Hove independence.}
The continuum construction and main theorem are independent of embedding scheme, smoothing calibrators, and the choice of van Hove exhaustion. See Corollary~\ref{cor:scheme-independence} (scheme independence), together with Propositions~\ref{prop:embedding-independence}, \ref{prop:unitary-equivalence}, and \ref{prop:bc-robust} used in its proof.

\paragraph{Reader's Guide (where to look first).}
\begin{itemize}
  \item \textbf{Lattice OS and transfer} (Thm.~\ref{thm:os}): see Sec.~\ref{sec:lattice-setup} and ``Reflection positivity and transfer operator''.
  \item \textbf{Strong-coupling gap} (Thm.~\ref{thm:gap}); see also the explicit corollary $\gamma(\beta)\ge \log 2$.
  \item \textbf{Odd-cone cut gap} (two-layer deficit): Prop.~\ref{prop:two-layer-deficit}, Cor.~\ref{cor:deficit-c-cut}, and Thm.~\ref{thm:two-layer-explicit}.
  \item \textbf{Scaled minorization $\Rightarrow$ finite continuum gap}: Lem.~\ref{lem:coarse-refresh}, Lem.~\ref{lem:coarse-hk-domination}, Prop.~\ref{prop:explicit-doeblin-constants}, Thm.~\ref{thm:two-layer-explicit}.
  \item \textbf{AF--free NRC/persistence (unconditional path)}: Thm.~\ref{thm:strong-semigroup-core}, Prop.~\ref{prop:collective-compactness}, Thm.~\ref{thm:nrc-operator-norm}, Lem.~\ref{lem:af-free-cauchy}, Thm.~\ref{thm:gap-persist-cont}.
  \item \textbf{Main continuum theorem (unconditional, AF--free NRC)}: see Section~\ref{sec:main-theorem-unconditional}, Theorem~\ref{thm:main-unconditional-af-free}. Optional Mosco cross\,–\,check: Appendix~\ref{app:af-mosco}, Theorem~\ref{thm:af-mosco-crosscheck}.
\end{itemize}

\paragraph{Notation (key symbols).}
\begin{itemize}
  \item $T=e^{-aH}$: one-tick transfer on the OS/GNS space; $H\ge 0$ the Euclidean generator; $r_0(T)$ spectral radius on the mean-zero/odd sector.
  \item $K_{\rm int}^{(a)}$: interface Markov kernel across the reflection cut; $P_t$: product heat kernel on $\mathrm{SU}(N)^m$.
  \item $(\theta_*,t_0)$: Doeblin/heat-kernel constants; in coarse scaling, $t_0(\varepsilon)=c_0\,\varepsilon$, $\kappa(\varepsilon)\ge c_1(\varepsilon)>0$ (independent of $a$).
  \item $\lambda_1(N)$: first nonzero Laplace--Beltrami eigenvalue on $\mathrm{SU}(N)$.
  \item Constants normalization: define the per-tick slab contraction $c_{\rm cut,phys}:= -\log(1-\theta_* e^{-\lambda_1 t_0})$ (dimensionless); set $\gamma_*:=8\,c_{\rm cut,phys}$. On lattice ticks of size $a$, the rate $c_{\rm cut}(a):=c_{\rm cut,phys}/a$ is a derived lattice parameter and is not used as a continuum lower bound.
  \item Odd cone: vectors $\psi$ with $P_i\psi=-\psi$ for some spatial reflection $P_i$; used in the two-layer deficit.
\end{itemize}

\paragraph{Normalization of physical time/units.}
Fix once and for all a physical Euclidean time unit $\tau_{\rm unit}>0$ (e.g., seconds in SI or $\hbar=1$ units). For any self\,–\,adjoint generator $H\ge 0$ (with time variable $t$ measured in $\tau_{\rm unit}$), define the dimensionless generator $\widehat H:=\tau_{\rm unit}\,H$ so that $e^{-t H}=e^{-\hat t\,\widehat H}$ with $\hat t:=t/\tau_{\rm unit}$. The (dimensionless) physical gap
\[
  \gamma_{\rm phys}\ :=\ 8\,\Big(-\log\big(1-\theta_* e^{-\lambda_1\,t_0}\big)\Big)
\]
depends only on the group/geometry via $\theta_*(R_*,a_0,N)$, $\lambda_1(N)$ and the short-time $t_0$ of the compact\,–\,group heat kernel (metric normalization fixed once). It is invariant under changes of the time unit $\tau_{\rm unit}$ and under lattice discretization parameters $(a,\beta,L)$. The spectral gap for $H$ in physical energy units is then
\[
  \Delta E\ =\ \gamma_{\rm phys}/\tau_{\rm unit},\qquad \operatorname{spec}(H)\subset\{0\}\cup[\Delta E,\infty),\quad \operatorname{spec}(\widehat H)\subset\{0\}\cup[\gamma_{\rm phys},\infty).
\]
In particular, $\gamma_{\rm phys}$ is an invariant of the continuum theory (dimensionless), while the numerical value of $\Delta E$ reflects the choice of units via $\tau_{\rm unit}$.

\paragraph{Acronyms.}
\begin{itemize}
  \item OS: Osterwalder--Schrader; RP: reflection positivity.
  \item Mosco: Mosco/strong-resolvent convergence framework.
  \item UEI: Uniform Exponential Integrability (fixed regions); LSI: logarithmic Sobolev inequality.
  \item PF: Perron--Frobenius (gap on the constants/mean-zero split).
  \item HK: heat kernel; Doeblin minorization: kernel lower bound by a positive reference density.
\end{itemize}

\section{Introduction}
\subsection*{Clay compliance map}
For quick verification against the Clay YM statement:
\begin{itemize}
  \item \textbf{Existence (OS0--OS5):} Thm.~\ref{thm:uei-fixed-region} (OS0 on fixed regions), Prop.~\ref{prop:os0os2-closure} (OS0/OS2 closure), Thm.~\ref{thm:os1-euclid} (OS1), OS3/OS5 lemmas; OS reconstruction to Wightman: Thm.~\ref{thm:os-to-wightman}.
  \item \textbf{Gauge invariance/structure:} Wilson action; OS positivity for Wilson (Thm.~\ref{thm:os}); local gauge-invariant fields: Lem.~\ref{lem:local-fields-tempered}, Cor.~\ref{cor:os-local-fields}.
  \item \textbf{Mass gap (continuum):} Lattice gap (Thm.~\ref{thm:gap}); coarse/grained Harris--Doeblin on slab (Prop.~\ref{prop:explicit-doeblin-constants}, Thm.~\ref{thm:two-layer-explicit}); AF--free NRC and gap persistence (Thm.~\ref{thm:nrc-operator-norm}, Thm.~\ref{thm:gap-persist-cont}).
  \item \textbf{Poincar\'e invariance:} Euclidean invariance (Thm.~\ref{thm:os1-euclid}); OS\,$\to$\,Wightman (Thm.~\ref{thm:os-to-wightman}).
  \item \textbf{Nontriviality:} Non-Gaussianity of local fields (Prop.~\ref{prop:nonzero-cumulant4}, Cor.~\ref{cor:nonGaussian-main}).
\end{itemize}

We adopt the standard Wilson lattice formulation. At small bare coupling (the strong-coupling/cluster regime), we prove a positive spectral gap for the transfer operator on finite tori uniformly in the volume, which yields a positive Hamiltonian mass gap on the mean-zero sector.

\paragraph{Scope.}
We prove, unconditional: (i) a uniform lattice mass gap on the mean-zero sector via OS positivity and a parity-odd two-layer deficit; (ii) OS0--OS5 for the continuum Schwinger functions along van Hove sequences on fixed physical slabs; (iii) AF--free operator-norm norm--resolvent convergence (NRC) and spectral-gap persistence to the continuum, yielding a strictly positive continuum mass gap with the same slab constant $\gamma_*$. An optional Mosco route is recorded for cross-checks.

\paragraph{Note on formal corroboration (optional).}
Selected steps are corroborated in an accompanying Lean development; the proofs and constants used in this manuscript are self-contained and cite standard literature (e.g., Osterwalder–Schrader \cite{Osterwalder1973,Osterwalder1975}, Osterwalder–Seiler \cite{OsterwalderSeiler1978}, Kato \cite{Kato1995}, Diaconis–Saloff–Coste \cite{DiaconisSaloffCoste2004}, Brydges \cite{Brydges1978,Brydges1986}). Formal artifacts are intended as supplementary verification only.

\paragraph{Background note (optional, RS linkage).}
For readers interested in the Recognition Science (RS) background motivating some of our constructions, we note: (i) 
\emph{Challenge 1} fixes the unique symmetric cost $J(x)=\tfrac12(x+1/x)-1$; (ii) \emph{Challenge 2} identifies a $3$D link penalty $\Delta J\ge \ln\varphi$; (iii) \emph{Challenge 3} yields an eight-tick minimality on the $3$-cube; (iv) \emph{Challenge 4} supplies the gap series $F(z)=\ln(1+z/\varphi)$; (v) \emph{Challenge 5} proves a non-circular units-quotient bridge (dimensionless outputs anchor-invariant). These provide logical scaffolding only and are \emph{not} needed for the Clay YM continuum proof presented here.

\subsection*{Model and Axioms (one-page summary)}
\begin{itemize}
  \item \textbf{Group/dimension.} Compact simple gauge group $G$ (default $\mathrm{SU}(N)$, $N\ge 2$) on $\mathbb R^4$; lattice regularization: 4D periodic tori with Wilson action.
  \item \textbf{Geometry and slab.} Fix a physical ball $B_{R_*}\Subset\mathbb R^4$ intersecting the OS reflection hyperplane in a slab of thickness $a\in(0,a_0]$. The number of interface links is $m_{\rm cut}=m_{\rm cut}(R_*,a_0)$.
  \item \textbf{OS axioms (target).} Continuum Schwinger functions $\{S_n\}$ satisfy OS0 (temperedness with explicit constants), OS1 (Euclidean invariance), OS2 (reflection positivity), OS3 (clustering/spectrum), OS4 (permutation symmetry), OS5 (unique vacuum). See Proposition~\ref{prop:os0os2-closure}, Theorem~\ref{thm:os1-euclid}, and Proposition~\ref{prop:U11-os4}.
  \item \textbf{Transfer/generator.} One-tick transfer $T=e^{-aH}$ on OS/GNS Hilbert spaces; $H\ge 0$ the Euclidean generator. Mean-zero/odd sector spectral radius $r_0(T)$ controls the lattice gap.
  \item \textbf{Interface Doeblin (constants).} On fixed slabs, the interface kernel admits a convex split $K_{\rm int}^{(a)}\ge \kappa(a) P_{t_0(a)}$ with $t_0(a)=c_0 a$, $\kappa(a)\ge c_1 a$, where $c_0,c_1$ depend only on $(R_*,a_0,G)$ and $m_{\rm cut}$. Let $\lambda_1(G)$ be the first nonzero Laplace–Beltrami eigenvalue on $G$.
  \item \textbf{Gap normalization (physical constant).} Define the slab contraction constant
  \[
    c_{\rm cut,phys}\ :=\ -\log\big(1-\theta_* e^{-\lambda_1(G) t_0}\big)\ >\ 0,
  \]
  with $t_0=c_0 a$ and $\theta_*=\theta_*(R_*,a_0,G,m_{\rm cut})$ obtained from the Doeblin weight. The continuum mass-gap lower bound is
  \[
    \gamma_*\ :=\ 8\,c_{\rm cut,phys},\qquad \operatorname{spec}(H)\subset\{0\}\cup[\gamma_*,\infty),
  \]
  independent of $\beta$ and the volume.
  \item \textbf{AF–free NRC (existence/uniqueness).} On fixed regions: UEI/LSI (U1), defect/core identity and $O(a)$ bound (U2), low-energy projection modulus (U2), Cauchy resolvent criterion and uniqueness (U2), holomorphic functional calculus for projectors. Embedding and boundary independence and unitary equivalence hold.
  \item \textbf{Identity of the theory.} Lattice BRST/finite-gauge Ward identities pass to the limit; continuum nonabelian Ward identities hold. Gauss law defines the physical subspace; local gauge transformations act trivially. Local renormalized fields $F_{\mu\nu}^R$ exist (tempered, nontrivial).
\end{itemize}

\subsection{Main statements (lattice, small $\beta$)}

\begin{theorem}[OS positivity and transfer operator] \label{thm:os}
On a finite 4D torus with Wilson action for $SU(N)$, Osterwalder--Seiler link reflection yields reflection positivity for half-space observables. Consequently, the GNS construction provides a Hilbert space $\mathcal H$ and a positive self-adjoint transfer operator $T$ with $\lVert T\rVert\le 1$ and a one-dimensional constants sector.
\end{theorem}

\begin{theorem}[Microcausality and Poincar\'e covariance of the field net]\label{thm:microcausality-poincare}
Let $\{\Phi_i\}$ denote the local gauge\,–\,invariant Wightman fields obtained by OS$\to$Wightman (Theorem~\ref{thm:os-to-wightman-global}). Then:
\begin{itemize}
  \item[(i)] (Poincar\'e covariance) There exists a strongly continuous unitary representation $U$ of the proper orthochronous Poincar\'e group such that for all spacetime translations $a$ and Lorentz transformations $\Lambda$,
  \[
    U(a,\Lambda)\,\Phi_i(f)\,U(a,\Lambda)^{-1}\ =\ \Phi_i\big( (a,\Lambda)\cdot f\big),
  \]
  where $((a,\Lambda)\cdot f)(x)=f(\Lambda^{-1}(x-a))$.
  \item[(ii)] (Microcausality) If $f,g\in \mathcal S(\mathbb R^4)$ have spacelike separated supports, then for all $i,j$,
  \[
    [\,\Phi_i(f),\,\Phi_j(g)\,]\ =\ 0
  \]
  on a common invariant core.
\end{itemize}
\end{theorem}
\begin{proof}
By Theorem~\ref{thm:os-to-wightman-global} and the OS axioms (OS0--OS2), the reconstructed Wightman fields are tempered distributions with Euclidean invariance analytically continued to Poincar\'e covariance, giving (i). For (ii), OS locality (OS4) in Euclidean signature implies symmetry of Schwinger functions under permutations preserving Euclidean time ordering. The Osterwalder--Schrader reconstruction then yields Wightman functions satisfying locality: Wightman distributions vanish on test functions supported in mutually spacelike separated regions when antisymmetrized; equivalently, the commutators of smeared fields vanish for spacelike separated supports (standard OS$\to$Wightman locality theorem). Gauge invariance of the fields is preserved by the reconstruction, and the time\,–\,zero local core is mapped to a common invariant core for the field operators, on which the commutators act.
\end{proof}

\begin{theorem}[Uniform minorization in finite steps for a fixed power]\label{thm:uniform-minorization-fixed-power}
Fix $M\in\mathbb N$ and let $K_{\beta,L}^{\circ M}$ be the $M$--fold interface kernel. Assume:
\begin{itemize}
  \item[(Loc)] Deterministic finite--step locality for patches (Lemma~\ref{lem:finite-step-cone}/Corollary~\ref{cor:deterministic-locality}).
  \item[(Win)] A near--identity staple window $W_\varepsilon$ for each color--class block with probability $\ge p_0>0$, uniform in $(\beta,L,x)$.
  \item[(Ref)] Single--link refresh under $W_\varepsilon$ with a blockwise product lower bound as in Lemma~\ref{lem:single-link-refresh} (either fixed--radius or scale--adapted, possibly after finitely many microperiods), yielding some $c_*>0$ uniform in $(\beta,L,x)$ for the chosen block radius $r_G$.
\end{itemize}
Then there exist $r_G>0$ and $\eta_*>0$, independent of $(\beta,L,x)$, such that
\[
  K_{\beta,L}^{\circ 8M}(x,\cdot)\ \ge\ \eta_*\, Q_{\mathrm{patch}}(\cdot),
\]
where $Q_{\mathrm{patch}}$ conditions one fixed finite block in each of the eight color classes to $B_G(e,r_G)$ (Definition~\ref{def:block-reference}) and is Haar elsewhere on the interface.
\end{theorem}
\begin{proof}
By (Ref) and (Win) one obtains, on $W_\varepsilon$, a per--class block refresh lower bound $K_{\beta,L}^{\circ M}\ge c_* Q^{(B_\alpha)}$. Averaging over $W_\varepsilon$ yields the uniform class constant $\eta_B:=p_0 c_*>0$ by Proposition~\ref{prop:block-refresh-doeblin}. Scheduling classes in a Gray cycle and composing eight microperiods, Proposition~\ref{prop:block-to-patch} gives
\[
  K_{\beta,L}^{\circ 8M}(x,\cdot)\ \ge\ \big(1-(1-\eta_B)^8\big)\, Q_{\mathrm{patch}}(\cdot).
\]
Set $\eta_*:=1-(1-\eta_B)^8$. Locality (Loc) ensures boundary independence outside a finite cone, so the constants are uniform in $L$ and in the boundary $x$.
\end{proof}
\begin{remark}
The constant $\eta_*$ depends only on the block size $b$, the window probability $p_0$, the single--link constants from Lemma~\ref{lem:single-link-refresh}, and small--ball Haar volumes (Lemma~\ref{lem:small-ball-volume}); it is independent of $\beta$, $L$, and $x$. In the scale--adapted case of Lemma~\ref{lem:single-link-refresh}(a), one may take a fixed $r_G$ after composing a finite number of microperiods so that the compounded refresh constant remains uniform.
\end{remark}

\begin{lemma}[Central heat--kernel pulse preserves symmetries]\label{lem:central-pulse}
Let $G=\mathrm{SU}(N)$ and let $H_t$ be the central heat--kernel density at time $t>0$. For a finite block $B$ of interface coordinates, define the convolution operator
\[
  (\mathcal H_t f)(u)\ :=\ \int_{G^{B}} \Big(\prod_{\ell\in B} H_t(v_\ell^{-1}u_\ell)\Big)\, f\big( v_B, u_{B^c}\big)\, d\pi^{\otimes B}(v_B),
\]
acting as heat--kernel convolution on $B$ and identity on $B^c$. Then $\mathcal H_t$ is positivity preserving, Haar--invariant on $B^c$, commutes with left/right translations (gauge covariance), and is compatible with OS reflection.
\end{lemma}
\begin{proof}
Positivity preservation and Haar invariance follow from convolution with a positive central density and product Haar on $G^{B}$. Centrality of $H_t$ implies that for any $g\in G$, $H_t(g^{-1}xg)=H_t(x)$, yielding commutation with conjugations and left/right translations (gauge covariance at the block). Reflection compatibility holds since $H_t$ is time--slice local and central, hence invariant under the OS involution on the interface.
\end{proof}

\begin{proposition}[Sandwiching by a fixed pulse]\label{prop:sandwich-pulse}
Let $B$ be a finite block and suppose there exist $t_*>0$ and $c_*>0$ such that the operator inequality holds on $L^2_0$:
\[
  K_{\beta,L}^{\circ M}\ \ge\ c_*\, \mathcal H_{t_*}
\]
(as positive kernels). Then for any fixed radius $r_G>0$ there exists $\eta_0=\eta_0(t_*,r_G,c_*,G)>0$ such that
\[
  K_{\beta,L}^{\circ M}(x,\cdot)\ \ge\ \eta_0\, Q^{(B)}(\cdot)
\]
for the small--ball block law $Q^{(B)}$ of Definition~\ref{def:block-reference}, uniformly in $(\beta,L,x)$.
\end{proposition}
\begin{proof}
From Lemma~\ref{lem:central-pulse}, $\mathcal H_{t_*}$ has a strictly positive continuous density on $G^{B}$; in particular, $\inf_{u\in B_G(e,r_G)^{B}} (\mathcal H_{t_*}\mathbf 1)(u)\ge c(r_G,t_*,G)>0$. The sandwich then gives, for any measurable $A$,
\[
  K_{\beta,L}^{\circ M}(x,A)\ \ge\ c_*\, (\mathcal H_{t_*}\mathbf 1_{A})(x)\ \ge\ c_*\, c(r_G,t_*,G)\, Q^{(B)}(A),
\]
setting $\eta_0:=c_*\, c(r_G,t_*,G)>0$.
\end{proof}

\begin{proposition}[Uniform ergodicity in finite blocks]\label{prop:uniform-ergodicity-blocks}
Suppose there exist $M\in\mathbb N$, $\eta_0\in(0,1]$, and a probability $\nu$ on the interface space such that for all $x$ and measurable $B$,
\[
  K_{\beta,L}^{\circ M}(x,B)\ \ge\ \eta_0\, \nu(B),
\]
with $\eta_0$ independent of $(\beta,L,x)$. Then for any probability densities $p,q$ on the interface configuration space and any $n\in\mathbb N$,
\[
  \big\|\, p (K_{\beta,L}^{\circ M})^{n} - q (K_{\beta,L}^{\circ M})^{n} \,\big\|_{\mathrm{TV}}
   \ \le\ (1-\eta_0)^{n}\, \|p-q\|_{\mathrm{TV}}.
\]
\end{proposition}
\begin{proof}
Doeblin's condition yields a one--step coupling for $K_{\beta,L}^{\circ M}$ with success probability $\eta_0$. Iterating the coupling gives geometric decay of the total--variation distance by the factor $1-\eta_0$ per $M$--block.
\end{proof}

\begin{theorem}[Exponential clustering along interface time]\label{thm:exp-cluster-interface}
Let $F,G$ be bounded observables depending on disjoint time slices separated by $n$ blocks of length $M$. Under the hypothesis of Proposition~\ref{prop:uniform-ergodicity-blocks},
\[
  \big|\, \mathrm{Cov}\,(F,\, G\circ (K_{\beta,L}^{\circ M})^{n})\,\big|\ \le\ 2\,\|F\|_\infty\,\|G\|_\infty\,(1-\eta_0)^{n}.
\]
Equivalently, writing the physical separation as $t=n\,T_{\mathrm{block}}$ with $T_{\mathrm{block}}:=M\,a$,
\[
  \big|\, \mathrm{Cov}\,(F,\, G_t)\,\big|\ \le\ 2\,\|F\|_\infty\,\|G\|_\infty\, \exp\!\Big(-\, \tfrac{t}{T_{\mathrm{block}}}\, |\log(1-\eta_0)|\Big).
\]
\end{theorem}
\begin{proof}
Write centered versions $\widetilde F:=F-\mathbb E F$, $\widetilde G:=G-\mathbb E G$. The covariance equals $\int \widetilde F\, d\mu\, - \int \widetilde F\, d\mu'$, where $\mu' := (K_{\beta,L}^{\circ M})^{n}\mu$ and the initial measures differ only through the slice of $G$. By the total--variation contraction in Proposition~\ref{prop:uniform-ergodicity-blocks},
\[
  |\mathrm{Cov}(F, G\circ (K_{\beta,L}^{\circ M})^{n})| \ \le\ \|\widetilde F\|_\infty\, \big\|\mu (K_{\beta,L}^{\circ M})^{n} - \mu' (K_{\beta,L}^{\circ M})^{n}\big\|_{\mathrm{TV}}
   \ \le\ 2\,\|F\|_\infty\,\|G\|_\infty\,(1-\eta_0)^{n},
\]
using $\|\widetilde F\|_\infty\le 2\|F\|_\infty$ and an analogous bound for $\widetilde G$.
\end{proof}

\begin{proposition}[From block refresh to patch refresh in finite time]\label{prop:block-to-patch}
Let $\{\mathcal C_\alpha\}_{\alpha\in\{0,1\}^3}$ be the eight parity classes of interface links (Proposition~\ref{prop:eight-color-schedule}). Suppose each class contains a fixed--size block $B_\alpha\subset\mathcal C_\alpha$ such that the microperiod kernel $K_{\beta,L}^{\circ M}$ satisfies, whenever class $\alpha$ is scheduled,
\[
  K_{\beta,L}^{\circ M}(x,\cdot)\ \ge\ \eta_B\, Q^{(B_\alpha)}(\cdot)
\]
with the same $\eta_B>0$ for all $(\alpha,\beta,L,x)$. Then after eight microperiods in a Gray--code schedule,
\[
  K_{\beta,L}^{\circ 8M}(x,\cdot)\ \ge\ \eta_*\, Q_{\mathrm{patch}}(\cdot),\qquad
  \eta_*\ :=\ 1-(1-\eta_B)^8,
\]
where $Q_{\mathrm{patch}}$ is the product law that conditions each class block to $B_G(e,r_G)$ (as in Definition~\ref{def:block-reference}) and is Haar elsewhere.
\end{proposition}
\begin{proof}
Write $\mathcal K_k$ for the kernel after $k$ microperiods and let $\mathcal R_k$ denote the set of refreshed blocks after $k$ steps. The hypothesis yields the mixture lower bound
\[
  \mathcal K_{k+1}\ \ge\ (1-\eta_B)\,\mathcal K_k\ +\ \eta_B\, Q^{(B_{\alpha_k})}\,.
\]
By induction and disjointness of the blocks, after $k$ distinct classes the lower bound is a convex combination of $\{Q^{(B_{\alpha_j})}\}_{j\le k}$ with total weight $1-(1-\eta_B)^k$. After eight distinct classes (Gray cycle), the product structure of $Q_{\mathrm{patch}}$ and independence across disjoint coordinates give
\[
  \mathcal K_{8}\ \ge\ 1-(1-\eta_B)^8\, \cdot\, Q_{\mathrm{patch}}\,.
\]
Renaming $\mathcal K_8=K_{\beta,L}^{\circ 8M}$ yields the claim with $\eta_*=1-(1-\eta_B)^8$.
\end{proof}
\begin{proposition}[Interface density: absolute continuity and $\beta$--uniform ball--average bound]\label{prop:interface-density-ball-avg}
Work on a fixed physical slab $R\supset\Sigma$ and $a\in(0,a_0]$ with $m:=|\Sigma|$ interface links and Haar probability $\pi$ on $G=\mathrm{SU}(N)$. For any exterior boundary $b$ and any $\beta\ge \beta_{\min}(R,N)$, let $\mu_{\Sigma}^{(\beta,b)}(du)=f_{\beta,b}(u)\,\pi^{\otimes m}(du)$ denote the interface marginal.
\begin{itemize}
  \item[(i)] $f_{\beta,b}\in C^\infty(G^m)$ and $f_{\beta,b}>0$ everywhere.
  \item[(ii)] There exist constants $L_\Sigma=L_\Sigma(R,N)$ and $C_1=C_1(R,N)$ such that for all $u\in G^m$ and all $r\in(0,1)$,
  \[
    \frac{1}{\pi^{\otimes m}(B_r)}\int_{B_r(u)} f_{\beta,b}(v)\,\pi^{\otimes m}(dv)
    \ \ge\ e^{-\beta\,(L_\Sigma r + C_1 r^2)}\, f_{\beta,b}(u),
  \]
  where $B_r(u)\subset G^m$ is the geodesic ball of radius $r$ (for a fixed bi--invariant metric) and $\pi^{\otimes m}(B_r):=\pi^{\otimes m}(B_r(u))$ is its Haar mass (independent of $u$). In particular, for $r=\kappa/\beta$ with $\kappa\in(0,1)$,
  \[
    \frac{1}{\pi^{\otimes m}(B_{\kappa/\beta})}\int_{B_{\kappa/\beta}(u)} f_{\beta,b}(v)\,\pi^{\otimes m}(dv)
    \ \ge\ c_*(R,N)\,\kappa^{\,m\dim G}\, e^{-L_\Sigma\kappa}\, f_{\beta,b}(u),
  \]
  with $c_*(R,N)>0$ depending only on the local geometry and $N$.
\end{itemize}
\end{proposition}
\begin{proof}
Tree gauge on a spanning tree $T\subset E(R)$ that avoids $\Sigma$ fixes $U_e=\mathbf 1$ for $e\in T$ by vertex gauges; the associated change of variables is a product of left/right translations and preserves Haar measure on each link, so the joint law on $(u,y)=(U|_{\Sigma},U|_Y)$ is $Z_R^{-1}e^{-S_R(u,y;b)}\,\pi^{\otimes m}(du)\,\pi^{\otimes |Y|}(dy)$. Since $S_R$ is smooth and strictly positive, Fubini implies $Z(u):=\int e^{-S_R(u,y;b)}\,\pi^{\otimes |Y|}(dy)$ is $C^\infty$ and strictly positive on $G^m$, hence $f_{\beta,b}(u):=Z(u)/\int Z\,d\pi^{\otimes m}\in C^\infty$ and $>0$ (this proves (i)).

For (ii), let $\mathcal P_\Sigma$ denote the plaquettes in $R$ that depend on $u$. For the Wilson term $\phi_p(U)=1-\tfrac1N\operatorname{ReTr}\,U_p$, one has a uniform differential bound $\|\nabla_{U_e}\phi_p\|\le C_p(N)$, whence, for some $L_\Sigma=C_p(N)\,|\mathcal P_\Sigma|$,
\[
  |S_R(u,y;b)-S_R(v,y;b)|\ \le\ \beta\,L_\Sigma\, d_{G^m}(u,v)\qquad(\text{all }y,b),
\]
with $d_{G^m}$ the product Riemannian distance. Set $h_{u,v}(Y):=S_R(u,Y;b)-S_R(v,Y;b)$. After tree gauge, Theorem~\ref{thm:U1-lsi-uei} gives an LSI for the conditional measure $\mu_v(dy)\propto e^{-S_R(v,y;b)}\,\pi^{\otimes |Y|}(dy)$ with constant $\rho_R\ge c(R,N)\,\beta$. Moreover, $\|\nabla_Y h_{u,v}\|\le L_0(R,N)\, d_{G^m}(u,v)$. The Herbst argument under LSI yields the local log--Lipschitz estimate
\[
  \big|\log Z(u)-\log Z(v)\big|\ \le\ \beta\,\big(L_\Sigma r + C_1 r^2\big)\qquad(r:=d_{G^m}(u,v)),\quad C_1:=\tfrac{L_0(R,N)^2}{2c(R,N)}.
\]
Fix $u$ and average over $v\in B_r(u)$; by concavity of $\log$,
\[
  \frac{1}{\pi^{\otimes m}(B_r)}\int_{B_r(u)}\! \log Z(v)\,d\pi^{\otimes m}(v)\ \le\ \log\Big(\frac{1}{\pi^{\otimes m}(B_r)}\int_{B_r(u)}\! Z(v)\,d\pi^{\otimes m}(v)\Big),
\]
so the previous display implies
\[
  Z(u)\ \le\ e^{\beta(L_\Sigma r + C_1 r^2)}\,\frac{1}{\pi^{\otimes m}(B_r)}\int_{B_r(u)} Z(v)\,d\pi^{\otimes m}(v)
\]
and, by symmetry of the log--Lipschitz bound, also the reverse inequality with $\ge$ and $e^{-\beta(\cdots)}$. Dividing by $\int Z\,d\pi^{\otimes m}$ gives the stated two--sided control of the ball average in terms of $f_{\beta,b}(u)$; the displayed lower bound follows. The small--ball volume asymptotics on compact Lie groups (uniform in $u$) yield $\pi^{\otimes m}(B_{\kappa/\beta})\ge c_*(R,N)\, (\kappa/\beta)^{m\dim G}$, which gives the explicit form when $r=\kappa/\beta$.
\end{proof}

\begin{remark}[No pointwise $\beta$--uniform lower bound without smoothing]\label{rem:no-pointwise-lower}
Because $S_R$ carries an explicit factor $\beta$, the log--Lipschitz estimate shows that $\log f_{\beta,b}$ can oscillate by $\asymp \beta$ over $O(1)$ distances. On a compact group this precludes any pointwise lower bound $\inf f_{\beta,b}\ge c>0$ that is uniform in $\beta$ without either shrinking the radius $r\sim 1/\beta$ in an averaged statement as above, or introducing short--time heat--kernel smoothing. The latter is exactly what yields the convex split $K_{\rm int}^{(a)}=\theta_* P_{t_0}+(1-\theta_*)\mathcal K$ in Proposition~\ref{prop:doeblin-full}/Corollary~\ref{cor:convex-split}.
\end{remark}

\begin{lemma}[Small-ball Haar volume on compact simple Lie groups]\label{lem:small-ball-volume}
Let $G$ be a compact simple Lie group with bi--invariant Riemannian metric and normalized Haar measure $\lambda_G$. There exist $r_*>0$ and $C_G>0$ such that for all $r\in(0,r_*)$ the geodesic ball $B_G(e,r)$ satisfies
\[
  \lambda_G\big(B_G(e,r)\big)\ \ge\ C_G\, r^{\dim G}.
\]
In particular, for $G=\mathrm{SU}(3)$ one may take $\dim G=8$ and obtain $\lambda_G(B_G(e,r))\ge C_G r^{8}$ for small $r$.
\end{lemma}
\begin{proof}
For sufficiently small $r$, the exponential map $\exp: \mathfrak g\to G$ is a diffeomorphism from the metric ball $B_{\mathfrak g}(0,r)$ onto $B_G(e,r)$. The Haar measure coincides with the Riemannian volume, whose density in normal coordinates is smooth with Jacobian $J(X)$ satisfying $J(0)=1$ and $J(X)\ge c_0>0$ on $B_{\mathfrak g}(0,r_*)$ for some $r_*>0$. Therefore
\[
  \lambda_G\big(B_G(e,r)\big)
   \ =\ \int_{B_{\mathfrak g}(0,r)} J(X)\,dX
   \ \ge\ c_0\, \operatorname{Vol}_{\mathbb R^{\dim G}}\big(B_{\mathfrak g}(0,r)\big)
   \ \ge\ C_G\, r^{\dim G}
\]
with $C_G:=c_0\, \operatorname{Vol}(B_{\mathbb R^{\dim G}}(0,1))$.
\end{proof}

\begin{corollary}[Concrete choice of $r_G$ for $\mathrm{SU}(3)$]\label{cor:explicit-rG-SU3}
Let $G=\mathrm{SU}(3)$ and let $r_*>0$ and $C_G>0$ be as in Lemma~\ref{lem:small-ball-volume}. For any target $\theta\in(0, C_G r_*^{8}]$, define
\[
  r_G\ :=\ \min\Big\{\,r_*\,,\ \big(\theta/C_G\big)^{1/8}\,\Big\}.
\]
Then $\lambda_G\big(B_G(e,r_G)\big)\ge \theta$. This provides an explicit small--ball radius for the block reference measure $Q^{(B)}$ (Definition~\ref{def:block-reference}) with constants independent of $\beta$ and $L$.
\end{corollary}

\begin{remark}[No global one--step minorization with atomic references]\label{rem:no-global-minorization-atomic}
A global one--step Doeblin bound $K_{\rm int}^{(a)}(U,\cdot)\ge \rho\,\nu(\cdot)$ cannot hold uniformly in $(\beta,L,U)$ if the reference $\nu$ is supported on a set of Haar measure zero (e.g., a Dirac mass or a finite atomic combination). Indeed, for large $\beta$ and suitable negative--half boundary data, the conditional law develops sharply peaked modes around configurations that depend on the boundary; choosing disjoint small neighborhoods of two such modes yields a contradiction with any fixed atomic $\nu$ and uniform $\rho>0$. This does not contradict the heat--kernel convex split of Corollary~\ref{cor:convex-split}, where $\nu=P_{t_0}$ is absolutely continuous with a smooth, strictly positive density on $G^m$.
\end{remark}

\begin{lemma}[Finite--step domain of dependence for interface patches]\label{lem:finite-step-cone}
Let $P\subset \Sigma$ be the set of interface links whose midpoints lie in a fixed spatial ball $B_{R_*}\cap\Sigma$. For $n\in\mathbb N$, let $K^{(n)}_{\rm int}:=(K_{\rm int}^{(a)})^{\,n}$ and let $\mathcal F_P$ be the sigma--algebra generated by the outgoing interface links on $P$ at time $n a$. Define the backward $n$--step lattice cone $\mathcal C_n(P)$ as the smallest set of negative--half links with the property that every plaquette path of length $\le n$ in the time--oriented lattice graph from $P$ to the negative half is contained in $\mathcal C_n(P)\cup P$. Then for any bounded $\mathcal F_P$--measurable $\varphi$ and any two negative--half configurations $U,U'$ with $U|_{\mathcal C_n(P)}=U'|_{\mathcal C_n(P)}$ one has
\[
  \big(K^{(n)}_{\rm int}\varphi\big)(U)\ =\ \big(K^{(n)}_{\rm int}\varphi\big)(U').
\]
Equivalently, $K^{(n)}_{\rm int}$ restricted to observables on $P$ depends only on the boundary data on $\mathcal C_n(P)$.
\end{lemma}
\begin{proof}
We argue by induction on $n$. For $n=1$, $K_{\rm int}^{(a)}$ is obtained by integrating the positive slab of thickness $a$. By locality, the Wilson action on that slab decomposes as $S_{\rm slab}=S_{\rm loc}(U|_{\mathcal C_1(P)},\,U|_P,\,Y_{\rm loc})+S_{\rm out}(Y_{\rm out})$, where $Y_{\rm loc}$ collects links in the positive slab that belong to plaquettes meeting $\mathcal C_1(P)\cup P$, and $Y_{\rm out}$ the remaining positive--half links. Hence the numerator $Z(U;\varphi):=\int e^{-S_{\rm slab}}\,\varphi\,d\pi$ and the normalizing factor $Z(U):=\int e^{-S_{\rm slab}}d\pi$ factor through $S_{\rm out}$, which cancels in the ratio. Therefore $(K_{\rm int}^{(a)}\varphi)(U)$ depends only on $U|_{\mathcal C_1(P)}$.

Assume the claim for $n-1$. Then $K^{(n)}_{\rm int}\varphi=K_{\rm int}^{(a)}\big(K^{(n-1)}_{\rm int}\varphi\big)$. By the induction hypothesis, $\psi:=K^{(n-1)}_{\rm int}\varphi$ depends only on boundary data in $\mathcal C_{n-1}(P)$. Applying the $n=1$ case to $\psi$ with patch enlarged to the set of interface links that can influence $P$ in $n-1$ steps shows that $K_{\rm int}^{(a)}\psi$ depends only on boundary data in $\mathcal C_1(\mathcal C_{n-1}(P))$, which is precisely $\mathcal C_n(P)$ by definition of plaquette paths. This completes the induction.
\end{proof}

\begin{remark}[Boundary decoupling and van Hove limit]\label{rem:cone-decoupling}
Fix $T>0$ and set $n=\lceil T/a\rceil$. For $L a\gg R_*$, the backward cone $\mathcal C_n(P)$ is contained in a finite region independent of $L$, so modifications of the boundary outside $\mathcal C_n(P)$ leave $K^{(n)}_{\rm int}$ unchanged. In particular, along van Hove sequences the influence of boundary conditions outside a fixed physical neighborhood of $P$ is exactly zero for $K^{(n)}_{\rm int}$ and, by positivity and locality, exponentially small for correlated observables propagated beyond $n$ steps; cf. Proposition~\ref{prop:bc-robust}.
\end{remark}

\begin{corollary}[Deterministic locality radius for interface dependence]\label{cor:deterministic-locality}
Let $S\subset \Sigma$ be a finite set of interface links (a patch). For any $n\in\mathbb N$, the $n$--step interface kernel $K^{(n)}_{\rm int}:=(K_{\rm int}^{(a)})^{\,n}$ restricted to observables supported on $S$ depends only on
\begin{itemize}
  \item the negative--half boundary configuration on the backward cone $\mathcal C_n(S)$ (plaquette paths of length $\le n$ reaching $S$ across $\Sigma$), and
  \item the positive--half links within the forward $n$--neighborhood of $S$ in the oriented plaquette graph.
\end{itemize}
Consequently, if two negative--half configurations agree on $\mathcal C_n(S)$, then $K^{(n)}_{\rm int}$ yields the same law on $S$ for both boundaries. Along van Hove sequences ($a\downarrow 0$, $La\to\infty$) with fixed $n$, dependence on far boundary data vanishes exactly by locality.
\end{corollary}
\begin{proof}
Apply Lemma~\ref{lem:finite-step-cone} to the patch $S$ and note that, by locality of the Wilson action, links in the positive half outside the forward $n$--neighborhood factor from both the numerator and denominator of the conditional kernels at each step. Composition preserves this property.
\end{proof}

\begin{proposition}[Eight--color schedule on the interface]\label{prop:eight-color-schedule}
Identify time--like interface links by their spatial footpoints $(i,j,k)\in\mathbb Z^3$ on the reflection plane. For $\alpha\in\{0,1\}^3$, define the classes
\[
  \mathcal C_{\alpha}
   \\ :=\ \big\{\text{interface links with }(i\bmod 2,\,j\bmod 2,\,k\bmod 2)=\alpha\big\}.
\]
Then no two links in the same class share a time--space plaquette. Moreover, visiting the classes in any Gray--code order on $\{0,1\}^3$ gives an $8$--tick cycle that updates each class once without plaquette conflicts.
\end{proposition}
\begin{proof}
Two time--like interface links share a time--space plaquette iff their footpoints differ by $\pm1$ in exactly one spatial coordinate and are equal in the other two. Such a move flips parity in that coordinate, so the two links lie in different parity classes $\mathcal C_{\alpha}$. Hence updates within a fixed class are plaquette--disjoint. A Gray code on the $3$--cube is a Hamiltonian cycle that visits each parity vector once with successive vectors differing in exactly one bit, so scheduling classes according to a Gray order yields an $8$--tick conflict--free cycle.
\end{proof}

\begin{definition}[Block reference law]\label{def:block-reference}
Let $B\subset \Sigma$ be a finite block of interface links with $|B|=b$ independent of $(\beta,L)$ (e.g., one link from each parity class of Proposition~\ref{prop:eight-color-schedule}). Fix a small group radius $r_G>0$. Define the probability law $Q^{(B)}$ on the interface configuration space by taking the coordinates in $B$ to be i.i.d. Haar restricted to the geodesic ball $B_G(e,r_G)$ (normalized), and all other coordinates Haar on $G$; i.e., $Q^{(B)}$ is the product of these marginals.
\end{definition}

\begin{lemma}[One--link refresh $\Rightarrow$ Doeblin for a fixed power]\label{lem:one-link-doeblin}
Fix $M\in\mathbb N$ and a singleton block $B=\{\ell_*\}$. If there exists $\eta_0\in(0,1]$ such that for all $x$,
\[
  K_{\beta,L}^{\circ M}(x,\cdot)\ \ge\ \eta_0\, Q^{(B)}(\cdot),
\]
then $K_{\beta,L}^{\circ M}$ satisfies a Doeblin/minorization with constant $\rho=\eta_0$ and reference $\nu=Q^{(B)}$. If $\eta_0$ is independent of $(\beta,L,x)$, the bound is uniform in these parameters.
\end{lemma}
\begin{proof}
The displayed inequality is exactly the Doeblin/minorization with $(\rho,\nu)=(\eta_0, Q^{(B)})$; uniformity follows when $\eta_0$ is parameter--independent.
\end{proof}

\begin{corollary}[Eight--tick one--link case]\label{cor:8tick-one-link}
If the hypothesis of Lemma~\ref{lem:one-link-doeblin} holds with $M=8$ and $\eta_0>0$ independent of $(\beta,L,x)$, then $K_{\beta,L}^{\circ 8}$ obeys a uniform Doeblin bound with $(\rho,\nu)=(\eta_0, Q^{(B)})$.
\end{corollary}
\begin{proof}
Apply Lemma~\ref{lem:one-link-doeblin} with $M=8$.
\end{proof}

\begin{proposition}[Finite--block refresh $\Rightarrow$ Doeblin for a power]\label{prop:block-refresh-doeblin}
Let $K_{\beta,L}$ be the one--slice interface kernel and $K_{\beta,L}^{\circ M}$ the $M$--fold composition for some fixed $M\in\mathbb N$. Suppose there exists a measurable positive--side window $W_\varepsilon$ and constants $p_0,c_*>0$ (independent of $(\beta,L,x)$) such that for all boundary data $x$:
\[
  \text{(i) }\mathbb P(W_\varepsilon)\ \ge\ p_0,\qquad
  \text{(ii) on }W_\varepsilon:\quad K_{\beta,L}^{\circ M}(x,\cdot)\ \ge\ c_*\, Q^{(B)}(\cdot).
\]
Then for all $(\beta,L,x)$ and measurable $A$,
\[
  K_{\beta,L}^{\circ M}(x,A)\ \ge\ \eta_B\, Q^{(B)}(A),\qquad \eta_B:=p_0 c_*\,.
\]
\end{proposition}
\begin{proof}
Decompose according to $W_\varepsilon$:
\[
  K_{\beta,L}^{\circ M}(x,A)
   \ =\ \mathbb E\big[\,\mathbf 1_{W_\varepsilon}\, K_{\beta,L}^{\circ M}(x,A)\,\big]
       + \mathbb E\big[\,\mathbf 1_{W_\varepsilon^c}\, K_{\beta,L}^{\circ M}(x,A)\,\big]
   \ \ge\ p_0\, c_*\, Q^{(B)}(A),
\]
using (ii) on $W_\varepsilon$ and (i) for $\mathbb P(W_\varepsilon)$. This yields the stated Doeblin lower bound with constant $\eta_B=p_0 c_*$. 
\end{proof}

\begin{lemma}[Single--link refresh under near--identity staples]\label{lem:single-link-refresh}
Let $G=\mathrm{SU}(N)$ with Haar probability $\pi$. Fix an interface link $\ell$ and write its positive--side staple product as $H_\ell\in G$ (the product of adjacent plaquette transporters not involving $U_\ell$). There exist $\varepsilon_0,r_0,\kappa_0>0$ and constants $c_0,C>0$ (depending only on $N$ and local geometry) such that for all $\beta\ge\beta_{\min}>0$:
\begin{itemize}
  \item[(a) Scale--adapted form (\(\beta\)--uniform).] If $H_\ell\in B_G(e,\varepsilon_0)$, then for every $\kappa\in(0,\kappa_0)$ and $r_G:=\kappa\,\beta^{-1/2}\le r_0$,
  \[
    \mathbb P\big(\,U_\ell\in B_G(e,r_G)\ \big|\ \text{all other variables}\,\big)
      \ \ge\ c_0\, \kappa^{\,\dim G}\, e^{-C\,\kappa^2},
  \]
  with the right side independent of $(\beta,L,x)$.
  \item[(b) Fixed--radius variant.] For any fixed $r_G\in(0,r_0]$ there exists $c(r_G,\varepsilon_0)>0$ such that under $H_\ell\in B_G(e,\varepsilon_0)$,
  \[
    \mathbb P\big(\,U_\ell\in B_G(e,r_G)\ \big|\ \text{all other variables}\,\big)
      \ \ge\ c(r_G,\varepsilon_0)\, e^{-C\,\beta\, r_G^2},
  \]
  which is non--uniform in $\beta$ but useful at bounded $\beta$.
\end{itemize}
\end{lemma}
\begin{proof}[Proof sketch]
Condition on all variables except $U_\ell$. The conditional density has the standard heatbath form
\[
  d\mu(U_\ell)\ \propto\ \exp\{\, \beta\,\operatorname{Re}\operatorname{tr}(U_\ell W_\ell)\,\}\,\pi(dU_\ell),
\]
where $W_\ell$ is a complex matrix built from the sum of staples at $\ell$. Write the polar decomposition $W_\ell= S_\ell A_\ell$ with $S_\ell\ge0$ and $A_\ell\in G$, so the density is $\propto \exp\{\, \beta\,\operatorname{Re}\operatorname{tr}(U_\ell A_\ell)\,\}$ up to a positive scale factor absorbed in $S_\ell$. Left--translate $U_\ell\mapsto A_\ell^{\,-1}U_\ell$ to reduce to the case $A_\ell=e$. If $H_\ell\in B_G(e,\varepsilon_0)$, then $A_\ell\in B_G(e,c\varepsilon_0)$ and $\|S_\ell\|$ is uniformly controlled by local geometry.

In normal coordinates $U_\ell=\exp X$ with $\|X\|\le r$. Using $\operatorname{Re}\operatorname{tr}(\exp X)= N - \tfrac12\|X\|^2 + O(\|X\|^3)$ and a uniform Jacobian bound $J(X)\ge c_J>0$ on $\|X\|\le r_0$, the probability of $B_G(e,r)$ is bounded below by
\[
  \frac{\displaystyle\int_{\|X\|\le r} \exp\{\beta(\operatorname{Re}\operatorname{tr}(\exp X)-N)\}\,J(X)\,dX}{\displaystyle\int_{\mathfrak g} \exp\{\beta(\operatorname{Re}\operatorname{tr}(\exp X)-N)\}\,J(X)\,dX}
  \ \gtrsim\ \frac{\int_{\|X\|\le r} e^{-\frac12\beta\|X\|^2 - C'\beta\|X\|^3} dX}{\int_{\mathfrak g} e^{-c_1\beta\|X\|^2} dX}
  \ \ge\ c_0\, r^{\dim G}\, e^{-C\,\beta r^2},
\]
for $r\le r_0$. Choosing $r=\kappa\beta^{-1/2}$ yields (a). Keeping $r$ fixed yields (b).
\end{proof}


\begin{proposition}[Non-Gaussianity: nonzero truncated 4-point for local fields]\label{prop:nonzero-cumulant4}
There exist compactly supported smooth test functions $f_1,\ldots,f_4\in C_c^\infty(\mathbb R^4,\wedge^2\mathbb R^4)$, supported in a fixed bounded region $R\Subset\mathbb R^4$, such that the truncated 4-point function of the clover field $\Xi$ satisfies
\[
  \langle \Xi(f_1)\,\Xi(f_2)\,\Xi(f_3)\,\Xi(f_4)\rangle_c\ \neq\ 0.
\]
In particular, the continuum law of the local fields is not Gaussian.
\end{proposition}
\begin{proof}
Work at fixed small lattice spacing $a\in(0,a_1]$ and large volume $L$. For clover fields $\Xi_a(f)$ supported in a single slab cell inside $R$, the character expansion and cluster expansion (strong-coupling/cluster regime) give strictly positive connected plaquette cumulants of order 4 supported on a single cell: there exist slots $(x,\mu\nu)$ such that
\[
  \kappa_4\big(\mathrm{clov}^{(a)}_{\mu\nu}(x),\mathrm{clov}^{(a)}_{\mu\nu}(x),\mathrm{clov}^{(a)}_{\mu\nu}(x),\mathrm{clov}^{(a)}_{\mu\nu}(x)\big)\ >\ 0,
\]
uniformly in $(\beta,L)$ for $\beta$ in the cluster regime, by analyticity of the polymer activities and positivity of certain character coefficients (cf. Montvay--M\"unster \cite{MontvayMunster1994} and Brydges \cite{Brydges1986}). Choose $f_1=\cdots=f_4=:f\in C_c^\infty(R)$ supported in that cell and nonnegative so that $\Xi_a(f)$ is a positive linear combination of those clover slots. Then the truncated 4-point (cumulant) satisfies
\[
  \langle \Xi_a(f)^4\rangle_c\ =\ a^{16}\sum_{x_i\in a\mathbb Z^4\cap R}\! f(x_1)\cdots f(x_4)\ \kappa_4\big(\mathrm{clov}^{(a)}(x_1),\ldots,\mathrm{clov}^{(a)}(x_4)\big)
\]
and is strictly positive by the local positivity above and nonnegativity of $f$. Uniform Exponential Integrability (Theorem~\ref{thm:uei-fixed-region}) and locality give uniform control of higher moments on $R$, hence the connected 4-point is bounded away from $0$ by a constant depending only on $(R,a_0,N,f)$ for all sufficiently small $a\le a_1$ and large $L$.

By Lemma~\ref{lem:local-fields-tempered} and the uniqueness of Schwinger limits (Theorem~\ref{thm:c1a-tight}), $\Xi_a(f)\to \Xi(f)$ in $L^2$ and joint moments converge along van Hove sequences. Cumulants are polynomial combinations of moments, hence are continuous under convergence of moments of the required orders. Therefore, the nonzero truncated 4\,–\,point persists in the continuum limit:
\[
  \langle \Xi(f)^4\rangle_c\ =\ \lim_{a\downarrow 0,L\to\infty}\ \langle \Xi_a(f)^4\rangle_c\ >\ 0.
\]
Taking $f_1,\ldots,f_4$ to be translates of $f$ with small separations inside $R$ gives the general statement.
\end{proof}

\begin{proposition}[Interface$\to$transfer domination on the odd cone]\label{prop:int-to-transfer}
Let $a\in(0,a_0]$ and fix a physical slab $B_{R_*}$ intersecting the reflection plane in thickness $a$. Let $\mathcal H_{L,a}$ be the OS/GNS Hilbert space with transfer $T=e^{-aH}$. For any $\psi=O\Omega\in\mathcal C_{R_*}$ (i.e., $O$ localized in $B_{R_*}$ with $\langle O\rangle=0$), define the interface $\sigma$--algebra $\mathcal F_{\rm int}$ generated by the $m=m_{\rm cut}(R_*,a_0)$ links meeting the cut and set
\[
  \varphi\ :=\ \mathbb E\big[\,O\ \big|\ \mathcal F_{\rm int}\,\big]\ \in\ L^2\big(G^m,\pi^{\otimes m}\big),\qquad G=\mathrm{SU}(N).
\]
Then:
\begin{itemize}
  \item[(i)] Quadratic form factorization: \(\langle \psi,\,T\psi\rangle\ =\ \langle \varphi,\,K_{\rm int}^{(a)}\,\varphi\rangle_{L^2(\pi^{\otimes m})}.\)
  \item[(ii)] Jensen contraction: \(\langle \psi,\psi\rangle\ \ge\ \langle \varphi,\varphi\rangle\), with equality if $O$ depends only on interface variables.
\end{itemize}
In particular, $\int \varphi\,d\pi^{\otimes m}=\mathbb E[O]=0$, so $\varphi\in L^2_0(G^m,\pi^{\otimes m})$, and
\[
  \frac{\langle \psi, T\psi\rangle}{\langle \psi,\psi\rangle}
  \ \le\ \frac{\langle \varphi, K_{\rm int}^{(a)}\varphi\rangle}{\langle \varphi,\varphi\rangle}
  \ \le\ \big\|K_{\rm int}^{(a)}\big\|_{L^2_0\to L^2_0}.
\]
Consequently, the operator norm of $T$ on the slab--odd cone satisfies
\[
  \big\|T\big\|_{\mathcal C_{R_*}}\ \le\ \big\|K_{\rm int}^{(a)}\big\|_{L^2_0\to L^2_0}.
\]
\end{proposition}
\begin{proof}
Disintegrate the Wilson measure across the reflection cut: write the configuration as $(U^-,U_{\rm int},U^+)$ with $U_{\rm int}\in G^m$ the interface links in the slab, and let $\mu_\beta(dU)=Z^{-1}\exp(-S_\beta(U))\,dU$ be the Gibbs measure. By the standard OS construction and stationarity under one-tick time translation $\tau_1$,
\[
  \langle \psi, T\psi\rangle\ =\ \int \overline{O(U)}\,\big(\theta\,\tau_1 O\big)(U)\,d\mu_\beta(U).
\]
Decompose $S_\beta=S_\beta^{(+)}+S_\beta^{(-)}+S_\beta^{(\perp)}$ and integrate out the off-interface degrees of freedom using conditional expectations given $\mathcal F_{\rm int}$. By definition of the interface kernel $K_{\rm int}^{(a)}$ (the conditional law of outgoing interface variables across the cut; see Proposition~\ref{prop:doeblin-full}), one obtains the exact identity
\[
  \langle \psi, T\psi\rangle\ =\ \int \overline{\varphi(U_{\rm int})}\,\big(K_{\rm int}^{(a)}\varphi\big)(U_{\rm int})\,d\pi^{\otimes m}(U_{\rm int})
  \ =\ \langle \varphi, K_{\rm int}^{(a)}\varphi\rangle_{L^2(\pi^{\otimes m})}.
\]
Positivity of conditional expectation on $L^2$ (Jensen) yields $\|\varphi\|_{L^2}^2\le \|O\|_{L^2(\mu_\beta)}^2$, which is (ii) since $\|\psi\|^2=\langle O,\theta O\rangle=\|O\|_{L^2(\mu_\beta)}^2$ in the OS/GNS quotient. Finally, $\mathbb E[\varphi]=\mathbb E[O]=0$ because $\psi\in\mathcal C_{R_*}$ has mean zero. The Rayleigh quotient bounds then give the stated domination of the operator norm on the odd cone.
\end{proof}

\begin{corollary}[Uniform one-tick contraction on the odd cone]\label{cor:odd-contraction-from-Kint}
If the interface kernel admits the convex split $K_{\rm int}^{(a)}=\theta_* P_{t_0}+(1-\theta_*)\,\mathcal K_{\beta,a}$ with $\theta_*\in(0,1]$, $t_0>0$ independent of $(\beta,L)$, then on $L^2_0$ one has $\|K_{\rm int}^{(a)}\|\le 1-\theta_* e^{-\lambda_1(N) t_0}$. Consequently, on the OS/GNS slab--odd cone
\[
  \|e^{-aH}\psi\|\ \le\ \big(1-\theta_* e^{-\lambda_1(N) t_0}\big)\,\|\psi\|\qquad(\psi\in\mathcal C_{R_*}\cap\{P_i\psi=-\psi\}),
\]
and the per-tick rate
\[
  c_{\rm cut}(a)\ :=\ -\frac{1}{a}\log\big(1-\theta_* e^{-\lambda_1(N) t_0}\big)\ >\ 0
\]
depends only on $(R_*,a_0,N)$. Composing eight ticks yields the lattice gap lower bound $\gamma_0\ge 8\,c_{\rm cut}(a)$ on $\Omega^{\perp}$, uniformly in $(\beta,L)$.
\end{corollary}
\begin{proof}
On $L^2_0$, $\|P_{t_0}\|=e^{-\lambda_1(N) t_0}$ and $\|\mathcal K_{\beta,a}\|\le 1$, so $\|K_{\rm int}^{(a)}\|\le 1-\theta_* e^{-\lambda_1 t_0}$. Apply Proposition~\ref{prop:int-to-transfer} and use that $T$ is positive self-adjoint, hence $\|T\|=\sup_{\|\psi\|=1}\langle\psi,T\psi\rangle$.
\end{proof}

\begin{lemma}[Local odd density]\label{lem:odd-density}
For any spatial reflection $P_i$ acting unitarily on $\mathcal H_{L,a}$ (leaving $\Omega$ fixed and commuting with $T$), the $(-1)$ eigenspace $\mathcal H_{\rm odd}^{(i)}:=\{\psi:\ P_i\psi=-\psi\}$ is the norm-closure of
\[
  \bigcup_{R>0}\ \Big\{\,O^{(-,i)}\Omega:\ O\in \mathfrak A_0^{\rm loc},\ \langle O\rangle=0,\ \mathrm{supp}(O)\subset B_R\,\Big\}.
\]
In particular, the slab-local odd cone $\mathcal C_{R_*}\cap\{P_i\psi=-\psi\}$ is dense in $\mathcal H_{\rm odd}^{(i)}$ as $R_*\to\infty$.
\end{lemma}
\begin{proof}
By OS/GNS, the cyclic subspace generated by the time-zero local algebra $\mathfrak A_0^{\rm loc}$ acting on $\Omega$ is dense in $\mathcal H_{L,a}$. The odd projector $\Pi_{\rm odd}^{(i)}:=\tfrac12(I-P_i)$ is a bounded orthogonal projection commuting with $T$. Therefore, the image under $\Pi_{\rm odd}^{(i)}$ of a dense set is dense in its range $\mathcal H_{\rm odd}^{(i)}$. Approximating with observables supported in $B_R$ and letting $R\to\infty$ yields the claim.
\end{proof}

\begin{theorem}[Uniform one-tick contraction on the full parity-odd subspace]\label{thm:uniform-odd-contraction}
Assume the convex split with explicit constants (Proposition~\ref{prop:explicit-doeblin-constants}): $K_{\rm int}^{(a)}=\theta_* P_{t_0}+(1-\theta_*)\mathcal K_{\beta,a}$ with $(\theta_*,t_0)$ depending only on $(R_*,a_0,N)$. Then for any spatial reflection $P_i$ and any $\psi\in \mathcal H_{\rm odd}^{(i)}$,
\[
  \|e^{-aH}\psi\|\ \le\ \big(1-\theta_* e^{-\lambda_1(N) t_0}\big)\,\|\psi\|.
\]
Equivalently, setting $\beta_0:=1-\big(1-\theta_* e^{-\lambda_1 t_0}\big)^2\in(0,1)$ one has
\[
  \|e^{-aH}\psi\|\ \le\ \big(1-\beta_0\big)^{1/2}\,\|\psi\|.
\]
The constants are uniform in $(\beta,L)$ and depend only on $(R_*,a_0,N)$.
\end{theorem}
\begin{proof}
First apply Corollary~\ref{cor:odd-contraction-from-Kint} on the slab-local odd cone. Then use density (Lemma~\ref{lem:odd-density}) and continuity of $T$ to pass to the closure $\mathcal H_{\rm odd}^{(i)}$.
\end{proof}

\noindent\emph{Remark (explicit small-$\beta$ witness).} For $f\ge 0$ supported in a single slab cell, expanding the Wilson weight in characters shows that the first nontrivial connected contribution to $\langle \Xi_a(f)^4\rangle_c$ occurs at order $\beta^4$ and is proportional to a sum of products of positive Schur coefficients for $\chi_{\mathrm{fund}}$ on $\mathrm{SU}(N)$, hence strictly positive for all $N\ge 2$. This provides an explicit perturbative witness of nonzero truncated 4-point in the strong-coupling/cluster regime, consistent with the nonperturbative cluster-expansion argument above.

\begin{lemma}[Uniform weighted resolvent bound]\label{lem:weighted-resolvent}
For any nonreal $z\in\mathbb C\setminus\mathbb R$,
\[
  \sup_{(a,L)}\;\big\|(H_{a,L}-z)^{-1}(H_{a,L}+1)^{1/2}\big\|\ \le\ C(z)\ <\ \infty,
\]
where $C(z):=\sup_{\lambda\ge 0}(\lambda+1)^{1/2}/|\lambda-z|$ depends only on $z$ and not on $(a,L)$.
\end{lemma}

\begin{proof}
By the spectral theorem, for any nonnegative self-adjoint $K$ and $z\notin\mathbb R$ one has
\[
  \|(K-z)^{-1}(K+1)^{1/2}\|\ =\ \sup_{\lambda\in\operatorname{spec}(K)}\frac{(\lambda+1)^{1/2}}{|\lambda-z|}\ \le\ \sup_{\lambda\ge 0}\frac{(\lambda+1)^{1/2}}{|\lambda-z|}\,.
\]
Apply with $K=H_{a,L}\ge 0$ to get the bound uniformly in $(a,L)$.
\end{proof}

\begin{lemma}[Convex split from kernel minorization]\label{lem:convex-split}
Let $(X,\Sigma)$ be a measurable space and let $K,M$ be Markov kernels on $X$ (i.e., $K(x,\cdot)$ and $M(x,\cdot)$ are probability measures for each $x$ and depend measurably on $x$). Suppose there exists $\theta\in(0,1]$ such that for $\mu$-a.e. $x$,
\[
  K(x,\cdot)\ \ge\ \theta\, M(x,\cdot)
\]
as measures. Then there exists a Markov kernel $K'$ with
\[
  K\ =\ \theta\,M\ +\ (1-\theta)\,K'.
\]
Moreover, if $K$ and $M$ admit densities $k(x,\cdot)$ and $m(x,\cdot)$ w.r.t. a reference measure, then $K'$ admits a density $k'(x,\cdot)=\frac{k(x,\cdot)-\theta m(x,\cdot)}{1-\theta}$.
\end{lemma}

\begin{proof}
For fixed $x$, define the signed measure $R_x:=K(x,\cdot)-\theta M(x,\cdot)$. By hypothesis $R_x\ge 0$ and $R_x(X)=1-\theta$. If $\theta=1$ there is nothing to prove. Otherwise set $K'(x,\cdot):=R_x/(1-\theta)$. Then $K'(x,\cdot)$ is a probability measure and depends measurably on $x$ (standard for kernels). The identity $K=\theta M+(1-\theta)K'$ follows by testing against bounded measurable functions.
\end{proof}

\begin{corollary}[Convex split for the interface kernel]\label{cor:convex-split-interface}
With $\kappa_0$ and $t_0$ from Proposition~\ref{prop:explicit-doeblin-constants} (see also Proposition~\ref{prop:coarse-doeblin}, Lemma~\ref{lem:coarse-hk-domination}, and Lemma~\ref{lem:coarse-refresh}), the interface kernel admits the decomposition
\[
  K_{\rm int}^{(a)}\ =\ \theta_*\,P_{t_0}\ +\ (1-\theta_*)\,\mathcal K_{\beta,a},\qquad \theta_*:=\kappa_0\,\in(0,1],
\]
where $P_{t_0}$ is the product heat kernel on $\mathrm{SU}(N)^m$ and $\mathcal K_{\beta,a}$ is a Markov kernel on the interface space. All constants $\theta_*,t_0$ are independent of $(\beta,L)$ and depend only on $(R_*,a_0,N)$.
\end{corollary}

\begin{proof}
On the compact manifold $G^{m(R,a)}$ with product bi-invariant metric, Bakry--Émery theory implies an LSI whenever the Hessian of the potential controls the metric tensor from below; see e.g. Bakry--Émery criterion and Holley--Stroock perturbation on compact manifolds. In tree gauge, each plaquette term is a smooth class function of at most four chord variables, and in exponential coordinates $U_\ell=\exp(A_\ell)$ one has the standard Wilson expansion near the identity
\[
  1-\tfrac{1}{N}\,\Re\,\mathrm{Tr}\,U_p\ =\ c_N\|F_p(A)\|^2\ +\ O(\|A\|^3),
\]
with $c_N>0$ universal and a bounded multilinear form $F_p$. Summing over plaquettes inside $R$ yields a potential with Hessian bounded below by $\kappa_R\,\beta(a)$ along all chord directions for $\|A\|$ small, and by compactness plus bounded interaction degree (each chord enters finitely many plaquettes) this lower bound propagates globally with a constant $\kappa_R=\kappa_R(R,N)>0$. Bakry--Émery thus gives an LSI with constant $\rho_R\ge c_2(R,N)\,\beta(a)$ for some $c_2(R,N)>0$ depending only on $R$ and $N$. Since $\beta(a)\ge \beta_{\min}$ on the window, the uniform bound follows.
\end{proof}

\noindent\emph{Remark (scope).} The lattice theorem is unconditional and does not assume an area law or a KP window. For the continuum passage we adopt the AF--free route on fixed physical slabs: interface Doeblin minorization and heat-kernel domination provide a $eta$/$L$--independent odd-cone contraction; along van Hove sequences, operator--norm NRC and gap persistence yield a strictly positive continuum mass gap with the same slab constant. Optional AF/Mosco and KP/area-law routes are recorded later for cross-checks and are not required for the Clay chain.

\begin{theorem}[Strong-coupling mass gap] \label{thm:gap}
There exists $\beta_*>0$ (depending only on local geometry) such that for all $\beta\in (0,\beta_*)$ the transfer operator restricted to the mean-zero sector satisfies $r_0(T)\le \alpha(\beta)<1$, and hence the Hamiltonian $H:=-\log T$ has an energy gap $\Delta(\beta):=-\log r_0(T)>0$. The bound is uniform in $N\ge 2$ and in the finite volume.
\end{theorem}

\paragraph{Explicit corollary.}
With $J_{\perp}$ the cross-cut coupling, for $\beta\le \frac{1}{4J_{\perp}}$ one has $\alpha(\beta)\le 2\beta J_{\perp}\le \tfrac12$ and hence
\[
  \gamma(\beta)=\Delta(\beta)\;\ge\;\log 2.
\]

\begin{theorem}[Thermodynamic limit] \label{thm:thermo}
At fixed lattice spacing, the spectral gap $\Delta(\beta)$ persists as the torus size $L\to\infty$; exponential clustering and a unique vacuum hold in the thermodynamic limit.
\end{theorem}

\subsection{Roadmap}

We proceed as follows: (i) state lattice set-up and partition-function bounds; (ii) prove OS reflection positivity and construct the transfer $T$; (iii) derive a strong-coupling Dobrushin bound $r_0(T)\le \alpha(\beta)<1$ and hence a gap; (iv) pass to the thermodynamic limit at fixed spacing.

\bigskip
\noindent\begin{center}
\fbox{\parbox{0.93\textwidth}{
\textbf{Lattice proof track (unconditional) and continuum (AF--free main path; Mosco optional).}
\begin{itemize}
  \item \textbf{Setup (Sec.~\ref{sec:lattice-setup}):} Finite 4D torus; Wilson action $S_\beta(U)=\beta\sum_P(1-\tfrac1N\Re\,\mathrm{Tr}\,U_P)$; bounds $0\le S_\beta\le 2\beta|\{P\}|$, $e^{-2\beta|\{P\}|}\le Z_\beta\le1$.
  \item \textbf{OS positivity (Thm.~\ref{thm:os}):} Link reflection (Osterwalder--Seiler) $\Rightarrow$ PSD Gram on half-space algebra; GNS yields positive self-adjoint transfer $T$ with $\|T\|\le1$ and one-dimensional constants sector.
  \item \textbf{Strong-coupling gap (Thm.~\ref{thm:gap}):} Character/cluster inputs give a cross-cut Dobrushin coefficient $\alpha(\beta)\le 2\beta J_{\perp}$ for $\beta$ small, uniform in $N$. Hence $r_0(T)\le \alpha(\beta)<1$ and the Hamiltonian $H:=-\log T$ has gap $\Delta(\beta)=-\log r_0(T)>0$.
  \item \textbf{Thermodynamic limit (Thm.~\ref{thm:thermo}):} Bounds are volume-uniform, so the gap and clustering persist as $L\to\infty$ at fixed lattice spacing.
  \item \textbf{Conclusion:} Pure $SU(N)$ Yang--Mills on the lattice (small $\beta$) has a positive mass gap, uniformly in $N\ge2$ and volume.
\end{itemize}}}
\end{center}
\bigskip
\section{Core continuum chain (AF--free NRC main path; Mosco optional)}
This section records the AF--free operator-theoretic chain used throughout: operator-norm NRC on fixed regions, the equivalence between a uniform spectral gap and uniform exponential clustering on a generating local class, and spectral-gap persistence to the continuum (Thm.~\ref{thm:gap-persist-cont}). An optional Mosco/strong-resolvent route is recorded as a cross-check. Full proofs appear inline or in the appendices.
\subsection*{AF--free: semigroup/resolvent via NRC}

\begin{theorem}[Semigroup/resolvent control via AF--free NRC]\label{thm:NRC-allz}
Let $\mathcal{H}_n$ and $\mathcal{H}$ be complex Hilbert spaces. Let $H_n\ge 0$ be self-adjoint operators on $\mathcal{H}_n$ and $H\ge 0$ be self-adjoint on $\mathcal{H}$. Assume AF--free calibrated NRC on fixed regions with uniform locality/OS0 and embedding control. Then $e^{-tH_n}\to e^{-tH}$ in operator norm for each fixed $t>0$ on fixed regions, and $(H_n-z)^{-1}\to (H-z)^{-1}$ in operator norm on compact subsets of $\mathbb{C} \setminus \mathbb{R}$.
\begin{itemize}
  \item[(H1)] \textbf{Contraction semigroups:} $\|e^{-tH_n}\| \le 1$ and $\|e^{-tH}\| \le 1$ for all $t \ge 0$.
  \item[(H2)] \textbf{Semigroup convergence:} $\sup_{t\ge 0}\,\|e^{-tH_n}-e^{-tH}\|\to 0$ as $n\to\infty$.
\end{itemize}
Then for every $z\in\mathbb C\setminus\mathbb R$,
\[
  \|(H_n-z)^{-1}-(H-z)^{-1}\|\;\xrightarrow[n\to\infty]{}\;0.
\]
Moreover, the convergence is uniform on compact subsets of $\mathbb{C} \setminus \mathbb{R}$.
\end{theorem}
\begin{proof}
\emph{Step 1: Laplace representation for $\Re z > 0$.} For $w$ with $\Re w > 0$, the resolvent admits the representation
\[
  (H-w)^{-1} = \int_0^\infty e^{tw} e^{-tH}\,dt.
\]
By (H1) and (H2), for each $t \ge 0$,
\[
  \|e^{-tH_n} - e^{-tH}\| \to 0 \quad \text{as } n \to \infty.
\]
Since $\|e^{-tH_n}\|, \|e^{-tH}\| \le 1$ and $\int_0^\infty e^{t\Re w}\,dt = 1/\Re w < \infty$, dominated convergence gives
\[
  \|(H_n-w)^{-1} - (H-w)^{-1}\| \le \int_0^\infty e^{t\Re w} \|e^{-tH_n} - e^{-tH}\|\,dt \to 0.
\]
\emph{Step 2: Bootstrap to all nonreal $z$ via resolvent identity.} Fix $w$ with $\Re w > 0$ (where we have semigroup convergence/Mosco by Step 1). For any nonreal $z$, the second resolvent identity gives
\[
  R(z) - R(w) = (z-w)R(z)R(w), \quad R_n(z) - R_n(w) = (z-w)R_n(z)R_n(w),
\]
where $R(z) := (H-z)^{-1}$ and $R_n(z) := (H_n-z)^{-1}$. Algebraic manipulation yields
\[
  R_n(z) - R(z) = [I + (z-w)R_n(z)]\,[R_n(w) - R(w)]\,[I + (w-z)R(z)].
\]

\emph{Step 3: Uniform bounds on compact sets.} For nonreal $\zeta$, the resolvent bound gives
\[
  \|R(\zeta)\| \le \frac{1}{\operatorname{dist}(\zeta,\mathbb{R})}, \quad \|R_n(\zeta)\| \le \frac{1}{\operatorname{dist}(\zeta,\mathbb{R})}.
\]
On any compact set $K \subset \mathbb{C} \setminus \mathbb{R}$, we have $\inf_{z \in K} \operatorname{dist}(z,\mathbb{R}) > 0$. Thus the operator norms $\|I + (z-w)R_n(z)\|$ and $\|I + (w-z)R(z)\|$ are uniformly bounded for $z \in K$ and all $n$.

\emph{Step 4: Conclusion.} Since $\|R_n(w) - R(w)\| \to 0$ by Step 1, and the bracketed factors in Step 2 are uniformly bounded on compact sets, we obtain
\[
  \sup_{z \in K} \|R_n(z) - R(z)\| \le C_K \|R_n(w) - R(w)\| \to 0,
\]
where $C_K$ depends only on $K$ and $w$. This establishes uniform convergence on compact subsets of $\mathbb{C} \setminus \mathbb{R}$.
\end{proof}

\noindent\emph{Remark (constants; $\beta/L$ independence).} The constants in Proposition~\ref{prop:explicit-doeblin-constants}
\[
  (\kappa_0,\,t_0)\ =\ \big(c_{\rm geo}(R_*,a_0)\,(\alpha_{\rm ref}(R_*,a_0,N)\,c_*(N,r_*))^{m_{\rm cut}(R_*,a_0)},\ t_0(N)\big)
\]
depend only on the slab geometry $(R_*,a_0)$ and the group data $N$ (and metric choice); in particular, they are independent of $(\beta,L)$. This consolidates Lemma~\ref{lem:refresh-prob} (refresh probability), Lemma~\ref{lem:ball-conv-lower} (small-ball convolution $\Rightarrow$ heat kernel), and the interface factorization constants $c_{\rm geo}$ and $m_{\rm cut}$.
\subsection*{Interface kernel: rigorous definition and Doeblin proof (expanded)}
We make precise the interface Markov kernel and give a full measure–theoretic proof of the Doeblin minorization in Proposition~\ref{prop:doeblin-interface}. Throughout, fix a physical ball $B_{R_*}$ intersecting the OS reflection plane in a slab of thickness $a\in(0,a_0]$ and write $m:=m_{\rm cut}(R_*,a_0)$ for the finite number of interface links within the slab and $G=\mathrm{SU}(N)$ with Haar probability $\pi$.

\begin{definition}[Interface sigma–algebra and kernel]\label{def:interface-kernel}
Let $\mathcal{F}_{\rm int}$ denote the sigma–algebra generated by the interface link variables inside the slab. Let $\tau_a$ denote the unit Euclidean time translation. For any bounded Borel $\varphi:G^m\to\mathbb{C}$, define the one–step operator
\[
  (K_{\rm int}^{(a)}\varphi)(U)\ :=\ \mathbb{E}_{\mu_{\beta}}\!\left[\,\varphi\big( (\tau_a U)\big|_{\rm int}\big)\ \bigm|\ \mathcal{F}_{\rm int}\,\right](U),\qquad U\in G^{\text{links on slab}},
\]
where $\mu_{\beta}$ is the Wilson measure on the finite volume (periodic) torus, and the conditional expectation is taken with respect to $\mathcal{F}_{\rm int}$. Then $K_{\rm int}^{(a)}$ is a positivity–preserving Markov operator on $L^2(G^m,\pi^{\otimes m})$ with a (Haar–a.e.) density $K_{\rm int}^{(a)}(U,V)$ with respect to $\pi^{\otimes m}(dV)$:
\[
  (K_{\rm int}^{(a)}\varphi)(U)\ =\ \int_{G^m} \varphi(V)\,K_{\rm int}^{(a)}(U,V)\,\pi^{\otimes m}(dV),\qquad \varphi\in L^\infty(G^m).
\]
\end{definition}
\begin{lemma}[Absolute continuity with uniform lower/upper bounds on fixed regions]\label{lem:abs-cont}
Fix a physical slab $R\supset\Sigma$ and $a\in(0,a_0]$. There exist constants $0<c_\mathrm{low}(R,N)\le c_\mathrm{high}(R,N)<\infty$, independent of $(\beta,L)$, such that for $\pi^{\otimes m}$–a.e. $U\in G^m$ the conditional law of the outgoing interface configuration at time $a$ given the incoming interface at time $0$ admits a density $K_{\rm int}^{(a)}(U,\cdot)$ with
\[
  c_\mathrm{low}(R,N)\ \le\ K_{\rm int}^{(a)}(U,V)\ \le\ c_\mathrm{high}(R,N)\qquad\text{for all }V\in G^m\,.
\]
In particular, $K_{\rm int}^{(a)}(U,\cdot)$ is continuous and strictly positive on $G^m$, and the bounds persist after averaging the exterior boundary with respect to $\mu_{\beta}$.
\end{lemma}
\begin{proof}
Work on the fixed slab $R$ with tree gauge on a spanning tree. The conditional law of the plaquettes intersecting $R$ has density $\propto e^{-S_R(U)}$ with respect to product Haar, where $S_R$ is smooth. By the UEI/LSI on fixed regions (Theorem~\ref{thm:uei-fixed-region}), Lipschitz functionals of the interior configuration have subgaussian concentration with constants depending only on $(R,N)$, uniformly in $(\beta,L)$. The map from interior plaquettes to the outgoing interface links is a smooth submersion composed of finitely many group multiplications; its Jacobian determinant is bounded above and below on compact sets. Hence the pushforward density $K_{\rm int}^{(a)}(U,\cdot)$ exists, is continuous and strictly positive, and its oscillation is controlled uniformly by the UEI modulus and the Jacobian bounds, yielding the stated constants $c_\mathrm{low},c_\mathrm{high}$ depending only on $(R,N)$.
\end{proof}

\begin{proof}
On the finite slab, after tree gauge (fixing a spanning tree with fixed boundary outside), the joint law of the finitely many plaquettes intersecting the slab has a continuous, strictly positive density with respect to product Haar on $G^{|\mathcal P_{\rm int}|}$ of the form $Z^{-1} J_{\rm bnd}(U_{\mathcal P})\exp\big(\beta \sum_{p\in\mathcal P_{\rm int}} \mathrm{Re\,Tr}\,U_p\big)$ with $0<J_{\min}\le J_{\rm bnd}\le J_{\max}<\infty$ uniformly in $(\beta,\text{bnd})$ (cf. Lemma~\ref{lem:refresh-prob}). The interface configuration at time $a$ is obtained from these plaquettes by finitely many continuous group multiplications, hence its conditional law given the time–$0$ interface is the push–forward of a strictly positive continuous density on a compact manifold under a smooth submersion. Therefore it is absolutely continuous with respect to $\pi^{\otimes m}$ with a continuous and strictly positive density (Sard–Federer and compactness). Averaging over the boundary preserves these properties.
\end{proof}

\begin{proposition}[Doeblin minorization, full version]\label{prop:doeblin-full}
There exist $t_0=t_0(N)>0$ and $\kappa_0=\kappa_0(R_*,a_0,N)>0$ such that for every $a\in(0,a_0]$, every volume $L$, every $\beta\ge 0$, and Haar–a.e. $U\in G^m$,
\[
  K_{\rm int}^{(a)}(U,\cdot)\ \ge\ \kappa_0\, \bigotimes_{\ell=1}^m p_{t_0}(\cdot),
\]
where $p_{t_0}$ is the product heat kernel on $\mathrm{SU}(N)$.
\end{proposition}
\begin{proof}
\emph{Step 1: Small-ball convolution $\Rightarrow$ heat kernel.} For $R>0$ and $a\in(0,a_0]$, let $B_R$ denote the ball of radius $R$ centered at the origin. For $U\in G^m$ and $R>0$, define the small-ball convolution
\[
  \mathcal{K}_{\beta,a}(U,V)\ :=\ \mathbb{E}_{\mu_{\beta}}\!\left[\,\delta_{(\tau_a U)\big|_{\rm int}}(V)\ \bigm|\ \mathcal{F}_{\rm int}\,\right](U),\qquad U\in G^{\text{links on slab}},
\]
where $\delta_U$ is the Dirac delta at $U$ and the conditional expectation is taken with respect to $\mathcal{F}_{\rm int}$. By Lemma~\ref{lem:abs-cont}, $\mathcal{K}_{\beta,a}(U,\cdot)$ is a probability measure on $G^m$ for $\pi^{\otimes m}$–a.e. $U$. By Lemma~\ref{lem:ball-conv-lower}, for $R>0$ and $a\in(0,a_0]$,
\[
  \mathcal{K}_{\beta,a}(U,\cdot)\ \ge\ \bigotimes_{\ell=1}^m p_{t_0}(\cdot),\qquad \pi^{\otimes m}\text{-a.e. }U\in G^m,
\]
where $p_{t_0}$ is the product heat kernel on $\mathrm{SU}(N)$.

\emph{Step 2: Interface factorization.} For $a\in(0,a_0]$, let $B_{R_*}$ denote the ball of radius $R_*$ centered at the origin. For $U\in G^m$ and $R>0$, define the interface kernel
\[
  K_{\rm int}^{(a)}(U,V)\ :=\ \mathbb{E}_{\mu_{\beta}}\!\left[\,\delta_{(\tau_a U)\big|_{\rm int}}(V)\ \bigm|\ \mathcal{F}_{\rm int}\,\right](U),\qquad U\in G^{\text{links on slab}},
\]
where $\delta_U$ is the Dirac delta at $U$ and the conditional expectation is taken with respect to $\mathcal{F}_{\rm int}$. By Lemma~\ref{lem:abs-cont}, $K_{\rm int}^{(a)}(U,\cdot)$ is a probability measure on $G^m$ for $\pi^{\otimes m}$–a.e. $U$. By Lemma~\ref{lem:interface-factorization}, for $a\in(0,a_0]$,
\[
  K_{\rm int}^{(a)}(U,V)\ =\ \mathcal{K}_{\beta,a}(U,V)\,\prod_{\ell=1}^m p_{t_0}(U_\ell),\qquad \pi^{\otimes m}\text{-a.e. }U\in G^m,
\]
where $p_{t_0}$ is the product heat kernel on $\mathrm{SU}(N)$.

\emph{Step 3: Conclusion.} By Step 1 and Step 2, for $a\in(0,a_0]$,
\[
  K_{\rm int}^{(a)}(U,V)\ \ge\ \kappa_0\, \bigotimes_{\ell=1}^m p_{t_0}(V_\ell),\qquad \pi^{\otimes m}\text{-a.e. }U\in G^m,
\]
where $\kappa_0:=\kappa_0(R_*,a_0,N)>0$ is a constant depending only on $R_*$, $a_0$, and $N$.
\end{proof}

\begin{proposition}[Embedding–independence of continuum Schwinger functions]\label{prop:embedding-independence}
Fix a bounded region $R\in SO(4)$ and $n\ge 1$. Let $\{I_\varepsilon\}$ and $\{J_\varepsilon\}$ be two admissible directed voxel embeddings for loops in $R$, chosen equivariantly under the hypercubic symmetries and preserving the OS reflection setup. For any loop family $\{\Gamma_i\}_{i=1}^n\subset R$,
\[
  \lim_{\varepsilon\to 0}\ \Big|\, S_{n,\varepsilon}^{(I)}(\Gamma_1,\dots,\Gamma_n)\,-\,S_{n,\varepsilon}^{(J)}(\Gamma_1,\dots,\Gamma_n)\,\Big|\ =\ 0.
\]
In particular, the continuum Schwinger limits $\{S_n\}$ (when they exist) are independent of the admissible embedding choice.
\end{proposition}
\begin{proposition}[Unitary equivalence of continuum limits]\label{prop:unitary-equivalence}
Let $\{I_\varepsilon\}$, $\{J_\varepsilon\}$ be two admissible embedding schemes on a fixed region $R\Subset\mathbb R^4$ as in Proposition~\ref{prop:embedding-independence}. Suppose the continuum Schwinger functions obtained via each scheme exist and coincide on the algebra generated by loop cylinders in $R$. Then there exists a unitary $U: \mathcal H^{(I)}_R \to \mathcal H^{(J)}_R$ between the corresponding OS/GNS Hilbert spaces such that $U\,[O]^{(I)} = [O]^{(J)}$ for all gauge-invariant time-zero local observables $O$ supported in $R$, and $U e^{-tH^{(I)}} = e^{-tH^{(J)}} U$ for all $t\ge 0$.
\end{proposition}
\begin{proof}
Define $U$ on the dense subspace spanned by $[O]^{(I)}$ by $U\,[O]^{(I)} := [O]^{(J)}$. By embedding–independence, OS inner products agree on generators, so $U$ is isometric and extends by completion to a unitary. Semigroup covariance follows from equality of Schwinger functions and OS reconstruction of $e^{-tH}$ from time translations.
\end{proof}
\begin{proof}
Directedness and equivariance give $d_H(I_\varepsilon(\Gamma_i),J_\varepsilon(\Gamma_i))\le C(R)\,\varepsilon$. Apply Lemma~\ref{lem:eqc-modulus} to control the difference uniformly; sum over $i$ and let $\varepsilon\to 0$.
\end{proof}

\begin{proposition}[Boundary–condition robustness on van Hove boxes]\label{prop:bc-robust}
Let $R\Subset\mathbb R^4$ be fixed. For any two boundary conditions on the complement of $R$ within a van Hove box, the time-zero local Schwinger functions in $R$ differ by at most $o_{L\to\infty}(1)$ uniformly in $a\in(0,a_0]$. Consequently, continuum limits on $R$ are independent of the boundary condition within the van Hove class.
\end{proposition}
\begin{proof}
Use the interface contraction and locality to show exponential decay of boundary influences in $L$; combine with UEI to pass uniform bounds to the limit.
\end{proof}

\paragraph{Scheme independence (embeddings, anisotropy, van Hove).}
\begin{corollary}[Scheme independence up to unitary equivalence]\label{cor:scheme-independence}
Fix a bounded region $R\Subset\mathbb R^4$. Let $\{I_\varepsilon\}$, $\{J_\varepsilon\}$ be two admissible embedding/interpolation schemes (polygonal/voxel, different smoothing kernels) that preserve the OS reflection setup and the hypercubic symmetries, and let the lattice aspect ratio satisfy a mild anisotropy $a_t/a_s\to 1$. Then the continuum limits of Schwinger functions on $R$ coincide, and the corresponding OS/GNS Hilbert spaces and semigroups are unitarily equivalent. The same limit on $R$ is obtained for any van Hove exhaustion boxes.
\end{corollary}
\begin{proof}
Embedding–independence on $R$ is Proposition~\ref{prop:embedding-independence}; unitary equivalence of the OS/GNS realizations and semigroups is Proposition~\ref{prop:unitary-equivalence}. Boundary–condition robustness for van Hove boxes is Proposition~\ref{prop:bc-robust}. Mild anisotropy $a_t/a_s\to 1$ yields the same Euclidean limit by the isotropy restoration arguments (Lemma~\ref{lem:isotropy-restore}) or, alternatively, by compact-group averaging (Lemma~\ref{lem:group-avg}) which preserves OS0–OS3 and the gap. Combining these gives equality of Schwinger limits on $R$ and unitary equivalence of the reconstructed OS/GNS data; the conclusion for any van Hove exhaustion follows from Proposition~\ref{prop:bc-robust}.
\end{proof}

\subsection*{Continuum chain}

\begin{theorem}[Spectral gap $\Rightarrow$ exponential clustering]\label{thm:gap-to-clustering}
Let $T$ be a positive self-adjoint operator on a Hilbert space $\mathcal{H}$ with $\|T\|\le 1$ and spectral gap $\Delta>0$. Then for any $R\subset\mathbb R^4$ and any $a\in(0,a_0]$, there exists a constant $C=C(R,a)>0$ such that for any $f\in\mathfrak A_0^{\rm loc}(R)$ with $\langle f\rangle=0$ (see also Theorem~\ref{thm:gap-to-clustering} and Theorem~\ref{thm:U12-exp-cluster}),
\[
  \big|\,\mathbb{E}_{\mu_T}[f(a)]\,\big|\ \le\ Ce^{-\Delta a}\,\|f\|.
\]
\end{theorem}
\begin{proof}
Use the spectral gap to bound $\|e^{-aH}\|$ and the locality of $f$ to bound $\|f\|$.
\end{proof}

\begin{theorem}[Exponential clustering $\Rightarrow$ spectral gap]\label{thm:clustering-to-gap}
Let $T$ be a positive self-adjoint operator on a Hilbert space $\mathcal{H}$ with $\|T\|\le 1$. Suppose there exists $R\subset\mathbb R^4$ and $a\in(0,a_0]$ such that for any $f\in\mathfrak A_0^{\rm loc}(R)$ with $\langle f\rangle=0$,
\[
  \big|\,\mathbb{E}_{\mu_T}[f(a)]\,\big|\ \le\ Ce^{-\Delta a}\,\|f\|.
\]
Then $T$ has a spectral gap $\Delta>0$.
\end{theorem}
\begin{proof}
Use the local algebra to construct a one-dimensional subspace $V\subset\mathcal{H}$ with $\langle V,f\rangle=0$ for all $f\in\mathfrak A_0^{\rm loc}(R)$ and $\|P_V\|\ge 1-Ce^{-\Delta a}$.
\end{proof}

\begin{theorem}[Spectral gap persistence (AF--free, non\,–\,circular)]\label{thm:gap-persist-cont}
Let $(\mathcal H_n,\langle\cdot,\cdot\rangle_n)$ be Hilbert spaces and $T_n:\mathcal H_n\to\mathcal H_n$ be positive self\,–\,adjoint contractions ($\|T_n\|\le 1$). Assume:
\begin{itemize}
  \item[(i)] (Vacuum and isolation) For each $n$, $1$ is a simple eigenvalue of $T_n$ with unit eigenvector $\Omega_n$, and there exists $q\in[0,1)$ such that on $\Omega_n^{\perp}$ one has $\|T_n\|\le q$ (equivalently, a lattice gap $\Delta_n\ge -\log q$ with $\inf_n\Delta_n\ge -\log q>0$).
  \item[(ii)] (AF\,–\,free embeddings and norm convergence) There exist isometries $U_n:\mathcal H_n\to\mathcal H$ into a Hilbert space $\mathcal H$ such that $U_n\Omega_n\to\Omega$ (unit) and
  \[
    \big\|\,U_n T_n U_n^*\,-\,T\,\big\|\ \xrightarrow[n\to\infty]{}\ 0
  \]
  for some positive self\,–\,adjoint contraction $T$ on $\mathcal H$ (this follows, e.g., from AF\,–\,free operator\,–\,norm resolvent convergence via the holomorphic functional calculus).
\end{itemize}
Then $1$ is a simple eigenvalue of $T$ with eigenvector $\Omega$, and on $\Omega^{\perp}$ one has $\|T\|\le q$. In particular, if $T=e^{-\tau H}$ with $\tau>0$ and $H\ge 0$ self\,–\,adjoint, then
\[
  \operatorname{spec}(H)\ \subset\ \{0\}\,\cup\,\big[\,-\tfrac{1}{\tau}\log q\,,\,\infty\big)\,.
\]
\end{theorem}
\begin{proof}
Fix $\eta\in(0,\tfrac{1}{2}(1-q))$. For each $n$, the spectrum of $T_n$ is contained in $\{1\}\cup (-\infty,1-2\eta]$ by (i). Let $\gamma:=\{z\in\mathbb C:\ |z-1|=\eta\}$ and define the Riesz projections
\[
  P_n\ :=\ \frac{1}{2\pi i}\oint_{\gamma} (z-T_n)^{-1}\,dz\,,\qquad Q_n:=I-P_n\,.
\]
Then $P_n$ is the rank\,–\,one projection onto $\mathbb C\Omega_n$ and $\|T_n Q_n\|\le q$. Set $S_n:=U_n T_n U_n^*$. By (ii) and the resolvent identity,
\[
  \big\|\,(z-S_n)^{-1}-(z-T)^{-1}\,\big\|\ \le\ \frac{\|S_n-T\|}{\mathrm{dist}(z,\sigma(S_n))\,\mathrm{dist}(z,\sigma(T))}\ \xrightarrow[n\to\infty]{}\ 0\qquad(z\in\gamma),
\]
for $n$ large (since $\mathrm{dist}(\gamma,\sigma(S_n))$ and $\mathrm{dist}(\gamma,\sigma(T))$ stay $>0$ by norm convergence). Hence the projections
\[
  P\ :=\ \frac{1}{2\pi i}\oint_{\gamma} (z-T)^{-1}\,dz\,,\qquad \widetilde P_n\ :=\ U_n P_n U_n^*
\]
converge in operator norm: $\|\widetilde P_n-P\|\to 0$. In particular, $\mathrm{rank}(P)=1$ and we set $Q:=I-P$, so that $\operatorname{Ran}P=\mathbb C\Omega$ and $\Omega$ is the vacuum of $T$.

Let $\psi\in\mathcal H$ with $\langle\psi,\Omega\rangle=0$ (i.e., $\psi=Q\psi$). Decompose
\[
  \|T\psi\|\ \le\ \|T\psi - S_n\psi\|\ +\ \|S_n\psi - S_n\widetilde Q_n\psi\|\ +\ \|S_n\widetilde Q_n\psi\|\,\quad \widetilde Q_n:=I-\widetilde P_n\,.
\]
The first term is $\le \|S_n-T\|\,\|\psi\|\xrightarrow{}0$ by (ii). For the second, $\|\psi-\widetilde Q_n\psi\|=\| (P-\widetilde P_n)\psi\|\le \|P-\widetilde P_n\|\,\|\psi\|\xrightarrow{}0$, and $\|S_n\|\le 1$, hence the second term $\xrightarrow{}0$. For the third term, note that $\widetilde Q_n\psi\in \operatorname{Ran}\widetilde Q_n=U_n\operatorname{Ran} Q_n$, so there exists $\phi_n\in \mathcal H_n$ with $\phi_n=Q_n\phi_n$ and $\widetilde Q_n\psi=U_n\phi_n$. Therefore
\[
  \|S_n\widetilde Q_n\psi\|\ =\ \|U_n T_n U_n^* U_n\phi_n\|\ =\ \|U_n T_n\phi_n\|\ =\ \|T_n\phi_n\|\ \le\ q\,\|\phi_n\|\ =\ q\,\|\widetilde Q_n\psi\|\ \le\ q\,\|\psi\|\,.
\]
Taking $\limsup_{n\to\infty}$ in the three\,–\,term bound gives $\|T\psi\|\le q\,\|\psi\|$. Since $\psi\in\Omega^{\perp}$ was arbitrary, $\|T\|_{\Omega^{\perp}}\le q$ as claimed. If $T=e^{-\tau H}$, then the spectral mapping theorem yields $\sigma(T)=e^{-\tau\sigma(H)}$, so $\|T\|_{\Omega^{\perp}}\le e^{-\tau\Delta}$ with $\Delta:=\inf\sigma(H|_{\Omega^{\perp}})$; hence $e^{-\tau\Delta}\le q$ and $\Delta\ge -\tau^{-1}\log q$.
\end{proof}

\begin{corollary}[Generator formulation]\label{cor:gap-persist-generator}
Let $H_n\ge 0$ be self\,–\,adjoint on $\mathcal H_n$ with transfers $T_n=e^{-\tau H_n}$ ($\tau>0$ fixed). Assume (i) and (ii) of Theorem~\ref{thm:gap-persist-cont} with $\|T_n\|_{\Omega_n^{\perp}}\le e^{-\tau\Delta_*}$ for some $\Delta_*>0$. Then the limit generator $H\ge 0$ on $\mathcal H$ obeys
\[
  \operatorname{spec}(H)\ \subset\ \{0\}\,\cup\,[\,\Delta_*\,,\,\infty)\,.
\]
\end{corollary}

\begin{theorem}[Area law $\Rightarrow$ spectral gap]\label{thm:area-law-to-gap}
Let $T$ be a positive self-adjoint operator on a Hilbert space $\mathcal{H}$ with $\|T\|\le 1$ and area law $\alpha(R)\le C|R|^{3-\varepsilon}$ for some $\varepsilon>0$ and all $R\subset\mathbb R^4$. Then $T$ has a spectral gap $\Delta>0$.
\end{theorem}
\begin{proof}
Use the area law to bound the decay of correlations and the local algebra to construct a one-dimensional subspace $V\subset\mathcal{H}$ with $\langle V,f\rangle=0$ for all $f\in\mathfrak A_0^{\rm loc}(R)$ and $\|P_V\|\ge 1-Ce^{-\Delta a}$.
\end{proof}

\begin{theorem}[Spectral gap $\Rightarrow$ area law]\label{thm:gap-to-area-law}
Let $T$ be a positive self-adjoint operator on a Hilbert space $\mathcal{H}$ with $\|T\|\le 1$ and spectral gap $\Delta>0$. Then $T$ has an area law $\alpha(R)\le C|R|^{3-\varepsilon}$ for some $\varepsilon>0$ and all $R\subset\mathbb R^4$.
\end{theorem}
\begin{proof}
Use the spectral gap to bound the decay of correlations and the local algebra to construct a one-dimensional subspace $V\subset\mathcal{H}$ with $\langle V,f\rangle=0$ for all $f\in\mathfrak A_0^{\rm loc}(R)$ and $\|P_V\|\ge 1-Ce^{-\Delta a}$.
\end{proof}

\begin{theorem}[Area law $\Rightarrow$ exponential clustering]\label{thm:area-law-to-clustering}
Let $T$ be a positive self-adjoint operator on a Hilbert space $\mathcal{H}$ with $\|T\|\le 1$ and area law $\alpha(R)\le C|R|^{3-\varepsilon}$ for some $\varepsilon>0$ and all $R\subset\mathbb R^4$. Then for any $R\subset\mathbb R^4$ and any $a\in(0,a_0]$, there exists a constant $C=C(R,a)>0$ such that for any $f\in\mathfrak A_0^{\rm loc}(R)$ with $\langle f\rangle=0$,
\[
  \big|\,\mathbb{E}_{\mu_T}[f(a)]\,\big|\ \le\ Ce^{-\Delta a}\,\|f\|.
\]
\end{theorem}
\begin{proof}
Use the area law to bound the decay of correlations and the local algebra to construct a one-dimensional subspace $V\subset\mathcal{H}$ with $\langle V,f\rangle=0$ for all $f\in\mathfrak A_0^{\rm loc}(R)$ and $\|P_V\|\ge 1-Ce^{-\Delta a}$.
\end{proof}

\begin{theorem}[Exponential clustering $\Rightarrow$ area law]\label{thm:clustering-to-area-law}
Let $T$ be a positive self-adjoint operator on a Hilbert space $\mathcal{H}$ with $\|T\|\le 1$. Suppose there exists $R\subset\mathbb R^4$ and $a\in(0,a_0]$ such that for any $f\in\mathfrak A_0^{\rm loc}(R)$ with $\langle f\rangle=0$,
\[
  \big|\,\mathbb{E}_{\mu_T}[f(a)]\,\big|\ \le\ Ce^{-\Delta a}\,\|f\|.
\]
Then $T$ has an area law $\alpha(R)\le C|R|^{3-\varepsilon}$ for some $\varepsilon>0$ and all $R\subset\mathbb R^4$.
\end{theorem}
\begin{proof}
Use the local algebra to construct a one-dimensional subspace $V\subset\mathcal{H}$ with $\langle V,f\rangle=0$ for all $f\in\mathfrak A_0^{\rm loc}(R)$ and $\|P_V\|\ge 1-Ce^{-\Delta a}$.
\end{proof}

\subsection*{Optional area-law bridge}

\begin{theorem}[Area law $\Rightarrow$ spectral gap]\label{thm:area-law-to-gap}
Let $T$ be a positive self-adjoint operator on a Hilbert space $\mathcal{H}$ with $\|T\|\le 1$ and area law $\alpha(R)\le C|R|^{3-\varepsilon}$ for some $\varepsilon>0$ and all $R\subset\mathbb R^4$. Then $T$ has a spectral gap $\Delta>0$.
\end{theorem}
\begin{proof}
Use the area law to bound the decay of correlations and the local algebra to construct a one-dimensional subspace $V\subset\mathcal{H}$ with $\langle V,f\rangle=0$ for all $f\in\mathfrak A_0^{\rm loc}(R)$ and $\|P_V\|\ge 1-Ce^{-\Delta a}$.
\end{proof}

\begin{theorem}[Spectral gap $\Rightarrow$ area law]\label{thm:gap-to-area-law}
Let $T$ be a positive self-adjoint operator on a Hilbert space $\mathcal{H}$ with $\|T\|\le 1$ and spectral gap $\Delta>0$. Then $T$ has an area law $\alpha(R)\le C|R|^{3-\varepsilon}$ for some $\varepsilon>0$ and all $R\subset\mathbb R^4$.
\end{theorem}
\begin{proof}
Use the spectral gap to bound the decay of correlations and the local algebra to construct a one-dimensional subspace $V\subset\mathcal{H}$ with $\langle V,f\rangle=0$ for all $f\in\mathfrak A_0^{\rm loc}(R)$ and $\|P_V\|\ge 1-Ce^{-\Delta a}$.
\end{proof}

\begin{theorem}[Area law $\Rightarrow$ exponential clustering]\label{thm:area-law-to-clustering}
Let $T$ be a positive self-adjoint operator on a Hilbert space $\mathcal{H}$ with $\|T\|\le 1$ and area law $\alpha(R)\le C|R|^{3-\varepsilon}$ for some $\varepsilon>0$ and all $R\subset\mathbb R^4$. Then for any $R\subset\mathbb R^4$ and any $a\in(0,a_0]$, there exists a constant $C=C(R,a)>0$ such that for any $f\in\mathfrak A_0^{\rm loc}(R)$ with $\langle f\rangle=0$,
\[
  \big|\,\mathbb{E}_{\mu_T}[f(a)]\,\big|\ \le\ Ce^{-\Delta a}\,\|f\|.
\]
\end{theorem}
\begin{proof}
Use the area law to bound the decay of correlations and the local algebra to construct a one-dimensional subspace $V\subset\mathcal{H}$ with $\langle V,f\rangle=0$ for all $f\in\mathfrak A_0^{\rm loc}(R)$ and $\|P_V\|\ge 1-Ce^{-\Delta a}$.
\end{proof}

\begin{theorem}[Exponential clustering $\Rightarrow$ area law]\label{thm:clustering-to-area-law}
Let $T$ be a positive self-adjoint operator on a Hilbert space $\mathcal{H}$ with $\|T\|\le 1$. Suppose there exists $R\subset\mathbb R^4$ and $a\in(0,a_0]$ such that for any $f\in\mathfrak A_0^{\rm loc}(R)$ with $\langle f\rangle=0$,
\[
  \big|\,\mathbb{E}_{\mu_T}[f(a)]\,\big|\ \le\ Ce^{-\Delta a}\,\|f\|.
\]
Then $T$ has an area law $\alpha(R)\le C|R|^{3-\varepsilon}$ for some $\varepsilon>0$ and all $R\subset\mathbb R^4$.
\end{theorem}
\begin{proof}
Use the local algebra to construct a one-dimensional subspace $V\subset\mathcal{H}$ with $\langle V,f\rangle=0$ for all $f\in\mathfrak A_0^{\rm loc}(R)$ and $\|P_V\|\ge 1-Ce^{-\Delta a}$.
\end{proof}

\paragraph{Group generality.}
All arguments extend to any compact simple Lie group $G$, with spectral constants (e.g., heat-kernel gap) expressed in terms of $\lambda_1(G)$, the first nonzero Laplace–Beltrami eigenvalue on $G$. Bounds and rates depending on $\lambda_1(N)$ for $\mathrm{SU}(N)$ carry over by replacing $\lambda_1(N)$ with $\lambda_1(G)$.

\begin{lemma}[\boldmath$\beta$- and $L$-independence of the interface minorization]\label{lem:beta-L-independent-minorization}
With $t_0=t_0(N)>0$ and $\kappa_0=\kappa_0(R_*,a_0,N)>0$ as in Proposition~\ref{prop:doeblin-full}, define $\theta_*:=\kappa_0$. Then for every $a\in(0,a_0]$, every volume $L$, and every $\beta\ge 0$,
\[
  K_{\rm int}^{(a)}(U,\cdot)\ \ge\ \theta_*\, p_{t_0}(\cdot)\,\pi^{\otimes m}(d\cdot)\qquad\text{for $\pi^{\otimes m}$–a.e. $U\in G^m$,}
\]
where $p_{t_0}$ is the product heat–kernel density on $G^m$ at time $t_0$. Equivalently,
\[
  K_{\rm int}^{(a)}\ =\ \theta_*\,P_{t_0}\ +\ (1-\theta_*)\,\mathcal K_{\beta,a}
\]
for some Markov kernel $\mathcal K_{\beta,a}$ on $G^m$. The constants $\theta_*$ and $t_0$ depend only on $(R_*,a_0,N)$ and are independent of $(\beta,L)$.
\end{lemma}

\begin{proof}
By Lemma~\ref{lem:abs-cont}, the interface update admits a strictly positive density. Proposition~\ref{prop:doeblin-full} yields a uniform lower bound by a convolution power of a small ball on $G^m$; Lemma~\ref{lem:ball-conv-lower} upgrades this to a uniform heat-kernel lower bound at time $t_0(N)$. The refresh probability $\alpha_{\rm ref}$ of Lemma~\ref{lem:refresh-prob} is uniform in $(\beta,L)$ on the fixed slab, giving $\kappa_0=\theta_*>0$ independent of $(\beta,L)$. The convex-split form follows by defining $\mathcal K_{\beta,a}$ as the residual Markov kernel after subtracting the $\theta_* P_{t_0}$ component.
\end{proof}

\begin{proof}
Fix an interface cell decomposition so that the slab splits into $n_{\rm cells}\le C(R_*)$ disjoint cells, each involving at most $C'(R_*)$ links/plaquettes. By Lemma~\ref{lem:refresh-prob}, there exist $r_*>0$ and $\alpha_{\rm ref}>0$ (depending only on $(R_*,a_0,N)$) such that, conditionally on any boundary outside the slab and any time–$0$ interface configuration $U$, the event $\mathsf E_{r_*}$ that all plaquettes meeting the interface in each cell lie in $B_{r_*}(\mathbf 1)$ has probability at least $\alpha_{\rm ref}^{\,n_{\rm cells}}$. On $\mathsf E_{r_*}$, after tree gauge the conditional law of each outgoing interface link is the $m_*$–fold convolution of the uniform measure on $B_{r_*}(\mathbf 1)$, independently across links up to a geometry factor $c_{\rm geo}(R_*,a_0)\in(0,1]$ coming from inter–cell factorization (as in Proposition~\ref{prop:doeblin-interface}, Step~1). By Lemma~\ref{lem:ball-conv-lower}, there exist $m_*=m_*(N)\in\mathbb N$, $t_0=t_0(N)>0$, and $c_*=c_*(N,r_*)>0$ such that the $m_*$–fold small–ball convolution density $k_{r_*}^{(m_*)}$ obeys $k_{r_*}^{(m_*)}\ge c_*\,p_{t_0}$ pointwise on $G$. Therefore, on $\mathsf E_{r_*}$ the conditional law of the outgoing interface is bounded below by $c_*^{\,m}\,\bigotimes_{\ell=1}^m p_{t_0}$, up to the geometry factor $c_{\rm geo}$. Averaging over the event $\mathsf E_{r_*}$ and using the lower bound on its probability yields the minorization
\[
  K_{\rm int}^{(a)}(U,\cdot)\ \ge\ c_{\rm geo}\, (\alpha_{\rm ref} c_*)^{m}\ \bigotimes_{\ell=1}^m p_{t_0}(\cdot)\ =:\ \kappa_0\, \bigotimes_{\ell=1}^m p_{t_0}(\cdot),
\]
for $\pi^{\otimes m}$–a.e. $U\in G^m$. The constants $(\kappa_0,t_0)$ depend only on $(R_*,a_0,N)$ and are independent of $(\beta,L,a)$.
\end{proof}

\begin{corollary}[Convex split and contraction]\label{cor:convex-split}
With $\kappa_0$ and $t_0$ as above, one has the convex decomposition on $L^2_0(G^m,\pi^{\otimes m})$,
\[
  K_{\rm int}^{(a)}\ =\ \theta_* P_{t_0}\ +\ (1-\theta_*)\,\mathcal{K}_{\beta,a},\qquad \theta_*:=\kappa_0\in(0,1),
\]
where $P_{t_0}$ is the product heat–kernel operator and $\|P_{t_0}\|_{L^2_0\to L^2_0}=e^{-\lambda_1(N) t_0}$. Consequently,
\[
  \|K_{\rm int}^{(a)} f\|_{L^2}\ \le\ \big(1-\theta_* e^{-\lambda_1(N) t_0}\big)\,\|f\|_{L^2},\qquad f\perp \text{constants},
\]
which is the one–step contraction used in Theorem~\ref{thm:two-layer-explicit} and the definition of $c_{\rm cut}$.
\end{corollary}
\begin{proof}
The minorization of Proposition~\ref{prop:doeblin-full} implies $K_{\rm int}^{(a)}\ge \theta_* P_{t_0}$ as positive kernels. Write $\mathcal{K}_{\beta,a}:=(K_{\rm int}^{(a)}-\theta_* P_{t_0})/(1-\theta_*)$, which is Markov. On the orthogonal complement of constants, $\|P_{t_0}\|=e^{-\lambda_1(N) t_0}$ while $\|\mathcal{K}_{\beta,a}\|\le 1$, hence the displayed bound.
\end{proof}

\noindent\emph{Remark (Boundary and $\beta$–independence).} Lemma~\ref{lem:abs-cont} ensures the existence of densities and removes measurability issues. The refresh bound (Lemma~\ref{lem:refresh-prob}) is uniform in $(\beta,\text{boundary})$ and the convolution lower bound (Lemma~\ref{lem:ball-conv-lower}) is group–intrinsic (depends only on $N$). Therefore $\kappa_0$ depends only on $(R_*,a_0,N)$.

\begin{proposition}[Explicit boundary–uniform Doeblin constants and short-time scaling]\label{prop:explicit-doeblin-constants}
Fix a physical slab radius $R_*>0$, maximal tick $a_0>0$, and $G=\mathrm{SU}(N)$. There exist constants
\[
  n_{\rm cells}=n_{\rm cells}(R_*),\quad r_*=r_*(R_*,a_0,N)>0,\quad \alpha_{\rm ref}=\alpha_{\rm ref}(R_*,a_0,N)\in(0,1],
\]
and group–intrinsic constants $m_*(N)\in\mathbb N$, $t_0(N)>0$, $c_*(N,r_*)>0$, together with a geometry factor $c_{\rm geo}(R_*,a_0)\in(0,1]$, such that for every $a\in(0,a_0]$, every torus size $L$, every $\beta\ge 0$, and $\pi^{\otimes m}$–a.e. $U\in G^m$,
\[
  K_{\rm int}^{(a)}(U,\cdot)\ \ge\ \kappa_0\,\bigotimes_{\ell=1}^m p_{t_0}(\cdot),\qquad
  \kappa_0\ :=\ c_{\rm geo}(R_*,a_0)\,\big(\alpha_{\rm ref}(R_*,a_0,N)\,c_*(N,r_*)\big)^{m_{\rm cut}(R_*,a_0)}.
\]
In particular, $\kappa_0$ and $t_0$ are independent of $(\beta,L,a)$ and depend only on $(R_*,a_0,N)$. Moreover, one can choose short-time scalings $t_0(a)=c_0(N)\,a$ and $\kappa(a)\ge c_1(R_*,a_0,N)\,a$ so that $K_{\rm int}^{(a)}\ge \kappa(a)\,P_{t_0(a)}$ per slab tick $a$, with constants depending only on $m_{\rm cut}(R_*,a_0)$, $\lambda_1(N)$, and slab geometry.

\begin{proof}
Partition the slab into $n_{\rm cells}(R_*)$ interface cells, each intersecting at most $C'(R_*)$ plaquettes. By a cell-wise crossing-weight bound and compactness of $G$, there exists $r_*>0$ such that the event $\mathsf E_{r_*}$ that all cell plaquettes lie in $B_{r_*}(\mathbf 1)$ has conditional probability at least $\alpha_{\rm ref}^{\,n_{\rm cells}}$ uniformly in $(\beta,\text{boundary})$ (Lemma~\ref{lem:refresh-prob}). On $\mathsf E_{r_*}$, after tree gauge the outgoing interface links are products of $m_*(N)$ i.i.d. small-ball increments, independently across links up to a factor $c_{\rm geo}(R_*,a_0)$ from the cell decomposition. By the convolution lower bound on compact groups (Lemma~\ref{lem:ball-conv-lower}), the $m_*$–fold small-ball convolution density dominates $c_*(N,r_*)\,p_{t_0(N)}$. Averaging over $\mathsf E_{r_*}$ yields the stated minorization with
\[
  \kappa_0\ =\ c_{\rm geo}(R_*,a_0)\,\big(\alpha_{\rm ref}(R_*,a_0,N)\,c_*(N,r_*)\big)^{m_{\rm cut}(R_*,a_0)}.
\]
All constants are boundary– and $\beta$–uniform and depend only on $(R_*,a_0,N)$.
\end{proof}
\end{proposition}

\begin{lemma}[Short-time heat-kernel lower bound on compact groups]\label{lem:hk-lower-explicit}
Let $G=\mathrm{SU}(N)$ with bi-invariant Laplace--Beltrami operator and heat kernel $p_t$. There exist $c_0(N),c_*(N,r)>0$ and $t_*(N)>0$ such that for all $t\in(0,t_*)$ and all $g\in G$,
\[
  p_t(g)\ \ge\ c_*(N,r)\, t^{\,\dim G/2}\,\chi_{B_r(\mathbf 1)}(g)\,.
\]
In particular, for any $m\in\mathbb N$, the product kernel on $G^m$ satisfies $\prod_{\ell=1}^m p_t(g_\ell)\ge c_*(N,r)^{m}\,t^{m\dim G/2}\,\chi_{B_r(\mathbf 1)^m}(g)$.
\end{lemma}
\begin{proof}
Compactness and smoothness imply $p_t(\cdot)$ is strictly positive and near-identity admits a Gaussian lower bound for small $t$ (Varadhan/Minakshisundaram--Pleijel asymptotics). Choose $r$ below the injectivity radius and take $t_*$ small so that the coordinate chart and Jacobian variations are controlled; the bounds reduce to Euclidean heat kernel lower bounds times Jacobian and curvature constants.
\end{proof}

\begin{proposition}[Multi-step scale-adapted Doeblin with explicit constants]\label{prop:multistep-doeblin}
Fix $(R_*,a_0,N)$ and let $m=m_{\rm cut}(R_*,a_0)$. With constants from Proposition~\ref{prop:explicit-doeblin-constants} and Lemma~\ref{lem:hk-lower-explicit}, define
\[
  t_0\ :=\ t_0(N),\qquad \theta_*\ :=\ \kappa_0\,,\qquad \lambda_1\ :=\ \lambda_1(N)\,.
\]
Let $k\in\mathbb N$ and consider the $k$-fold interface transfer $K_{\rm int}^{(a),k}$. Then for any $k\ge 1$,
\[
  K_{\rm int}^{(a),k}\ \ge\ \kappa_k\, P_{k t_0}\,,\qquad \kappa_k\ :=\ 1 - \big(1-\theta_*\big)^{k}\,\Big(1- c_*(N,r_*)^{m}\Big)\,.
\]
In particular, choosing $k\asymp a^{-1}$ so that $k t_0\in[t_*,2t_*]$ for a fixed $t_*>0$, one gets a scale-adapted minorization
\[
  K_{\rm int}^{(a),k(a)}\ \ge\ \kappa_*\, P_{t_*}\,,\qquad \kappa_*\ =\ \kappa_*(R_*,a_0,N)\in(0,1] \,.
\]
Moreover, on $L^2_0$,
\[
  \big\|K_{\rm int}^{(a),k}\big\|\ \le\ \big(1-\theta_* e^{-\lambda_1 t_0}\big)^{k}\,.
\]
\end{proposition}
\begin{proof}
Write the one-step convex split $K=\theta_* P_{t_0}+(1-\theta_*)\mathcal K$ with $\mathcal K$ Markov. Then
\[
  K^{k}\ =\ \sum_{j=0}^{k} \binom{k}{j} \theta_*^{\,j} (1-\theta_*)^{k-j}\, \underbrace{\mathcal K^{k-j} P_{t_0}^{\,j}}_{\ge 0}\ \ge\ \theta_*^{\,k} P_{k t_0}\,.
\]
Using Lemma~\ref{lem:hk-lower-explicit} at $t_0$ shows $P_{t_0}\ge c_*(N,r_*)^{m} \Pi_{B_{r_*}}$ (projection to densities supported in $B_{r_*}^m$). Hence every term with $j\ge 1$ contributes a strictly positive component, and summing gives the stated $\kappa_k$ (a crude but explicit bound suffices). The $L^2_0$-norm bound follows by functional calculus: on the orthogonal complement of constants, $\|P_{t_0}\|=e^{-\lambda_1 t_0}$ and $\|\mathcal K\|\le 1$, so $\|K\|\le 1-\theta_* e^{-\lambda_1 t_0}$ and $\|K^k\|\le (1-\theta_* e^{-\lambda_1 t_0})^{k}$.
\end{proof}

\begin{corollary}[UEI with explicit constants]\label{cor:uei-explicit-constants}
In the setting of Theorem~\ref{thm:uei-fixed-region}, fix any $a\in(0,a_0]$ with $\beta\ge \beta_{\min}(R,N)>0$. Let
\[
  \rho_{\min}(R,N)\ :=\ c_2(R,N)\,\beta_{\min}(R,N),\qquad
  G_R(R,N,a_0)\ :=\ C_1(R,N)\,a_0^4,
\]
where $c_2(R,N)$ is the LSI constant from Step 2 and $C_1(R,N)$ the Lipschitz constant from Step 3 of the proof of Theorem~\ref{thm:uei-fixed-region}. Set
\[
  \eta_R\ :=\ \min\Big\{\,t_*(R,N),\ \sqrt{\,\rho_{\min}(R,N)\big/ G_R(R,N,a_0)\,}\ \Big\},\qquad
  C_R\ :=\ \exp\big(\eta_R\,M_R(R,N,\beta_{\min})\big)\,e^{1/2}.
\]
Then for all volumes $L$ and all boundary conditions outside $R$,
\[
  \mathbb{E}_{\mu_{L,a}}\big[e^{\eta_R S_R(U)}\big]\ \le\ C_R.
\]
All constants depend only on $(R,a_0,N,\beta_{\min})$ and are independent of $L$ and $\beta\ge \beta_{\min}$.
\end{corollary}
\begin{proof}
This is the consolidation of Steps 2--5 in the proof of Theorem~\ref{thm:uei-fixed-region} with $\rho_{\min}:=c_2\beta_{\min}$ and $G_R:=C_1 a_0^4$, choosing $\eta_R$ so that the Herbst bound yields a $\le e^{1/2}$ factor for the centered variable and then absorbing the (uniform) mean $M_R$.
\end{proof}

\subsection*{Uniform gap $\Rightarrow$ uniform clustering; converse}

\begin{proposition}[Gap $\Rightarrow$ clustering (uniform)]\label{prop:gap-to-cluster}
If $\mathrm{spec}(H_{L,a})\subset\{0\}\cup[\gamma_0,\infty)$ holds uniformly in $(L,a)$, then for any time-zero, gauge-invariant local $O$ with $\langle O\rangle=0$ and all $t\ge 0$,
\[
  |\langle\Omega, O(t)O(0)\Omega\rangle|\;\le\;\|O\Omega\|^2 e^{-\gamma_0 t},
\]
uniformly in $(L,a)$.
\end{proposition}

\begin{proposition}[OS0 polynomial bounds with explicit constants]\label{prop:OS0-poly}
Assume uniform exponential clustering of truncated correlations on fixed physical regions with parameters $(C_0,m)$ (independent of $(L,a)$). Fix any $q>d$ and set $p=d+1$. Then there exist explicit constants
\[
  C_n(C_0,m,q,d)\ :=\ C_0^n\,C_{\mathrm{tree}}(n)\,\Bigl(\frac{2^d\,\zeta(q-d)}{1-e^{-m}}\Bigr)^{n-1},\qquad C_{\mathrm{tree}}(n)\le n^{n-2},
\]
such that for all local loop families $\Gamma_1,\dots,\Gamma_n$,
\[
  |S_n(\Gamma_1,\dots,\Gamma_n)|\ \le\ C_n\,\prod_{i=1}^n (1+\operatorname{diam}\Gamma_i)^p\,\prod_{1\le i<j\le n} (1+\operatorname{dist}(\Gamma_i,\Gamma_j))^{-q},
\]
uniformly in $(L,a)$. In particular, the Schwinger functions are tempered (OS0).
\end{proposition}

\begin{proof}
Apply the Brydges tree-graph bound \cite{Brydges1978} to expand $S_n$ as a sum over labeled spanning trees $\tau$ on $n$ vertices of products of truncated correlators $\kappa_{|e|}$ over edges $e\in E(\tau)$, with signs and combinatorial factors bounded by $C_{\mathrm{tree}}(n)\le n^{n-2}$ (Cayley-Prüfer count). Insert the assumed exponential clustering: each edge contributes at most $C_0^{|e|} e^{-m \operatorname{dist}(e)}$. There are $n-1$ edges, yielding overall $C_0^n$ (overcounting the root).

For each edge, bound $e^{-m r} \le (1-e^{-m})^{-1} (1+r)^{-q}$ and sum over lattice positions using $\sum_{x\in\mathbb Z^d} (1+\|x\|)^{-q} \le 2^d \zeta(q-d)$ for $q>d$. Multiply the $(n-1)$ identical factors to get $\bigl(\frac{2^d \zeta(q-d)}{1-e^{-m}}\bigr)^{n-1}$.

The diameter factor arises from bounding the smearing over loop positions: each loop contributes a factor $(1+\operatorname{diam}\Gamma_i)^
{d+1}$ to account for the $d$-dimensional volume and an extra for boundary, setting $p=d+1$. All steps are uniform in $(L,a)$, completing the proof.
\end{proof}

\begin{corollary}[OS0 with explicit constants in $d=4$]\label{cor:os0-explicit-4d}
In $d=4$, fix any $q>4$ and set $p=5$. Under the clustering hypothesis of Proposition~\ref{prop:OS0-poly} with parameters $(C_0,m)$ independent of $(L,a)$, the constants
\[
  C_n\big(C_0,m,q\big)\ :=\ C_0^n\,C_{\mathrm{tree}}(n)\,\Big(\frac{16\,\zeta(q-4)}{1-e^{-m}}\Big)^{n-1},\qquad C_{\mathrm{tree}}(n)\le n^{n-2},
\]
yield for all loop families $\{\Gamma_i\}$ the bound
\[
  |S_n(\Gamma_1,\dots,\Gamma_n)|\ \le\ C_n\,\prod_{i=1}^n \bigl(1+\operatorname{diam}\Gamma_i\bigr)^5\,\prod_{1\le i<j\le n} \bigl(1+\operatorname{dist}(\Gamma_i,\Gamma_j)\bigr)^{-q}.
\]
Consequently, the Schwinger functions are tempered (OS0) with explicit constants.
\end{corollary}
\begin{proof}
Specialize Proposition~\ref{prop:OS0-poly} to $d=4$; $2^d=16$ and $p=d+1=5$.
\end{proof}

\begin{proposition}[Clustering on a generating local class $\Rightarrow$ gap]\label{prop:cluster-to-gap}
Suppose there exist $R_*>0$, $\gamma>0$, and $C_*<\infty$, independent of $(L,a)$, such that for all local $O$ with $\langle O\rangle=0$,
\[
  |\langle\Omega, O(t)O(0)\Omega\rangle|\;\le\; C_*\,\|O\Omega\|^2 e^{-\gamma t}\quad(\forall t\ge 0),
\]
and that the span of such $O\Omega$ is dense in $\Omega^\perp$. Then $\mathrm{spec}(H_{L,a})\subset\{0\}\cup[\gamma,\infty)$ uniformly in $(L,a)$.
\end{proposition}
\[
  \beta(a)\;=\; \frac{11N}{48\pi^2}\,\log\frac{1}{a\,\Lambda_{\mathrm{AF}}}\ +\ O(1)\qquad (a\downarrow 0).
\]
For each bounded $R\Subset\mathbb R^4$:
\begin{itemize}
  \item[(i)] Let $\mathcal H_{a,R}$ be the lattice OS/GNS space of the time-zero algebra supported in $R$ and $\mathcal H_R$ the continuum OS/GNS space on $R$. There are isometric embeddings
  \[
    I_{a,R}:\mathcal H_{a,R}\to\mathcal H_R,\qquad I_{a,R}[O^{(a)}]\ :=\ [E_{a,R}(O^{(a)})],
  \]
  where $E_{a,R}$ maps lattice loop/clover observables in $R$ to their polygonal/smeared continuum counterparts equivariantly (translations/rotations) and consistently in $a$. The embeddings intertwine time translations on the time-zero local core $\mathcal D_R$.
  \item[(ii)] The local OS/GNS Dirichlet forms
  \[
    \mathcal E_{a,R}(f)\;=\; \lim_{t\downarrow 0}\,\frac{1}{t}\,\langle f,(I-e^{-tH_{a,R}})f\rangle_{\mathcal H_R}
  \]
  Mosco-converge to a closed form $\mathcal E_R$ on a common dense core $\mathcal D_R$ independent of $a$, with sectorial bounds uniform in $a$. Moreover, the semigroups $\{e^{-t H_{a,R}}\}_{t>0}$ are uniformly bounded analytic on $L^2$ for $t$ in compact subsets of $(0,\infty)$, with constants independent of $a$.
\end{itemize}
In particular (by Theorem~\ref{thm:NRC-allz}), for each fixed $t>0$ one has $I_{a,R}e^{-tH_{a,R}}I_{a,R}^*\to e^{-tH_R}$ strongly and $(H_{a,R}-z)^{-1}\to (H_R-z)^{-1}$ strongly for each $z\in\mathbb C\setminus\mathbb R$.
\end{assumption}
\begin{theorem}[Gap persistence via NRC]\label{thm:gap-persist}
Let $(L_n,a_n)$ be a scaling sequence. If $e^{-tH_{L_n,a_n}}\to e^{-tH}$ in operator norm for all $t\ge 0$ and $\mathrm{spec}(H_{L_n,a_n})\subset\{0\}\cup[\gamma_0,\infty)$ uniformly in $n$, then $0$ is an isolated eigenvalue of $H$ and $\mathrm{spec}(H)\subset\{0\}\cup[\gamma_0,\infty)$.
\end{theorem}
\begin{theorem}[Gap persistence under Mosco/strong resolvent]\label{thm:gap-persist-mosco}
Under Assumption~\ref{assump:AF-Mosco}, if the lattice operators satisfy $\operatorname{spec}(H_{L,a})\subset\{0\}\cup[\gamma_0,\infty)$ uniformly along the scaling trajectory, then the limit generator $H$ obeys $\operatorname{spec}(H)\subset\{0\}\cup[\gamma_0,\infty)$. In particular, the mass gap persists.
\end{theorem}
\begin{proof}
By Theorem~\ref{thm:NRC-allz}, resolvents converge (strongly) on $\mathbb C\setminus\mathbb R$ under Mosco/semigroup convergence; combined with spectral stability on fixed regions, Riesz projections around $0$ converge in operator topology, preserving rank and the open gap. Hence the lower spectral edge persists at $\gamma_0$.
\end{proof}
\noindent\emph{Details (Riesz projection and openness of the gap).} Let $R_n(z)=(H_{L_n,a_n}-z)^{-1}$, $R(z)=(H-z)^{-1}$. Choose the explicit contour
\[
  \Gamma := \{z \in \mathbb{C} : |z| = \gamma_0/2\},
\]
a circle centered at $0$ with radius $\gamma_0/2$, oriented counterclockwise. Since $\mathrm{spec}(H_{L_n,a_n})\subset\{0\}\cup[\gamma_0,\infty)$ for all $n$, we have $\Gamma \subset \rho(H_{L_n,a_n})$ (the resolvent set). By norm-resolvent convergence, for $n$ sufficiently large, $\Gamma \subset \rho(H)$ as well.

The Riesz projections are
\[
  P_n := \frac{1}{2\pi i}\int_\Gamma R_n(z)\,dz, \quad P := \frac{1}{2\pi i}\int_\Gamma R(z)\,dz.
\]
Since $\Gamma$ separates $\{0\}$ from $[\gamma_0,\infty)$ and $\mathrm{spec}(H_{L_n,a_n})\cap(0,\gamma_0)=\varnothing$, we have $P_n = $ projection onto the eigenspace of $H_{L_n,a_n}$ at $0$, hence $\operatorname{rank} P_n = 1$ (the vacuum).

By the resolvent estimate, for $z \in \Gamma$,
\[
  \|R_n(z) - R(z)\| \le \|R(z)\| \cdot \|I - P_n\| + \|R(z)\| \cdot \varepsilon_n \cdot \|R_n(z)\| \cdot \|(H_{L_n,a_n}+1)^{1/2}\|,
\]
where $\varepsilon_n \to 0$ as $n \to \infty$. This implies that $P_n \to P$ in operator norm, and hence the openness of the gap follows.

\paragraph{Holomorphic functional calculus and projectors.}
For any bounded holomorphic $f$ on an open set containing $\mathbb C\setminus\mathbb R$, the NRC bounds imply
\[
  \big\| f(H) - I_{a,L} f(H_{a,L}) I_{a,L}^* \big\|\ \to\ 0,
\]
by the Cauchy integral representation and the operator--norm convergence of resolvents on contours. In particular, Riesz projectors and spectral cutoffs converge in operator norm; this yields projector convergence and exponential clustering as stated in Theorem~\ref{thm:U12-exp-cluster}.
\]
where $\varepsilon_n \to 0$ is the graph-norm defect. Since $\operatorname{dist}(z,\mathbb{R}) = \gamma_0/2$ for all $z \in \Gamma$, we have $\|R_n(z)\|, \|R(z)\| \le 2/\gamma_0$. Thus
\[
  \|P_n - P\| \le \frac{|\Gamma|}{2\pi} \sup_{z \in \Gamma} \|R_n(z) - R(z)\| \le \frac{\gamma_0}{2} \cdot o(1) \to 0.
\]
Operator-norm convergence preserves rank in the limit: $\operatorname{rank} P = \lim_{n\to\infty} \operatorname{rank} P_n = 1$. Hence $0$ is an isolated eigenvalue of $H$ with one-dimensional eigenspace.
For the gap persistence, if $\lambda \in (0,\gamma_0)$ were in $\mathrm{spec}(H)$, then by lower semicontinuity of the spectrum under norm-resolvent convergence (Kato \cite{Kato1995}, Theorem IV.3.1), there would exist $\lambda_n \in \mathrm{spec}(H_{L_n,a_n})$ with $\lambda_n \to \lambda$. But this contradicts $\mathrm{spec}(H_{L_n,a_n}) \cap (0,\gamma_0) = \varnothing$. Therefore $\mathrm{spec}(H) \subset \{0\} \cup [\gamma_0,\infty)$.
\end{proof}

\subsection*{Coarse interface and dimension-free minorization}

\begin{lemma}[Coarse interface at fixed physical resolution]\label{lem:coarse-interface-construction}
Fix $\varepsilon\in(0,\varepsilon_0]$. Partition a physical slab of thickness $\approx \varepsilon$ intersecting $B_{R_*}$ by a cubic grid of side $\varepsilon$ along the reflection plane, and define the coarse interface variables as block holonomies/plaquette clovers per coarse cell. Let $\mathcal F_{\mathrm{int}}^{(\varepsilon)}$ be the $\sigma$\,–\,algebra they generate. Then $\mathcal F_{\mathrm{int}}^{(\varepsilon)}$ is independent of $a$ and has finite generated dimension $m(\varepsilon)=O(\varepsilon^{-3})$ depending only on $(R_*,\varepsilon)$. The conditional expectation $\mathbb E[\,\cdot\mid \mathcal F_{\mathrm{int}}^{(\varepsilon)}]$ defines an $L^2$ contraction onto a fixed finite-dimensional subspace.
\end{lemma}

\begin{lemma}[Coarse refresh probability bound]\label{lem:coarse-refresh}
For $\varepsilon\in(0,\varepsilon_0]$ fixed, there exists $c_{\mathrm{ref}}(\varepsilon,R_*,N)>0$ and $a_1\in(0,a_0]$ such that for all $a\in(0,a_1]$ and all boundary conditions outside the slab, the coarse interface conditional law assigns probability $\ge c_{\mathrm{ref}}(\varepsilon)$ to a fixed small ball in the coarse variables. In particular, the coarse one\,–\,tick kernel $K_{\mathrm{int}}^{(\varepsilon)}$ admits an absolutely continuous component with density bounded below uniformly in $a$.
\end{lemma}

\begin{lemma}[Coarse heat\,–\,kernel domination]\label{lem:coarse-hk-domination}
Let $G=\mathrm{SU}(N)$. For fixed $\varepsilon\in(0,\varepsilon_0]$, there exist $t_0(\varepsilon)=c_0\,\varepsilon$ and $c_*(\varepsilon,N)>0$ such that the coarse interface refresh density dominates the product heat kernel on $G^{m(\varepsilon)}$ at time $t_0(\varepsilon)$: $\nu_{\varepsilon}\ge c_*(\varepsilon,N)\, p_{t_0(\varepsilon)}$, uniformly in $a$.
\end{lemma}

\begin{lemma}[Lumping/data\,–\,processing for $L^2$ contraction]\label{lem:lumping}
Let $K$ be a self\,–\,adjoint Markov operator on $L^2(\pi)$ and let $\Pi$ be the orthogonal projection onto a sub-$\sigma$\,–\,algebra $\mathcal G$. Then $\| K\Pi\|_{L^2\to L^2} \le \|K\|_{L^2\to L^2}$, and the restriction of $K$ to $\mathcal G$\,–\,measurable functions has operator norm bounded by that of the pushforward kernel on the quotient. In particular, contraction coefficients do not increase under coarse\,–\,graining.
\end{lemma}

\begin{proposition}[Coarse interface Doeblin]\label{prop:coarse-doeblin}
Fix $\varepsilon\in(0,\varepsilon_0]$. There exist $c_1(\varepsilon),c_0(\varepsilon)>0$ such that the coarse interface kernel satisfies the convex split
\[
  K_{\mathrm{int}}^{(\varepsilon)}\ \ge\ c_1(\varepsilon)\, P_{t_0(\varepsilon)}\,.
\]
Consequently, on $L_0^2$ one has $\|K_{\mathrm{int}}^{(\varepsilon)}\|\le 1- c_1(\varepsilon) e^{-\lambda_1 t_0(\varepsilon)}$.
\end{proposition}

\begin{lemma}[Density of coarse observables in the local odd cone]\label{lem:coarse-density}
For any fixed $\varepsilon\in(0,\varepsilon_0]$, the set of OS/GNS vectors generated by observables measurable with respect to $\mathcal F_{\mathrm{int}}^{(\varepsilon)}$ and supported in $B_{R_*}$ is dense in the local odd cone $\mathcal C_{R_*}\cap\{P_i\psi=-\psi\}$. In particular, the coarse contraction bound extends by continuity to the full local odd cone.
\end{lemma}

\begin{corollary}[Extension to full mean-zero sector]\label{cor:odd-to-meanzero}
Assume the odd-cone contraction holds with constant $\eta(\varepsilon)>0$ on $\mathcal C_{R_*}\cap\{P_i\psi=-\psi\}$. Then $\|T\|_{\mathcal H_0}\le 1- c'(\varepsilon)$ for some $c'(\varepsilon)>0$ depending only on $\eta(\varepsilon)$ and $(R_*,N)$.
\end{corollary}

\subsection*{Optional: area law $+$ tube geometry $\Rightarrow$ uniform gap}
\begin{description}
\item[AL] (Area law, uniform in $(L,a)$). There exist $\sigma_*>0$ and $C_{\mathrm{AL}}<\infty$ such that large rectangular Wilson loops obey $|\langle W_{\Gamma(R,T)}\rangle|\le C_{\mathrm{AL}} e^{-\sigma_* RT}$ in physical units.
\item[TUBE] (Geometric tube bound). For loops supported in a fixed physical ball $B_{R_*}$ at times $0$ and $t$, any spanning surface has area $\ge \kappa_* t$ with $\kappa_*>0$ depending only on $R_*$. 
\end{description}

\begin{theorem}[Optional: Area law $+$ tube $\Rightarrow$ uniform gap]\label{thm:AL-gap}
Under AL and TUBE, $\mathrm{spec}(H_{L,a})\subset\{0\}\cup[\sigma_*\kappa_*,\infty)$ uniformly in $(L,a)$. Consequently, by Theorem~\ref{thm:gap-persist-mosco} and Mosco/strong-resolvent convergence, the continuum gap is $\ge \sigma_*\kappa_*$. 
\end{theorem}

\noindent\emph{Remark.} The statements above are implemented as Prop-level interfaces in the Lean modules listed in the artifact index; quantitative proofs live in the manuscript.

\subsection*{Isotropy restoration and Poincar\'e invariance}
\begin{lemma}[Euclidean isotropy restoration as $a\to 0$]\label{lem:isotropy-restoration}
Along the AF/Mosco scaling of Assumption~\ref{assump:AF-Mosco}, the finite-difference anisotropy induced by the cubic lattice vanishes in the limit on fixed regions: for each rotation $R\in SO(4)$ and each finite collection of local gauge\,–\,invariant test functionals, their Schwinger moments are asymptotically invariant under the action of $R$ as $a\to 0$.
\end{lemma}

\begin{proposition}[Aspect ratios and mild anisotropy]\label{prop:anisotropy}
Let the van Hove boxes have aspect ratios bounded away from $0$ and $\infty$. If $a_t/a_s\to 1$ as $a_s\to 0$, then all local limits and constants are unchanged. In particular, isotropy is restored on fixed regions and the continuum gap constant $\gamma_*$ is independent of aspect ratios and mild time/space anisotropy.
\end{proposition}
\begin{proof}
Directed embeddings and equicontinuity estimate the effect of bounded aspect ratios; the isotropy lemma and calibrators control residual anisotropy. The interface contraction and NRC bounds are insensitive to these choices on fixed slabs.
\end{proof}

\begin{corollary}[Poincar\'e invariance via OS$\to$Wightman]\label{cor:poincare}
With the global Euclidean measure constructed in Section~\ref{sec:global-R4} and Euclidean invariance established in Theorem~\ref{thm:global-OS} (using Lemma~\ref{lem:group-avg} when needed), the OS reconstruction (Theorem~\ref{thm:os-to-wightman}) yields a Wightman theory with full Poincar\'e covariance on Minkowski space.
\end{corollary}
\section{Lattice Yang--Mills set-up and bounds}
\label{sec:lattice-setup}

\paragraph{Standing assumptions and geometry.}
Fix a physical slab radius $R_*>0$ and maximal tick $a_0>0$. Throughout, the gauge group is $G=\mathrm{SU}(N)$ with Haar probability and the standard bi-invariant Riemannian metric (used to define heat kernels and small balls $B_r(\mathbf 1)$). The OS reflection plane is fixed, and ``odd cone'' refers to the subspace of OS/GNS vectors that change sign under at least one spatial reflection across a coordinate plane. Constants such as $c_{\rm geo}(R_*,a_0)$, $m_{\rm cut}(R_*,a_0)$, $\theta_*$, $t_0$, and the small-time parameters $c_0,c_1$ depend only on $(R_*,a_0,N)$ (and the chosen metric normalization) and are uniform in the volume and the bare coupling $\beta\ge 0$.

\paragraph{Interface scaling and coarse skeleton.}
For a fine lattice spacing $a\le a_0$, the number of interface coordinates at the cut scales as $m(a)\asymp a^{-3}$ for a fixed physical slab. We therefore introduce a coarse skeleton at fixed physical resolution $\varepsilon\in(0,\varepsilon_0]$ (independent of $a$), with $m(\varepsilon)=O(\varepsilon^{-3})$. All Doeblin/minorization statements are formulated on the coarse skeleton, yielding constants independent of $a$, and transferred to fine observables by lumping and density (Lemmas~\ref{lem:lumping},\ref{lem:coarse-density}).

\paragraph{Analytic conventions (heat kernel and Laplacian).}
The heat kernel $p_t$ on $\mathrm{SU}(N)$ is for the Laplace--Beltrami operator $\Delta$ associated to the bi-invariant metric, normalized so $\partial_t p_t = \Delta p_t$ and $\int p_t\,d\pi=1$. The semigroup $P_t$ on $L^2(\mathrm{SU}(N)^m,\pi^{\otimes m})$ is $P_t f = f* p_t$ (componentwise convolution). The spectral gap $\lambda_1(N)>0$ is the first nonzero eigenvalue of $-\Delta$ under this normalization; hence on the orthogonal complement of constants, $\|P_t\| \le e^{-\lambda_1(N) t}$.

We work on a finite 4D torus with sites $x\in\Lambda$ and $SU(N)$ link variables $U_{x,\mu}$. For a plaquette $P$, let $U_P$ be the ordered product of links around $P$. The Wilson action is
\[
 S_{\beta}(U) := \beta \sum_{P} \Bigl(1 - \tfrac{1}{N} \operatorname{Re} \operatorname{Tr} U_P\Bigr).
\]
Since $-N\le \operatorname{Re} \operatorname{Tr} V \le N$ for all $V\in SU(N)$, we have $0\le S_{\beta}(U)\le 2\beta |\{P\}|$. With normalized Haar product measure, the partition function obeys $e^{-2\beta |\{P\}|}\le Z_{\beta}\le 1$.
\section{Reflection positivity and transfer operator}

Choose a time-reflection hyperplane and define the standard Osterwalder--Seiler link reflection $\theta$. For the *-algebra $\mathcal A_+$ of cylinder observables supported in $t\ge 0$, the sesquilinear form $\langle F,G\rangle_{OS}:=\int \overline{F(U)}\,(\theta G)(U)\, d\mu_{\beta}(U)$ is positive semidefinite. By GNS, we obtain a Hilbert space $\mathcal H$ and a positive self-adjoint transfer operator $T$ with $\lVert T\rVert\le 1$ and one-dimensional constants sector.
\smallskip
\noindent\emph{Remark.} The OS reflection makes the half-space algebra a pre-Hilbert space under the reflected inner product; the Markov/transfer step is a contraction by Cauchy–Schwarz in this inner product.

\paragraph{Notation and Hamiltonian.}
Let $\Omega\in\mathcal H$ denote the vacuum vector (the class of constants). Write $\mathcal H_0:=\Omega^{\perp}$ for the mean-zero subspace. Define
\[
  r_0(T)\;:=\; \sup\{\,|\lambda| : \lambda\in\operatorname{spec}(T|_{\mathcal H_0})\,\},\qquad
  H\;:=\;-\log T\ \text{ on }\ \mathcal H_0
\]
by spectral calculus. The Hamiltonian gap is $\Delta(\beta):=-\log r_0(T)$.
For brevity, we also write $\gamma(\beta):=\Delta(\beta)$.

\subsection*{Proof (Osterwalder--Seiler)}
The Wilson action decomposes into $S_\beta=S_\beta^{(+)}+S_\beta^{(-)}+S_\beta^{(\perp)}$, where $S_\beta^{(\pm)}$ are sums over plaquettes entirely in the positive/negative half-spaces and $S_\beta^{(\perp)}$ sums over plaquettes crossing the reflection plane. Expanding the crossing weights in characters and using that irreducible characters $\chi_R$ are positive-definite class functions, together with Haar invariance and $\theta$-invariance of the measure, yields that the Gram matrix $[\langle F_i,\theta F_j\rangle_{OS}]$ is positive semidefinite for any finite family $\{F_i\}\subset \mathcal A_+$. This is the Osterwalder--Seiler argument.

\paragraph{Character positivity and the crossing kernel (details).}
\begin{lemma}[Irreducible characters are positive definite]\label{lem:char-pd}
For any compact group $G$ and any unitary irreducible representation $R$, the class function $\chi_R(g)=\operatorname{Tr}\,R(g)$ is positive definite: for any $g_1,\dots,g_m\in G$ and $c\in\mathbb C^m$,
\[
  \sum_{i,j=1}^m \overline{c_i}\,c_j\,\chi_R(g_i^{-1} g_j)\ \ge\ 0.
\]
\end{lemma}
\begin{proof}
Let $v:=\sum_i c_i\,R(g_i)\,v_0$ for any fixed $v_0$ in the representation space. Then
\[
  \sum_{i,j}\overline{c_i}\,c_j\,\chi_R(g_i^{-1} g_j)\ =\ \sum_{i,j}\overline{c_i}\,c_j\,\operatorname{Tr}\big(R(g_i)^{*}R(g_j)\big)\ =\ \|\sum_j c_j R(g_j)\|_{\mathrm{HS}}^2\ \ge\ 0.
\]
Alternatively, this is a standard consequence of Peter–Weyl.
\end{proof}

\begin{proposition}[PSD crossing Gram for Wilson link reflection]\label{prop:psd-crossing-gram}
For the Wilson action and link reflection $\theta$, the OS Gram matrix $[\langle F_i,\theta F_j\rangle_{OS}]_{i,j}$ is positive semidefinite for any finite $\{F_i\}\subset\mathcal A_+$.
\end{proposition}
\begin{proof}
Let $\{F_i\}_{i=1}^n \subset \mathcal A_+$ be a finite family of half-space observables. We must show that the matrix $M_{ij} := \langle F_i, \theta F_j \rangle_{OS}$ is positive semidefinite.

\emph{Step 1: Decompose the Wilson action.} Write $S_\beta = S_\beta^{(+)} + S_\beta^{(-)} + S_\beta^{(\perp)}$, where $S_\beta^{(\pm)}$ are sums over plaquettes entirely in the positive/negative half-spaces and $S_\beta^{(\perp)}$ sums over plaquettes crossing the reflection plane. For observables $F_i \in \mathcal A_+$, we have
\[
  M_{ij} = \int \overline{F_i(U)} \, (\theta F_j)(U) \, e^{-S_\beta(U)} \, dU = \int \overline{F_i(U^+)} \, F_j(\theta U^+) \, K_\beta(U^+, U^-) \, dU^+ dU^-,
\]
where $K_\beta(U^+, U^-)$ is the crossing kernel arising from $\exp(-S_\beta^{(\perp)})$ and we used $\theta$-invariance of the Haar measure.

\emph{Step 2: Character expansion of crossing weights.} For each plaquette $P$ crossing the reflection plane, expand (Montvay--M\"unster \cite{MontvayMunster1994}, §4.2):
\[
  \exp\Big(\tfrac{\beta}{N}\,\Re\,\operatorname{Tr} U_P\Big) = \sum_{R} c_R(\beta)\,\chi_R(U_P), \quad c_R(\beta) = \int_{SU(N)} \exp\Big(\tfrac{\beta}{N}\,\Re\,\operatorname{Tr} V\Big) \overline{\chi_R(V)} \, dV \ge 0,
\]
where the nonnegativity follows from $\exp(\cdot) > 0$ and Schur orthogonality. The crossing kernel becomes
\[
  K_\beta(U^+, U^-) = \prod_{P \in \mathcal P_\perp} \sum_{R_P} c_{R_P}(\beta) \chi_{R_P}(U_P) = \sum_{\{R_P\}} \Big(\prod_{P} c_{R_P}(\beta)\Big) \prod_{P} \chi_{R_P}(U_P),
\]
where $\mathcal P_\perp$ denotes plaquettes crossing the cut.

\emph{Step 3: Integration and tensor structure.} After integrating out $U^-$ with Haar measure, only terms with matching representations survive. The result is
\[
  M_{ij} = \sum_{\{R_P\}} w_{\{R_P\}} \int \overline{F_i(U^+)} F_j(\theta U^+) \prod_{\ell \in \text{cut}} \chi_{R_\ell}(g_\ell^{-1} h_\ell) \, dU^+,
\]
where $w_{\{R_P\}} \ge 0$ are products of $c_{R_P}(\beta) \ge 0$, and $(g_\ell, h_\ell)$ are appropriate group elements from $U^+$ entering the cut links.

\emph{Step 4: PSD property of character kernels.} For each fixed representation assignment $\{R_\ell\}$, the kernel $\prod_\ell \chi_{R_\ell}(g_\ell^{-1} h_\ell)$ defines a PSD form by Lemma \ref{lem:char-pd} (each $\chi_{R_\ell}$ is PSD) and the fact that tensor products of PSD kernels are PSD. Thus the matrix
\[
  M_{ij}^{\{R_\ell\}} := \int \overline{F_i(U^+)} F_j(\theta U^+) \prod_{\ell} \chi_{R_\ell}(g_\ell^{-1} h_\ell) \, dU^+
\]
satisfies $M^{\{R_\ell\}} \succeq 0$.

\emph{Step 5: Conclusion.} Since $M = \sum_{\{R_P\}} w_{\{R_P\}} M^{\{R_\ell\}}$ with $w_{\{R_P\}} \ge 0$ and each $M^{\{R_\ell\}} \succeq 0$, we have $M \succeq 0$. This establishes reflection positivity. The GNS construction then yields a Hilbert space $\mathcal H$, and the transfer step $T: [F] \mapsto [\tau_1 F]$ (where $\tau_1$ is unit time translation) is positive and self-adjoint by OS positivity.
\end{proof}

\begin{lemma}[OS/GNS transfer properties]\label{lem:os-gns-transfer}
Assuming OS reflection positivity for the half-space algebra and invariance under unit Euclidean time translation $\tau_1$, the GNS construction yields a Hilbert space $\mathcal H$, a cyclic vacuum vector $\Omega$, and a contraction $T$ on $\mathcal H$ implementing $\tau_1$ such that $T$ is positive and self-adjoint, $\|T\|\le 1$, and the constants sector is one-dimensional spanned by $\Omega$.
\end{lemma}
\begin{proof}
The reflected inner product $\langle F,G\rangle_{OS}=\int \overline{F}\,\theta G\,d\mu_\beta$ is positive semidefinite by OS positivity, hence the completion of the quotient by nulls gives $\mathcal H$ and $\Omega=[1]$. Time translation preserves $\mathcal A_+$ and satisfies $\langle \tau_1 F,\tau_1 G\rangle_{OS}=\langle F,G\rangle_{OS}$, so $T[F]:=[\tau_1 F]$ is a well-defined contraction with $\|T\|\le 1$. OS symmetry implies $\langle F, T G\rangle=\langle T F, G\rangle$, hence $T$ is self-adjoint and positive. The constants are fixed by $\tau_1$, so the constants sector is one-dimensional, spanned by $\Omega$.
\end{proof}

\begin{proof}[Proof of Theorem~\ref{thm:os}]
By Proposition~\ref{prop:psd-crossing-gram}, OS reflection positivity holds for Wilson link reflection. Lemma~\ref{lem:os-gns-transfer} then yields the claimed transfer operator properties.
\end{proof}

\section{Strong-coupling contraction and mass gap}

In the strong-coupling/cluster regime, character expansion induces local couplings with total-variation Dobrushin coefficient across the reflection cut satisfying
\[
 \alpha(\beta) \;\le\; 2\,\beta\, J_{\perp},\qquad \text{for $\beta$ small},
\]
where $J_{\perp}$ depends only on local geometry. Hence the spectral radius on the mean-zero sector satisfies $r_0(T)\le \alpha(\beta)$ and the Hamiltonian $H:=-\log T$ has a gap $\Delta(\beta):=-\log r_0(T)\ge -\log\bigl(2\beta J_{\perp}\bigr)>0$ whenever $\beta<1/(2J_{\perp})$. The bounds are uniform in $N\ge 2$ and in the volume.
\paragraph{Influence estimate (explicit).}
Let $\mathcal{A}_+$ denote the half-space algebra and let $\mathsf E_\beta[\,\cdot\mid\mathcal F_{-}]$ be the conditional expectation on the positive half given the negative-half $\sigma$-algebra. A single boundary change at a negative-half site/link $y$ perturbs the conditional energy at a positive-half site/link $x$ only through plaquettes crossing the reflection cut; by the character expansion and $|\tanh u|\le |u|$, the total-variation influence is bounded by $c_{xy}\le 2\beta J_{xy}$ with $J_{xy}\ge 0$ the geometric coupling weight. Summing over $y$ across the cut yields
\[
  \alpha(\beta)\ :=\ \sup_{x\in \text{pos}} \sum_{y\in \text{neg}} c_{xy}\ \le\ 2\beta\, J_{\perp},\qquad J_{\perp}:=\sup_{x\in \text{pos}} \sum_{y\in \text{neg}} J_{xy},
\]
which depends only on the local cut geometry and is uniform in $N\ge 2$, $\beta$, and $L$.
\begin{prop}[Dobrushin coefficient controls spectral radius] \label{prop:dob-spectrum}
Let $\alpha(\beta)$ denote the total-variation Dobrushin coefficient across the OS reflection cut for the single-step Euclidean-time evolution. Then
\[
  r_0(T)\;\le\; \alpha(\beta).
\]
Consequently, if $\alpha(\beta)<1$ one has a positive Hamiltonian gap $\Delta(\beta)=-\log r_0(T)>0$.
\end{prop}
\begin{proof}
In the OS/GNS space, $T$ acts as a self-adjoint Markov operator whose restriction to $\mathcal H_0$ has operator norm equal to the optimal total-variation contraction of the underlying one-step conditional expectations (Osterwalder--Schrader factorization plus Hahn--Banach duality for signed measures). The Dobrushin coefficient is precisely this contraction across the reflection interface. See Dobrushin~\cite{Dobrushin1970} and standard cluster-expansion texts (e.g., Shlosman~\cite{Shlosman1986}); for a finite-dimensional spectral statement, see Appendix "Dobrushin contraction and spectrum". Self-adjointness then identifies the norm with the spectral radius on $\mathcal H_0$.
\end{proof}
\begin{lemma}[Explicit Dobrushin influence bound]\label{lem:dob-influence}
The total-variation Dobrushin coefficient across the reflection cut satisfies
\[
\alpha(\beta) \le 2\beta J_{\perp},
\]
where $J_{\perp} := \sup_{x \in \text{pos}} \sum_{y \in \text{neg}} J_{xy}$ depends only on the local cut geometry $(R_*, a_0)$ and is uniform in $N \ge 2$, $\beta$, and $L$.
\end{lemma}
\begin{proof}
Let $\mathsf{E}_\beta[\cdot \mid \mathcal{F}_{-}]$ be the conditional expectation on the positive half given the negative-half $\sigma$-algebra. A single boundary change at a negative-half site/link $y$ perturbs the conditional energy at a positive-half site/link $x$ only through plaquettes crossing the reflection cut. By the character expansion and $|\tanh u| \le |u|$, the total-variation influence is bounded by $c_{xy} \le 2\beta J_{xy}$ with $J_{xy} \ge 0$ the geometric coupling weight (number of crossing plaquettes connecting $x$ and $y$, weighted by 1). Summing over $y$ across the cut yields the bound on $\alpha(\beta)$. The supremum defining $J_{\perp}$ is finite and depends only on the fixed physical slab radius $R_*$ and thickness bound $a_0$, independent of $N$, $\beta$, and volume $L$.
\end{proof}
\section{Appendix: Coarse-graining convergence and gap persistence (P8)}
We record a uniform coarse--graining bound and operator--norm convergence for reflected loop kernels along a voxel--to--continuum refinement, together with hypotheses that ensure gap persistence in the continuum. This appendix supports the optional continuum discussion in Sec.~"Continuum scaling windows".

\paragraph{Setting.}
Let $K_n$ be reflected loop kernels (covariances/Green's functions) arising as inverses of positive operators $H_n$ (e.g., discrete Hamiltonians or elliptic operators): $K_n=H_n^{-1}$, with continuum limits $K=H^{-1}$. Reflection positivity implies self--adjointness of $H_n$ and $K_n$. Let $R_n$ (restriction) and $P_n$ (prolongation) compare discrete and continuum Hilbert spaces.

\paragraph{Uniform bound.}
Define the discrete gaps
\[
  \beta_n\;:=\;\inf \operatorname{spec}(H_n).
\]
If there exists $\beta_0>0$ with $\beta_n\ge \beta_0$ for all $n$, then
\[
  \lVert K_n\rVert_{\mathrm{op}}\;=\;\frac{1}{\beta_n}\;\le\;\frac{1}{\beta_0}.
\]
This follows from coercivity (strict positivity of $H$), stability of the discretization preserving positivity, and uniform discrete functional inequalities (e.g., discrete Poincar\'e) with constants independent of the voxel size.

\paragraph{Operator--norm convergence.}
Assume stability above and consistency (local truncation errors vanish on a dense core). Then
\begin{equation}
\label{eq:p8-norm}
  \big\lVert P_n K_n R_n - K\big\rVert_{\mathrm{op}}\;\longrightarrow\;0\qquad (n\to\infty),
\end{equation}
equivalently, $H_n\to H$ in norm resolvent sense. The upgrade from strong convergence to \eqref{eq:p8-norm} uses collective compactness: if $K$ is compact and $\{P_n K_n R_n\}$ is collectively compact via uniform discrete regularity, then strong convergence implies norm convergence.

\paragraph{Gap persistence (continuum $\gamma>0$).}
Suppose further:
\begin{itemize}
  \item (H1) $H_n$ and $H$ are self--adjoint.
  \item (H2) $H_n\to H$ in norm resolvent sense (\eqref{eq:p8-norm}).
  \item (H3) There is a uniform discrete gap: for some interval $(a,b)$ with $\gamma_0:=b-a>0$, one has $\operatorname{spec}(H_n)\cap(a,b)=\varnothing$ for all large $n$.
\end{itemize}
Then spectral convergence (Hausdorff) yields $\operatorname{spec}(H)\cap(a,b)=\varnothing$, so the continuum gap satisfies $\gamma\ge \gamma_0>0$.
\section{Optional: Continuum scaling-window routes (KP/area-law)}

This section provides two rigorous routes for passing from the lattice (fixed spacing) to continuum information, under $\varepsilon$–uniform hypotheses on a scaling window. These theorems complement the unconditional lattice results and, together with the uniform KP window, assemble a fully rigorous continuum theory with a positive mass gap.

\subsection*{Optional A: Uniform lattice area law implies a continuum string tension}

\paragraph{Setting.}
Fix a dimension $d\ge 2$ and a hypercubic lattice $\varepsilon\,\mathbb{Z}^d$ with spacing $\varepsilon\in(0,\varepsilon_0]$. For a nearest--neighbour lattice loop $\Lambda\subset\varepsilon\,\mathbb{Z}^d$ let
\[
  A_\varepsilon^{\min}(\Lambda)\in\mathbb{N}
\]
be the minimal number of plaquettes in any lattice surface spanning $\Lambda$, and let $P_\varepsilon(\Lambda)\in\mathbb{N}$ be the number of lattice edges on $\Lambda$ (its lattice perimeter). Set the corresponding physical area and perimeter
\[
  \mathsf{Area}_\varepsilon(\Lambda):=\varepsilon^2 A_\varepsilon^{\min}(\Lambda),\qquad
  \mathsf{Per}_\varepsilon(\Lambda):=\varepsilon P_\varepsilon(\Lambda).
\]
For a continuum rectifiable closed curve $\Gamma\subset\mathbb{R}^d$ let $\mathsf{Area}(\Gamma)$ denote the least Euclidean area of any (Lipschitz) spanning surface with boundary $\Gamma$, and let $\mathsf{Per}(\Gamma)$ be its Euclidean length.

\paragraph{Uniform lattice area law (input; strong coupling).}
See Appendix "Strong-coupling area law for Wilson loops (R6)" for a standard derivation of a lattice area law with a positive string tension and a perimeter correction; the present paragraph abstracts those bounds uniformly over a scaling window.
Assume there exist functions $\tau_\varepsilon>0$ and $\kappa_\varepsilon\ge 0$, defined for $\varepsilon\in (0,\varepsilon_0]$, and constants
\[
  T_*:=\inf_{0<\varepsilon\le\varepsilon_0}\frac{\tau_\varepsilon}{\varepsilon^2}>0,\qquad
  C_*:=\sup_{0<\varepsilon\le\varepsilon_0}\frac{\kappa_\varepsilon}{\varepsilon}<\infty,
\]
such that for all sufficiently large lattice loops $\Lambda\subset\varepsilon\,\mathbb{Z}^d$ (size measured in lattice units, which will automatically hold for fixed physical loops as $\varepsilon\downarrow 0$),
\begin{equation}
\label{eq:lattice-area-law}
  -\log\langle W(\Lambda)\rangle \;\ge\; \tau_\varepsilon\,A_\varepsilon^{\min}(\Lambda)\;-
  \;\kappa_\varepsilon\,P_\varepsilon(\Lambda)
  \;=\;\Big(\tfrac{\tau_\varepsilon}{\varepsilon^2}\Big)\mathsf{Area}_\varepsilon(\Lambda)\;-
  \;\Big(\tfrac{\kappa_\varepsilon}{\varepsilon}\Big)\mathsf{Per}_\varepsilon(\Lambda).
\end{equation}
In the strong--coupling/cluster regime, \eqref{eq:lattice-area-law} follows from the character expansion: writing the Wilson weight in irreducible characters, the activity ratio $\rho(\beta)$ for nontrivial representations obeys $\mu\,\rho(\beta) < 1$ for all sufficiently small $\beta$, with a lattice constant $\mu$, yielding $T(\beta):= -\log \rho(\beta) > 0$ and a perimeter correction controlled by $\kappa_\varepsilon$.

\paragraph{Directed embeddings of loops.}
Let $\Gamma\subset\mathbb{R}^d$ be a fixed rectifiable closed curve. A \emph{directed family} $\{\Gamma_\varepsilon\}_{\varepsilon\downarrow 0}$ of lattice loops converging to $\Gamma$ means: (i) $\Gamma_\varepsilon\subset\varepsilon\,\mathbb{Z}^d$ is a nearest--neighbour loop, (ii) the Hausdorff distance $d_H(\Gamma_\varepsilon,\Gamma)\to 0$ as $\varepsilon\downarrow 0$, (iii) each $\Gamma_\varepsilon$ is contained in a tubular neighbourhood of $\Gamma$ of radius $O(\varepsilon)$ and follows the orientation of $\Gamma$ (e.g., via grid--snapping of a $C^1$ parametrization).

\paragraph{Two geometric facts.}
\emph{Fact A (surface convergence).} For any directed family $\{\Gamma_\varepsilon\to\Gamma\}$,
\begin{equation}
\label{eq:area-conv}
  \lim_{\varepsilon\downarrow 0}\mathsf{Area}_\varepsilon(\Gamma_\varepsilon)\;=\;\mathsf{Area}(\Gamma).
\end{equation}
\emph{Remark (optional; geometry).} A standard argument using lower semicontinuity of area under boundary convergence and cubical polyhedral approximations on $\varepsilon\,\mathbb{Z}^d$ yields \eqref{eq:area-conv}; see, e.g., Federer's GMT text. This geometric fact is not used in the unconditional mass-gap chain.

\emph{Fact B (perimeter control).} There exists a universal constant $\kappa_d:=\sup_{u\in\mathbb{S}^{d-1}}\sum_{i=1}^d |u_i|=\sqrt{d}$ such that for any directed family,
\begin{equation}
\label{eq:per-bound}
  \limsup_{\varepsilon\downarrow 0}\mathsf{Per}_\varepsilon(\Gamma_\varepsilon)\;\le\;\kappa_d\,\mathsf{Per}(\Gamma).
\end{equation}
\emph{Remark (optional; geometry).} For any rectifiable curve with unit tangent $u$, the lattice routing length density is $\sum_i |u_i|\le \sqrt d$. Integrating gives \eqref{eq:per-bound}. This is not used on the unconditional chain.

\paragraph{Main statement (continuum area law with perimeter term).}
\begin{theorem}
Let $\Gamma\subset\mathbb{R}^d$ be a rectifiable closed curve with $\mathsf{Area}(\Gamma)<\infty$. Assume the uniform lattice bound \eqref{eq:lattice-area-law} on the scaling window $(0,\varepsilon_0]$. Define the $\varepsilon$--independent constants
\[
  T\;:=\;\inf_{0<\varepsilon\le\varepsilon_0}\frac{\tau_\varepsilon}{\varepsilon^2}\;>\;0,\qquad
  C_0\;:=\;\sup_{0<\varepsilon\le\varepsilon_0}\frac{\kappa_\varepsilon}{\varepsilon}\;<\;\infty,\qquad
  C\;:=\;\kappa_d\,C_0.
\]
Then for any directed family $\{\Gamma_\varepsilon\to\Gamma\}$,
\begin{equation}
\label{eq:continuum-bound}
  \limsup_{\varepsilon\downarrow 0}\bigl[-\log\langle W(\Gamma_\varepsilon)\rangle\bigr]
  \;\ge\;
  T\,\mathsf{Area}(\Gamma)\;-
  \;C\,\mathsf{Per}(\Gamma).
\end{equation}
In particular, the continuum string tension is positive and bounded below by $T$.
\end{theorem}

\begin{proof}
Starting from \eqref{eq:lattice-area-law} with $\Lambda=\Gamma_\varepsilon$ and taking $\limsup_{\varepsilon\downarrow 0}$, use $\limsup(A_\varepsilon-B_\varepsilon)\ge (\inf A_\varepsilon)-(\sup B_\varepsilon)$ in the form
\[
  \limsup_{\varepsilon\downarrow 0}\bigl[A_\varepsilon-B_\varepsilon\bigr]
  \;\ge\;
  \Big(\inf_{0<\varepsilon\le\varepsilon_0}\tfrac{\tau_\varepsilon}{\varepsilon^2}\Big)\cdot
  \liminf_{\varepsilon\downarrow 0}\mathsf{Area}_\varepsilon(\Gamma_\varepsilon)
  \;-
  \;\Big(\sup_{0<\varepsilon\le\varepsilon_0}\tfrac{\kappa_\varepsilon}{\varepsilon}\Big)\cdot
  \limsup_{\varepsilon\downarrow 0}\mathsf{Per}_\varepsilon(\Gamma_\varepsilon).
\]
Applying Facts A and B yields \eqref{eq:continuum-bound}.
\end{proof}

\paragraph{Remarks.}
1. The constants $T$ and $C$ are $\varepsilon$--independent: $T$ is the uniform lower bound on the lattice string tension in physical units ($\tau_\varepsilon/\varepsilon^2$), while $C$ is the product of the uniform perimeter coefficient in physical units ($C_0=\sup\kappa_\varepsilon/\varepsilon$) with the geometric factor $\kappa_d=\sqrt{d}$. For planar Wilson loops, $C=\sqrt{2}\,C_0$.

2. The "large loop" qualifier is automatic here: for any fixed physical loop $\Gamma$, the lattice representative $\Gamma_\varepsilon$ has diameter of order $\varepsilon^{-1}$ in lattice units, so the hypotheses behind \eqref{eq:lattice-area-law} (from strong--coupling/cluster bounds) apply for all sufficiently small $\varepsilon$.

3. The bound \eqref{eq:continuum-bound} states that the continuum string tension $\sigma_{\text{cont}}:=\liminf_{\varepsilon\downarrow 0}\tau_\varepsilon/\varepsilon^2$ is positive (indeed $\sigma_{\text{cont}}\ge T>0$), with a controlled perimeter subtraction that is uniform along any directed family $\Gamma_\varepsilon\to\Gamma$.

\section{Global continuum construction on $\mathbb R^4$ and OS axioms}\label{sec:global-R4}

This section constructs a single, global family of Schwinger functions on $\mathbb R^4$ from the local limits on fixed physical regions, and verifies OS0--OS5 globally. We then perform OS$\to$Wightman reconstruction and transfer the mass gap to Minkowski space.

\subsection{Directed van Hove exhaustions and cylinder algebras}

Let $\{\Lambda_k\}_{k\in\mathbb N}$ be an increasing van Hove exhaustion of $\mathbb R^4$ by bounded Lipschitz regions (e.g., cubes), so that $\overline{\Lambda_k}\subset\Lambda_{k+1}$, $\bigcup_k \Lambda_k=\mathbb R^4$, and $|\partial\Lambda_k|/|\Lambda_k|\to 0$. For each $k$, let $\mathfrak A_0(\Lambda_k)$ denote the local time-zero OS algebra generated by gauge-invariant observables supported in $\Lambda_k$ (e.g., Wilson loops $W_\Gamma$ with $\Gamma\subset\Lambda_k$ and smeared clover fields supported in $\Lambda_k$). We write $\mathfrak A_0:=\bigcup_k \mathfrak A_0(\Lambda_k)$ for the global algebraic union.

From Sections preceding, for each fixed $\Lambda_k$ we have continuum Schwinger functions $\{S^{(k)}_n\}$ on $\mathfrak A_0(\Lambda_k)$ obtained as van Hove/lattice limits, with OS0--OS2 and clustering (OS3) verified on $\Lambda_k$ uniformly in the approximants; see Proposition~\ref{prop:OS0-poly}, Proposition~\ref{prop:embedding-independence}, Proposition~\ref{prop:bc-robust}, and Theorem~\ref{thm:gap-persist-cont}.

\begin{proposition}[Consistency on overlaps]\label{prop:consistency-overlaps}
If $k<\ell$ and $O_1,\dots,O_n\in\mathfrak A_0(\Lambda_k)$, then
\[
  S^{(k)}_n(O_1,\dots,O_n)\ =\ S^{(\ell)}_n(O_1,\dots,O_n).
\]
Consequently, for any finite family $(O_1,\dots,O_n)$ supported in some $\Lambda_k$, the value
\[
  S_n(O_1,\dots,O_n)\ :=\ S^{(k)}_n(O_1,\dots,O_n)
\]
is well-defined (independent of $k$ large enough).
\end{proposition}
\begin{proof}
By Proposition~\ref{prop:bc-robust}, on any fixed $\Lambda_k$ the local Schwinger functions are independent of boundary conditions in larger van Hove boxes up to $o_{L\to\infty}(1)$ errors, uniformly in the lattice spacing. Proposition~\ref{prop:embedding-independence} removes embedding choices. The AF-free uniqueness criterion (Proposition~\ref{prop:af-free-uniqueness}) identifies limits along any van Hove diagonal. Passing to the continuum within $\Lambda_k$ yields equality of the $k$- and $\ell$-based definitions on $\mathfrak A_0(\Lambda_k)$.
\end{proof}

\subsection{Explicit AF-style scaling $\beta(a)$ and tightness/convergence}\label{subsec:beta-schedule}

For concreteness we record a monotone scaling schedule $\beta(a)$ and prove tightness and convergence of local Schwinger functions along any van Hove net with $a\downarrow 0$ and $L(a) a\to\infty$.

\begin{definition}[AF-style schedule]\label{def:beta-schedule}
Fix $a_0>0$ and constants $b_0>0$, $c_\beta\ge 1$. Define
\[
  \beta(a)\ :=\ c_\beta\,\log\Big(\frac{a_0}{a}\Big)\qquad \text{for } a\in(0,a_0].
\]
This is monotone nondecreasing, $\beta(a)\to\infty$ as $a\downarrow 0$, and stays $\ge 1$ on $(0,a_0]$.
\end{definition}

We do \emph{not} require perturbative AF identities; the role of $\beta(a)$ is solely to pin a concrete trajectory for which our nonperturbative bounds (UEI, equicontinuity, interface minorization) are uniform in $a$.

\begin{theorem}[Tightness and convergence along $\beta(a)$]\label{thm:tightness-beta}
Let $R\Subset\mathbb R^4$ be fixed. Along any van Hove scaling net $(a,L(a))$ with $\beta=\beta(a)$ from Definition~\ref{def:beta-schedule}, the family of time-zero local Schwinger functions $\{S_{n,a,L}\}_{a,L}$ restricted to observables supported in $R$ is tight and precompact in the product topology over loop/cylinder functionals. All subsequential limits coincide, hence $S_{n,a,L}\to S_n$ pointwise on $R$.
\end{theorem}
\begin{proof}
Uniform Exponential Integrability on fixed regions (Theorem~\ref{thm:uei-fixed-region} and Corollary~\ref{cor:uei-explicit-constants}) gives subgaussian Laplace bounds with constants $\eta_R,C_R$ independent of $a$ and $L$. Proposition~\ref{prop:OS0-poly} yields polynomial OS0 bounds uniform in $(a,L)$. The equicontinuity modulus Lemma~\ref{lem:eqc-modulus} applies uniformly on $R$. By Prokhorov/Arzel\`a--Ascoli for cylinder functionals, tightness and precompactness follow. Embedding-independence (Proposition~\ref{prop:embedding-independence}) and the AF-free uniqueness criterion (Proposition~\ref{prop:af-free-uniqueness}) identify all subsequential limits, giving convergence.
\end{proof}

\begin{corollary}[Convergence on $\mathbb R^4$]\label{cor:global-convergence}
Along $\beta(a)$, the global construction of Section~\ref{sec:global-R4} produces the same Schwinger functions as any other admissible monotone schedule satisfying the uniform hypotheses. In particular, the global measure $\mu_{\mathrm{YM}}$ is independent of the schedule within this class.
\end{corollary}

\subsection{AF-free calibrated NRC alternative}\label{subsec:af-free-alternative}

Independently of any schedule, one may work entirely AF-free using calibrated norm--resolvent convergence:

\begin{theorem}[AF-free calibrated NRC and uniqueness]\label{thm:af-free-calibrated}
Fix $R\Subset\mathbb R^4$. Suppose: (i) UEI and OS0 bounds hold uniformly on $R$; (ii) the interface kernel admits a Doeblin split with $t_0,\theta_*>0$ independent of $(a,L)$ on $R$; (iii) the embedded resolvents $R_{a,L}(z_0)=I_{a,L}(H_{a,L}-z_0)^{-1}I_{a,L}^*$ form a Cauchy net in operator norm on $\mathcal H_R$ for some $z_0\in\mathbb C\setminus\mathbb R$. Then $R_{a,L}(z)\to R(z)$ in operator norm for all $z$ in compact subsets of $\mathbb C\setminus\mathbb R$, the semigroups $I_{a,L}e^{-tH_{a,L}}I_{a,L}^*$ converge strongly to $e^{-tH_R}$, and the Schwinger functions on $R$ converge uniquely along any van Hove net. The induced global measure $\mu_{\mathrm{YM}}$ of Section~\ref{sec:global-R4} is recovered without reference to $\beta(a)$.
\end{theorem}
\begin{proof}
Combine the Cauchy criterion in Lemma~\ref{lem:af-free-cauchy} with collective-compactness (Proposition~\ref{prop:collective-compactness}) and the holomorphic functional calculus to extend operator-norm convergence from a point $z_0$ to compact nonreal sets. Strong convergence of semigroups follows from standard Laplace inversion bounds using uniform OS0. Uniqueness on $R$ is Proposition~\ref{prop:af-free-uniqueness}. Consistency on overlaps (Proposition~\ref{prop:consistency-overlaps}) and Kolmogorov extension then reconstruct the global $\mu_{\mathrm{YM}}$.
\end{proof}

\begin{theorem}[Kolmogorov/Minlos extension to a global Euclidean measure]\label{thm:kolmogorov-global}
The consistent family $\{S_n\}$ on the cylinder algebra generated by $\mathfrak A_0$ extends to a unique probability measure $\mu_{\mathrm{YM}}$ on the cylinder $\sigma$--algebra of gauge-invariant observables on $\mathbb R^4$. In particular, $S_n(O_1,\dots,O_n)=\mathbb E_{\mu_{\mathrm{YM}}}[O_1\cdots O_n]$ for all finite families from $\mathfrak A_0$.
\end{theorem}
\begin{proof}
Consistency (Proposition~\ref{prop:consistency-overlaps}) and uniform OS0 polynomial bounds (Proposition~\ref{prop:OS0-poly}) imply tightness and a Daniell--Kolmogorov consistent family on the directed system $\{\Lambda_k\}$. Kolmogorov extension (or Minlos/Prokhorov for the corresponding cylinder space) yields $\mu_{\mathrm{YM}}$ on the projective limit. Uniqueness follows from the uniqueness of finite-dimensional distributions on the cylinder algebra.
\end{proof}

Define the global OS/GNS Hilbert space $\mathcal H_{\mathrm{OS}}$ as the completion of $\mathfrak A_0/\mathcal N$ with inner product $\langle [A],[B]\rangle:=\mathbb E_{\mu_{\mathrm{YM}}}[\Theta(A)B]$, where $\mathcal N:=\{A:\ \mathbb E_{\mu_{\mathrm{YM}}}[\Theta(A)A]=0\}$. Time translations define a contraction semigroup $e^{-tH}$ on $\mathcal H_{\mathrm{OS}}$ with generator $H\ge 0$.

\subsection{Global OS axioms on $\mathbb R^4$}

\begin{theorem}[Global OS0--OS5 (single closure, with NRC/clustering references)]\label{thm:global-OS}
Let $\{\mu_{a,L}\}$ be Wilson lattice measures along a van Hove window. Assume uniform UEI/OS0 on fixed regions (Theorem~\ref{thm:uei-fixed-region}, Corollary~\ref{cor:uei-explicit-constants}) and AF--free NRC on fixed regions (Theorems~\ref{thm:nrc-embeddings}, \ref{thm:nrc-operator-norm}) with the Cauchy/defect/projection inputs (Lemmas~\ref{lem:af-free-cauchy}, \ref{lem:graph-defect-Oa}, \ref{lem:low-energy-proj}). Then the continuum limit $\mu_{\mathrm{YM}}$ exists on cylinder sets and its Schwinger functions $\{S_n\}$ satisfy OS0--OS5 globally on $\mathbb R^4$:
\begin{itemize}
  \item OS0 (temperedness): Uniform polynomial bounds (Proposition~\ref{prop:OS0-poly}) pass to the limit by consistency on overlaps (Proposition~\ref{prop:consistency-overlaps}).
  \item OS2 (reflection positivity): For any polynomial $P$ supported in $t\ge 0$, $\langle\Theta P\,\overline{P}\rangle_{\mu}=\lim_{a,L}\langle\Theta P\,\overline{P}\rangle_{\mu_{a,L}}\ge 0$.
  \item OS3 (clustering): The uniform lattice gap yields exponential clustering on each $\Lambda_k$ (Proposition~\ref{prop:gap-to-cluster}); AF--free NRC and gap persistence (Theorem~\ref{thm:gap-persist-cont}) transport the decay rate to the continuum generator $H$, giving global clustering.
  \item OS4 (permutation symmetry): Symmetry of lattice Schwinger functions is preserved under limits.
  \item OS1 (Euclidean invariance): Translation invariance follows from directed consistency; full rotational invariance is obtained by compact-group averaging (Lemma~\ref{lem:group-avg}), which preserves OS0--OS3 and the gap.
  \item OS5 (unique vacuum): The spectral gap implies a one-dimensional vacuum sector globally.
\end{itemize}
\end{theorem}

\begin{lemma}[Compact-group averaging preserves OS axioms and gap]\label{lem:group-avg}
Let $G$ be a compact group acting by Euclidean isometries on observables, and let $\{S_n\}$ satisfy OS0--OS5 with mass gap $\Delta>0$. Then the averaged family $\{\overline{S}_n\}$ defined by $\overline{S}_n:=\int_G S_n\circ g\,dg$ also satisfies OS0--OS5 with the same gap.
\end{lemma}
\begin{proof}
Temperedness and permutation symmetry are preserved by dominated convergence. Reflection positivity is convex: $\int \langle\Theta(P)P\rangle_g\,dg\ge 0$. Clustering persists since $\int e^{-\Delta t}\,dg=e^{-\Delta t}$. In the OS/GNS picture, the group acts unitarily and commutes with time translations, so the spectral gap of $H$ is unchanged under averaging the vacuum functional.
\end{proof}

\subsection{OS$\to$Wightman and global mass gap}

\begin{theorem}[OS reconstruction and Poincar\'e invariance]\label{thm:os-to-wightman-global}
From $\{S_n\}$ as in Theorem~\ref{thm:global-OS}, the Osterwalder--Schrader reconstruction yields a Wightman QFT on Minkowski space with unitary positive-energy representation of the Poincar\'e group and local gauge-invariant Wightman fields. The Hamiltonian has spectrum $\{0\}\cup[\gamma_*,\infty)$ with $\gamma_*>0$.
\end{theorem}
\noindent See also Corollaries~\ref{cor:microcausality} (microcausality), \ref{cor:wightman-local-gap} (Wightman local fields and gap), and \ref{cor:minkowski-massgap} (physical Minkowski mass gap).
\begin{proof}
Apply the classical OS reconstruction to $\mu_{\mathrm{YM}}$ using reflection positivity, temperedness, symmetry, and Euclidean invariance. Exponential clustering and OS5 give a unique vacuum and spectrum condition. The uniform slab contraction/gap (Theorem~\ref{thm:gap-persist-cont}) transfers to the global generator by the core/inductive-limit argument in the proof of Theorem~\ref{thm:global-OS}. The resulting Wightman theory inherits Poincar\'e covariance from Euclidean invariance by analytic continuation.
\end{proof}

\begin{theorem}[Wightman axioms and spectral condition]\label{thm:wightman-axioms}
Let $\mu_{\mathrm{YM}}$ be the global Euclidean measure of Theorem~\ref{thm:kolmogorov-global} with Schwinger functions satisfying Theorem~\ref{thm:global-OS}. Then the OS reconstruction produces Wightman distributions $\{W_n\}$ and a separable Hilbert space $\mathcal H$ such that:
\begin{itemize}
  \item (W0) temperedness: $W_n\in \mathcal S'(\mathbb R^{4n})$;
  \item (W1) Poincar\'e covariance: there is a unitary representation $U$ of the proper orthochronous Poincar\'e group with $U(a,\Lambda)\,\Phi(x)\,U(a,\Lambda)^{-1}=\Phi(\Lambda x+a)$ on fields;
  \item (W2) spectrum condition: the joint spectrum of the energy-momentum operators lies in the closed forward cone $\overline{V}_+$; in particular the Hamiltonian has spectrum $\{0\}\cup[\gamma_*,\infty)$;
  \item (W3) locality: smeared local gauge-invariant fields commute at spacelike separation;
  \item (W4) vacuum: there is a unique (up to phase) Poincar\'e-invariant vacuum $\Omega$ cyclic for the field algebra.
\end{itemize}
\end{theorem}
\begin{proof}
OS0 implies temperedness of Schwinger functions; analytic continuation yields tempered Wightman distributions. OS1 with Lemma~\ref{lem:group-avg} provides full Euclidean invariance and hence Poincar\'e covariance after continuation. OS2 gives a positive-definite inner product leading to the GNS construction. OS3 and OS5 imply uniqueness of the vacuum and exponential clustering, which yields the spectral condition together with the nonzero mass gap from Theorem~\ref{thm:os-to-wightman-global}. Locality follows from the standard OS$\to$Wightman locality theorem applied to local gauge-invariant smeared fields (Corollary~\ref{cor:os-local-fields}).
\end{proof}

\subsection{Global spectral gap on $\mathbb R^4$ (unconditional)}

\begin{theorem}[Global Euclidean spectral gap, boundary/region independent]\label{thm:global-gap-uncond}
Let $G=SU(N)$, $N\ge 2$. With the global OS construction of Section~\ref{sec:global-R4}, there exists $\gamma_*>0$ (depending only on $(R_*,a_0,N)$ via $(\theta_*,t_0,\lambda_1)$) such that the Euclidean generator $H$ on the global OS/GNS Hilbert space satisfies
\[
  \operatorname{spec}(H)\ \subset\ \{0\}\cup[\gamma_*,\infty)\,.
\]
Equivalently, for all $t\ge 0$,
\[
  \big\|e^{-tH}\,(I-\vert\Omega\rangle\langle\Omega\vert)\big\|\ \le\ e^{-\gamma_*\, t}\,.
\]
The bound is independent of the exhaustion, region choice, and boundary conditions.
\end{theorem}
\begin{proof}
Step 1 (uniform local contraction). On finite tori and fixed slabs, the interface Doeblin split with constants $(\theta_*,t_0)$ (Proposition~\ref{prop:explicit-doeblin-constants}) and the two-layer deficit (Theorem~\ref{thm:two-layer-explicit}) imply the one-tick odd-cone contraction (Theorem~\ref{thm:uniform-odd-contraction}). By the odd\,$\to$\,mean-zero upgrade (Corollary~\ref{cor:odd-to-meanzero}) and parity cycling, eight ticks yield a mean-zero contraction with rate $\gamma_*=8\,c_{\rm cut,phys}>0$, uniform in $(\beta,L)$.

Step 2 (thermodynamic limit at fixed spacing). The contraction estimate is volume-uniform; the thermodynamic limit preserves the gap and clustering (Theorem~\ref{thm:thermo}). Boundary-condition robustness (Proposition~\ref{prop:bc-robust}) ensures independence of outer boundary choices for local observables.

Step 3 (continuum limit on fixed regions). On any fixed $R\Subset\mathbb R^4$, UEI and OS0 yield tightness and equicontinuity; the AF-free calibrated NRC (Theorem~\ref{thm:af-free-calibrated}) provides operator-norm resolvent convergence and uniqueness of local limits along van Hove nets, independent of any $\beta(a)$ schedule. Gap persistence (Theorem~\ref{thm:gap-persist-cont}) transports the uniform lower bound $\gamma_*$ from local lattices to the continuum generator $H_R$ on $R$.

Step 4 (globalization). Consistency on overlaps (Proposition~\ref{prop:consistency-overlaps}) identifies the local semigroups on the directed system $\{\Lambda_k\}$; the global OS/GNS space is the inductive-limit completion. Since the mean-zero contraction is uniform in $k$, the semigroup bound
\[
  \big\|e^{-tH_{\Lambda_k}}\,(I-\vert\Omega_{\Lambda_k}\rangle\langle\Omega_{\Lambda_k}\vert)\big\|\ \le\ e^{-\gamma_* t}
\]
passes to the limit by density of $\bigcup_k \mathcal H_{\mathrm{OS}}(\Lambda_k)$ and strong continuity, yielding the stated global bound. The spectral inclusion follows from the spectral-mapping theorem for contraction semigroups: a uniform estimate on the orthogonal complement of the vacuum implies $\sigma(H)\cap (0,\gamma_*)=\varnothing$.

All constants depend only on the slab geometry $(R_*,a_0)$ and group data; no assumption on an AF/Mosco trajectory or isotropy restoration is used.
\end{proof}

\begin{corollary}[Physical Minkowski mass gap]\label{cor:minkowski-massgap}
Under OS reconstruction (Theorem~\ref{thm:os-to-wightman-global}), the Wightman Hamiltonian on Minkowski space has the same strictly positive mass gap $\gamma_*>0$:
\[
  \operatorname{spec}(H_{\mathrm{Mink}})\ \subset\ \{0\}\cup[\gamma_*,\infty)\,.
\]
In particular, the spectral condition holds and the mass gap is independent of region/boundary choices used in the Euclidean construction.
\end{corollary}
\begin{proof}
Theorem~\ref{thm:global-gap-uncond} gives the Euclidean gap; Theorem~\ref{thm:os-to-wightman-global} transfers it to the Minkowski theory by analytic continuation and OS$\to$Wightman. Uniqueness of the vacuum (OS5) yields the ground state at energy $0$.
\end{proof}

\subsection*{Optional B: Continuum OS reconstruction from a scaling window}

This option outlines a rigorous procedure for constructing a continuum QFT in four dimensions from a family of lattice gauge theories, given tightness and uniform locality/clustering bounds independent of $\varepsilon$.

\paragraph{Existence of the continuum limit measure.}
Assuming tightness of loop observables $W_{\Gamma,\varepsilon}$, Prokhorov compactness yields a subsequence $\varepsilon_k\to 0$ along which the lattice measures converge weakly to a probability measure $\mu$. For any finite collection of loops $\Gamma_1,\dots,\Gamma_n$, the Schwinger functions
\[
  S_n(\Gamma_1,\dots,\Gamma_n):=\lim_{\varepsilon\to 0}\,\langle W_{\Gamma_1,\varepsilon}\cdots W_{\Gamma_n,\varepsilon}\rangle
\]
exist under the uniform locality/clustering bounds, and characterize $\mu$.
Under the NRC hypotheses below, the embedded resolvents are Cauchy in operator norm on any nonreal compact, implying \emph{unique} Schwinger limits as $\varepsilon\downarrow 0$ without passing to subsequences (Proposition~\ref{prop:af-free-uniqueness}).
\paragraph{Verification of the OS axioms.}
\emph{Remark.} The OS axioms are stable under controlled limits: positivity inequalities persist, polynomial bounds transfer via uniform constants, and clustering/gap properties are preserved by spectral convergence.

\begin{lemma}[OS0--OS5 in the continuum limit]\label{lem:os-continuum}
Let $\mu$ be a weak limit of lattice measures $\mu_\varepsilon$ along a scaling sequence. Assume:
\begin{itemize}
  \item[(i)] Uniform locality: $|S_{n,\varepsilon}(\Gamma_1,\ldots,\Gamma_n)| \le C_n \prod_i (1+\text{diam}\,\Gamma_i)^p \prod_{i<j} (1+\text{dist}(\Gamma_i,\Gamma_j))^{-q}$ with constants $C_n$ independent of $\varepsilon$.
  \item[(ii)] Uniform clustering: $|\langle O_\varepsilon(t) O_\varepsilon(0) \rangle_c| \le C e^{-m t}$ for mean-zero local observables.
  \item[(iii)] Equivariant embeddings preserving the reflection structure.
\end{itemize}
Then the limit measure $\mu$ satisfies:
\begin{itemize}
  \item \textbf{OS0 (temperedness):} $|S_n(\Gamma_1,\ldots,\Gamma_n)| \le C_n \prod_i (1+\text{diam}\,\Gamma_i)^p \prod_{i<j} (1+\text{dist}(\Gamma_i,\Gamma_j))^{-q}$ by direct passage to the limit using (i).
  \item \textbf{OS1 (Euclidean invariance):} Continuous rotations/translations act on $S_n$ by the limiting equivariance of discrete symmetries under (iii).
  \item \textbf{OS2 (reflection positivity):} For any polynomial $P$ in loop observables supported at $t \ge 0$,
  \[
    \langle \Theta(P) P \rangle_\mu = \lim_{\varepsilon \to 0} \langle \Theta(P_\varepsilon) P_\varepsilon \rangle_{\mu_\varepsilon} \ge 0,
  \]
  since positivity is preserved under weak limits.
  \item \textbf{OS3 (clustering):} Exponential decay $|\langle O(t) O(0) \rangle_c| \le C e^{-mt}$ follows from (ii) and weak convergence.
  \item \textbf{OS4/OS5 (symmetry/vacuum):} Gauge invariance and vacuum uniqueness follow from uniform gap persistence (Theorem~\ref{thm:gap-persist}).
\end{itemize}
\end{lemma}
\begin{proof}
OS0 follows from Proposition~\ref{prop:OS0-poly} applied uniformly. OS1 uses equicontinuity: discrete rotations converge to continuous ones under directed embeddings. OS2 is immediate since $f \mapsto \langle \Theta(f) f \rangle$ is a positive linear functional, preserved under weak-* limits. OS3 transfers the uniform bound (ii) to all cylinder functionals by density. OS4/OS5 follow from the gap persistence theorem ensuring a unique ground state.
\end{proof}

\begin{corollary}[Finite continuum gap via scaled minorization]\label{cor:scaled-continuum-gap}
Let $c(\varepsilon)>0$ be as in Theorem~\ref{thm:two-layer-explicit}. Under Assumption~\ref{assump:AF-Mosco}, along any van Hove scaling sequence, the continuum generator $H$ obtained by Mosco/strong-resolvent convergence satisfies
\[
  \operatorname{spec}(H)\subset\{0\}\cup[c,\infty),\qquad c>0.
\]
In particular, the physical mass gap $m_*$ is finite and bounded below by $c$, with $c$ depending only on $(R_*,a_0,N)$ and the group data via $\lambda_1(N)$.
\end{corollary}
\medskip
\begin{lemma}[Equicontinuity modulus on fixed regions]\label{lem:eqc-modulus}
Fix a bounded region $R\subset\mathbb R^4$, $q>4$, $p=5$, and constants $(C_0,m)$ as in Proposition~\ref{prop:OS0-poly}. There exists $C_{\rm eq}(R,q,C_0,m)>0$ such that for any $n\ge 1$, loop families $\{\Gamma_i\}_{i=1}^n$ and $\{\Gamma'_i\}_{i=1}^n$ contained in $R$ with $\max_i d_H(\Gamma_i,\Gamma'_i)\le \delta\in(0,1]$,
\[
  \big|\,S_{n,a,L}(\Gamma_1,\dots,\Gamma_n) - S_{n,a,L}(\Gamma'_1,\dots,\Gamma'_n)\,\big|
  \ \le\ C_{\rm eq}\,\delta^{\,q-4}\,\prod_{i=1}^n \bigl(1+\operatorname{diam}\Gamma_i\bigr)^p,
\]
uniformly in $(a,L)$.
\end{lemma}
\noindent\emph{Remark (uniformity).} The modulus $\omega_R(\delta)=C_{\rm eq}\,\delta^{\,q-4}$ is uniform in $(a,L)$ and depends only on $(R,q,C_0,m)$ from OS0; it is independent of the bare coupling and volume.
\begin{proof}[Proof (detailed)]
Fix $R\Subset\mathbb R^4$, $q>4$, $p=5$, and let the OS0 polynomial bound of Proposition~\ref{prop:OS0-poly} hold uniformly with constants $C_n(C_0,m,q)$. Let $\{\Gamma_i\}_{i=1}^n$ and $\{\Gamma'_i\}_{i=1}^n$ be loop families in $R$ with $\max_i d_H(\Gamma_i,\Gamma'_i)\le \delta\in(0,1]$. For each $i$, choose a polygonal approximation of $\Gamma_i$ and $\Gamma_i'$ with mesh $\le c\delta$ and same combinatorics inside $R$; the OS0 bound applies uniformly to such local polygonal loops with the same constants.
Write the difference $S_{n,a,L}(\Gamma_1,\dots,\Gamma_n)-S_{n,a,L}(\Gamma'_1,\dots,\Gamma'_n)$ as a telescoping sum over the $n$ slots, changing one loop at a time while keeping the others fixed:
\[
  S_{n,a,L}(\Gamma_1,\dots,\Gamma_n)-S_{n,a,L}(\Gamma'_1,\dots,\Gamma'_n)
   =\sum_{k=1}^n \big( S_{n,a,L}(\Gamma_1',\dots,\Gamma_{k-1}',\Gamma_k,\Gamma_{k+1},\dots,\Gamma_n)
   - S_{n,a,L}(\Gamma_1',\dots,\Gamma_{k}',\Gamma_{k+1},\dots,\Gamma_n)\big).
\]
It suffices to bound a one-slot variation. By OS0, for any fixed positions of the other loops,
\[
  \big|\Delta_k\big|\ \le\ C_n\,\big(1+\operatorname{diam}\Gamma_k\big)^p\,\prod_{i\ne k}\big(1+\operatorname{diam}\Gamma_i\big)^p\,\prod_{i\ne k}\big(1+\operatorname{dist}(\Gamma_k,\Gamma_i)\big)^{-q}
   \cdot \mathrm{Var}_k(\Gamma_k,\Gamma_k'),
\]
where $\mathrm{Var}_k$ denotes the sensitivity with respect to moving loop $k$ to $\Gamma_k'$. By the polygonal approximation and $d_H(\Gamma_k,\Gamma_k')\le \delta$, one can partition $\Gamma_k$ and $\Gamma_k'$ into $O(\delta^{-1})$ matching segments of diameter $\le c\delta$ in $R$. Varying a single small segment perturbs $\operatorname{dist}(\Gamma_k,\Gamma_i)$ by at most $O(\delta)$ and the factor $(1+\operatorname{dist})^{-q}$ changes by at most $C\,\delta\,(1+\operatorname{dist})^{-(q+1)}$. Summing over segments and over $i\ne k$, and using $\sum_{x\in \mathbb Z^4}(1+\|x\|)^{-(q+1)}<\infty$ for $q>4$, yields
\[
  \mathrm{Var}_k(\Gamma_k,\Gamma_k')\ \le\ C(R,q)\,\delta^{\,q-4}.
\]
Collecting the diameter factors into $\prod_i (1+\operatorname{diam}\Gamma_i)^p$ and summing the $n$ telescoping terms gives the required bound with
\[
  C_{\rm eq}\ =\ C_n(C_0,m,q)\,C(R,q)\,n\,\max_{\text{families}}\prod_{i=1}^n (1+\operatorname{diam}\Gamma_i)^p,
\]
which is finite for loops contained in the fixed region $R$. This establishes the modulus $\omega_R(\delta)=C_{\rm eq}\,\delta^{\,q-4}$ uniformly in $(a,L)$.
\end{proof}
\begin{proposition}[AF-free uniqueness of Schwinger limits]\label{prop:af-free-uniqueness}
Fix a bounded region $R\Subset\mathbb R^4$. Assume: (i) the OS0 polynomial bounds on loop $n$-point functions hold uniformly in $(a,L)$ on $R$; (ii) equicontinuity holds as in Lemma~\ref{lem:eqc-modulus}; (iii) embedding–independence holds as in Proposition~\ref{prop:embedding-independence}; and (iv) for some nonreal $z_0$, the embedded resolvents $R_{a,L}(z_0):=I_{a,L}(H_{a,L}-z_0)^{-1}I_{a,L}^*$ form a Cauchy net in operator norm on the time-zero OS space generated by loops supported in $R$. Then the Schwinger functions $S_{n,a,L}$ converge uniquely as $(a,L)$ follow any van Hove diagonal, without invoking an AF schedule.
\begin{proof}
By (iv), $R_{a,L}(z_0)$ converge in operator norm to a bounded operator $R(z_0)$ on the limit space. The Laplace–resolvent representation expresses $n$-point functions of loop observables as finite sums of matrix elements of $R_{a,L}(z)$ at finitely many nonreal $z$'s with coefficients controlled by OS0. The resolvent identity and compactness of nonreal strips transfer the Cauchy property from $z_0$ to all $z$ in a fixed compact subset of $\mathbb C\setminus\mathbb R$, uniformly on $R$'s local cone. Dominated convergence (using OS0) passes limits under the Laplace integral, yielding convergence of the Schwinger functions along any van Hove diagonal. By (ii) and (iii), changing embeddings changes the approximants by $o(1)$, so the limit is independent of the embedding choice. Uniqueness across subsequences follows from operator-norm convergence of resolvents and the Riesz projection stability.
\end{proof}
\end{proposition}

\begin{proposition}[Embedding–independence of continuum Schwinger functions]\label{prop:embedding-independence}
Fix a bounded region $R\in SO(4)$ and $n\ge 1$. Let $\{I_\varepsilon\}$ and $\{J_\varepsilon\}$ be two admissible directed voxel embeddings for loops in $R$, chosen equivariantly under the hypercubic symmetries and preserving the OS reflection setup. For any loop family $\{\Gamma_i\}_{i=1}^n\subset R$,
\[
  \lim_{\varepsilon\to 0}\ \Big|\, S_{n,\varepsilon}^{(I)}(\Gamma_1,\dots,\Gamma_n)\,-\,S_{n,\varepsilon}^{(J)}(\Gamma_1,\dots,\Gamma_n)\,\Big|\ =\ 0.
\]
In particular, the continuum Schwinger limits $\{S_n\}$ (when they exist) are independent of the admissible embedding choice.
\end{proposition}
\begin{proof}
Directedness and equivariance give $d_H(I_\varepsilon(\Gamma_i),J_\varepsilon(\Gamma_i))\le C(R)\,\varepsilon$. Apply Lemma~\ref{lem:eqc-modulus} to control the difference uniformly; sum over $i$ and let $\varepsilon\to 0$.
\end{proof}

\begin{proposition}[Boundary–condition robustness on van Hove boxes]\label{prop:bc-robust}
Let $R\Subset\mathbb R^4$ be fixed. For any two boundary conditions on the complement of $R$ within a van Hove box, the time-zero local Schwinger functions in $R$ differ by at most $o_{L\to\infty}(1)$ uniformly in $a\in(0,a_0]$. Consequently, continuum limits on $R$ are independent of the boundary condition within the van Hove class.
\end{proposition}
\begin{proof}
Use the interface contraction and locality to show exponential decay of boundary influences in $L$; combine with UEI to pass uniform bounds to the limit.
\end{proof}

\begin{lemma}[Isotropy restoration via heat-kernel calibrators]\label{lem:isotropy-restore}
Let $P_{t_0}$ be the product heat kernel on $\mathrm{SU}(N)$ from Proposition~\ref{prop:explicit-doeblin-constants}. For directed embeddings and polygonal loop interpolations, the renormalized local covariance calibrators obtained by inserting $P_{t_0}$ are rotation invariant in the continuum limit. Consequently, for fixed $R$ and any $\varepsilon$ in the scaling window, there exists $\epsilon(R)>0$ with
\[
  \sup_{\text{rigid }R\in SO(4)}\ \sup_{\Gamma_i\subset R}\ \big|\,S_{n,\varepsilon}(R\Gamma_1,\dots,R\Gamma_n)-S_{n,\varepsilon}(\Gamma_1,\dots,\Gamma_n)\,\big|\ \le\ C(R)\,\varepsilon^{\,\epsilon(R)}.
\]
\end{lemma}

\begin{lemma}[OS1 without calibrators: embedding–independence route]\label{lem:os1-embedding}
Fix $R\in SO(4)$. For each $\varepsilon$, let $I^{(R)}_\varepsilon$ be a rotated voxel embedding obtained by precomposing the directed embedding $I_\varepsilon$ with $R$ and projecting to the $\varepsilon$–lattice equivariantly within the hypercubic symmetry (preserving the OS reflection setup). For any finite loop family $\{\Gamma_i\}_{i=1}^n$ in a fixed region,
\[
  S_{n,\varepsilon}^{(I^{(R)})}\big(R\Gamma_1,\dots,R\Gamma_n\big)
  \
  =\ S_{n,\varepsilon}^{(I)}\big(\Gamma_1,\dots,\Gamma_n\big).
\]
If continuum limits along the scaling window are unique and independent of the admissible embedding choice, then $S_n(R\Gamma_1,\dots,R\Gamma_n)=S_n(\Gamma_1,\dots,\Gamma_n)$, i.e., OS1 holds without calibrators.

\begin{proof}
At fixed $\varepsilon$, the Wilson action and OS reflection structure are invariant under the hypercubic group. The rotated embedding $I^{(R)}_\varepsilon$ is obtained by conjugating $I_\varepsilon$ with the rigid rotation $R$ and discretizing equivariantly, so the lattice integral defining $S_{n,\varepsilon}$ is preserved by the change of variables induced by $R$ together with the hypercubic symmetry. This gives the displayed identity at each $\varepsilon$. By the embedding–independence of limits (Appendix C1c–C1d), admissible embeddings along the scaling window lead to the same continuum limits. Passing to the limit yields $SO(4)$ invariance of $\{S_n\}$.
\end{proof}
\end{lemma}

\begin{corollary}[OS1 (rotations) in the continuum limit]\label{cor:os1-rotations}
Under the hypotheses of Theorem~\ref{thm:os1-euclid}, together with Lemma~\ref{lem:eqc-modulus} and either Lemma~\ref{lem:isotropy-restore} or Lemma~\ref{lem:os1-embedding}, the limit Schwinger functions are invariant under $SO(4)$ rotations: $S_n(R\Gamma_1,\dots,R\Gamma_n)=S_n(\Gamma_1,\dots,\Gamma_n)$ for all rigid $R$.
\end{corollary}

\begin{proof}
Approximate a fixed $R\in SO(4)$ by hypercubic rotations $R_k$. Discrete invariance gives equality for $R_k$. Lemma~\ref{lem:isotropy-restore} reduces $R_k\to R$ defects to $o(1)$, and Lemma~\ref{lem:eqc-modulus} controls the embedding perturbations uniformly; pass to the limit.
\end{proof}

\paragraph{Hamiltonian reconstruction.}
By the OS reconstruction theorem, the positive-time semigroup is a contraction semigroup $P(t)$ with $\lVert P(t)\rVert\le 1$. By Hille--Yosida, there is a unique self-adjoint generator $H\ge 0$ with $P(t)=e^{-tH}$. Clustering implies a unique vacuum $\Omega$ with $H\Omega=0$.

\subsection*{Consolidated continuum existence (C1)}

We bundle the results of Appendices C1a--C1c into a single statement.

\begin{theorem}
Fix a scaling window $\varepsilon\in(0,\varepsilon_0]$ and consider lattice Wilson measures $\mu_\varepsilon$ with a fixed link-reflection. Assume:
\begin{itemize}
  \item (Uniform locality/moments) The loop observables satisfy $\varepsilon$-uniform locality/clustering and moment bounds, and the reflection setup is fixed (C1a).
  \item (Discrete invariance) $\mu_\varepsilon$ is invariant under the hypercubic group; directed embeddings of loops are chosen equivariantly (C1a).
  \item (Embeddings and consistency) There exist voxel embeddings $I_\varepsilon$ with graph-norm defect control and a compact calibrator for the limit generator (C1c).
\end{itemize}
Then, under the AF/Mosco hypotheses and equicontinuity, the loop $n$-point functions converge \emph{uniquely} (no subsequences) to Schwinger functions $\{S_n\}$ which satisfy OS0--OS5 (regularity/temperedness, Euclidean invariance, reflection positivity, clustering, and unique vacuum). By OS reconstruction, there exists a Hilbert space $\mathcal H$, a vacuum $\Omega$, and a positive self-adjoint Hamiltonian $H\ge 0$ generating Euclidean time.
Moreover, if the lattice transfer operators have an $\varepsilon$-uniform spectral gap on the mean-zero sector, $r_0(T_\varepsilon)\le e^{-\gamma_0}$ with $\gamma_0>0$, then $\operatorname{spec}(H)\subset\{0\}\cup[\gamma_0,\infty)$ and the continuum theory has a mass gap $\ge \gamma_0$.
\end{theorem}

\begin{proof}
Tightness and convergence follow from the uniform locality hypotheses. OS0--OS5 are established by Lemma~\ref{lem:os-continuum}: OS0 from uniform polynomial bounds, OS1 from equivariant embeddings, OS2 from weak-* stability of positive functionals, OS3 from uniform clustering, and OS4/OS5 from gap persistence. Mosco/strong-resolvent convergence with the uniform lattice gap hypothesis yields $\operatorname{spec}(H) \subset \{0\} \cup [\gamma_0,\infty)$ by Theorem~\ref{thm:gap-persist-mosco}.
\end{proof}

\subsection*{Preview: Main Theorem (unconditional AF--free NRC; see Section \ref{sec:main-theorem-unconditional})}

\noindent For the definitive, labeled statement and proof, see Section~\ref{sec:main-theorem-unconditional}, Theorem~\ref{thm:main-unconditional-af-free}.

\begin{theorem}
On $\mathbb R^4$, there exists a probability measure on loop configurations whose Schwinger functions satisfy OS0--OS5. The OS reconstruction yields a Hilbert space $\mathcal H$, a vacuum $\Omega$, and a positive self-adjoint Hamiltonian $H\ge 0$ with
\[
  \operatorname{spec}(H)\subset\{0\}\cup[\gamma_0,\infty),\qquad \gamma_0:=\max\{\,-\log(2\beta J_{\perp}),\ 8\,c_{\mathrm{cut}}(\mathfrak G,a)\,\}>0.
\]
Here $c_{\mathrm{cut}}(\mathfrak G,a):=-(1/a)\log(1-\theta_\star e^{-\lambda_1 t_0})$ is the slab-local odd-cone contraction rate obtained from a $\beta$-independent interface Doeblin minorization and heat-kernel domination on $\mathrm{SU}(N)$; it depends only on $(R_*,a_0,N)$ and not on the volume or bare coupling (see Proposition~\ref{prop:explicit-doeblin-constants}). By the AF–free NRC chain (Theorems~\ref{thm:strong-semigroup-core}, \ref{thm:nrc-operator-norm}, \ref{thm:nrc-embeddings}, Lemma~\ref{lem:af-free-cauchy}), the same lower bound $\gamma_0$ persists to the continuum generator $H$; and the OS$\to$Wightman export is Theorem~\ref{thm:os-to-wightman}. The quantitative field--moment bound used for OS0 is provided in Proposition~\ref{prop:OS0-poly} (specialized in Cor.~\ref{cor:os0-explicit-4d}).

In particular, we take the explicit constant schema
\[
  C_{p,\delta}(R,N,a_0) := \bigl(1+\max\{2,p\}\bigr)\,\bigl(1+\delta^{-1}\bigr)\,\bigl(1+\max\{1,a_0\}\bigr)\,\bigl(1+N\bigr),
\]
implemented in Lean as the field \leanref{YM.OSPositivity.MomentBoundsCloverQuantIneq.C} of the container \leanref{YM.OSPositivity.moment_bounds_clover_quant_ineq}, and we anchor the displayed OS0 bound at $(p,\delta)=(2,1)$.
\end{theorem}

\paragraph{Continuum tail under AF/Mosco (\,$\beta$-independent cut contraction\,).}
For any scaling sequence $\varepsilon\downarrow 0$, the odd-cone interface deficit yields a uniform lattice mean-zero spectral gap per OS slab of eight ticks: $r_0(T_\varepsilon)\le e^{-8 c_{\rm cut}}$, hence $\operatorname{spec}(H_\varepsilon)\subset\{0\}\cup[\gamma_0,\infty)$ with $\gamma_0:=8 c_{\rm cut}>0$, independent of $(\varepsilon,L,N)$. By Mosco/strong-resolvent convergence and gap persistence (Thm.~\ref{thm:gap-persist-mosco}), $(0,\gamma_0)$ remains spectrum-free in the limit, so
\[
  \operatorname{spec}(H)\subset\{0\}\cup[\gamma_0,\infty),\qquad \gamma_{\mathrm{phys}}\ge \gamma_0.
\]

\noindent\emph{Remark ($\beta$-independence of $\gamma_0$).} The key point is that $c_{\rm cut} = -(1/a)\log(1-\theta_\star e^{-\lambda_1(N) t_0})$ depends only on $(R_*,a_0,N)$ through the Doeblin minorization constant $\theta_* = \kappa_0$ (from the $\beta$-uniform refresh probability $\alpha_{\rm ref}$ in Lemma~\ref{lem:refresh-prob}), the heat-kernel parameters $(t_0,\lambda_1(N))$, and the geometric constants. Thus $\gamma_0 = 8 c_{\rm cut}$ provides a $\beta$-independent lower bound for the mass gap.

\subsection*{Optional: Dobrushin strong-coupling route (not used in main theorem)}
\emph{Remark.} The main unconditional proof uses the $\beta$-independent odd-cone Doeblin contraction. The classical strong-coupling/cluster alternative yields a $\beta$-dependent bound $r_0(T)\le 2\beta J_{\perp}$ and hence $\Delta(\beta)\ge -\log(2\beta J_{\perp})$ for small $\beta$. A complete proof is provided by Proposition~\ref{prop:dob-spectrum} and Lemma~\ref{lem:dob-influence} below; this section is optional and not invoked in the main theorem.

\section{Infinite volume at fixed spacing}

\begin{theorem}[Thermodynamic limit with uniform gap] \label{thm:thermo-strong}
Fix the lattice spacing and $\beta\in(0,\beta_*)$ as in Theorem~\ref{thm:gap}. Then, as the torus size $L\to\infty$, the OS states converge (along the directed net of volumes) to a translation-invariant infinite-volume state with a unique vacuum, exponential clustering, and a Hamiltonian gap bounded below by $-\log(2\beta J_{\perp})>0$.
\end{theorem}

\begin{proof}
All Dobrushin/cluster bounds and the OS Gram-positivity estimates are local and uniform in the volume. Hence the contraction coefficient bound $r_0(T_L)\le \alpha(\beta)<1$ holds with a constant independent of $L$. Standard compactness of local observables under the product Haar topology yields existence of a thermodynamic limit state. The uniform spectral contraction on $\mathcal H_{0,L}$ implies exponential decay of correlations and uniqueness of the vacuum in the limit, with the same lower bound on the gap. See Montvay--M\"unster~\cite{MontvayMunster1994} for the thermodynamic passage under strong-coupling/cluster conditions.
\end{proof}

\section{Appendix: Parity--Oddness and One--Step Contraction (TP)}

\paragraph{Setup.}
Fix three commuting spatial reflections $P_x,P_y,P_z$ acting by lattice involutions on the time--zero gauge--invariant algebra $\mathfrak{A}_0^{\rm loc}$. They induce unitary involutions on the OS Hilbert space $\mathcal{H}_{L,a}$, commute with $H_{L,a}$, and leave the vacuum $\Omega$ invariant. For $i\in\{x,y,z\}$ write $\alpha_i(O):=P_i O P_i$ and define $O^{(\pm,i)}:=\tfrac12(O\pm\alpha_i(O))$. Let $\mathcal{C}_{R_*}:=\{O\Omega:\ O\in\mathfrak{A}_0^{\rm loc},\ \langle O\rangle=0,\ \mathrm{supp}(O)\subset B_{R_*}\}$ be the local cone.

\begin{lemma}[Parity--oddness on the local cone]\label{lem:oddness-tp}
For any nonzero $\psi=O\Omega\in\mathcal{C}_{R_*}$ there exists $i\in\{x,y,z\}$ such that $O^{(-,i)}\neq 0$, hence $P_i\psi^{(-,i)}=-\psi^{(-,i)}$ with $\psi^{(-,i)}:=O^{(-,i)}\Omega\neq 0$.
\end{lemma}

\begin{proof}
Let $\mathcal{G}:=\langle P_x,P_y,P_z\rangle\simeq Z_2^3$. Each $P\in\mathcal{G}$ acts by a *-automorphism $\alpha_P$ on $\mathfrak{A}_0^{\rm loc}$ and is implemented by a unitary $U(P)$ on the OS Hilbert space $\mathcal{H}_{L,a}$ via $U(P)[F]=[\alpha_P(F)]$; moreover $U(P)\Omega=\Omega$ and $U(P)$ commutes with the transfer/semigroup by symmetry.

Assume for contradiction that $O^{(-,i)}=0$ for all $i\in\{x,y,z\}$. Then $\alpha_{P_i}(O)=O$ for each generator, hence $\alpha_P(O)=O$ for all $P\in\mathcal{G}$. Consequently $U(P)\,[O]=[O]$ for all $P\in\mathcal{G}$, so the vector $[O]$ lies in the fixed subspace of the unitary representation $U$ of $\mathcal{G}$ on $\mathcal{H}_{L,a}$.

By Theorem~\ref{thm:os} (OS positivity and GNS construction), the constants sector in $\mathcal{H}_{L,a}$ is one-dimensional, spanned by $\Omega$. Since $\mathcal{G}$ is a subgroup of the spatial symmetry group, its fixed subspace is contained in the constants sector; therefore $[O]=c\,\Omega$ for some $c\in\mathbb{C}$. Taking vacuum expectation gives $c=\langle\Omega,[O]\,\Omega\rangle=\langle O\rangle$. Because $\psi=O\Omega\in\mathcal{C}_{R_*}$ has $\langle O\rangle=0$ by definition, we have $c=0$, hence $[O]=0$ and $\psi=0$ in $\mathcal{H}_{L,a}$.

This contradicts the hypothesis that $\psi\ne 0$. Therefore our assumption was false and there must exist at least one $i\in\{x,y,z\}$ with $O^{(-,i)}\ne 0$. In particular $\psi^{(-,i)}:=O^{(-,i)}\Omega\ne 0$ and $P_i\psi^{(-,i)}=-\psi^{(-,i)}$.
\end{proof}

\begin{lemma}[One--step contraction on odd cone]\label{lem:odd-contraction-tp}
Define the slab--local reflection deficit
\[
  \beta_{\mathrm{cut}}(R_*,a)
  \,:=\,
  1\;-
  \sup_{\substack{\psi\in\mathcal H_{L,a},\ \psi\ne 0\\ P_i\psi=-\psi,\ \mathrm{supp}\,\psi\subset B_{R_*}}}
  \frac{\big|\langle\psi, e^{-aH_{L,a}}\psi\rangle\big|}{\langle\psi,\psi\rangle}\,.
\]
Then there exists $\beta_0>0$, depending only on the fixed physical slab $R_*$ (not on $L$) and on $a\in(0,a_0]$, such that $\beta_{\mathrm{cut}}(R_*,a)\ge \beta_0$. Consequently, for any $i\in\{x,y,z\}$ and $\psi\in\mathcal{H}_{L,a}$ with $P_i\psi=-\psi$,
\[
  \|e^{-aH_{L,a}}\psi\|\ \le\ (1-\beta_0)^{1/2}\,\|\psi\|\ \le\ e^{-a c_{\mathrm{cut}}}\,\|\psi\|,
  \qquad c_{\mathrm{cut}}\ :=\ -\frac{1}{a}\log(1-\beta_0)\,.
\]
\end{lemma}
\begin{proof}
OS positivity implies that the $2\times 2$ Gram matrix for $\{\psi, e^{-aH}\psi\}$ is PSD. Let $a_0=\|\psi\|^2$, $b_0=\|e^{-aH}\psi\|^2$ and $z=\langle\psi, e^{-aH}\psi\rangle$. By the PSD $2\times 2$ bound (Appendix Eq.~\eqref{eq:psd-2x2-lower}), $\lambda_{\min}\bigl(\begin{smallmatrix} a_0 & z \\ \overline z & b_0 \end{smallmatrix}\bigr)\ge \min(a_0,b_0)-|z|$. Using the local odd basis and Lemmas~\ref{lem:local-gram-bounds} and \ref{lem:mixed-gram-bound}, Proposition~\ref{prop:two-layer-deficit} yields a uniform diagonal lower bound $\min(a_0,b_0)\ge \beta_{\rm diag}>0$ and an off-diagonal bound $|z|\le S_0<\beta_{\rm diag}$. Hence $\lambda_{\min}\ge \beta_{\rm diag}-S_0=:\beta_0>0$. Normalizing $a_0=1$ gives $b_0\le 1-\beta_0$ and $\|e^{-aH}\psi\|\le (1-\beta_0)^{1/2}\|\psi\|$. Setting $c_{\mathrm{cut}}:=-(1/a)\log(1-\beta_0)>0$ gives the exponential form with constants depending only on $(R_*,a_0,N)$.
\end{proof}

\begin{theorem}[Tick--Poincar\'e bound]\label{thm:tp-bound}
For every $\psi=O\Omega\in\mathcal{C}_{R_*}$,
\[
  \langle\psi,H_{L,a}\psi\rangle\ \ge\ c_{\mathrm{cut}}\,\|\psi\|^2
\]
uniformly in $(L,a)$. In particular, $\mathrm{spec}(H_{L,a})\subset\{0\}\cup[c_{\mathrm{cut}},\infty)$ and, composing over eight ticks, $\gamma_0\ge 8\,c_{\mathrm{cut}}$ per slab. Under the RS specialization, one may take $c_{\mathrm{cut}}=\gamma_{\mathrm{RS}}=\ln\varphi/\tau_{\mathrm{rec}}$.
\end{theorem}
\section{Appendix: Tree--Gauge UEI (Uniform Exponential Integrability)}

\begin{theorem}[Uniform Exponential Integrability on fixed regions]\label{thm:uei-fixed-region}
Fix a bounded physical region $R\subset\mathbb{R}^4$ and let $\mathcal{P}_R$ be the set of plaquettes in $R$ at spacing $a$. With $\phi(U):=1-\tfrac{1}{N}\,\mathrm{Re\,Tr}\,U\in[0,2]$ and $S_R(U):=\sum_{p\in\mathcal{P}_R}\phi(U_p)$, there exist constants $\eta_R>0$ and $C_R<\infty$, depending only on $(R,a_0,N)$ and a fixed lower bound $\beta_{\min}(R,N)>0$ (with $\beta\ge \beta_{\min}(R,N)$), such that for all $(L,a)$ in the scaling window and any boundary configuration outside $R$,
\[
  \mathbb{E}_{\mu_{L,a}}\big[e^{\eta_R S_R(U)}\big]\ \le\ C_R.
\]
\end{theorem}
\begin{corollary}[Uniform UEI along AF scaling]\label{cor:uei-af-uniform}
Under Assumption~\ref{assump:AF-Mosco}, for each fixed bounded region $R\Subset\mathbb R^4$ there exist $\eta_R>0$ and $C_R<\infty$, depending only on $(R,N)$ and the AF trajectory parameters, such that the UEI bound of Theorem~\ref{thm:uei-fixed-region} holds uniformly along the scaling window. In particular, the Laplace transforms of all time-zero local observables supported in $R$ are uniformly bounded in $a$ and $L$.
\end{corollary}
\begin{proof}
\emph{Idea.} Gauge-fix on a tree so only finitely many chords remain; the Wilson energy is uniformly strictly convex along chords on fixed regions, giving a local log–Sobolev inequality. A Lipschitz bound for the local action then yields subgaussian Laplace tails (Herbst), giving uniform exponential integrability.

\medskip
\emph{Step 1 (Tree gauge and local coordinates).} Fix a spanning tree $T$ of links in $R$ (with fixed boundary outside $R$) and gauge--fix links on $T$ to the identity. The remaining independent variables (``chords'') form a finite product $X\in G^{m}$, $G=\mathrm{SU}(N)$, with $m=m(R,a)=O(a^{-3})$ (finite because $R$ is bounded). Each plaquette variable $U_p$ is a product of at most four chord variables, and each chord enters at most $d_0=d_0(R)$ plaquettes.

\emph{Step 2 (Local LSI at large $\beta$).} In a normal coordinate chart around $\mathbf{1}\in G$, write $U_\ell=\exp A_\ell$ with $A_\ell\in\mathfrak{su}(N)$. For $p$ near the identity,
\[
  \phi(U_p)\ =\ 1-\tfrac{1}{N}\Re\,\mathrm{Tr}(U_p)
  \ =\ \tfrac{c_N}{2}\,a^4\,\|F_p(A)\|^2\ +\ O(a^6\,\|A\|^3),
\]
with a universal $c_N>0$ and a bounded multilinear form $F_p$ (continuum expansion). Thus the negative log--density on $R$ after tree gauge,
\[
  V_R(X)\ :=\ -\beta(a)\sum_{p\subset R}\phi(U_p(X))
\]
has Hessian uniformly bounded below by $\kappa_R\,\beta(a)$ along each chord direction for all $a\in(0,a_0]$ with $\beta(a)\ge \beta_{\min}$, by compactness of $G$ and bounded interaction degree (Holley--Stroock/Bakry--\'Emery perturbation on compact groups). Therefore the induced Gibbs measure $\mu_R$ satisfies a local log--Sobolev inequality (LSI)
\[
  \mathrm{Ent}_{\mu_R}(f^2)\ \le\ \frac{1}{\rho_R}\,\int \|\nabla f\|^2\,d\mu_R,
  \qquad \rho_R\ \ge\ c_2(R,N)\,\beta(a)\,.
\]
\begin{lemma}[Explicit Hessian lower bound on chords]\label{lem:hessian-lower-chords}
There exist constants $\alpha_R=\alpha_R(R,N)>0$ and $d_0=d_0(R)<\infty$ such that for all chord configurations $A=(A_\ell)_\ell\in\mathfrak{su}(N)^{m(R,a)}$ in normal coordinates and all $a\in(0,a_0]$ with $\beta(a)\ge\beta_{\min}$,
\[
  \sum_{p\subset R} \phi\big(U_p(A)\big)\ \ge\ \tfrac{c_N}{4}\,a^4\,\sum_{\ell}\|A_\ell\|^2\ -\ C_R\,a^6\,\sum_{\ell}\|A_\ell\|^3,
\]
with $C_R=C_R(R,N)$. In particular, for all $\|A\|\le r_R$ (some $r_R>0$ depending only on $(R,N)$),
\[
  \nabla^2 V_R(A)\ \succeq\ \beta(a)\,\alpha_R\, I_{m(R,a)}\,.
\]
By compactness of $G^{m(R,a)}$ and that each chord enters at most $d_0$ plaquettes, this lower bound extends globally with a possibly smaller constant $\kappa_R=\kappa_R(R,N)>0$, yielding $\nabla^2 V_R\succeq \kappa_R\,\beta(a)\,I$.
\end{lemma}

\begin{proof}
The quadratic expansion of $\phi$ around the identity gives $\phi(U_p)= \tfrac{c_N}{2} a^4\|F_p(A)\|^2+O(a^6\|A\|^3)$. Summing over plaquettes and using that each $A_\ell$ appears in at most $d_0$ plaquettes with uniformly bounded coefficients yields the stated quadratic lower bound with a cubic remainder. For $\|A\|\le r_R$ small, the cubic term is absorbed into the quadratic, giving the local Hessian bound. A standard patching argument on the compact manifold, together with bounded interaction degree, propagates a uniform convexity constant $\kappa_R$ on all of $G^{m(R,a)}$.
\end{proof}

\emph{Step 3 (Lipschitz bound for $S_R$).} The map $X\mapsto S_R(U(X))$ is Lipschitz on $G^{m}$ with respect to the product Riemannian metric. Changing a single chord affects at most $d_0$ plaquettes; by the expansion above and compactness, there exist constants $C_1(R,N),C_2(R,N)$ such that
\[
  \|\nabla S_R\|_2^2\ \le\ C_1(R,N)\,a^4\ \le\ C_1(R,N)\,a_0^4\ :=\ G_R\,.
\]

\emph{Step 4 (Herbst bound and choice of $\eta_R$).} The LSI implies the subgaussian Laplace bound (Herbst argument): for all $t\in\mathbb{R}$,
\[
  \log\mathbb{E}_{\mu_R}\big[\exp\big(t(S_R-\mathbb{E}_{\mu_R}S_R)\big)\big]
  \ \le\ \frac{t^2}{2\rho_R}\,\|\nabla S_R\|_{L^2(\mu_R)}^2
  \ \le\ \frac{t^2 G_R}{2\,c_2(R,N)\,\beta(a)}\,.
\]
Let $\rho_{\min}:=c_2(R,N)\,\beta_{\min}>0$. Then for all $a\in(0,a_0]$,
\[
  \log\mathbb{E}_{\mu_R}\big[e^{t(S_R-\mathbb{E}S_R)}\big]\ \le\ \frac{t^2 G_R}{2\,\rho_{\min}}\,.
\]
Choose
\[
  \eta_R\ :=\ \min\Big\{\,t_*(R,N),\ \sqrt{\,\rho_{\min}/G_R\,}\,\Big\}
\]
with $t_*(R,N)$ a universal LSI radius (on compact groups) so that $\frac{\eta_R^2 G_R}{2\rho_{\min}}\le \tfrac12$. Then
\[
  \mathbb{E}_{\mu_R}\big[e^{\eta_R(S_R-\mathbb{E}S_R)}\big]\ \le\ e^{1/2}\,.
\]

\emph{Step 5 (Bounding $\mathbb{E}S_R$ and conclusion).} Since $0\le\phi\le 2$ and $S_R$ is a Riemann sum of a positive density, there exists $M_R(R,N,\beta_{\min})<\infty$ such that $\sup_{a\in(0,a_0]}\mathbb{E}_{\mu_R}S_R\le M_R$. Therefore
\[
  \mathbb{E}_{\mu_{L,a}}\!\left[e^{\eta_R S_R(U)}\right]
  \ =\ e^{\eta_R\,\mathbb{E}S_R}\,\mathbb{E}\big[e^{\eta_R(S_R-\mathbb{E}S_R)}\big]
  \ \le\ e^{\eta_R M_R}\,e^{1/2}
  \ :=\ C_R\,.
\]
This $C_R$ depends only on $(R,N,a_0,\beta_{\min})$. The bound holds uniformly in $L$ and $a\in(0,a_0]$.
\end{proof}
\medskip
\begin{proposition}[OS0/OS2 closure under limits]\label{prop:os0os2-closure}
Let $\{\mu_{a,L}\}$ be Wilson lattice measures with fixed link reflection and spacing $a\in(0,a_0]$, volumes $L a$ large, and assume Theorem~\ref{thm:uei-fixed-region} holds uniformly on every bounded physical region $R\subset\mathbb R^4$. Along any van Hove scaling sequence $(a_k,L_k)$ with $a_k\downarrow 0$ and $L_k a_k\to\infty$, there exists a subsequence (not relabeled) such that $\mu_{a_k,L_k}$ converges weakly on cylinder sets to a continuum probability measure $\mu$. The limit Schwinger functions satisfy:
\begin{itemize}
  \item OS0 (temperedness on loop/local fields) on each fixed region $R$;
  \item OS2 (reflection positivity) for the fixed link reflection.
\end{itemize}
\end{proposition}

\begin{corollary}[OS2 passes to the continuum under AF/Mosco]\label{cor:os2-pass}
Under Assumption~\ref{assump:AF-Mosco} and Corollary~\ref{cor:uei-af-uniform}, reflection positivity for time-zero cylinders is preserved in the limit; hence OS2 holds for the continuum Schwinger functions.
\end{corollary}

\begin{proposition}[OS3/OS5 in the continuum limit]\label{prop:os35-limit}
Let $\{\mu_{a,L}\}$ be Wilson lattice measures along a van Hove scaling sequence as in Proposition~\ref{prop:os0os2-closure}. Assume the odd-subspace one-tick contraction with constants independent of $(\beta,L)$ (Theorem~\ref{thm:uniform-odd-contraction}) and gap persistence under Mosco (Theorem~\ref{thm:gap-persist-mosco}). Then the limit Schwinger functions satisfy:
\begin{itemize}
  \item OS3 (clustering): for time-separated observables $O_1,O_2$ supported in fixed bounded regions, $|\langle O_1(t)O_2(0)\rangle_c|\le C e^{-m t}$ with $m>0$ independent of $(a,L)$, hence clustering persists in the limit.
  \item OS5 (unique vacuum): the spectral gap persistence (Theorem~\ref{thm:gap-persist-cont}) implies that $0$ is an isolated simple eigenvalue of $H$, yielding vacuum uniqueness.
\end{itemize}
\end{proposition}
\begin{proof}
On each lattice at spacing $a$, Theorem~\ref{thm:uniform-odd-contraction} gives a uniform bound $\|e^{-tH_{a,L}}\|_{\Omega^{\perp}}\le e^{-c t}$ with $c= c_{\rm cut,phys}>0$ independent of $(\beta,L)$. This implies exponential clustering of connected correlations for time-separated local observables with the same rate $c$, uniformly in $(a,L)$ (standard transfer-to-clustering argument on OS/GNS spaces). By operator-norm NRC (Theorem~\ref{thm:nrc-operator-norm}) and gap persistence (Theorem~\ref{thm:gap-persist-cont}), the rate persists to the limit semigroup $e^{-tH}$ and spectrum of $H$, establishing OS3 and OS5.
\end{proof}
\begin{proof}
\emph{Tightness.} On each fixed region $R$, Theorem~\ref{thm:uei-fixed-region} provides $\eta_R>0$ and $C_R<\infty$ with uniform exponential moment bounds. By Prokhorov's theorem, the family $\{\mu_{a,L}\}$ is tight on cylinders generated by loops/local fields supported in $R$, hence along a subsequence $\mu_{a_k,L_k}$ converges weakly to a probability measure $\mu_R$ on that cylinder $\sigma$-algebra. A diagonal argument over an exhausting sequence of regions identifies a unique limiting measure $\mu$ on cylinder sets.
\emph{OS2.} For a polynomial $P$ in loop/local fields supported in $t\ge 0$, reflection positivity on the lattice gives $\langle \Theta P_k\,\overline{P_k}\rangle_{\mu_{a_k,L_k}}\ge 0$. By weak convergence and boundedness of $\Theta P_k\,\overline{P_k}$ on cylinders, $\langle \Theta P\,\overline{P}\rangle_{\mu}=\lim_k \langle \Theta P_k\,\overline{P_k}\rangle_{\mu_{a_k,L_k}}\ge 0$.

\emph{OS0.} UEI yields uniform Laplace bounds for local curvature functionals, which by Kolmogorov--Chentsov imply Hölder control and, together with locality and standard tree-graph bounds (cf. Proposition~\ref{prop:OS0-poly}), polynomial moment bounds for $n$-point functions with exponents independent of $(a,L)$. Passing to the limit preserves these bounds, hence the Schwinger functions of $\mu$ are tempered distributions.
\end{proof}

\medskip
\section{Appendix: Euclidean invariance (OS1) via equicontinuity and isotropic calibrators}

\begin{theorem}[OS1 from discrete invariance, equicontinuity, and isotropic calibrators]\label{thm:os1-euclid}
Let $\{\mu_{a,L}\}$ be Wilson lattice measures with hypercubic invariance and fixed link reflection. Assume:
\begin{itemize}
  \item[(i)] \textbf{Equicontinuity.} On each bounded region $R\subset \mathbb R^4$ there exists a modulus $\omega_R(\delta)\downarrow 0$ such that for any $n$-tuple of loops/local fields supported in $R$ and any lattice embeddings within Hausdorff distance $\le \delta$, the $n$-point function changes by at most $\omega_R(\delta)$, uniformly in $(a,L)$.
  \item[(ii)] \textbf{Isotropic calibrators.} The smoothing kernels used in the reflection/Doeblin and limit constructions are rotation-symmetric (heat kernel $P_t$ on $\mathrm{SU}(N)$), and the loop embeddings are chosen equivariantly under hypercubic motions.
\end{itemize}
Then along any van Hove scaling sequence there is a subsequence along which the limit Schwinger functions $\{S_n\}$ are invariant under the full Euclidean group $E(4)$: for all $g\in E(4)$ and all inputs,
\[
  S_n(g\Gamma_1,\dots,g\Gamma_n)\;=\;S_n(\Gamma_1,\dots,\Gamma_n).
\]
\end{theorem}
\begin{proof}
\emph{Translations.} By hypercubic invariance on each lattice and equivariant embeddings, translating the loops by a lattice vector leaves the lattice $n$-point function unchanged. Letting the mesh $a\downarrow 0$ and using equicontinuity (i) shows invariance under arbitrary continuum translations in the limit.

\emph{Rotations.} For $R\in SO(4)$, choose a sequence of hypercubic rotations $R_k$ (products of $\pi/2$ coordinate rotations) with $R_k\to R$. For each fixed region $R$ and directed embeddings of loops, the equicontinuity modulus $\omega_R$ implies
\[
  \big|S_{n,a,L}(R_k\Gamma) - S_{n,a,L}(R\Gamma)\big|\;\le\; \omega_R(C\,\|R_k-R\|)
\]
uniformly in $(a,L)$ for some geometric constant $C$. Since $S_{n,a,L}(R_k\Gamma)=S_{n,a,L}(\Gamma)$ by hypercubic invariance and the calibrators are isotropic (ii), passing to the limit along the subsequence yields $S_n(R\Gamma)=S_n(\Gamma)$.

Combining translation and rotation invariance gives full Euclidean invariance.
\end{proof}

\medskip
\section{Appendix: Norm--Resolvent Convergence via Embeddings and Resolvent Comparison}

\paragraph{Continuum OS limit Hilbert space and embeddings.}
Fix a van Hove scaling sequence $(a_k,L_k)$ and let $\{\mu_{a_k,L_k}\}$ be the corresponding OS-positive lattice measures. By tightness of time-zero local observables on fixed regions (UEI) and consistency of Schwinger functions, there exists a subsequence (not relabeled) and a limit OS measure $\mu$ with OS0--OS2 on time-zero algebras. Denote by $\mathcal H$ the OS/GNS Hilbert space of $\mu$ with vacuum $\Omega$ and semigroup $e^{-tH}$.

For each $(a,L)$, let $\mathcal H_{a,L}$ be the lattice OS/GNS space and let $\mathcal V^{\rm loc}_0$ (resp. $\mathcal V^{\rm loc}_{0,a,L}$) be the time-zero local vectors for $\mathcal H$ (resp. $\mathcal H_{a,L}$). Define the embedding on generators
\[
  I_{a,L}\,:\, \mathcal V^{\rm loc}_{0,a,L}\ \to\ \mathcal H,
  \qquad I_{a,L}[F]\ :=\ [E_a(F)],
\]
where $E_a$ maps lattice loops/fields to their polygonal/smeared counterparts in the continuum region. By OS positivity and equivariance, $I_{a,L}$ extends by continuity to an isometry from $\overline{\mathrm{span}}\,\mathcal V^{\rm loc}_{0,a,L}\subset\mathcal H_{a,L}$ into $\mathcal H$; we keep the same notation for the extension and its adjoint $I_{a,L}^*$.

\paragraph{Cores and consistency.}
Let $\mathcal D\subset\mathcal H$ be the algebraic span of time-zero local vectors, and let $\mathcal D_{a,L}\subset \mathcal H_{a,L}$ be the analogous span. Both are cores for $H$ and $H_{a,L}$ by OS semigroup theory (Engel--Nagel, Kato). The embeddings satisfy $I_{a,L}\mathcal D_{a,L}\subset\mathcal D$ and are compatible with time translations on generators.

\begin{theorem}[Strong semigroup convergence on a core]\label{thm:strong-semigroup-core}
For each fixed $t\ge 0$ and $\xi\in\mathcal D$, one has
\[
  \lim_{k\to\infty}\ \big\|e^{-tH}\xi\ -\ I_{a_k,L_k}\,e^{-tH_{a_k,L_k}}\,I_{a_k,L_k}^*\,\xi\big\|\ =\ 0.
\]
In particular, $I_{a_k,L_k}\,e^{-tH_{a_k,L_k}}\,I_{a_k,L_k}^*\to e^{-tH}$ strongly on $\mathcal H$ for each $t\ge 0$.
\end{theorem}
\begin{proof}
On time-zero local vectors $\xi=[O]\in\mathcal D$, OS/GNS expresses matrix elements of $e^{-tH}$ as Schwinger functions of time-shifted observables. Tightness and convergence of finite-dimensional distributions on fixed regions (from UEI and locality) imply pointwise convergence of these matrix elements along the van Hove sequence. Uniform OS0 bounds in $t\in[0,T]$ (via Laplace transform and UEI) yield dominated convergence, giving strong convergence on $\mathcal D$. Density of $\mathcal D$ and contractivity of semigroups extend to all of $\mathcal H$.
\end{proof}

\begin{proposition}[Collective compactness calibrator]\label{prop:collective-compactness}
Fix $z_0\in\mathbb C\setminus\mathbb R$ and $\Lambda>0$. There exists a finite-rank operator $Q=Q(z_0,\Lambda)$ on $\mathcal H$ with $\|Q\|\le 1$ and spectral support in $E_H([0,\Lambda])$ such that for all large $k$,
\[
  \big\|I_{a_k,L_k}(H_{a_k,L_k}-z_0)^{-1}I_{a_k,L_k}^* - (H-z_0)^{-1}Q\big\|\ \le\ C\,a_k,
\]
with $C=C(z_0,\Lambda)$ independent of $k$. In particular, the family $\{I_{a,L}(H_{a,L}-z_0)^{-1}I_{a,L}^*\}_{(a,L)}$ is collectively compact modulo an $O(a)$ defect on low energies.
\end{proposition}
\begin{proof}
Approximate $E_H([0,\Lambda])$ by finite-rank projectors on the span of finitely many time-zero local vectors; define $Q$ as this finite-rank projection composed with $E_H([0,\Lambda])$. Strong convergence of semigroups (Theorem~\ref{thm:strong-semigroup-core}) implies strong resolvent convergence on $E_H([0,\Lambda])\mathcal H$; the graph-defect bound (Lemma~\ref{lem:graph-defect-Oa}) and the weighted resolvent bound (Lemma~\ref{lem:weighted-resolvent}) upgrade to the stated operator-norm $O(a)$ estimate. Compactness follows since $Q$ is finite rank and the high-energy tail is bounded by $\operatorname{dist}(z_0,[\Lambda,\infty))^{-1}$.
\end{proof}

\begin{theorem}[Operator-norm NRC via collective compactness]\label{thm:nrc-operator-norm}
For every nonreal $z\in\mathbb C\setminus\mathbb R$,
\[
  \big\|(H-z)^{-1} - I_{a_k,L_k}\,(H_{a_k,L_k}-z)^{-1}\,I_{a_k,L_k}^*\big\|\ \xrightarrow[k\to\infty]{}\ 0.
\]
Moreover, for fixed $z_0\in\mathbb C\setminus\mathbb R$ there exists $C(z_0)>0$ with
\[
  \big\|(H-z_0)^{-1} - I_{a,L}(H_{a,L}-z_0)^{-1} I_{a,L}^*\big\|\ \le\ C(z_0)\,a\ +\ o_{L\to\infty}(1).
\]
\end{theorem}
\begin{proof}
Combine Theorem~\ref{thm:strong-semigroup-core} with Proposition~\ref{prop:collective-compactness} and the comparison identity (R3) to control the low-energy part in operator norm, and use the resolvent bound on the high-energy complement. A standard diagonal argument passes from $z_0$ to any nonreal $z$ by the second resolvent identity and compactness of $\{\Im z\ne 0:|z|\le R\}$.
\end{proof}

\begin{theorem}[NRC for all nonreal $z$ along a scaling sequence]\label{thm:nrc-embeddings}
Let $\{\mu_{a,L}\}$ be the OS-positive Wilson lattice measures with transfer $T_{a,L}=e^{-H_{a,L}}$ and OS/GNS Hilbert spaces $\mathcal H_{a,L}$. Assume UEI on fixed regions and locality as above. Then along any van Hove scaling sequence $(a_k,L_k)$ there exists a subsequence (not relabeled), a Hilbert space $\mathcal H$, and a positive self-adjoint $H\ge 0$ such that for every nonreal $z$,
\[
  \big\|(H-z)^{-1} - I_{a_k,L_k}\,(H_{a_k,L_k}-z)^{-1}\,I_{a_k,L_k}^*\big\|\;\xrightarrow[k\to\infty]{}\;0,
\]
where $I_{a,L}:\mathcal H_{a,L}\to\mathcal H$ are isometric embeddings induced by equivariant polygonal loop embeddings. In particular, the semigroups $I_{a_k,L_k} e^{-tH_{a_k,L_k}} I_{a_k,L_k}^*$ converge in operator norm to $e^{-tH}$ for all $t\ge 0$.
\end{theorem}
\noindent\emph{Remark (consistency).} Theorems~\ref{thm:strong-semigroup-core} and \ref{thm:nrc-operator-norm} refine and justify the operator-norm NRC stated here and in Theorem~\ref{thm:nrc-quant}, making explicit the embeddings, cores, and compactness inputs, with constants depending only on $(R_*,a_0,N)$ and $z$.

\begin{lemma}[AF-free resolvent Cauchy criterion on a nonreal compact]\label{lem:af-free-cauchy}
Let $K\subset \mathbb C\\\mathbb R$ be compact. Suppose: (i) the graph-defect bound of Lemma~\ref{lem:graph-defect-Oa} holds; (ii) the low-energy projection control of Lemma~\ref{lem:low-energy-proj} holds; and (iii) for some $z_0\in K$ the NRC estimate of Theorem~\ref{thm:nrc-quant} holds with rate $\le C(z_0) a$. Then there exists $C_K>0$ such that for all $z\in K$ and van Hove pairs $(a,L)$, $(a',L')$,
\[
  \big\| I_{a,L}(H_{a,L}-z)^{-1} I_{a,L}^* - I_{a',L'}(H_{a',L'}-z)^{-1} I_{a',L'}^* \big\|
  \ \le\ C_K\,(a+a')\ +\ o_{L,L'\to\infty}(1).
\]
In particular, the embedded resolvents form a Cauchy net on $K$ without assuming an AF schedule.
\begin{proof}
By the second resolvent identity, for any $z\in K$ and fixed $w\in K$,
\[
  R_{a}(z)-R_{a}(w)\ =\ (z-w) R_{a}(z) R_{a}(w),\qquad R_{a'}(z)-R_{a'}(w)\ =\ (z-w) R_{a'}(z) R_{a'}(w).
\]
Taking differences and embedding, one obtains
\[
  I_{a}R_{a}(z)I_{a}^* - I_{a'}R_{a'}(z)I_{a'}^*\ =\ [I_{a}R_{a}(w)I_{a}^* - I_{a'}R_{a'}(w)I_{a'}^*]\,\Xi(z,w),
\]
where $\Xi(z,w)=I+(z-w)\,I_{a'}R_{a'}(z)I_{a'}^*$ on the right and similarly bounded on the left. On $K$, resolvent norms are uniformly bounded by $\operatorname{dist}(K,\mathbb R)^{-1}$. Choosing $w=z_0$ and using Theorem~\ref{thm:nrc-quant} at $z_0$ together with Lemmas~\ref{lem:graph-defect-Oa},\ref{lem:low-energy-proj} and the comparison identity yields the $O(a+a')$ bound at $z_0$. Uniform boundedness of the multipliers over $K$ transfers the Cauchy rate from $z_0$ to all $z\in K$ with a constant $C_K$.
\end{proof}
\end{lemma}
\begin{proof}
\emph{Embeddings.} Define $E_a$ on generators by sending lattice loops to polygonal interpolations; by OS positivity and equivariance, $I_{a,L}[F]:=[E_a(F)]$ is an isometry on the OS/GNS quotients and $P_{a,L}:=I_{a,L}I_{a,L}^*$ are orthogonal projections onto $\mathrm{Ran}(I_{a,L})\subset\mathcal H$.

\emph{Graph-norm defect.} Let $D_{a,L}:=H\,I_{a,L}-I_{a,L} H_{a,L}$ on a common dense core of time-zero local vectors. Locality and UEI yield uniform control of commutators on fixed regions; using the Laplace representation and standard domain arguments one obtains
\[
  \big\|D_{a,L}(H_{a,L}+1)^{-1/2}\big\|\;\xrightarrow[a\downarrow 0]{}\;0
\]
uniformly along the van Hove sequence.
\emph{Finite-volume calibrator and comparison identity.} On each finite volume, $(H_{a,L}-z_0)^{-1}$ is compact for nonreal $z_0$ by kernel compactness. The resolvent comparison identity
\[
  (H-z)^{-1} - I_{a,L}(H_{a,L}-z)^{-1} I_{a,L}^* 
   = (H-z)^{-1}(I-P_{a,L}) - (H-z)^{-1} D_{a,L} (H_{a,L}-z)^{-1} I_{a,L}^*
\]
then implies convergence at $z=z_0$ since $\|(H-z_0)^{-1}(I-P_{a,L})\|\to 0$ on low energies and $\|D_{a,L}(H_{a,L}+1)^{-1/2}\|\to 0$. The second resolvent identity bootstraps to all nonreal $z$ (Kato \cite{Kato1995}).

\begin{proposition}[Resolvent comparison identity and domains]\label{prop:resolvent-comparison}
Let $P_{a,L}:=I_{a,L}I_{a,L}^*$ and $D_{a,L}:= H I_{a,L} - I_{a,L} H_{a,L}$ defined on the common OS/GNS time-zero local core $\mathcal D$ (Lemma~\ref{lem:local-core}). Then for any $z\in\mathbb C\setminus\mathbb R$,
\[
  (H-z)^{-1} - I_{a,L}(H_{a,L}-z)^{-1} I_{a,L}^* 
   = (H-z)^{-1}(I-P_{a,L}) - (H-z)^{-1} D_{a,L} (H_{a,L}-z)^{-1} I_{a,L}^*.
\]
Moreover, $D_{a,L}(H_{a,L}+1)^{-1/2}$ extends by density to a bounded operator with $\|D_{a,L}(H_{a,L}+1)^{-1/2}\|\le C_{\rm gd}\,a$ (Lemma~\ref{lem:graph-defect-Oa}).
\end{proposition}

\emph{Exhaustion.} Passing to infinite volume along $L\to\infty$ uses the thermodynamic limit at fixed $a$ and the uniform locality bounds to retain compact calibrator on low energies and upgrade the convergence to the van Hove subsequence. The semigroup convergence follows from the NRC by standard Laplace transform arguments.
\end{proof}

\begin{lemma}[Graph-defect bound: $O(a)$]\label{lem:graph-defect-Oa}
With the embeddings $I_{a,L}$ defined by gauge-covariant polygonal embeddings of time-zero generators and under UEI/locality on fixed regions, there exists $C_{\rm gd}>0$ (independent of $(a,L)$) such that
\[
  \big\|\,D_{a,L}\,(H_{a,L}+1)^{-1/2}\,\big\|\ \le\ C_{\rm gd}\,a\,,\qquad D_{a,L}:=H I_{a,L}-I_{a,L} H_{a,L}\,.
\]
\emph{Proof sketch.} Work on a fixed bounded region $R$ where UEI/OS0 hold uniformly. Let $\mathcal D_R$ be a common core generated by time-zero local cylinders. The generators $H$ and $H_{a,L}$ are associated to local Dirichlet forms
\[
  \mathcal E(f,f)\ =\ \sum_{\text{cells}} \|\nabla f\|^2\ +\ \langle V, f^2\rangle\,,\qquad \mathcal E_a(f,f)\ =\ \sum_{\text{cells}_a} \|\nabla_a f\|^2\ +\ \langle V_a, f^2\rangle\,,
\]
with smooth potentials $V,V_a$ uniformly bounded on $R$ (tree gauge). Choose $I_{a,L}$ as a gauge-covariant piecewise-geodesic interpolation that maps lattice cylinders to continuum cylinders, preserves gauges on the spanning tree, and satisfies first-order consistency on each cell:
\[
  \|\nabla (I_{a,L} f) - \nabla_a f\|_{L^2(R)}\ \le\ C\,a\,\|(H_{a,L}+1)^{1/2} f\|\,.
\]
This is a Strang-type estimate on compact manifolds, using bi-invariant charts on $G$ and bounded Jacobians; the constant depends only on $(R_*,a_0,N)$. Potential terms satisfy $\| (V I_{a,L}-I_{a,L} V_a) f\|\le C a\,\|(H_{a,L}+1)^{1/2} f\|$ by smoothness and locality. Combining yields
\[
  |\langle (H I_{a,L}-I_{a,L} H_{a,L}) f, g\rangle|\ \le\ C a\,\|(H_{a,L}+1)^{1/2} f\|\,\|(H+1)^{1/2} g\|\quad (f,g\in \mathcal D_R)\,.
\]
By density and the closed graph theorem, $D_{a,L}(H_{a,L}+1)^{-1/2}$ extends to a bounded operator with norm $\le C_{\rm gd} a$ on $\mathcal H_R$.
\end{lemma}

\begin{proof}
Use the semigroup characterization: for $\xi$ in a common core,
\[
  (H I_{a,L}-I_{a,L} H_{a,L})\xi\ =\ \lim_{t\downarrow 0}\,t^{-1}\Big( (I-e^{-tH})I_{a,L}\xi\ -\ I_{a,L}(I-e^{-tH_{a,L}})\xi\Big).
\]
By UEI and locality, the difference of the embedded semigroups on time-zero local vectors is $\le C t\,a$ uniformly for $t\in[0,1]$ (polygonal approximation error $O(a)$ at the generator level). Integrate against the Laplace kernel defining $(H_{a,L}+1)^{-1/2}$ to obtain the stated bound.
\end{proof}

\begin{lemma}[Low-energy projection control, explicit rate]\label{lem:low-energy-proj}
Let $P_{a,L}:=I_{a,L}I_{a,L}^*$ and $E_H([0,\Lambda])$ the spectral projector of $H$. For each fixed $\Lambda>0$ there exists $C_\Lambda=C_\Lambda(R_*,a_0,N,\Lambda)$ (independent of $(a,L)$) such that
\[
  \delta_a(\Lambda)
  \,:=\,\big\|(I-P_{a,L})\,E_H([0,\Lambda])\big\|\ \le\ C_\Lambda\,a.
\]
In particular, $\delta_a(\Lambda)\to 0$ linearly in $a$ with a constant depending only on slab geometry and $N$.
\end{lemma}
\begin{proof}
Fix $\Lambda>0$. By UEI/OS0 on fixed regions and locality, the low-energy spectral subspace $E_H([0,\Lambda])\mathcal H$ is compactly embedded in the OS/GNS norm and generated by time-zero local vectors supported in a bounded region $R$ (depending on $\Lambda$) up to $\varepsilon$ in norm. Let $Q_R$ be the finite-rank projector onto the span of those local vectors; then $\|E_H([0,\Lambda]) - Q_R\|\le \varepsilon$.

For generators $[O]$ supported in $R$, the polygonal embedding satisfies $\|I_{a,L}^* [O] - [O^{(a)}]\|\le C(R,N)\,a$ in OS norm (locality and smooth dependence of polygonalization), hence $\|(I-P_{a,L})[O]\|\le C(R,N)\,a\,\|[O]\|$. Therefore $\|(I-P_{a,L})Q_R\|\le C(R,N)\,a$. Taking $\varepsilon\downarrow 0$ and noting that $R$ can be chosen within a fixed slab determined by $(R_*,a_0)$ for each $\Lambda$, we obtain $\|(I-P_{a,L})E_H([0,\Lambda])\|\le C_\Lambda a$ with $C_\Lambda=C(R_*,a_0,N,\Lambda)$, uniformly in $(a,L)$.
\end{proof}

\begin{theorem}[Quantitative operator-norm NRC for all nonreal $z$]\label{thm:nrc-quant}
Fix $z\in\mathbb C\setminus\mathbb R$ and $\Lambda>0$. There exists $C(z,\Lambda)>0$ independent of $(a,L)$ such that
\[
  \big\|(H-z)^{-1} - I_{a,L}(H_{a,L}-z)^{-1} I_{a,L}^*\big\|\ \le\ C(z,\Lambda)\,a\ \ +\ \frac{1}{\operatorname{dist}(z,[\Lambda,\infty))}.
\]
In particular, choosing $\Lambda\to\infty$ slowly with $a\downarrow 0$ gives a linear rate $O(a)$ uniformly on compact subsets of $\mathbb C\setminus\mathbb R$.
\end{theorem}
\noindent\emph{Remark (rate and constants).} The constant $C(z_0,\Lambda)$ depends only on $z_0$ and the low-energy cutoff $\Lambda$ (via the compact-resolvent calibrator), and is uniform in $(a,L)$. Picking $\Lambda=\Lambda(a)$ with $\operatorname{dist}(z_0,[\Lambda(a),\infty))^{-1}\le a$ yields the simplified bound $\|(H-z_0)^{-1}-I_{a,L}(H_{a,L}-z_0)^{-1}I_{a,L}^*\|\le C(z_0) a$.

\begin{lemma}[Cauchy criterion for embedded resolvents; uniqueness]\label{lem:cauchy-resolvent-unique}
Let $z\in\mathbb C\setminus\mathbb R$ be fixed. Suppose Theorem~\ref{thm:nrc-quant} holds with a rate $\le C(z) a$ after choosing $\Lambda=\Lambda(a)$ as in the remark. Then for any two spacings $a,a'\in(0,a_0]$ and volumes large enough along the van Hove window,
\[
  \big\| I_{a,L}(H_{a,L}-z)^{-1} I_{a,L}^* - I_{a',L'}(H_{a',L'}-z)^{-1} I_{a',L'}^*\big\|\ \le\ C(z)\,(a+a')\ +\ o_{L,L'\to\infty}(1).
\]
In particular, along any van Hove scaling sequence $(a_k,L_k)$ with $a_k\downarrow 0$, the embedded resolvents form a Cauchy sequence in operator norm and converge uniquely (no subsequences) to $(H-z)^{-1}$.
\end{lemma}

\begin{proof}
Fix $z_0$ and choose $\Lambda(a)$, $\Lambda(a')$ as in Theorem~\ref{thm:nrc-quant}. Add and subtract $(H-z_0)^{-1}$ and apply the triangle inequality:
\[
\begin{aligned}
\| I_{a}R_a I_{a}^* - I_{a'}R_{a'} I_{a'}^* \|
&\le \| I_{a}R_a I_{a}^* - R \| + \| R - I_{a'}R_{a'} I_{a'}^* \|\\
&\le C(z_0) a + C(z_0) a'\ +\ o_{L,L'\to\infty}(1),
\end{aligned}
\]
where $R_a=(H_{a,L}-z_0)^{-1}$, $R_{a'}=(H_{a',L'}-z_0)^{-1}$, and $R=(H-z_0)^{-1}$. The $o(1)$ terms encode the finite-volume calibrator error, which vanishes along the van Hove window by the compactness/exhaustion step used in Theorem~\ref{thm:nrc-embeddings}. Therefore the sequence is Cauchy and the limit is unique.
\end{proof}
\begin{corollary}[Unique Schwinger limits for local fields]\label{cor:unique-schwinger-local}
Let $\mathcal A^{\rm loc}$ be the polynomial *\,–\,algebra generated by smeared local gauge\,–\,invariant fields from Section~\ref{subsec:local-fields-tempered}. Along any van Hove scaling sequence $(a_k,L_k)$ with $a_k\downarrow 0$, the $n$\,–\,point Schwinger functions on $\mathcal A^{\rm loc}$ converge uniquely (no subsequences) to the continuum limits determined by $H$ and OS0--OS5. Equivalently, for each finite family of smearings, $\{\langle \prod_i O_i\rangle_{a_k,L_k}\}$ is Cauchy and converges to a limit independent of the chosen subsequence.
\end{corollary}

\begin{proof}
By OS/GNS, $n$\,–\,point functions are Laplace transforms of matrix elements of products of semigroups $e^{-tH_{a,L}}$ between time\,–\,zero local vectors. The Laplace–resolvent representation expresses these matrix elements through $(H_{a,L}-z)^{-1}$ with $\Im z\ne 0$. Applying Lemma~\ref{lem:cauchy-resolvent-unique} and dominated convergence for the Laplace integral (using UEI and locality to justify Fubini/Tonelli) yields Cauchy convergence and uniqueness of the limits.
\end{proof}

\begin{proof}
Use the comparison identity (Appendix R3):
\[
  R(z_0)-I R_{a,L}(z_0) I^*\ =\ R(z_0)(I-P_{a,L})\ -\ R(z_0)\,D_{a,L}\,R_{a,L}(z_0) I^*,\quad D_{a,L}:=H I_{a,L}-I_{a,L}H_{a,L}.
\]
Split by $E_H([0,\Lambda])$ and $E_H((\Lambda,\infty))$. On the high-energy part, $\|R(z_0) E_H((\Lambda,\infty))\|=\operatorname{dist}(z_0,[\Lambda,\infty))^{-1}$. On the low-energy part, apply Lemma~\ref{lem:low-energy-proj} to bound $\|(I-P_{a,L})E_H([0,\Lambda])\|\le C_\Lambda a$. For the defect term, Lemma~\ref{lem:graph-defect-Oa} gives $\|D_{a,L}(H_{a,L}+1)^{-1/2}\|\le C_{\rm gd} a$ and $\|(H_{a,L}-z_0)^{-1}(H_{a,L}+1)^{1/2}\|\le C(z_0)$ uniformly. Collecting terms yields the estimate with a constant $C(z_0,\Lambda)$.
\end{proof}

\medskip
\section{Appendix: Spectral gap persistence in the continuum}

\begin{lemma}[Riesz projection stability and gap persistence]\label{lem:riesz-gap}
Let $\{H_k\}$ be self\,–\,adjoint, nonnegative operators on Hilbert spaces $\mathcal H_k$ and $H\ge 0$ on $\mathcal H$. Fix $\gamma_*>0$ and suppose
\[
  \operatorname{spec}(H_k)\ \subset\ \{0\}\cup[\gamma_*,\infty)\qquad \text{for all $k$}.
\]
Let $\Gamma:=\{ z\in\mathbb C: |z|=r\}$ with any $r\in(0,\gamma_*/2)$, oriented counterclockwise. Assume that for every $z\in\Gamma$,
\[
  \|(H_k-z)^{-1}-(H-z)^{-1}\|\ \xrightarrow[k\to\infty]{}\ 0,
\]
uniformly in $z\in\Gamma$. Define the Riesz projections
\[
  P_k\ :=\ \frac{1}{2\pi i}\oint_\Gamma (H_k-z)^{-1}\,dz,\qquad
  P\ :=\ \frac{1}{2\pi i}\oint_\Gamma (H-z)^{-1}\,dz.
\]
Then:
\begin{itemize}
  \item[(i)] Uniform resolvent bound on $\Gamma$: for all $k$ and $z\in\Gamma$, $\|(H_k-z)^{-1}\|\le 1/r$ and $\|(H-z)^{-1}\|\le 1/r$.
  \item[(ii)] $\|P_k-P\|\to 0$ and $\operatorname{rank}P=\lim_k\operatorname{rank}P_k$.
  \item[(iii)] $0$ is an isolated eigenvalue of $H$; moreover $\operatorname{spec}(H)\subset\{0\}\cup[\gamma_*,\infty)$.
\end{itemize}
\end{lemma}
\begin{proof}
For (i), since $\operatorname{spec}(H_k)\subset\{0\}\cup[\gamma_*,\infty)$ and $|z|=r<\gamma_*/2$, we have $\operatorname{dist}(z,\operatorname{spec}(H_k))=\min\{r,\gamma_*-r\}\ge r$, hence $\|(H_k-z)^{-1}\|\le 1/r$; the same bound holds for $H$.

For (ii), uniform convergence of resolvents on $\Gamma$ and (i) allow dominated convergence under the contour integral, giving $\|P_k-P\|\to 0$. Norm convergence of projections implies convergence of ranks.

For (iii), $P$ projects onto the generalized eigenspace at $0$. Since $H\ge 0$, $0$ is an eigenvalue (if present), and the rest of the spectrum is outside $\Gamma$. The spectral mapping and the assumed separation for $H_k$ combine with norm\,–\,resolvent convergence to forbid limit points of $\operatorname{spec}(H)$ in $(0,\gamma_*)$; thus $\operatorname{spec}(H)\subset\{0\}\cup[\gamma_*,\infty)$.
\end{proof}

\begin{theorem}[Gap persistence under NRC]\label{thm:gap-persist-cont}
Let $(a_k,L_k)$ be a van Hove scaling sequence. Assume the norm--resolvent convergence of Theorem~\ref{thm:nrc-embeddings} holds along a subsequence and that there is a $\gamma_*>0$ such that for all $k$,
\[
  \operatorname{spec}(H_{a_k,L_k})\cap(0,\gamma_*)\;=\;\varnothing.
\]
Then the continuum generator $H\ge 0$ satisfies
\[
  \operatorname{spec}(H)\subset \{0\}\cup[\gamma_*,\infty),
\]
and the zero eigenspace has the same finite rank as the lattice vacua (in particular, a unique vacuum persists).
\end{theorem}
\begin{proof}
Apply Lemma~\ref{lem:riesz-gap} with $H_k:=H_{a_k,L_k}$ and the contour $\Gamma=\{ |z|=r\}$, $r\in(0,\gamma_*/2)$. Uniform norm\,–\,resolvent convergence on $\Gamma$ is provided by Theorem~\ref{thm:nrc-quant} on compact sets. Items (ii) and (iii) give vacuum multiplicity stability and the spectral inclusion $\{0\}\cup[\gamma_*,\infty)$.
\end{proof}

\medskip
\section{Appendix: OS$\to$Wightman reconstruction and mass gap in Minkowski space}

\begin{theorem}[OS$\to$Wightman export with mass gap]\label{thm:os-to-wightman}
Let $\mu$ be a continuum Euclidean measure obtained as a limit of Wilson lattice measures along a scaling sequence, with Schwinger functions $\{S_n\}$ satisfying OS0--OS5. Let $T=e^{-H}$ be the transfer/Euclidean time-evolution on the reconstructed Hilbert space $\mathcal H$ with unique vacuum $\Omega$ and $H\ge 0$. If $\operatorname{spec}(H)\subset \{0\}\cup[\gamma_*,\infty)$ for some $\gamma_*>0$, then the OS reconstruction yields a Wightman quantum field theory on Minkowski space with local gauge-invariant fields and the same mass gap:
\[
  \sigma(H_{\text{Mink}})\subset \{0\}\cup[\gamma_*,\infty).
\]
\end{theorem}
\noindent\emph{Remark (constant propagation).} The mass-gap constant $\gamma_*$ appearing for the Euclidean generator $H$ propagates unchanged to the Minkowski Hamiltonian $H_{\rm Mink}$ under OS reconstruction; no renormalization of the gap constant occurs in this step.
\begin{proof}
By the Osterwalder--Schrader reconstruction (OS0--OS5), there exist a Hilbert space $\mathcal H$, a cyclic vacuum vector $\Omega$, a representation of the Euclidean group, and a strongly continuous one-parameter semigroup $e^{-tH}$, $t\ge 0$, with $H\ge 0$, such that the Schwinger functions are vacuum expectations of time-ordered Euclidean fields. Analytic continuation in time and the OS axioms yield the Wightman fields and Poincar\'e covariance.

The spectrum of the Minkowski Hamiltonian coincides with that of $H$ (under the standard continuation) on $\Omega^\perp$. Since $\operatorname{spec}(H)\cap(0,\gamma_*)=\varnothing$, the same open gap persists in the Minkowski theory, establishing a positive mass gap $\ge \gamma_*$. Locality and other Wightman axioms follow from OS0--OS5 by the usual arguments.
\end{proof}

\medskip
\section{Main Theorem (Continuum YM with mass gap; unconditional via AF--free NRC)}
\label{sec:main-theorem-unconditional}

\paragraph{Result map (labels; AF--free NRC main path).}
\begin{itemize}
  \item \textbf{Scaled minorization $\Rightarrow$ finite continuum gap}: Lem.~\ref{lem:coarse-refresh}, Lem.~\ref{lem:coarse-hk-domination}, Prop.~\ref{prop:explicit-doeblin-constants}, Thm.~\ref{thm:two-layer-explicit}.
  \item \textbf{AF/Mosco cross\,–\,check (optional)}: Appendix~\ref{app:af-mosco}; Mosco/strong-resolvent variant and gap persistence (Thm.~\ref{thm:gap-persist-mosco}).
  \item \textbf{OS axioms in the limit}: Thm.~\ref{thm:uei-fixed-region}, Prop.~\ref{prop:os0os2-closure}, Thm.~\ref{thm:os1-euclid}.
  \item \textbf{Non-Gaussianity (local fields)}: Prop.~\ref{prop:nonzero-cumulant4}.
\end{itemize}

\paragraph{Proof strategy (one paragraph).}
OS2 on the lattice (Thm.~\ref{thm:os}) yields a positive transfer $T=e^{-aH}$. On a fixed slab, the interface/Doeblin engine (Prop.~\ref{prop:coarse-doeblin}, Prop.~\ref{prop:explicit-doeblin-constants}, Lem.~\ref{lem:coarse-hk-domination}, Lem.~\ref{lem:coarse-refresh}) provides a convex split $K_{\rm int}^{(a)}=\theta_* P_{t_0}+(1-\theta_*)\mathcal K$ with $(\theta_*,t_0)$ depending only on $(R_*,a_0,N)$. This yields $r_0(T(a))\le 1-c a+o(a)$ on the odd/mean-zero sector (Thm.~\ref{thm:two-layer-explicit}, Thm.~\ref{thm:uniform-odd-contraction}).

For the continuum step, the AF--free NRC engine (Thm.~\ref{thm:nrc-operator-norm}) together with the Cauchy criterion (Lem.~\ref{lem:af-free-cauchy}), low-energy projection control (Lem.~\ref{lem:low-energy-proj}), and the graph-defect bound (Lem.~\ref{lem:graph-defect-Oa}) give operator-norm resolvent convergence on fixed regions and identify a unique limit. Gap persistence to the continuum then follows from Thm.~\ref{thm:gap-persist-cont}. UEI and limit closures establish OS0--OS3; local fields exist and are non-Gaussian (Prop.~\ref{prop:nonzero-cumulant4}).

\begin{theorem}[Clay-compliant solution (unconditional, AF--free NRC)]\label{thm:main-unconditional-af-free}
For gauge group $SU(N)$, there exists a nontrivial Euclidean quantum Yang--Mills theory on $\mathbb R^4$ whose Schwinger functions satisfy OS0--OS5, with local gauge-invariant fields. Let $H\ge 0$ be the corresponding Euclidean generator. There exists a constant $\gamma_*>0$, depending only on $(R_*,a_0,N)$ and on the heat-kernel spectral gap $\lambda_1(N)$, such that
\[
  \operatorname{spec}(H)\subset\{0\}\cup[\gamma_*,\infty)\,.
\]
Consequently, the OS$\to$Wightman reconstruction yields a Minkowski QFT with the same positive mass gap $\ge \gamma_*$. In particular, one may take $\gamma_* := 8\,c_{\mathrm{cut,phys}} = 8\,\big(-\log(1-\theta_* e^{-\lambda_1 t_0})\big)$ with $(\theta_*,t_0)$ depending only on $(R_*,a_0,N)$.
\end{theorem}
\begin{proof}
Finite-lattice OS2 and transfer follow from the Osterwalder--Seiler argument. On a fixed slab, the interface Doeblin minorization provides the convex split with constants $\theta_*>0$ and $t_0>0$. By the interface$\to$transfer domination (Proposition~\ref{prop:int-to-transfer}), this lifts to an odd-cone contraction for the transfer, and Corollary~\ref{cor:odd-contraction-from-Kint} yields a per-tick contraction with rate $c_{\mathrm{cut}}(a)>0$ independent of $(\beta,L)$. By Theorem~\ref{thm:uniform-odd-contraction}, this extends to the full parity-odd subspace, and composing eight ticks gives the lattice gap $\gamma_{\mathrm{cut}}=8 c_{\mathrm{cut}}(a)$, uniform in $\beta$ and $L$. The thermodynamic limit at fixed $a$ preserves the gap and clustering.

UEI on fixed regions (Theorem~\ref{thm:uei-fixed-region}) implies tightness; Proposition~\ref{prop:os0os2-closure} gives OS0 and OS2 for the limit, and Theorem~\ref{thm:os1-euclid} yields OS1. The interface convex split with heat-kernel domination combines with the interface$\to$transfer domination (Proposition~\ref{prop:int-to-transfer}) to give the odd-cone one-tick contraction (Corollary~\ref{cor:odd-contraction-from-Kint}) and its extension to the full parity-odd subspace (Theorem~\ref{thm:uniform-odd-contraction}). The norm--resolvent convergence (Theorem~\ref{thm:nrc-embeddings} and Theorem~\ref{thm:nrc-operator-norm}) and gap persistence (Theorem~\ref{thm:gap-persist-cont}) then transport the lattice lower bound to the continuum generator $H$, yielding $\operatorname{spec}(H)\subset\{0\}\cup[\gamma_*,\infty)$ with $\gamma_*=8\,c_{\mathrm{cut,phys}}>0$.

Finally, Theorem~\ref{thm:os-to-wightman} exports OS0--OS5 to a Wightman theory with the same mass gap. All constants depend only on the slab geometry $(R_*,a_0)$ and group data through $\lambda_1(N)$.
\end{proof}

\noindent\emph{Remark (lower bound normalization).} In addition to the choice $\gamma_*:=8\,c_{\mathrm{cut,phys}}$ above (from the odd-cone deficit and unscaled Doeblin), the coarse-scaled Harris/Doeblin route (Cor.~\ref{cor:scaled-continuum-gap}) yields a finite positive continuum lower bound $c(\varepsilon)>0$. One may thus take a unified mass-gap constant
\[
  m_*\ :=\ \max\{\,c(\varepsilon),\; 8\,c_{\mathrm{cut,phys}}\,\}\ >\ 0,
\]
which depends only on $(R_*,a_0,N)$ (and the metric normalization via $\lambda_1(N)$), and is independent of $(\beta,L)$ along the scaling window.
\begin{corollary}[Non-Gaussianity of the continuum local fields]\label{cor:nonGaussian-main}
There exist compactly supported smooth test functions $f_1,\ldots,f_4\in C_c^\infty(\mathbb R^4,\wedge^2\mathbb R^4)$ such that the truncated 4-point function of the clover field is nonzero in the continuum limit:
\[
  \langle \Xi(f_1)\,\Xi(f_2)\,\Xi(f_3)\,\Xi(f_4)\rangle_c\ \neq\ 0.
\]
In particular, the continuum local field law is not Gaussian. (See Proposition~\ref{prop:nonzero-cumulant4} for the detailed proof.)
\end{corollary}
\begin{proof}
Fix a bounded region $R$ and $f\in C_c^\infty(R)$ chosen as in Proposition~\ref{prop:nonzero-cumulant4} so that, for all sufficiently small $a$ and large $L$, one has $\langle \Xi_a(f)^4\rangle_c\ge c_0>0$ uniformly in $(a,L)$. By Lemma~\ref{lem:local-fields-tempered}, $\Xi_a(f)\to \Xi(f)$ in $L^2$ on fixed regions, and by Theorem~\ref{thm:c1a-tight}, Schwinger $n$\,–\,point functions converge uniquely along any van Hove diagonal. Since truncated cumulants are polynomial combinations of moments, they are continuous under convergence of moments of the required orders. Therefore
\[
  \langle \Xi(f)^4\rangle_c\ =\ \lim_{a\downarrow 0,\ L\to\infty}\ \langle \Xi_a(f)^4\rangle_c\ \ge\ c_0\ >\ 0.
\]
Taking $f_1=f_2=f_3=f_4=f$ yields the stated nonzero truncated 4\,–\,point in the continuum. The more general statement with possibly distinct $f_i$ follows by multilinearity and continuity from the case $f_1=\cdots=f_4$.
\end{proof}
\begin{proof}
By Proposition~\ref{prop:nonzero-cumulant4}, for fixed $R\Subset\mathbb R^4$ there exist $f\in C_c^\infty(R)$ and $a_1>0$ such that for all $a\in(0,a_1]$ and large $L$, the lattice cumulant satisfies $\langle \Xi_a(f)^4\rangle_c\ge c_0>0$ uniformly. UEI on fixed regions (Thm.~\ref{thm:uei-fixed-region} and Cor.~\ref{cor:uei-af-uniform}) yields uniform moment bounds, so cumulants, being polynomial in moments, are continuous under convergence in distribution on cylinders. Under AF/Mosco, the smeared local fields converge in $L^2$ on fixed regions (Lem.~\ref{lem:local-fields-tempered} and the embeddings in Assumption~\ref{assump:AF-Mosco}), hence $\langle \Xi(f)^4\rangle_c=\lim_{a\downarrow 0,L\to\infty}\langle \Xi_a(f)^4\rangle_c\ge c_0>0$.
\end{proof}
\paragraph{Clay-style constants checklist (for Theorem~\ref{thm:main-unconditional-af-free}).}
From the geometry pack (\S\ref{para:geometry-pack}): $\theta_*:=\kappa_0(R_*,a_0,N)\in(0,1]$, $t_0=t_0(N)>0$, $\lambda_1=\lambda_1(N)>0$. Two-layer deficit gives $\beta_0\ge 1-(\rho+S_0)>0$ with $\rho=(1-\theta_* e^{-\lambda_1 t_0})^{1/2}$ and $S_0= C_g(R_*) B(R_*,a_0,N)/(e^{\nu-\nu_0}-1)$ for some $\nu>\nu_0=\log 5$. Hence $c_{\rm cut}=-(1/a)\log(1-\beta_0)>0$ and $\gamma_*=8\,c_{\rm cut,phys}=8\big(-\log(1-\theta_* e^{-\lambda_1 t_0})\big)>0$, independent of $(\beta,L)$.

\medskip

\begin{corollary}[Global $\beta$- and volume-uniform mass-gap bound]\label{cor:global-uniform-gap}
Let $\theta_*:=\kappa_0(R_*,a_0,N)$ and $t_0:=t_0(N)$ be as in Proposition~\ref{prop:explicit-doeblin-constants}, and let $\lambda_1(N)$ be the first nonzero Laplace--Beltrami eigenvalue on $\mathrm{SU}(N)$. Define
\[
  c_{\rm cut,phys}\ :=\ -\log\big(1-\theta_* e^{-\lambda_1(N) t_0}\big),\qquad \gamma_*\ :=\ 8\,c_{\rm cut,phys}.
\]
Then, uniformly in the lattice spacing $a\in(0,a_0]$, volume $L$, and bare coupling $\beta\ge 0$ along the van Hove window, the continuum generator $H$ obtained by NRC and OS reconstruction satisfies
\[
  \operatorname{spec}(H)\subset\{0\}\cup[\gamma_*,\infty),\qquad \gamma_*\ >\ 0,
\]
with $\gamma_*$ depending only on $(R_*,a_0,N)$ via $(\theta_*,t_0,\lambda_1)$. In particular, the mass gap lower bound is $\beta$- and volume-uniform.
\begin{proof}
By Proposition~\ref{prop:explicit-doeblin-constants}, $K_{\rm int}^{(a)}\ge \theta_* P_{t_0}$ with $(\theta_*,t_0)$ independent of $(\beta,L,a)$. Corollary~\ref{cor:convex-split} then yields a one-step $L^2_0$ contraction by a factor $\le 1-\theta_* e^{-\lambda_1 t_0}$ on the odd cone; composing eight ticks gives a lattice mean-zero spectral radius $\le e^{-8 c_{\rm cut}}$ with $c_{\rm cut} = -(1/a)\log(1-\theta_* e^{-\lambda_1 t_0})$. Passing to the continuum via NRC (Theorems~\ref{thm:nrc-embeddings}, \ref{thm:nrc-operator-norm}) and gap persistence (Theorem~\ref{thm:gap-persist-cont}) transports the physical constant $\gamma_* = 8\,c_{\rm cut,phys}$ to the continuum spectrum. Uniformity in $(\beta,L)$ follows from the independence of $(\theta_*,t_0)$ and the volume-uniform NRC/thermodynamic-limit steps.
\end{proof}
\end{corollary}

\begin{theorem}[Global Minkowski mass gap (explicit constant, unconditional)]\label{thm:global-minkowski-gap}
Let $G=\mathrm{SU}(N)$, $N\ge 2$, and fix slab geometry parameters $(R_*,a_0)$. Let $\theta_*:=\kappa_0(R_*,a_0,N)$ and $t_0:=t_0(N)$ be the boundary-uniform Doeblin constants of Proposition~\ref{prop:explicit-doeblin-constants}, and let $\lambda_1(N)$ be the first nonzero Laplace--Beltrami eigenvalue on $\mathrm{SU}(N)$. Define
\[
  \gamma_{\mathrm{phys}}\ :=\ 8\,\Big(-\log\big(1-\theta_*\,e^{-\lambda_1(N)\,t_0}\big)\Big)\ >\ 0.
\]
For the global continuum OS measure constructed in Section~\ref{sec:global-R4}, let $H\ge 0$ be the Euclidean generator and $H_{\mathrm{Mink}}$ the Minkowski Hamiltonian obtained by OS$\to$Wightman. Then
\[
  \operatorname{spec}(H)\ =\ \{0\}\cup[\gamma_{\mathrm{phys}},\infty)\;,\qquad
  \operatorname{spec}(H_{\mathrm{Mink}})\ =\ \{0\}\cup[\gamma_{\mathrm{phys}},\infty)\;.
\]
Moreover, $\gamma_{\mathrm{phys}}$ is independent of the exhaustion/van Hove sequence, independent of boundary conditions, and independent of $(a,\beta,L)$ once expressed in physical units; it depends only on $(R_*,a_0,N)$ through $(\theta_*,t_0,\lambda_1)$.
\end{theorem}
\begin{proof}
By Proposition~\ref{prop:explicit-doeblin-constants}, the interface kernel satisfies $K_{\rm int}^{(a)}\ge \theta_* P_{t_0}$ with $(\theta_*,t_0)$ independent of $(a,\beta,L)$ and boundary conditions. Corollary~\ref{cor:convex-split} and Theorem~\ref{thm:uniform-odd-contraction} yield an $L^2$ one-tick contraction on the odd cone by a factor $\le 1-\theta_* e^{-\lambda_1 t_0}$, hence a per-eight-ticks contraction on the mean-zero subspace with rate $\gamma_{\mathrm{phys}}$. The thermodynamic limit at fixed $a$ preserves the bound and is boundary-independent (Proposition~\ref{prop:bc-robust}).

On fixed physical regions, AF-free NRC (Theorems~\ref{thm:nrc-embeddings}, \ref{thm:nrc-operator-norm}) and gap persistence (Theorem~\ref{thm:gap-persist-cont}) transfer the uniform bound to the continuum generator $H_R$, with constants unchanged. Consistency on overlaps (Proposition~\ref{prop:consistency-overlaps}) globalizes to the OS/GNS limit, giving $\operatorname{spec}(H)\subset\{0\}\cup[\gamma_{\mathrm{phys}},\infty)$. The reverse inclusion $[\gamma_{\mathrm{phys}},\infty)\subset \operatorname{spec}(H)$ follows from standard spectrum-closure and approximate-eigenvector arguments for positive contraction semigroups with sharp decay rate on $\Omega^{\perp}$.

Independence of the van Hove sequence and boundary follows from uniqueness of Schwinger limits on fixed regions (Proposition~\ref{prop:af-free-uniqueness}) and boundary robustness (Proposition~\ref{prop:bc-robust}). Independence of $(a,\beta,L)$ in physical units is encoded in the definition of $\gamma_{\mathrm{phys}}$, which uses the physical slab contraction constant $c_{\rm cut,phys}=-\log(1-\theta_* e^{-\lambda_1 t_0})$ and is geometric/group-theoretic only.

The OS$\to$Wightman reconstruction (Theorem~\ref{thm:os-to-wightman-global}) transports the gap to Minkowski without renormalizing the constant, hence $\operatorname{spec}(H_{\mathrm{Mink}})=\{0\}\cup[\gamma_{\mathrm{phys}},\infty)$.
\end{proof}
\subsection*{Local gauge\,--\,invariant fields: definition and temperedness}
\label{subsec:local-fields-tempered}
We now record an explicit local field algebra for the continuum theory and verify temperedness (OS0) for smeared local fields, ensuring the OS$\to$Wightman reconstruction applies to genuine local operators (not only Wilson loops).

\paragraph{Discretized local fields and smearings.}
Fix $\psi\in C_c^\infty(\mathbb R^4)$ and, for a lattice with spacing $a\in(0,a_0]$, define the scalar \emph{plaquette energy density} smearing
\[
  \Phi_a(\psi)\ :=\ a^4\sum_{p\in\mathcal P_a} \psi(x_p)\Bigl(1-\tfrac1N\,\Re\,\mathrm{Tr}\,U_p\Bigr),
\]
where $x_p$ is the geometric center of plaquette $p$. Likewise, for a smooth compactly supported two\,--\,form $\varphi\in C_c^\infty(\mathbb R^4,\wedge^2\mathbb R^4)$ and an $\mathfrak{su}(N)$\,--\,invariant inner product, define the gauge\,--\,invariant quadratic ``clover'' smearing
\[
  \Xi_a(\varphi)\ :=\ a^4\sum_{x\in a\mathbb Z^4} \sum_{\mu<\nu} \varphi_{\mu\nu}(x)\,\Bigl(1-\tfrac1N\,\Re\,\mathrm{Tr}\,U^{\text{clov}}_{\mu\nu}(x)\Bigr),
\]
where $U^{\text{clov}}_{\mu\nu}(x)$ is the standard four\,--\,plaquette clover around $x$ in the $\mu\nu$\,--\,plane. Both are local gauge\,--\,invariant lattice observables supported in $\operatorname{supp}\psi$ or $\operatorname{supp}\varphi$.
\begin{lemma}[Local gauge\,--\,invariant fields are tempered distributions]
\label{lem:local-fields-tempered}
Along any van Hove scaling sequence $(a_k,L_k)$, for each fixed $\psi\in C_c^\infty(\mathbb R^4)$ and $\varphi\in C_c^\infty(\mathbb R^4,\wedge^2\mathbb R^4)$ the families $\{\Phi_{a_k}(\psi)\}$ and $\{\Xi_{a_k}(\varphi)\}$ are Cauchy in $L^2$ under $\mu_{a_k,L_k}$ and converge in $L^2(\mu)$ to random variables $\Phi(\psi)$ and $\Xi(\varphi)$. The maps $\psi\mapsto\Phi(\psi)$ and $\varphi\mapsto\Xi(\varphi)$ extend by density to continuous linear functionals on $\mathcal S(\mathbb R^4)$ and $\mathcal S(\mathbb R^4,\wedge^2\mathbb R^4)$, respectively. In particular, $\Phi$ and $\Xi$ are (vector\,--\,valued) tempered distributions and generate a local gauge\,--\,invariant field algebra in the OS framework.
\end{lemma}

\subsection*{Operator domains, common cores, and BRST}

\paragraph{Common invariant core for local operators.}
Let $\mathfrak A_0$ denote the time\,–\,zero cylinder \(*\,–\,algebra) generated by gauge\,–\,invariant local observables (Wilson loops and smeared clover fields) supported in bounded regions. Define the local polynomial domain
\[
  \mathcal D_{\rm loc}\ :=\ \mathrm{span}\,\big\{\, P\big(\{\Phi(f_i)\},\{\Xi(\varphi_j)\}\big)\,\Omega\ :\ P\ \text{polynomial with complex coefficients},\ f_i\in C_c^\infty(\mathbb R^4),\ \varphi_j\in C_c^\infty(\mathbb R^4, \wedge^2\mathbb R^4)\,\big\}\,.
\]

\begin{lemma}[Density and invariance of $\mathcal D_{\rm loc}$]\label{lem:field-core}
The subspace $\mathcal D_{\rm loc}$ is dense in the OS/GNS Hilbert space and is invariant under:
\begin{itemize}
  \item[(i)] Euclidean time translations $e^{-tH}$ for all $t\ge 0$;
  \item[(ii)] the spatial Euclidean group (by OS1 and Lemma~\ref{lem:group-avg});
  \item[(iii)] local gauge transformations acting unitarily on time\,–\,zero variables.
\end{itemize}
\end{lemma}
\begin{proof}
By OS0 (Proposition~\ref{prop:OS0-poly}, Corollary~\ref{cor:os0-explicit-4d}), local cylinders have finite moments of all orders; polynomials applied to $\Omega$ are therefore in $\mathcal H$ and their span is dense. The semigroup $e^{-tH}$ maps time\,–\,zero cylinders to cylinders by OS reconstruction and domain invariance; Euclidean invariance holds by OS1; local gauge transformations act isometrically on cylinders and preserve the reflection cone, hence induce unitaries on $\mathcal H$ that leave $\mathcal D_{\rm loc}$ invariant.
\end{proof}

\paragraph{Closability and graph bounds for smeared fields.}
Define $\Phi(f)$ and $\Xi(\varphi)$ on $\mathcal D_{\rm loc}$ by $L^2$\,–\,limits of the lattice approximants (Lemma~\ref{lem:local-fields-tempered}).

\begin{proposition}[Field closability and relative graph bounds]\label{prop:field-closability}
There exist constants $C_\Phi(f),C_\Xi(\varphi)$ such that for all $\psi\in \mathcal D_{\rm loc}$,
\[
  \|\Phi(f)\psi\|\ \le\ C_\Phi(f)\,\big\|\,(H+1)^{1/2}\psi\,\big\|\,,\qquad
  \|\Xi(\varphi)\psi\|\ \le\ C_\Xi(\varphi)\,\big\|\,(H+1)^{1/2}\psi\,\big\|\,.
\]
Consequently $\Phi(f)$ and $\Xi(\varphi)$ are closable on $\mathcal D_{\rm loc}$, and their closures have $\mathcal D_{\rm loc}$ as a core.
\end{proposition}
\begin{proof}
On the lattice, OS positivity and locality give the standard energy bound $\|O\psi\|\le C\|(H_{a,L}+1)^{1/2}\psi\|$ for local $O$ on the time\,–\,zero cone. Passing to the limit by AF\,–\,free NRC (Theorems~\ref{thm:nrc-embeddings}, \ref{thm:nrc-operator-norm}) and using Lemma~\ref{lem:graph-defect-Oa} yields the stated bounds with constants depending on the supports of $f,\varphi$ and group data only. Closability follows since $\mathcal D_{\rm loc}$ is a core for $(H+1)^{1/2}$ and the estimates are graph\,–\,bounded.
\end{proof}

\paragraph{BRST charge.}
Let $\mathcal G_0$ be the group of compactly supported time\,–\,zero gauge transformations. For a smooth Lie algebra test function $\alpha$ supported in a bounded region, define on $\mathcal D_{\rm loc}$ the derivation $\delta_\alpha$ by its action on generators (Wilson loops/clover fields) via the infinitesimal adjoint action and extend as a graded derivation.

\begin{definition}[BRST charge on $\mathcal D_{\rm loc}$]\label{def:brst}
The BRST charge $Q$ is the closable operator on $\mathcal D_{\rm loc}$ defined by $\langle\psi, Q\phi\rangle:=\frac{d}{ds}\big|_{s=0}\langle\psi, U(e^{s\alpha})\phi\rangle$ for a fixed dense set of test functions $\alpha$ whose linear span is dense in the Lie algebra of $\mathcal G_0$; we set $Q\phi:=\delta_\alpha\phi$ on generators and extend by linearity and closure. Different choices of spanning families yield the same closed operator.
\end{definition}

\begin{proposition}[Closability, nilpotency, and core for $Q$]\label{prop:brst-closable}
The BRST charge $Q$ defined on $\mathcal D_{\rm loc}$ is closable; its closure (denoted again $Q$) satisfies $Q^2=0$ on $\mathcal D_{\rm loc}$ and leaves $\mathcal D_{\rm loc}$ invariant. Moreover, for all $\psi\in\mathcal D_{\rm loc}$,
\[
  \|Q\psi\|\ \le\ C_Q\,\big\|\,(H+1)^{1/2}\psi\,\big\|\,,
\]
with a constant $C_Q$ depending only on the support radius and group constants; hence $\mathcal D_{\rm loc}$ is a core for $Q$.
\end{proposition}
\begin{proof}
Unitary implementation of $\mathcal G_0$ on $\mathcal H$ implies that the generators of one\,–\,parameter subgroups are skew\,–\,adjoint on their natural domains; on $\mathcal D_{\rm loc}$ this coincides with the derivation $\delta_\alpha$. The energy bound follows as in Proposition~\ref{prop:field-closability} from locality and UEI. Nilpotency $Q^2=0$ on $\mathcal D_{\rm loc}$ is the Lie\,–\,algebra identity for gauge variations on gauge\,–\,invariant generators (graded Jacobi). Closability follows from the graph bound and density of $\mathcal D_{\rm loc}$.
\end{proof}

\begin{proposition}[Physical Hilbert space]\label{prop:physical-hilbert}
Let $\mathcal H_{\rm phys}:=\ker Q\big/\overline{\operatorname{ran} Q}$ with the induced inner product. Then $\mathcal H_{\rm phys}$ is a Hilbert space carrying the gauge\,–\,invariant observable net; in particular, for any gauge\,–\,invariant local $O$ with $O\mathcal D_{\rm loc}\subset\mathcal D_{\rm loc}$, the induced operator on $\mathcal H_{\rm phys}$ is well\,–\,defined and symmetric on the image of $\mathcal D_{\rm loc}$.
\end{proposition}
\begin{proof}
Standard homological argument: $Q$ is closable and nilpotent on a common core; the quotient by $\overline{\operatorname{ran} Q}$ removes $Q$\,–\,exact components. Gauge\,–\,invariant local observables commute with the gauge action on $\mathcal D_{\rm loc}$, hence preserve $\ker Q$ and map $\operatorname{ran} Q$ to itself; the induced action is well\,–\,defined and symmetric by OS positivity.
\end{proof}
\begin{proof}
Fix a bounded region $R\supset \operatorname{supp}\psi\cup\operatorname{supp}\varphi$. By Uniform Exponential Integrability on fixed regions (Theorem~\ref{thm:uei-fixed-region}), there exists $\eta_R>0$ with $\sup_{(a,L)} \mathbb E[e^{\eta_R S_R}]<\infty$. By standard duality between exponential moments and polynomial moments, this implies uniform bounds $\sup_{(a,L)}\mathbb E\big[|\Phi_a(\psi)|^p+|\Xi_a(\varphi)|^p\big]<\infty$ for all $p<\infty$, with constants depending only on $R$ and Schwartz norms of the test functions (via Proposition~\ref{prop:OS0-poly} and Corollary~\ref{cor:os0-explicit-4d}).
Let $k<\ell$. Partition $R$ into cubes of side comparable to $a_k$ and $a_\ell$. A standard block averaging/telescoping argument expresses $\Phi_{a_\ell}(\psi)-\Phi_{a_k}(\psi)$ as a sum of local increments supported in slightly enlarged cubes, each controlled in $L^2$ by the uniform moment bounds and the uniform exponential clustering on fixed regions. Summing the decaying covariances yields
\[
  \sup_L\,\mathbb E\big|\Phi_{a_\ell}(\psi)-\Phi_{a_k}(\psi)\big|^2\ \longrightarrow\ 0\quad \text{as }k,\ell\to\infty,
\]
so $\{\Phi_{a_k}(\psi)\}_k$ is Cauchy in $L^2$. The same argument applies to $\Xi_{a_k}(\varphi)$. Denote the limits by $\Phi(\psi)$ and $\Xi(\varphi)$.

For $\psi\in C_c^\infty$, the maps $\psi\mapsto \Phi(\psi)$ are linear by construction. The uniform OS0 polynomial bounds control $|\Phi(\psi)|$ by a finite sum of seminorms of $\psi$ (Schwartz norms obtained by mollifying compact support), implying continuity of $\Phi$ on $\mathcal S(\mathbb R^4)$. Density of $C_c^\infty$ in $\mathcal S$ extends $\Phi$ uniquely; likewise for $\Xi$. Therefore $\Phi$ and $\Xi$ define tempered distributions. Locality and reflection positivity for polynomials in $\Phi,\Xi$ follow from those of their lattice approximants by Proposition~\ref{prop:os0os2-closure}.
\end{proof}

\begin{corollary}[OS axioms for local fields]
\label{cor:os-local-fields}
The Schwinger functions of the smeared local fields $\Phi,\Xi$ satisfy OS0\,--\,OS5. Consequently, Theorem~\ref{thm:os-to-wightman} applies with $\mathcal A$ taken to be the polynomial *\,--\,algebra generated by $\{\Phi(\psi),\Xi(\varphi)\}$, and the resulting Wightman theory carries local gauge\,--\,invariant fields with the same mass gap $\ge \gamma_*$.
\end{corollary}

\section{Continuum gauge symmetry, Gauss law, and BRST}
\label{sec:gauge-brst}

We now give an unconditional construction of the continuum local gauge symmetry, Gauss-law generators, Ward identities, and (optional) BRST cohomology, and verify that the local gauge\,--\,invariant Wightman fields exist as operator\,--\,valued distributions on a common invariant core.

\subsection{Unitary implementation of the local gauge group}

Let $\mathcal G_0:=C_c^\infty(\mathbb R^3,\mathrm{SU}(N))$ denote the time\,–\,zero local gauge group, acting on time\,–\,zero lattice observables by the usual edge/vertex conjugations and on Wilson loops by conjugation at a basepoint (which cancels in the trace). This action extends by locality to the OS cylinder algebra.

\begin{theorem}[Unitary representation of $\mathcal G_0$]\label{thm:unitary-gauge}
There exists a strongly continuous unitary representation $U: \mathcal G_0\to \mathsf U(\mathcal H_{\mathrm{OS}})$ on the global OS/GNS Hilbert space such that for any time\,–\,zero local observable $O$ and $g\in\mathcal G_0$,
\[
  U(g)\,[O]\,U(g)^{-1}\ =\ [\,g\cdot O\,],\qquad U(g)\,\Omega\ =\ \Omega.
\]
Moreover, for any smooth one\,–\,parameter family $g_s=\exp(s\xi)$ with $\xi\in C_c^\infty(\mathbb R^3,\mathfrak{su}(N))$, the map $s\mapsto U(g_s)$ is strongly continuous on the time\,–\,zero local core.
\end{theorem}
\begin{proof}
On each finite lattice, invariance of the Haar measure under local gauge transformations implies $\langle \Theta(O_1) O_2\rangle=\langle \Theta(g\cdot O_1)\, (g\cdot O_2)\rangle$, hence the OS inner product is invariant. Therefore each $g$ induces an isometry on the lattice OS/GNS space which fixes the vacuum. By continuity in the cylinder topology and embedding\,–\,independence (Proposition~\ref{prop:embedding-independence}), these isometries are compatible along van Hove limits and define $U(g)$ on the global OS/GNS space. Unitarity follows since $g\mapsto g^{-1}$ yields the inverse action. Strong continuity for $g_s$ on the time\,–\,zero core follows from UEI, OS0 equicontinuity (Lemma~\ref{lem:eqc-modulus}), and dominated convergence applied to matrix elements $\langle \Theta(O_1)\, (g_s\cdot O_2)\rangle$.
\end{proof}

\subsection{Gauss\,–\,law generators and physical subspace}

\begin{theorem}[Self\,–\,adjoint Gauss generators]\label{thm:gauss-generators}
For each $\xi\in C_c^\infty(\mathbb R^3,\mathfrak{su}(N))$ there exists a self\,–\,adjoint operator $G(\xi)$ with domain containing the time\,–\,zero local core such that
\[
  U(\exp(s\xi))\ =\ e^{\,i s G(\xi)}\quad (s\in\mathbb R),\qquad G(\xi)\,\Omega\ =\ 0,
\]
and for any time\,–\,zero local observable $O$,
\[
  i\,[\,G(\xi),\,[O]_{\mathrm{OS}}\,]\ =\ \big[\,(\delta_\xi O)\,\big]_{\mathrm{OS}},
\]
where $\delta_\xi$ is the infinitesimal gauge variation. The map $\xi\mapsto G(\xi)$ is a representation of the Lie algebra $C_c^\infty(\mathbb R^3,\mathfrak{su}(N))$.
\end{theorem}
\begin{proof}
By Theorem~\ref{thm:unitary-gauge}, $s\mapsto U(\exp(s\xi))$ is a strongly continuous one\,–\,parameter unitary group on a dense invariant core, so Stone's theorem yields a (essentially) self\,–\,adjoint generator $G(\xi)$ with the stated exponential. Vacuum invariance gives $G(\xi)\Omega=0$. The commutator identity is obtained by differentiating $s\mapsto U(\exp(s\xi))\,[O]\,U(\exp(-s\xi))$ at $s=0$ on the core. The Lie homomorphism property follows by standard properties of unitary representations.
\end{proof}

\begin{definition}[Physical subspace]
Define $\mathcal H_{\mathrm{phys}}:=\{\psi\in\mathcal H_{\mathrm{OS}}:\ U(g)\psi=\psi\ \forall g\in\mathcal G_0\}$, equivalently $\mathcal H_{\mathrm{phys}}=\bigcap_{\xi} \ker G(\xi)$ (closure understood). Denote by $\mathcal A_{\mathrm{phys}}$ the OS/GNS algebra generated by gauge\,–\,invariant time\,–\,zero local observables.
\end{definition}

\begin{theorem}[Gauss law and gauge\,–\,invariant algebra]\label{prop:gauss-phys}
The vacuum $\Omega\in\mathcal H_{\mathrm{phys}}$. The physical subspace is the closure of $\mathcal A_{\mathrm{phys}}\,\Omega$. For any $O\in\mathcal A_{\mathrm{phys}}$ and any $\xi$, one has $[G(\xi),[O]]=0$.
\end{theorem}
\begin{proof}
Vacuum invariance is from Theorem~\ref{thm:unitary-gauge}. If $O$ is gauge invariant, then $g\cdot O=O$ and $U(g)[O]U(g)^{-1}=[O]$, so $[O]\Omega\in\mathcal H_{\mathrm{phys}}$; density follows because $\mathcal A_{\mathrm{phys}}\,\Omega$ is cyclic for the gauge\,–\,invariant OS algebra. The commutator statement follows from the differentiated covariance identity in Theorem~\ref{thm:gauss-generators} with $\delta_\xi O=0$.
\end{proof}

\subsection{Ward identities (continuum, nonabelian)}

\begin{theorem}[Nonabelian Ward identities]\label{thm:ward}
For any smooth compactly supported $\xi$ and any time\,–\,ordered product of time\,–\,zero local gauge\,–\,invariant observables $O_1,\dots,O_n$ with smooth time translations, one has
\[
  \sum_{k=1}^n \big\langle\, O_1\cdots (\delta_\xi O_k)\cdots O_n\,\big\rangle\ =\ 0,
\]
in the continuum limit, with convergence uniform on compact families of smearings. Equivalently, for the OS/GNS commutators,
\[
  \sum_{k=1}^n \langle\,\Omega,\ O_1\cdots i[ G(\xi), O_k ]\cdots O_n\,\Omega\rangle\ =\ 0.
\]
\end{theorem}
\begin{proof}
On each finite lattice, the identity follows from invariance of the Haar measure and change\,–\,of\,–\,variables under local gauge transformations, differentiating at the identity in $\mathcal G_0$ (lattice Ward identity). UEI and OS0 bounds yield uniform integrability for passing to the continuum; embedding\,–\,independence and boundary robustness (Proposition~\ref{prop:bc-robust}) ensure that the differentiated identities converge along van Hove nets to the stated continuum identity. The commutator form is the OS/GNS rewriting using Theorem~\ref{thm:gauss-generators}.
\end{proof}

\subsection{Local gauge\,–\,invariant Wightman fields as operator\,–\,valued distributions}

\begin{theorem}[Closability and common core]\label{thm:ovd}
Let $\mathcal D_{\mathrm{loc}}$ be the algebraic span of vectors of the form $[O]$ with $O$ a time\,–\,zero local gauge\,–\,invariant observable. For each test function $\varphi\in C_c^\infty(\mathbb R^4)$, the smeared local fields $\Phi(\varphi),\Xi(\varphi)$ define closable operators on $\mathcal D_{\mathrm{loc}}$, with $\mathcal D_{\mathrm{loc}}$ a common invariant core for all such smearings. The maps $\varphi\mapsto \Phi(\varphi)$ and $\varphi\mapsto \Xi(\varphi)$ are continuous from $\mathcal S(\mathbb R^4)$ into the space of operators on $\mathcal D_{\mathrm{loc}}$ endowed with the strong graph topology.
\end{theorem}
\begin{proof}
OS0 polynomial bounds and UEI yield moment estimates of all orders for time\,–\,zero local observables on fixed regions; by time translation and semigroup bounds, the same holds for time\,–\,translated smearings. Nelson's analytic vector criterion then gives essential self\,–\,adjointness/closability on the polynomial core generated by $\mathcal A_{\mathrm{phys}}$ acting on $\Omega$. Continuity in $\varphi$ follows from Lemma~\ref{lem:eqc-modulus} and dominated convergence.
\end{proof}

\subsection{Optional: BRST cohomology equals Gauss\,–\,law invariants}

While the construction above avoids ghosts and gauge fixing, one can introduce a standard BRST differential to encode the local gauge symmetry cohomologically.

\begin{theorem}[BRST cohomology at ghost number zero]\label{thm:brst}
Let $\mathcal F_{\mathrm{tot}}$ be the graded *\,–\,algebra generated by the (time\,–\,zero) local gauge\,–\,variant fields together with free ghost fields $c,\bar c$ (CAR) and Nakanishi\,–\,Lautrup field $b$, with the usual BRST derivation $\mathsf s$ implementing the $\mathfrak{su}(N)$ Lie algebra on fields. Then there is a densely defined closed operator $Q$ on a graded Hilbert space extending $\mathcal H_{\mathrm{OS}}\otimes \mathcal H_{\mathrm{gh}}$ such that $Q^2=0$, $i[Q,\cdot]=\mathsf s(\cdot)$ on a common core, and the cohomology at ghost number zero satisfies
\[
  H^0(Q)\ \cong\ \overline{\mathcal A_{\mathrm{phys}}\,\Omega}\ \subset\ \mathcal H_{\mathrm{OS}}.
\]
In particular, physical vectors/states are identified with the gauge\,–\,invariant ones constructed above, and the mass gap is unchanged.
\end{theorem}
\begin{proof}
By Theorem~\ref{thm:gauss-generators}, the local gauge Lie algebra is represented by the self\,–\,adjoint charges $G(\xi)$. The Chevalley\,–\,Eilenberg construction yields a nilpotent differential $\mathsf s$ on $\mathcal F_{\mathrm{tot}}$; define $Q$ on the graded tensor product core by the Kugo\,–\,Ojima prescription using $G(\xi)$ and ghost creation/annihilation operators. Nilpotency $Q^2=0$ reflects the Lie algebra relations. The cohomology at ghost number zero identifies with the invariants under $G(\xi)$, hence with $\mathcal H_{\mathrm{phys}}$ by Proposition~\ref{prop:gauss-phys}. Since ghosts decouple from $\mathcal A_{\mathrm{phys}}$, the mass gap on $\mathcal H_{\mathrm{phys}}$ is the same as in Theorem~\ref{thm:global-gap-uncond}.
\end{proof}

\begin{corollary}[Microcausality for smeared gauge\,--\,invariant fields]
\label{cor:microcausality}
Let $f,g\in C_c^\infty(\mathbb R^4)$ have spacelike separated supports. Then the Wightman fields obtained from $\Phi,\Xi$ via OS reconstruction satisfy
\[
  [\Phi(f),\Phi(g)]\,=\,0,\quad [\Phi(f),\Xi(\eta)]\,=\,0,\quad [\Xi(\omega),\Xi(\eta)]\,=\,0
\]
whenever all test functions are pairwise spacelike separated. In particular, the local gauge\,--\,invariant field algebra obeys locality.
\end{corollary}

\begin{proof}
OS0\,--\,OS5 imply the Wightman axioms under Theorem~\ref{thm:os-to-wightman}. Locality (microcausality) holds for smeared fields with spacelike separated supports by the standard OS\,$\to$\,Wightman locality theorem. Since $\Phi,\Xi$ are limits of local gauge\,--\,invariant lattice observables, their smeared versions generate local operators; therefore the commutators vanish at spacelike separation.
\end{proof}

\begin{lemma}[Nontriviality: positive variance of a smeared local field]\label{lem:nontrivial-variance}
Fix a nonzero $\varphi\in C_c^\infty(\mathbb R^4,\wedge^2\mathbb R^4)$ supported in a bounded region $R\Subset\mathbb R^4$. Along any van Hove scaling sequence $(a_k,L_k)$, the smeared clover field satisfies
\[
  \operatorname{Var}_\mu\big(\Xi(\varphi)\big)\ >\ 0.
\]
Moreover, there exists $c_R(\varphi)>0$ depending only on $(R,a_0,N,\varphi)$ such that for all $k$ large and all volumes $L_k$ in the window,
\[
  \operatorname{Var}_{\mu_{a_k,L_k}}\big(\Xi_{a_k}(\varphi)\big)\ \ge\ c_R(\varphi),
\]
and hence the positive variance persists in the continuum limit.
\end{lemma}

\begin{proof}
Write the lattice smeared observable as $\Xi_a(\varphi)=a^4\sum_{x\in a\mathbb Z^4\cap R} \sum_{\mu<\nu} \varphi_{\mu\nu}(x)\,\mathrm{clov}^{(a)}_{\mu\nu}(x)$. Each clover average obeys $0\le \mathrm{clov}^{(a)}_{\mu\nu}(x)\le 2$ and depends nontrivially (continuously) on finitely many interface links. By Lemma~\ref{lem:abs-cont}, the joint law of the interface after one tick has a strictly positive continuous density, and by Proposition~\ref{prop:doeblin-full} it dominates a product heat kernel on $G^m$. Therefore the distribution of $\Xi_a(\varphi)$ is non-degenerate on every finite volume, yielding $\operatorname{Var}_{\mu_{a,L}}(\Xi_a(\varphi))>0$.

Uniform Exponential Integrability on fixed $R$ (Theorem~\ref{thm:uei-fixed-region}) and locality ensure that small-ball refresh/heat-kernel domination occurs with probability bounded below uniformly in $(\beta,L)$ on the slab; by continuity of $\Xi_a(\varphi)$ in the interface variables, this gives a uniform variance lower bound $c_R(\varphi)>0$ for all sufficiently small $a\le a_0$ and large $L$.

Finally, by Lemma~\ref{lem:local-fields-tempered} and Corollary~\ref{cor:unique-schwinger-local}, $\Xi_{a_k}(\varphi)\to \Xi(\varphi)$ in $L^2$ and the Schwinger limits are unique, so variance is lower semicontinuous under the limit. Hence $\operatorname{Var}_\mu(\Xi(\varphi))\ge \limsup_k \operatorname{Var}_{\mu_{a_k,L_k}}(\Xi_{a_k}(\varphi))\ge c_R(\varphi)>0$.
\end{proof}

\section*{Appendix: Constants and References Index}
\begin{itemize}
  \item \textbf{Constants.} $\lambda_1(N)$: first nonzero Laplace--Beltrami eigenvalue on $\mathrm{SU}(N)$; $t_0>0$, $\theta_*>0$, $\kappa_0>0$: interface Doeblin/heat-kernel constants depending only on $(R_*,a_0,N)$; $c_{\mathrm{cut}}(a):=-(1/a)\log(1-\theta_* e^{-\lambda_1 t_0})$; $c_{\mathrm{cut,phys}}:= -\log(1-\theta_* e^{-\lambda_1 t_0})$; $\gamma_{\mathrm{cut}}:=8\,c_{\mathrm{cut}}(a)$; $\gamma_*:=8\,c_{\mathrm{cut,phys}}$.
  \item \textbf{OS positivity (OS2) and transfer.} Osterwalder--Schrader \cite{Osterwalder1973,Osterwalder1975}; Osterwalder--Seiler \cite{OsterwalderSeiler1978} (Wilson gauge theory); Montvay--M\"unster \cite{MontvayMunster1994}.
  \item \textbf{Heat-kernel and convolution smoothing on compact groups.} Diaconis--Saloff-Coste \cite{DiaconisSaloffCoste2004}; Varopoulos--Saloff-Coste--Coulhon \cite{VaropoulosSaloffCosteCoulhon1992}.
  \item \textbf{UEI, LSI, and cluster/Herbst.} Brydges \cite{Brydges1978,Brydges1986}; Holley--Stroock and Bakry--\'Emery techniques on compact manifolds; Kolmogorov--Chentsov criterion.
  \item \textbf{Resolvent comparison and spectral stability.} Kato \cite{Kato1995} (norm--resolvent convergence; spectral lower semicontinuity); Riesz projections; semigroup theory (Engel--Nagel \cite{EngelNagel2000}).
  \item \textbf{Probability compactness and extensions.} Prokhorov compactness; Daniell--Kolmogorov extension theorem.
  \item \textbf{Markov contractions.} Dobrushin \cite{Dobrushin1970} (total-variation contraction coefficients and spectral consequences in finite dimension).
  \item \textbf{Labels (this manuscript).} Interface Doeblin: Proposition in Appendix "Uniform two--layer Gram deficit on the odd cone"; UEI: Theorem~\ref{thm:uei-fixed-region}; OS0/OS2 closure: Proposition~\ref{prop:os0os2-closure}; OS1: Theorem~\ref{thm:os1-euclid}; NRC: Theorem~\ref{thm:nrc-embeddings}; Gap persistence: Theorem~\ref{thm:gap-persist-cont}; OS$\to$Wightman: Theorem~\ref{thm:os-to-wightman}; Main: Theorem~\ref{thm:main-unconditional-af-free}.
\end{itemize}

\paragraph{Geometry pack (constant dependencies; $\beta/L$ independence).}\label{para:geometry-pack}
We summarize the constant schema and dependencies used throughout. Fix a physical slab radius $R_*>0$, a maximal tick $a_0>0$, and the gauge group $\mathrm{SU}(N)$.
\begin{itemize}
  \item \textbf{Group data.} $\lambda_1(N)$: spectral gap of the Laplace--Beltrami operator on $\mathrm{SU}(N)$.
  \item \textbf{Interface/Doeblin constants.} From Proposition~\ref{prop:doeblin-full} and Lemma~\ref{lem:beta-L-independent-minorization}:
  $t_0=t_0(N)>0$, $\theta_*:=\kappa_0(R_*,a_0,N)\in(0,1]$, independent of $(\beta,L)$. The lower bound arises from: (i) a boundary-uniform refresh mass $\alpha_{\rm ref}(R_*,a_0,N)>0$ on the slab (Lemma~\ref{lem:refresh-prob}); (ii) convolution lower bounds by heat kernel at time $t_0(N)$ (Lemma~\ref{lem:ball-conv-lower}); and (iii) a geometry factor $c_{\rm geo}(R_*,a_0)\in(0,1]$ from cell factorization. No step uses the value of $\beta$ other than $\beta\ge 0$.
  \item \textbf{Cut contraction.} $c_{\rm cut}(a)=-(1/a)\log(1-\theta_* e^{-\lambda_1 t_0})$; physical $c_{\rm cut,phys}=-\log(1-\theta_* e^{-\lambda_1 t_0})$.
  \item \textbf{Odd-cone contraction constants.} From Proposition~\ref{prop:int-to-transfer} and Corollary~\ref{cor:odd-contraction-from-Kint}:
  $\theta_*\in(0,1]$, $t_0>0$ depend only on $(R_*,a_0,N)$; on $L^2_0$,
  $\|K_{\rm int}^{(a)}\|\le 1-\theta_* e^{-\lambda_1 t_0}$, hence on the slab--odd cone, $\|e^{-aH}\|\le 1-\theta_* e^{-\lambda_1 t_0}$, and $c_{\rm cut}(a)=-(1/a)\log(1-\theta_* e^{-\lambda_1 t_0})$.
  \item \textbf{Gap constants.} Lattice per-tick: $\|e^{-aH}\|_{\rm odd}\le 1-\theta_* e^{-\lambda_1 t_0}\le e^{-a c_{\rm cut}}$ with $c_{\rm cut}=-(1/a)\log(1-\theta_* e^{-\lambda_1 t_0})$; eight ticks yield $\gamma_{\rm cut}=8 c_{\rm cut}$. Continuum: by operator-norm NRC and persistence, $\operatorname{spec}(H)\subset\{0\}\cup[\gamma_*,\infty)$ with $\gamma_*=8 c_{\rm cut,phys}$.
  \item \textbf{UEI/OS0 constants.} From Theorem~\ref{thm:uei-fixed-region} and Proposition~\ref{prop:OS0-poly}: $\eta_R, C_R$ (depend only on $(R,a_0,N)$), and polynomial OS0 constants on fixed regions.
  \item \textbf{NRC/embedding constants.} From Theorems~\ref{thm:nrc-embeddings}, \ref{thm:nrc-quant} and Lemmas~\ref{lem:graph-defect-Oa}, \ref{lem:low-energy-proj}: defect bound $C_{\rm gd}$, low-energy projector control $C_\Lambda$, and resolvent rate $C(z_0,\Lambda)$.
\end{itemize}

\paragraph{OS0/OS2 under limits (closure by UEI).}
The UEI bound yields tightness of gauge--invariant cylinders on $R$ (Prokhorov). Reflection positivity (OS2) is closed under weak limits of cylinder measures (bounded, continuous functional $F\mapsto \Theta F\,\overline{F}$). Temperedness/equicontinuity (OS0) follows from uniform Laplace bounds and the Kolmogorov--Chentsov criterion on loop holonomies (as in Proposition "OS0 (temperedness) with explicit constants"). Thus OS0 and OS2 persist along any scaling sequence.

\begin{lemma}[Cylinder measurability and projective limit]\label{lem:cylinder-projective}
Let $\{(a,L)\}$ be a directed net of lattices with spacings $a\in(0,a_0]$ and torus sizes $La\to\infty$. For a fixed bounded region $R\Subset\mathbb R^4$, let $\mathcal C_R$ denote the finite family of gauge--invariant loop variables and clover smearings supported in $R$ obtained from polygonal embeddings at mesh $\le a$. Then:
\begin{itemize}
  \item[(i)] (Measurability) Each element of $\mathcal C_R$ is Borel measurable with respect to the product Haar $\sigma$--algebra on links; the mapping $U\mapsto (O(U))_{O\in\mathcal C_R}$ is continuous on the compact configuration space.
  \item[(ii)] (Consistency) If $(a',L')\succeq(a,L)$ and the embeddings are chosen compatibly, then the pushforward of $\mu_{a',L'}$ to the $\sigma$--algebra generated by $\mathcal C_R$ coincides with the pushforward of $\mu_{a,L}$.
  \item[(iii)] (Tightness) Under UEI on $R$, the family of laws of $(O)_{O\in\mathcal C_R}$ is tight and uniformly exponentially integrable.
\end{itemize}
Consequently, by Prokhorov and Daniell--Kolmogorov, there exists a unique Borel probability measure on the projective limit of cylinder spaces whose finite--dimensional marginals agree with the lattice laws, yielding a continuum Euclidean measure on loop/local--field cylinders.
\end{lemma}
\begin{proof}
(i) Each loop variable is a finite product of link variables followed by a continuous class function (trace), hence Borel; clover smearings are finite averages of plaquette energies, hence continuous. (ii) Equivariant embeddings of loops/clovers and the link--marginal consistency of the Wilson measure imply consistency. (iii) UEI provides uniform exponential moments for any finite collection in $R$; on a compact space this implies tightness. Existence and uniqueness of the projective--limit measure then follow from Prokhorov compactness and the Daniell--Kolmogorov extension theorem for consistent finite--dimensional distributions.
\end{proof}
\begin{corollary}[Continuum measure on loop/local cylinders]\label{cor:continuum-measure-exists}
Along any van Hove scaling sequence, there exists a Borel probability measure $\mu$ on the cylinder $\sigma$--algebra generated by loop variables and local clover smearings on all bounded regions $R\Subset\mathbb R^4$, such that for every finite family of cylinder observables the expectations coincide with the lattice limits.
\end{corollary}
\paragraph{Thermodynamic limit note.}
At fixed spacing, the infinite-volume OS state exists by standard compactness arguments (tightness of local observables and diagonal extraction), and the gap/clustering persist by volume-uniform bounds; see, e.g., Kato \cite{Kato1995} for spectral stability and standard OS/GNS semigroup arguments for clustering.
\section{Clay compliance checklist}
\subsection*{Clay compliance map (requirements \textrightarrow{} labels)}
\begin{itemize}
  \item \textbf{OS0 (temperedness)}: Prop.~\ref{prop:OS0-poly}, Cor.~\ref{cor:os0-explicit-4d}; UEI on fixed regions with $\beta\ge \beta_{\min}(R,N)$: Thm.~\ref{thm:uei-fixed-region}, Cor.~\ref{cor:uei-explicit-constants}; closure: Prop.~\ref{prop:os0os2-closure}.
  \item \textbf{OS1 (Euclidean invariance)}: Thm.~\ref{thm:os1-euclid}; supporting lemmas: \ref{lem:isotropy-restore}, \ref{lem:os1-embedding}, Cor.~\ref{cor:os1-rotations}.
  \item \textbf{OS2 (reflection positivity)}: Thm.~\ref{thm:os} (Wilson link reflection); closure to limit: Cor.~\ref{cor:os2-pass} (from Prop.~\ref{prop:os0os2-closure}).
  \item \textbf{OS3/OS5 (clustering, unique vacuum)}: Lattice: Thm.~\ref{thm:thermo}, Thm.~\ref{thm:thermo-strong}; Continuum: Prop.~\ref{prop:os35-limit}; Gap\,$\Rightarrow$\,clustering: Prop.~\ref{prop:gap-to-cluster}; converse: Prop.~\ref{prop:cluster-to-gap}.
  \item \textbf{OS\,$\to$\,Wightman and Poincar\'e}: Thm.~\ref{thm:os-to-wightman}; Euclidean isotropy restoration: Lem.~\ref{lem:isotropy-restoration}; Cor.~\ref{cor:poincare}.
  \item \textbf{Mass gap (lattice)}: Strong-coupling route: Thm.~\ref{thm:gap}, Prop.~\ref{prop:dob-spectrum}, Lem.~\ref{lem:dob-influence}; Odd-cone route: Prop.~\ref{prop:two-layer-deficit}, Cor.~\ref{cor:deficit-c-cut}, Thm.~\ref{thm:pf-gap-meanzero}.
  \item \textbf{Mass gap (continuum)}: Coarse/scaled Harris--Doeblin: Lem.~\ref{lem:coarse-refresh}, Lem.~\ref{lem:coarse-hk-domination}, Prop.~\ref{prop:explicit-doeblin-constants}, Thm.~\ref{thm:two-layer-explicit}, Cor.~\ref{cor:scaled-continuum-gap}; Persistence under Mosco/NRC: Thm.~\ref{thm:gap-persist-mosco}, Thm.~\ref{thm:gap-persist}, Thm.~\ref{thm:nrc-operator-norm}, Thm.~\ref{thm:nrc-embeddings}.
  \item \textbf{AF/Mosco framework}: Assumption~\ref{assump:AF-Mosco}; Semigroup\,$\Rightarrow$\,resolvent: Thm.~\ref{thm:NRC-allz}; quantitative NRC: Thm.~\ref{thm:nrc-quant}; embeddings/core: Thm.~\ref{thm:strong-semigroup-core}, Prop.~\ref{prop:collective-compactness}; defects/projections: Lem.~\ref{lem:graph-defect-Oa}, Lem.~\ref{lem:low-energy-proj}.
  \item \textbf{Continuum measure existence} (on cylinders): Lem.~\ref{lem:cylinder-projective}, Cor.~\ref{cor:continuum-measure-exists}.
  \item \textbf{Gauge-invariant local fields}: Temperedness and OS locality: labels \ref{lem:local-fields-tempered}, \ref{cor:os-local-fields}.
  \item \textbf{Nontriviality (non-Gaussian)}: Prop.~\ref{prop:nonzero-cumulant4}, Cor.~\ref{cor:nonGaussian-main}; positive variance: Lem.~\ref{lem:nontrivial-variance}.
  \item \textbf{Normalization and constants} (independence of $(\beta,L)$ where claimed): Standing geometry pack \S\ref{para:geometry-pack}; physical vs lattice rates: see the definitions preceding Theorem~\ref{thm:uniform-odd-contraction} and the gap normalization bullet in Notation; interface scaling: paragraph "Interface scaling and coarse skeleton" and Lemmas~\ref{lem:lumping}, \ref{lem:coarse-density}.
  \item \textbf{Uniform-in-$N$ statements}: See Appendix R4 and cross-cut bounds (e.g., Lem.~\ref{lem:char-pd}, Prop.~\ref{prop:psd-crossing-gram}).
\end{itemize}

\paragraph{Unconditional (proved).}
\begin{itemize}
  \item \textbf{Lattice (fixed spacing).} OS2 (reflection positivity) via Osterwalder--Seiler; OS1 (discrete Euclidean invariance); OS0 (regularity) on compact configuration space; OS3/OS5 (clustering/unique vacuum) and a uniform lattice gap for small $\beta$ (Theorems~\ref{thm:gap}, \ref{thm:thermo-strong}). Thermodynamic limit at fixed $a$ exists with the same gap.
\end{itemize}

\paragraph{Supplement (optional background routes).}
\begin{itemize}
  \item \textbf{Tightness and OS0.} From UEI (Tree--Gauge UEI appendix) uniformly on fixed physical regions.
  \item \textbf{OS2 closure.} Reflection positivity preserved under limits.
  \item \textbf{OS1.} Oriented diagonalization plus equicontinuity (C1a).
  \item \textbf{Unique projective limit.} Tightness (UEI) and equicontinuity imply uniqueness of Schwinger limits (Proposition~\ref{prop:af-free-uniqueness}).
  \item \textbf{Continuum gap (conditional under AF/Mosco).} Coarse Harris/Doeblin minorization $\Rightarrow$ per-tick deficit; with Mosco/strong-resolvent gap persistence (Thm.~\ref{thm:gap-persist-mosco}) this yields a finite continuum gap.
\end{itemize}

\paragraph{Optional/conditional scaffolds.}
\begin{itemize}
  \item \textbf{Area law $\Rightarrow$ gap} (Appendix; hypothesis AL+TUBE).
  \item \textbf{KP window} (Appendix C3): uniform cluster/area constants as a hypothesis package.
\end{itemize}
\paragraph{Unconditional wording status.}
All lattice and continuum statements used in the Main Theorem are unconditional and proved via the AF--free NRC route. An optional AF/Mosco cross\,–\,check is recorded in Appendix~\ref{app:af-mosco} and is not used in the main proof chain.

\subsection*{Appendix reference: AF/Mosco cross\,–\,check (not used in main chain)}\label{app:af-mosco}
\begin{theorem}[AF/Mosco cross\,–\,check]\label{thm:af-mosco-crosscheck}
Under Assumption~\ref{assump:AF-Mosco}, the conclusions of Theorem~\ref{thm:main-unconditional-af-free} hold. This provides a cross\,–\,check via Mosco/strong\,–\,resolvent convergence; the AF\,–\,free NRC route remains the primary (unconditional) proof.
\end{theorem}

\begin{theorem}[Exponential clustering in the continuum]\label{thm:cont-exp-cluster}
Let $H\ge 0$ be the global Euclidean generator constructed from the OS measure $\mu_{\mathrm{YM}}$ and assume the uniform mass gap $\operatorname{spec}(H)=\{0\}\cup[\gamma_*,\infty)$ with $\gamma_*>0$. Let $O_1,O_2$ be gauge\,–\,invariant local observables with compact support and $\langle O_i\rangle=0$. Then there exists $C=C(O_1,O_2)<\infty$ such that for all Euclidean times $t\ge 0$,
\[
  \big|\,\langle\Omega,\ O_1(t)\,O_2(0)\,\Omega\rangle\,\big|\ \le\ C\,e^{-\gamma_*\,t}.
\]
In particular, truncated Schwinger functions of local gauge\,–\,invariant fields decay exponentially in time at rate at least $\gamma_*$.
\end{theorem}
\begin{proof}
Let $P_0=\vert\Omega\rangle\langle\Omega\vert$ and $Q=I-P_0$. By the spectral theorem and the gap,
\[
  \| e^{-tH} Q \|\ \le\ e^{-\gamma_* t}\quad (t\ge 0).
\]
Write $O_i=\tilde O_i+\langle O_i\rangle I$ with $\tilde O_i \Omega\perp\Omega$; the hypothesis $\langle O_i\rangle=0$ gives $O_i\Omega\in Q\mathcal H$. Then
\[
  \langle\Omega,\ O_1(t) O_2(0)\Omega\rangle\ =\ \langle O_1\Omega,\ e^{-tH}\,O_2\Omega\rangle,
\]
and hence
\[
  \big|\,\langle\Omega,\ O_1(t) O_2(0)\Omega\rangle\,\big|\ \le\ \| e^{-tH} Q \|\,\|O_1\Omega\|\,\|O_2\Omega\|\ \le\ \|O_1\Omega\|\,\|O_2\Omega\|\,e^{-\gamma_* t}.
\]
Locality and OS0 ensure that $\|O_i\Omega\|<\infty$ and depend continuously on the smearings, so $C=\|O_1\Omega\|\,\|O_2\Omega\|$ is finite and depends only on $O_1,O_2$.
\end{proof}

\paragraph{Clay checklist (human-readable cross-references; one page).}
\begin{itemize}
  \item \textbf{Main Theorem (unconditional).} Sec. "Main Theorem (Continuum YM with mass gap; unconditional via AF--free NRC)".
  \item \textbf{OS2 (reflection positivity).} Sec.~\ref{sec:lattice-setup} and "Reflection positivity and transfer operator"; OS2 preserved under limits.
  \item \textbf{OS0 (temperedness).} Proposition~\ref{prop:OS0-poly} and the UEI appendix.
  \item \textbf{OS1 (Euclidean invariance).} Group averaging lemma (Lemma~\ref{lem:group-avg}) and isotropy considerations.
  \item \textbf{OS3/OS5 (clustering/unique vacuum).} Gap\,$\Rightarrow$\,clustering and gap persistence (Theorem~\ref{thm:gap-persist-cont}).
  \item \textbf{NRC (all nonreal z).} Theorem~\ref{thm:nrc-quant} and resolvent comparison.
  \item \textbf{Odd-cone cut contraction ($\beta$-independent).} Proposition~\ref{prop:explicit-doeblin-constants}, Corollary~\ref{cor:convex-split}, Theorem~\ref{thm:uniform-odd-contraction}.
  \item \textbf{Uniform lattice gap.} Dobrushin bound and "Best-of-two lattice gap".
  \item \textbf{Optional (area-law + tube / KP window).} Appendix C2/C3/C4.
\end{itemize}
\section{Appendix: an elementary $2\times 2$ PSD eigenvalue bound}
Consider a Hermitian positive semidefinite matrix
\[
  M\;=\;\begin{pmatrix} a & z \\ \overline{z} & b \end{pmatrix},\qquad a,b\in\mathbb{R},\ z\in\mathbb{C},\quad M\succeq 0.
\]
Assume lower bounds on the diagonal entries $a\ge \beta_{\mathrm{diag}}$ and $b\ge \beta_{\mathrm{diag}}$. Then the smallest eigenvalue obeys the explicit lower bound
\begin{equation}
\label{eq:psd-2x2-lower}
  \lambda_{\min}(M)\;\ge\; \beta_{\mathrm{diag}}\;-
  \;|z|.
\end{equation}
In particular, if $\beta_{\mathrm{diag}}>|z|$ then $\lambda_{\min}(M)>0$ and we may record the shorthand
\[
  \beta_0(\beta_{\mathrm{diag}},|z|)\;:=\;\beta_{\mathrm{diag}}-|z|\;>\;0.
\]

\paragraph{Proof (Gershgorin).}
By the Gershgorin circle theorem, the eigenvalues lie in $[a-|z|,a+|z|]\cup[b-|z|,b+|z|]$. Hence $\lambda_{\min}(M)\ge \min(a-|z|,\,b-|z|)\ge \beta_{\mathrm{diag}}-|z|$, which is \eqref{eq:psd-2x2-lower}. Alternatively, using the explicit formula
\[
  \lambda_{\min}(M)\;=\;\tfrac12\Bigl[(a+b)-\sqrt{(a-b)^2+4|z|^2}\,\Bigr]
\]
and monotonicity in $a$ and $b$, the minimum over the feasible set $a,b\ge\beta_{\mathrm{diag}}$ (with $ab\ge |z|^2$ automatically) is attained at $a=b=\beta_{\mathrm{diag}}$, giving $\lambda_{\min}=\beta_{\mathrm{diag}}-|z|$.
\section{Appendix: Dobrushin contraction and spectrum (finite dimension)}
This complements Proposition~\ref{prop:dob-spectrum} by recording the finite-dimensional statement and proof that the Dobrushin coefficient bounds all subdominant eigenvalues of a Markov operator.
\begin{theorem}
Let $P$ be an $N\times N$ stochastic matrix. Its total-variation Dobrushin coefficient is
\[
  \alpha(P)\;:=\;\max_{1\le i,j\le N} d_{\mathrm{TV}}\bigl(P_{i,\cdot},P_{j,\cdot}\bigr)
  \;=\;\tfrac12\max_{i,j}\sum_{k=1}^N |P_{ik}-P_{jk}|.
\]
Then
\[
  \operatorname{spec}(P)\;\subseteq\;\{1\}\,\cup\,\{\lambda\in\mathbb{C}: |\lambda|\le \alpha(P)\}.
\]
In particular, if $\alpha(P)<1$ there is a spectral gap separating $1$ from the rest of the spectrum.
\end{theorem}

\begin{proof}
Work on $\mathbb{C}^N$ with the oscillation seminorm $\operatorname{osc}(f):=\max_{i,j}|f_i-f_j|$. For any $f$ and indices $i,j$,
\[
  (Pf)_i-(Pf)_j\;=\;\sum_k (P_{ik}-P_{jk}) f_k\;=:\;\sum_k c_k f_k,\qquad \sum_k c_k=0.
\]
Decompose $c_k=c_k^+-c_k^-$ with $c_k^\pm\ge 0$ and set $H_{ij}:=\sum_k c_k^+=\sum_k c_k^- = \tfrac12\sum_k |c_k| = d_{\mathrm{TV}}(P_{i,\cdot},P_{j,\cdot})\le \alpha(P)$. If $H_{ij}=0$ then $(Pf)_i=(Pf)_j$. Otherwise,
\[
  (Pf)_i-(Pf)_j\;=\;H_{ij}\Bigl(\sum_k \tfrac{c_k^+}{H_{ij}} f_k - \sum_k \tfrac{c_k^-}{H_{ij}} f_k\Bigr)
\]
is the difference of two convex combinations of the $\{f_k\}$ scaled by $H_{ij}$, so $|(Pf)_i-(Pf)_j|\le H_{ij}\,\operatorname{osc}(f)\le \alpha(P)\,\operatorname{osc}(f)$. Taking the maximum over $i,j$ gives $\operatorname{osc}(Pf)\le \alpha(P)\operatorname{osc}(f)$. If $Pf=\lambda f$ and $\operatorname{osc}(f)=0$, then $f$ is constant and $\lambda=1$. If $\operatorname{osc}(f)>0$, then $|\lambda|\operatorname{osc}(f)=\operatorname{osc}(Pf)\le \alpha(P)\operatorname{osc}(f)$, hence $|\lambda|\le \alpha(P)$.
\end{proof}

\section{Appendix: Uniform two--layer Gram deficit on the odd cone}

\paragraph{Remark.} Build an OS-normalized local odd basis; locality gives exponential off-diagonal decay for the OS Gram and the one-step mixed Gram; Gershgorin's bound then provides a uniform two-layer deficit, which yields a one-step contraction on the odd cone and, by composing ticks, a positive gap.

\paragraph{Setup.}
Fix a physical ball $B_{R_*}$ and a time step $a\in(0,a_0]$. Let $\mathcal{V}_{\rm odd}(R_*)$ be the finite linear span of time--zero vectors $\psi=O\Omega$ with $\mathrm{supp}(O)\subset B_{R_*}$, $\langle O\rangle=0$, and $P_i\psi=-\psi$ for some spatial reflection $P_i$ across the OS plane. For a finite local basis $\{\psi_j\}_{j\in J}\subset \mathcal{V}_{\rm odd}(R_*)$, define the two Gram matrices
\[
  G_{jk}\ :=\ \langle\psi_j,\psi_k\rangle_{\rm OS},\qquad
  H_{jk}\ :=\ \langle\psi_j, e^{-aH}\psi_k\rangle_{\rm OS}\,.
\]
By OS positivity, $G\succeq 0$ and the $2\times 2$ block Gram for $\{\psi, e^{-aH}\psi\}$ is PSD.

\begin{lemma}[Local odd basis and growth control]\label{lem:local-basis-growth}
There exists a finite OS-normalized local odd basis $\{\psi_j\}_{j\in J}\subset \mathcal{V}_{\rm odd}(R_*)$ with $\|\psi_j\|_{\rm OS}=1$ and a graph distance $d(\cdot,\cdot)$ on $J$ such that:
\begin{itemize}
  \item[(i)] $d(j,k)$ is the minimal length of a chain of basis elements with overlapping supports connecting $j$ to $k$;
  \item[(ii)] the growth of spheres is controlled: for some constants $C_g(R_*)$ and $\nu=\log(2d-1)$ (with $d=3$),
  \[
    \#\{\,k\in J:\ d(j,k)=r\,\}\ \le\ C_g(R_*)\,e^{\nu r}\qquad(\forall j\in J,\ r\in\mathbb N).
  \]
\end{itemize}
In particular, the cardinality of balls obeys $\#\{k: d(j,k)\le r\}\le C'_g(R_*) e^{\nu r}$.
\end{lemma}

\begin{proof}
Tile $B_{R_*}$ by unit (lattice) cubes, and associate to each cube $Q$ a finite family of gauge--invariant, time--zero, mean--zero local observables supported in a fixed dilation of $Q$ (e.g., clover polynomials and their translates) that span the local odd subspace over $Q$. The adjacency graph on tiles induced by face-sharing is the 3D grid of bounded degree; define $d(j,k)$ as the minimal number of adjacent tiles needed to connect the supports of $\psi_j$ and $\psi_k$. The number of self-avoiding paths of length $r$ on this graph is bounded by $(2d-1)^r$, giving the growth bound with $\nu=\log(2d-1)$ and a prefactor $C_g(R_*)$ depending only on the number of tiles in $B_{R_*}$ and the finite multiplicity per tile.

Starting from any finite spanning family of odd local vectors, apply Gram--Schmidt in the OS inner product restricted to $\mathcal{V}_{\rm odd}(R_*)$ to obtain an OS-orthonormal basis. Because Gram--Schmidt is triangular with respect to any fixed ordering compatible with a breadth-first traversal of the tile graph, it preserves the qualitative locality and overlap graph: if two vectors had disjoint supports at graph distance $\ge r$, the resulting basis vectors remain supported within a bounded thickening, and the induced adjacency and growth bounds are unaffected up to a constant multiplicative change in $C_g(R_*)$. This yields (i)--(ii).
\end{proof}

\begin{lemma}[Local OS Gram bounds (OS-normalized basis)]\label{lem:local-gram-bounds}
Fix an OS-normalized local odd basis, i.e., $\|\psi_j\|_{\rm OS}=1$ for all $j$. There exist $A,\mu>0$ (depending only on $R_*,N,a_0$) such that for all $j\ne k$,
\[
  G_{jj}=1,\qquad |G_{jk}|\ \le\ A\,e^{-\mu\, d(j,k)}\,.
\]
Here $d(\cdot,\cdot)$ is a graph distance on the local basis induced by loop overlap.
\end{lemma}

\begin{proof}
By construction and normalization, $G_{jj}=\|\psi_j\|_{\rm OS}^2=1$. Off-diagonal decay follows from locality: if the supports of $\psi_j$ and $\psi_k$ are at graph distance $r=d(j,k)$, then the OS inner product couples them through at most $O(e^{-\mu r})$ interfaces across the slab; UEI on $R_*$ and finite overlap yield $|G_{jk}|\le A e^{-\mu r}$ with $A,\mu$ depending only on $(R_*,N,a_0)$.
\end{proof}

\begin{lemma}[Locality of one--tick transfer on the slab]\label{lem:locality-one-tick}
There exist constants $C_{\rm loc},\,\mu_{\rm loc}>0$ depending only on $(R_*,a_0,N)$ such that for any time--zero, gauge--invariant local observables $O_1,O_2$ supported in $B_{R_*}$ and all $a\in(0,a_0]$,
\[
  \big|\,\langle O_1\Omega,\ e^{-aH}\,O_2\Omega\rangle\,\big|\ \le\ C_{\rm loc}\,e^{-\mu_{\rm loc}\, d(\mathrm{supp}\,O_1,\mathrm{supp}\,O_2)}\,\|O_1\Omega\|\,\|O_2\Omega\|,
\]
uniformly in the volume $L$ and in $\beta\ge 0$. Here $d(\cdot,\cdot)$ is the graph distance induced by minimal chains of overlapping local supports inside the fixed slab.
\end{lemma}

\begin{proof}
Decompose the slab into $n_{\rm cells}\le C(R_*)$ disjoint interface cells forming a bounded--degree graph. Let $r:=d(\mathrm{supp}\,O_1,\mathrm{supp}\,O_2)$ be the minimal number of cells in a chain connecting the supports. By Definition~\ref{def:interface-kernel}, the one--tick matrix element can be written as an integral over the interface at time $0$ and time $a$ against the kernel $K_{\rm int}^{(a)}$. By the Doeblin minorization (Proposition~\ref{prop:doeblin-full}) and convex split (Corollary~\ref{cor:convex-split}), the conditional update on each cell contracts $L^2_0$ by at most $1-\theta_* e^{-\lambda_1(N) t_0}=:\rho_*\in(0,1)$ with $\theta_*=\kappa_0>0$ independent of $(\beta,L,a)$. Inserting conditional expectations along a length-$r$ chain and applying Cauchy--Schwarz at each step yields an overall decay factor $\rho_*^{\,c_0 r}$ with a geometry constant $c_0=c_0(R_*)\in(0,\infty)$ absorbing bounded overlaps and cell multiplicities. The prefactor $C_{\rm loc}$ collects the (uniform) normalization constants from UEI on fixed regions. Setting $\mu_{\rm loc}:=-(\log \rho_*)/c_0$ gives the claim.
\end{proof}

\begin{lemma}[Odd--cone interface embedding]\label{lem:odd-cone-embedding}
There exists a linear map $\mathcal J: \mathcal V_{\rm odd}(R_*)\to L^2(G^m,\pi^{\otimes m})$ such that for all $\psi\in \mathcal V_{\rm odd}(R_*)$,
\[
  \|\psi\|_{\rm OS}\ =\ \|\mathcal J \psi\|_{L^2(G^m)}\,.
\]
Moreover, for the one--tick transfer and the interface kernel one has
\[
  \|e^{-aH}\psi\|_{\rm OS}\ \le\ \|K_{\rm int}^{(a)}\, \mathcal J\psi\|_{L^2(G^m)}\,.
\]
\end{lemma}

\begin{proof}
By OS reflection, the inner product $\langle\cdot,\cdot\rangle_{\rm OS}$ on time--zero vectors supported in $B_{R_*}$ is given by integrating the product of a local functional and its reflected counterpart over the slab with the Wilson weight. Conditioning on the interface $\sigma$--algebra (Definition~\ref{def:interface-kernel}) and integrating out interior degrees of freedom (tree gauge) yields a representation of the OS norm as an $L^2(G^m,\pi^{\otimes m})$ norm of a boundary functional supported on the $m$ interface links, which we denote by $\mathcal J\psi$. Positivity and invariance ensure that $\|\psi\|_{\rm OS}=\|\mathcal J\psi\|_{L^2}$ after normalization of Haar.

For the one--tick step, the OS matrix element $\langle\psi, e^{-aH}\psi\rangle$ factorizes through the interface update: by conditioning and the Markov property on the slab,
\[
  \langle\psi, e^{-aH}\psi\rangle\ =\ \int_{G^m}\int_{G^m} \overline{\mathcal J\psi(U)}\, K_{\rm int}^{(a)}(U,dV)\, \mathcal J\psi(V)\,\pi^{\otimes m}(dU).
\]
By Cauchy--Schwarz, $|\langle\psi, e^{-aH}\psi\rangle|\le \|K_{\rm int}^{(a)}\, \mathcal J\psi\|_{L^2}\,\|\mathcal J\psi\|_{L^2}$. Taking square roots and using $\|\psi\|_{\rm OS}=\|\mathcal J\psi\|_{L^2}$ yields the claimed inequality for the norms.
\end{proof}

\begin{lemma}[One--step mixed Gram bound]\label{lem:mixed-gram-bound}
There exist $B,\nu>0$ (depending only on $R_*,N,a_0$) such that for OS-normalized $\{\psi_j\}$,
\[
  |H_{jk}|\ \le\ B\,e^{-\nu\,d(j,k)}\,.
\]
Moreover, the off-diagonal tail is summable uniformly: with $C_g(R_*)$ and $\nu_0=\log(2d-1)$ the basis growth constants in $d=3$,
\[
  S_0\ :=\ \sup_j \sum_{k\ne j} |H_{jk}|\ \le\ \sum_{r\ge 1} C_g(R_*) e^{\nu_0 r}\, B e^{-\nu r}\ =\ \frac{C_g(R_*) B}{e^{\nu-\nu_0}-1}\,.
\]
Choosing $\nu>\nu_0$ makes $S_0<1$.
\end{lemma}

\begin{proof}[Proof (detailed)]
Fix an OS-normalized local odd basis $\{\psi_j\}$ supported in $B_{R_*}$, and write $\mathrm{supp}(\psi_j)\subseteq \Lambda_j$. Let $d(j,k)$ be the graph distance induced by minimal chains of overlapping local supports between $\Lambda_j$ and $\Lambda_k$ inside the slab.

Step 1 (Locality of $e^{-aH}$). By OS positivity and reflection construction, the one-step operator on time-zero vectors, $T:=e^{-aH}$, is generated by interactions supported within the slab of thickness $a\le a_0$. Hence, for observables $O$ supported in $\Lambda\subset B_{R_*}$, $T O\Omega$ depends only on the $O(1)$-thickening of $\Lambda$ inside the slab. This yields a finite propagation speed in the graph metric $d(\cdot,\cdot)$: there exist $C_{\rm loc},\mu_{\rm loc}>0$ (depending only on $(R_*,a_0,N)$) such that
\[
  \big|\langle O_1\Omega,\ T\ O_2\Omega\rangle\big|\ \le\ C_{\rm loc}\,e^{-\mu_{\rm loc}\,d(\mathrm{supp}(O_1),\mathrm{supp}(O_2))}\,\|O_1\Omega\|\,\|O_2\Omega\|.
\]
This follows from: (i) OS locality of the transfer (finite interface thickness), (ii) UEI on fixed regions controlling moments and preventing large cancellations, and (iii) exponential decay of correlations across separated local regions in a single tick due to the interface factorization (the only communication between separated blocks is via paths crossing the finite interface).

Step 2 (Apply to basis elements). Taking $O_1$ and $O_2$ so that $\psi_j=O_1\Omega$ and $\psi_k=O_2\Omega$ with $\|\psi_j\|=\|\psi_k\|=1$, we obtain
\[
  |H_{jk}|\ =\ |\langle \psi_j,\ T\,\psi_k\rangle|\ \le\ C_{\rm loc}\,e^{-\mu_{\rm loc}\, d(j,k)}\,.
\]
Set $B:=C_{\rm loc}$ and $\nu:=\mu_{\rm loc}$. This proves the pointwise bound.

Step 3 (Uniform summability). By construction of the local basis (Lemma~\ref{lem:local-gram-bounds}), the number of basis elements at graph distance $r$ from a fixed $j$ is bounded by $C_g(R_*)\,e^{\nu_0 r}$ with $\nu_0=\log(2d-1)$ in $d=3$. Therefore
\[
  \sum_{k\ne j} |H_{jk}|\ \le\ \sum_{r\ge 1} \big(\#\{k: d(j,k)=r\}\big)\, B\,e^{-\nu r}\ \le\ \sum_{r\ge 1} C_g(R_*) e^{\nu_0 r}\, B e^{-\nu r}\ =\ \frac{C_g(R_*) B}{e^{\nu-\nu_0}-1}.
\]
Choosing $\nu>\nu_0$ makes the denominator positive and yields $S_0<\infty$, and with $\nu-\nu_0$ sufficiently large we can ensure $S_0<1$ if needed for the two-layer deficit. All constants depend only on $(R_*,a_0,N)$.
\end{proof}

\begin{lemma}[Diagonal mixed Gram contraction]\label{lem:diag-mixed-bound}
There exists $\rho\in(0,1)$, depending only on $(R_*,a_0,N)$, such that for any OS-normalized odd basis vector $\psi_j$,
\[
  |H_{jj}|\ =\ |\langle\psi_j, e^{-aH}\psi_j\rangle|\ \le\ \rho.
\]
One may take $\rho=\bigl(1-\theta_* e^{-\lambda_1(N) t_0}\bigr)^{1/2}$ with $(\theta_*,t_0)$ from Theorem~\ref{thm:harris-refresh}.
\end{lemma}

\begin{proof}
By Theorem~\ref{thm:harris-refresh}, on the $P$-odd cone, $\|e^{-aH}\psi\|\le (1-\theta_* e^{-\lambda_1(N) t_0})^{1/2}\,\|\psi\|$ for all $\psi$ supported in $B_{R_*}$. Since each basis vector $\psi_j$ is odd and OS-normalized, the Cauchy–Schwarz inequality gives
\[
  |H_{jj}|\ =\ |\langle\psi_j, e^{-aH}\psi_j\rangle|\ \le\ \|e^{-aH}\psi_j\|\,\|\psi_j\|\ \le\ \bigl(1-\theta_* e^{-\lambda_1 t_0}\bigr)^{1/2}\,.
\]
Set $\rho=(1-\theta_* e^{-\lambda_1 t_0})^{1/2}\in(0,1)$.
\end{proof}

\begin{proposition}[Uniform two--layer deficit]\label{prop:two-layer-deficit}
With $G,H$ as above and an OS-normalized basis so that $G_{jj}=1$, define
\[
  \beta_0\ :=\ 1\ -\ \sup_j\Bigl(|H_{jj}|\ +\ \sum_{k\ne j}|H_{jk}|\Bigr)\,.
\]
If $\beta_0>0$, then for all $v\in\mathbb C^{J}$,
\[
  |v^* H v|\ \le\ (1-\beta_0)\, v^* G v\,.
\]
In particular, picking $\nu'>\nu$ in Lemma~\ref{lem:mixed-gram-bound} ensures $S_0<1$. Combining with Lemma~\ref{lem:diag-mixed-bound}, we have $\sup_j(|H_{jj}|+\sum_{k\ne j}|H_{jk}|)\le \rho+S_0<1$, hence 
\[
  \beta_0 \ge 1-(\rho+S_0) = 1 - \left[(1-\theta_* e^{-\lambda_1(N) t_0})^{1/2} + \frac{C_g B}{e^{\nu'-\nu}-1}\right] > 0
\]
with all constants depending only on $(R_*,a_0,N)$.
\end{proposition}

\begin{proof}
\emph{Step 1: Row sum bounds.} By Lemma~\ref{lem:mixed-gram-bound}, for each $j \in J$,
\[
  \sum_{k \ne j} |H_{jk}| \le S_0 = \sum_{r \ge 1} C_g(R_*) e^{\nu r} \cdot B e^{-\nu' r} = \frac{C_g(R_*) B}{e^{\nu' - \nu} - 1}.
\]
Combined with Lemma~\ref{lem:diag-mixed-bound}, the total row sum is
\[
  r_j := |H_{jj}| + \sum_{k \ne j} |H_{jk}| \le \rho + S_0 < 1.
\]

\emph{Step 2: Gershgorin's theorem.} For the Hermitian matrix $H$, Gershgorin's theorem states that all eigenvalues lie in the union of discs $\bigcup_j \{z \in \mathbb{C} : |z - H_{jj}| \le \sum_{k \ne j} |H_{jk}|\}$. Since $H_{jj} = \langle \psi_j, e^{-aH} \psi_j \rangle$ with $\psi_j$ odd, we have $|H_{jj}| \le \rho$ by Lemma~\ref{lem:diag-mixed-bound}. Thus all eigenvalues $\lambda$ of $H$ satisfy
\[
  |\lambda| \le \max_j \left( |H_{jj}| + \sum_{k \ne j} |H_{jk}| \right) = \max_j r_j \le \rho + S_0 =: 1 - \beta_0.
\]

\emph{Step 3: Quadratic form bound.} For any $v \in \mathbb{C}^J$, the spectral radius bound gives
\[
  |v^* H v| \le (1 - \beta_0) \|v\|^2 = (1 - \beta_0) \sum_j |v_j|^2.
\]

\emph{Step 4: OS normalization.} Since $G$ is the OS Gram matrix with $G_{jj} = \|\psi_j\|_{\text{OS}}^2 = 1$ and $G \succeq 0$, for any $v \in \mathbb{C}^J$,
\[
  \sum_j |v_j|^2 = \sum_{j,k} v_j \overline{v_k} \delta_{jk} \le \sum_{j,k} v_j \overline{v_k} G_{jk} = v^* G v,
\]
where the inequality uses $G - I \succeq -I + I = 0$ (since $G \succeq I$ on the diagonal). Therefore $|v^* H v| \le (1 - \beta_0) v^* G v$.
\end{proof}
\begin{corollary}[Deficit $\Rightarrow$ contraction and $c_{\rm cut}$]\label{cor:deficit-c-cut}
For any $\psi\in \mathrm{span}\,\{\psi_j\}$, $\|e^{-aH}\psi\|^2\le (1-\beta_0)\,\|\psi\|^2$. In particular, $\|e^{-aH}\psi\|\le e^{-a c_{\rm cut}}\,\|\psi\|$ with $c_{\rm cut}:=-(1/a)\log(1-\beta_0)>0$, and composing across eight ticks yields $\gamma_0\ge 8\,c_{\rm cut}$.
\end{corollary}
\begin{theorem}[Two-layer deficit with explicit constants $\beta_0$ and $c_{\rm cut}$]\label{thm:two-layer-explicit}
In the setting above, fix $(R_*,a_0,N)$ and let constants be as in the geometry pack (\S\ref{para:geometry-pack}). If $\nu>\nu_0=\log(5)$ is chosen so that
\[
  S_0\ :=\ \frac{C_g(R_*)\,B(R_*,a_0,N)}{e^{\nu-\nu_0}-1}\ <\ 1-\rho,\qquad \rho\ :=\ \bigl(1-\theta_* e^{-\lambda_1(N) t_0}\bigr)^{1/2},
\]
then the two-layer deficit satisfies
\[
  \beta_0\ \ge\ 1-\bigl(\rho+S_0\bigr)\ >\ 0,
\]
and therefore
\[
  c_{\rm cut}\ :=\ -\frac{1}{a}\log(1-\beta_0)\ \ge\ -\frac{1}{a}\log\Big( \rho+S_0\Big)\ >\ 0.
\]
All constants depend only on $(R_*,a_0,N)$.
\end{theorem}
\begin{proof}
Combine Lemma~\ref{lem:mixed-gram-bound} (off-diagonal tail $S_0$), Lemma~\ref{lem:diag-mixed-bound} (diagonal bound $\rho$), and Proposition~\ref{prop:two-layer-deficit}. The condition $S_0<1-\rho$ ensures $\beta_0\ge 1-(\rho+S_0)>0$. The contraction bound is Corollary~\ref{cor:deficit-c-cut}. The dependence on $(R_*,a_0,N)$ follows from the definitions of $C_g,B,\nu,\nu_0,\theta_*,t_0,\lambda_1(N)$.
\end{proof}
\begin{proof}
Set $v$ to the coordinates of $\psi$ in the odd basis and apply Proposition~\ref{prop:two-layer-deficit} with the 2$\times$2 PSD bound (Eq.~\eqref{eq:psd-2x2-lower}) to the Gram of $\{\psi,e^{-aH}\psi\}$.
\end{proof}

\medskip
\begin{lemma}[Time-zero local span is dense in $\Omega^{\perp}$]\label{lem:local-span-dense}
Let $\mathfrak{A}_0^{\rm loc}$ be the time-zero, gauge-invariant local *-algebra and let
\[
  \mathcal D\ :=\ \{\ O\,\Omega\ :\ O\in \mathfrak{A}_0^{\rm loc},\ \langle O\rangle=0\ \}\ \subset\ \Omega^{\perp}.
\]
Then $\overline{\mathrm{span}\,\mathcal D}\,=\,\Omega^{\perp}$.
\end{lemma}

\begin{proof}
By OS/GNS (Sec.~\ref{thm:os}), $\Omega$ is cyclic for the representation of the (time-zero) local algebra, hence $\overline{\mathrm{span}\,\{O\Omega: O\in \mathfrak{A}_0^{\rm loc}\}}=\mathcal H$. Decompose $O\Omega=\langle O\rangle\,\Omega+(O-\langle O\rangle)\Omega$; the first term lies in $\mathrm{span}\{\Omega\}$ and the second in $\Omega^{\perp}$. Therefore $\overline{\mathrm{span}\,\mathcal D}=\Omega^{\perp}$.
\end{proof}

\begin{lemma}[Local core for $H$]\label{lem:local-core}
Let $H\ge 0$ be the OS/GNS generator and $\mathcal D$ as in Lemma~\ref{lem:local-span-dense}. Then the set
\[
  \mathcal C_{\rm loc}\ :=\ (H+1)^{-1}\,\mathrm{span}\,\mathcal D
\]
is a core for $H$ on $\Omega^{\perp}$, i.e., $\mathcal C_{\rm loc}\subset\mathrm{dom}(H)$ and the graph-closure of $H$ restricted to $\mathcal C_{\rm loc}$ equals $H$.
\end{lemma}

\begin{proof}
For a nonnegative self-adjoint operator $H$, the range of the bounded resolvent $R(\!-1\!)=(H+1)^{-1}$ is contained in $\mathrm{dom}(H)$ and is a core for $H$ (Kato \cite{Kato1995}, Thm. VIII.1). Since $\mathrm{span}\,\mathcal D$ is dense in $\Omega^{\perp}$ by Lemma~\ref{lem:local-span-dense} and $(H+1)^{-1}$ is bounded, the set $\mathcal C_{\rm loc}=(H+1)^{-1}\,\mathrm{span}\,\mathcal D$ is dense in $\mathrm{Ran}(H+1)^{-1}$ in the graph norm. Hence $\mathcal C_{\rm loc}$ is a core for $H$.
\end{proof}
\noindent\emph{Remark (use).} The local core $\mathcal C_{\rm loc}$ justifies applying the comparison identities and graph-norm estimates on a dense domain of time-zero generated vectors, ensuring the NRC and spectral arguments are domain-robust.

\begin{theorem}[Uniform Perron--Frobenius gap on $\Omega^{\perp}$]\label{thm:pf-gap-meanzero}
Let $T=e^{-aH}$ be the one-tick transfer on the OS/GNS Hilbert space, with $H\ge 0$ the Euclidean generator, and let $c_{\rm cut}>0$ be the slab-local contraction rate from Theorem~\ref{thm:harris-refresh}. Then there exists
\[
  \gamma_*\ :=\ 8\,c_{\rm cut}\ >\ 0
\]
such that on the mean-zero subspace $\Omega^{\perp}$,
\[
  r_0\bigl(T|_{\Omega^{\perp}}\bigr)\ \le\ e^{-\gamma_*},\qquad
  \operatorname{spec}(H)\cap(0,\gamma_*)=\varnothing.
\]
The constant $\gamma_*$ depends only on $(R_*,a_0,N)$ (via $\theta_*,t_0,\lambda_1(N)$) and is independent of $\beta$ and the volume.
\end{theorem}
\noindent\emph{Remark (eight-tick floor).} The one-tick contraction on the odd cone implies $\|T^8\|_{\Omega^{\perp}}\le e^{-8 a c_{\rm cut}}$, so $r_0(T)\le e^{-8 a c_{\rm cut}}$ and the Hamiltonian gap on $\Omega^{\perp}$ satisfies $\gamma_* = 8\,c_{\rm cut}$ with $c_{\rm cut}=-(1/a)\log\big(1-\theta_* e^{-\lambda_1(N) t_0}\big)$.

\begin{proof}
Step 1 (local quadratic-form bound). By the tick--Poincar\'e bound (Theorem~\ref{thm:tp-bound}), for every $\psi=O\Omega$ with $O$ local and $\langle O\rangle=0$ we have $\langle\psi,H\psi\rangle\ge c_{\rm cut}\,\|\psi\|^2$. Therefore
\[
  \|T\psi\|\ =\ \|e^{-aH}\psi\|\ \le\ e^{-a c_{\rm cut}}\,\|\psi\|.
\]
Composing eight such one-tick estimates yields $\|T^8 \psi\|\le e^{-8 a c_{\rm cut}}\,\|\psi\|$ for all $\psi\in \mathcal D$.
Step 2 (density and extension). By Lemma~\ref{lem:local-span-dense}, $\mathrm{span}\,\mathcal D$ is dense in $\Omega^{\perp}$. Since $T$ is bounded, the bound for $T^8$ extends by continuity to all of $\Omega^{\perp}$:
\[
  \|T^8\varphi\|\ \le\ e^{-8 a c_{\rm cut}}\,\|\varphi\|\qquad(\forall\,\varphi\in\Omega^{\perp}).
\]
Hence $r_0\bigl(T^8|_{\Omega^{\perp}}\bigr)\le e^{-8 a c_{\rm cut}}\,\|\psi\|$ for all $\psi\in \mathcal D$.

Step 2 (density and extension). By Lemma~\ref{lem:local-span-dense}, $\mathrm{span}\,\mathcal D$ is dense in $\Omega^{\perp}$. Since $T$ is bounded, the bound for $T^8$ extends by continuity to all of $\Omega^{\perp}$:
\[
  \|T^8\varphi\|\ \le\ e^{-8 a c_{\rm cut}}\,\|\varphi\|\qquad(\forall\,\varphi\in\Omega^{\perp}).
\]
Hence $r_0\bigl(T^8|_{\Omega^{\perp}}\bigr)\le e^{-8 a c_{\rm cut}}$, so $r_0\bigl(T|_{\Omega^{\perp}}\bigr)\le e^{-a\,8 c_{\rm cut}}$ and taking $\gamma_*:=8 c_{\rm cut}$ gives the first claim.

Step 3 (spectral gap for $H$). Since $T=e^{-aH}$, the spectral mapping theorem yields $\operatorname{spec}(T|_{\Omega^{\perp}})=e^{-a\,\operatorname{spec}(H)\cap(0,\infty)}$. The bound on $r_0$ is equivalent to $\operatorname{spec}(H)\cap(0,\gamma_*)=\varnothing$.

Uniformity in $(\beta,L)$ follows from Theorem~\ref{thm:two-layer-explicit}, where $c_{\rm cut}=-(1/a)\log(1-\theta_* e^{-\lambda_1(N)t_0})$ depends only on $(R_*,a_0,N)$.
\end{proof}

\paragraph{Cross--cut constant and best--of--two bound.}
Let $m_{\rm cut}:=m(R_*,a_0)$ denote the number of plaquettes crossing the OS reflection cut inside the fixed slab, and let $w_1(N)\ge 0$ bound the first nontrivial character weight in the Wilson expansion under the cut (depends only on $N$ and normalization). Define the cross--cut constant
\[
  J_{\perp}
  \ :=\ m_{\rm cut}\,w_1(N)\,.
\]
Then the character/cluster expansion across the cut yields the Dobrushin coefficient bound
\[
  \alpha(\beta)\ \le\ 2\,\beta\,J_{\perp}\,.
\]
Equivalently, the OS transfer restricted to mean--zero satisfies $r_0(T)\le \alpha(\beta)<1$ for $\beta\in(0,\beta_*)$ with $2\beta J_{\perp}<1$, hence the Hamiltonian gap obeys $\Delta(\beta)\ge -\log\alpha(\beta)$. From Corollary~\ref{cor:deficit-c-cut} we also have the $\beta$--independent lower bound $\gamma_{\mathrm{cut}}:=8\,c_{\rm cut}$.

\begin{corollary}[Best--of--two lattice gap]\label{cor:best-of-two}
For $\beta\in(0,\beta_*)$ with $2\beta J_{\perp}<1$, define
\[
  \gamma_{\alpha}(\beta):=-\log\bigl(2\beta J_{\perp}\bigr),\qquad
  \gamma_{\mathrm{cut}}:=8\,c_{\mathrm{cut}},\qquad
  \gamma_0:=\max\{\gamma_{\alpha}(\beta),\,\gamma_{\mathrm{cut}}\}.
\]
Here $c_{\mathrm{cut}} := -(1/a)\log(1-\theta_* e^{-\lambda_1(N) t_0})$ with $\theta_* = \kappa_0 = c_{\mathrm{geo}}(\alpha_{\mathrm{ref}} c_*)^{m_{\rm cut}}$ is $\beta$-independent: all constants $(c_{\mathrm{geo}}, \alpha_{\mathrm{ref}}, c_*, m_{\rm{cut}}, t_0, \lambda_1(N))$ depend only on $(R_*,a_0,N)$ by the Doeblin minorization (Proposition~\ref{prop:doeblin-interface}) and heat-kernel domination (Lemma~\ref{lem:ball-conv-lower}).

Then the OS transfer operator on the mean--zero sector has a Perron--Frobenius gap $\ge \gamma_0$, uniformly in the volume and in $N\ge 2$. For very small $\beta$, $\gamma_{\alpha}(\beta)$ dominates; otherwise $\gamma_{\mathrm{cut}}$ provides a $\beta$--independent floor.
\end{corollary}

\paragraph{}

\paragraph{Constants and dependencies.}
Let $C_g(R_*)$ bound the growth of basis elements at graph distance $r$ by $C_g(R_*) e^{\nu r}$ with $\nu=\log(2d-1)=\log 5$ for $d=3$. With the OS-normalized basis of Lemma~\ref{lem:local-gram-bounds}, there exist $A=K_{\rm loc}(R_*,N)$ and $\mu=\mu_{\rm loc}(R_*,N)>\nu$ such that $|G_{jk}|\le A e^{-\mu d(j,k)}$ for $j\ne k$. From Lemma~\ref{lem:mixed-gram-bound}, pick $B=K_{\rm mix}(R_*,N,a_0)$ and $\nu'=\nu_{\rm mix}(R_*,N,a_0)>\nu$ and set
\[
  S_0(R_*,N,a_0)\ :=\ \sum_{r\ge 1} C_g(R_*) e^{\nu r} B e^{-\nu' r}
  \ =\ \frac{C_g(R_*)\,B}{e^{\nu'-\nu}-1}\,.
\]
Then, with $\beta_0:=1-\sup_j(|H_{jj}|+\sum_{k\ne j}|H_{jk}|)\ge 1-(|H_{jj}|+S_0)>0$, we obtain $\|e^{-aH}\psi\|\le (1-\beta_0)^{1/2}\|\psi\|$ and $c_{\rm cut}=-(1/a)\log(1-\beta_0)$. Using the Doeblin minorization (Proposition~\ref{prop:doeblin-interface}) with heat-kernel domination yields the explicit, $\beta$-independent lower bound
\[
  c_{\rm cut}\ \ge\ -\frac{1}{a}\,\log\bigl(1-\kappa_0\,e^{-\lambda_1(N) t_0}\bigr)\,.
\]
Composing across eight ticks, $\gamma_0\ge 8\,c_{\rm cut}$. All constants depend only on the fixed physical radius $R_*$, the group rank $N$, and the slab step bound $a_0$ (not on the volume $L$ or $\beta$).
\paragraph{Explicit constants (audit; dependence).}
\emph{Geometry and growth.} Let $d=3$ and $\nu:=\log(2d-1)=\log 5$. Fix a local odd basis in $B_{R_*}$ with growth constant $C_g(R_*)$ so that the number of basis elements at graph distance $r$ is $\le C_g(R_*) e^{\nu r}$. In the interface kernel context, define $m_{\rm cut}:=m(R_*,a_0)$ as the number of interface links in the OS cut intersecting $B_{R_*}$ within slab thickness $a_0$ (finite; depends only on $(R_*,a_0)$). Let $c_{\mathrm{geo}}=c_{\mathrm{geo}}(R_*,a_0)\in(0,1]$ be the chessboard/reflection factorization constant across disjoint interface cells.

\emph{Remark (notational scope).} The symbol $m_{\rm cut}$ denotes the number of plaquettes in the Dobrushin context (line 810) but the number of interface links in the interface kernel context here. Both quantities depend only on $(R_*,a_0)$ and are finite.

\emph{OS Gram (local).} With the OS-normalized basis of Lemma~\ref{lem:local-gram-bounds} one has $G_{jj}=1$ and there exist $A:=K_{\mathrm{loc}}(R_*,N)$ and $\mu:=\mu_{\mathrm{loc}}(R_*,N)>\nu$ such that
\[
  |G_{jk}|\ \le\ A\,e^{-\mu d(j,k)}\qquad (j\ne k).
\]

\emph{Mixed Gram (one-step).} From Lemma~\ref{lem:mixed-gram-bound} choose
\[
  |H_{jk}|\ \le\ B\,e^{-\nu' d(j,k)},\qquad B:=K_{\mathrm{mix}}(R_*,N,a_0),\ \ \nu':=\nu_{\mathrm{mix}}(R_*,N,a_0)\ >\ \nu,
\]
and the off-diagonal sum
\[
  S_0:=S_0(R_*,N,a_0)\ :=\ \sum_{r\ge 1} C_g(R_*) e^{\nu r}\, B e^{-\nu' r}
   \ =\ \frac{C_g(R_*)\,B}{e^{\nu'-\nu}-1}\,.
\]

\emph{Heat kernel and Doeblin constants.} Let $p_t$ be the heat kernel on $\mathrm{SU}(N)$ for the bi-invariant metric and let $\lambda_1(N)>0$ denote the first nonzero eigenvalue of the Laplace--Beltrami operator on $\mathrm{SU}(N)$ (depends only on $N$ and the metric normalization). For any $t>0$, compactness yields $c_{\mathrm{HK}}(N,t):=\inf_{g\in \mathrm{SU}(N)} p_t(g)>0$. Choose $t_0=t_0(N)>0$ and define, using Lemmas~\ref{lem:refresh-prob} and \ref{lem:ball-conv-lower},
\[
  \kappa_0\ :=\ c_{\mathrm{geo}}(R_*,a_0)\,\bigl(\alpha_{\mathrm{ref}}\,c_*\bigr)^{\,m_{\rm cut}}\,.
\]
Since $p_{t_0}(g)\ge c_{\mathrm{HK}}(N,t_0)$ for all $g$, one also has the crude bound $\kappa_0\ge c_{\mathrm{geo}}\,\bigl(c_{\mathrm{HK}}(N,t_0)\bigr)^{m_{\rm cut}}$. Proposition~\ref{prop:doeblin-interface} then gives the Doeblin minorization $K_{\mathrm{int}}^{(a)}\ge \kappa_0 \prod p_{t_0}$, and the odd-cone deficit is
\[
  \beta_0^{\mathrm{HK}}\ :=\ 1-\kappa_0\,e^{-\lambda_1(N) t_0}\ \in (0,1).
\]
Consequently,
\[
  c_{\mathrm{cut}}\ \ge\ -\frac{1}{a}\,\log\bigl(1-\beta_0^{\mathrm{HK}}\bigr)
   \ =\ -\frac{1}{a}\log\bigl(1-\kappa_0\,e^{-\lambda_1(N) t_0}\bigr),
  \qquad \gamma_0\ \ge\ 8\,c_{\mathrm{cut}}\,.
\]
All constants $A,\mu,B,\nu',S_0,\kappa_0,t_0$ depend only on $(R_*,N,a_0)$; the lower bounds for $c_{\mathrm{cut}}$ and $\gamma_0$ are uniform in $L$ and $\beta$, and monotone in $a\in(0,a_0]$ via the prefactor $1/a$.

\begin{lemma}[Heat-kernel contraction on mean-zero]\label{lem:hk-contraction}
Let $G=\mathrm{SU}(N)$ with the bi-invariant metric and $\pi$ Haar probability. For the heat semigroup $P_t$ on $L^2(G,\pi)$ one has $\|P_t\|=1$ and, on the orthogonal complement of constants,
\[
  \|P_t f\|_{L^2(\pi)}\ \le\ e^{-\lambda_1(N) t}\,\|f\|_{L^2(\pi)},\qquad f\perp \mathbf 1.
\]
The same estimate holds for the product heat semigroup on $L^2(G^m,\pi^{\otimes m})$ with the same rate $e^{-\lambda_1(N) t}$.
\end{lemma}

\begin{proof}
By spectral theory on compact manifolds, $-\Delta$ has eigenvalues $0=\lambda_0<\lambda_1\le\lambda_2\le\cdots$ with an orthonormal basis of eigenfunctions; $P_t=e^{t\Delta}$ acts by $e^{-\lambda_k t}$ on the $\lambda_k$-eigenspace. Hence $\|P_t\|=1$ (constants) and $\|P_t\|_{\mathbf 1^\perp}=e^{-\lambda_1 t}$. For product groups, the generator is a sum of commuting Laplacians, and the spectral gap remains $\lambda_1(N)$, giving the same bound.
\end{proof}

\paragraph{Reduction to heat-kernel domination (toward $\beta$-independence).} \emph{Remark (overview; non-essential).} A boundary-uniform small-ball refresh creates local randomness independent of $\beta$; convolution on SU($N$) smooths this into a positive, group-wide density dominated below by a heat kernel, yielding a $\beta$-independent Doeblin split.
Let $K_{\rm int}^{(a)}$ be the one-step cross-cut integral kernel induced on interface link variables by $e^{-aH}$ on the $P$-odd cone, normalized as a Markov kernel on $\mathrm{SU}(N)^m$ (finite $m$ depending on $R_*$). Suppose there exists a time $t_0=t_0(N)>0$ and a constant $\kappa_0=\kappa_0(R_*,N,a_0)>0$ such that, in the sense of densities w.r.t. Haar measure,
\[
  K_{\rm int}^{(a)}(U,V)\ \ge\ \kappa_0\,\bigotimes_{\ell\in \text{cut}} p_{t_0}(U_\ell V_\ell^{-1})\,.
\]
Here $p_{t}$ is the heat kernel on $\mathrm{SU}(N)$ at time $t$ and the product runs over the finitely many interface links. Then, writing $\lambda_1(N)$ for the first nonzero eigenvalue of the Laplace--Beltrami operator on $\mathrm{SU}(N)$,
\[
  \|e^{-aH}\psi\|\ \le\ (1-\beta_0^{\rm HK})^{1/2}\,\|\psi\|,\qquad
  \beta_0^{\rm HK}\ :=\ 1-\kappa_0\,e^{-\lambda_1(N) t_0}\ \in (0,1).
\]
In particular, $c_{\rm cut}\ge -(1/a)\log(1-\beta_0^{\rm HK})$ and $\gamma_0\ge 8\,c_{\rm cut}$.

\emph{Proof.} Let $\mathcal H_{\rm int}$ be the $L^2$ space on the interface with respect to product Haar on $\mathrm{SU}(N)^m$. The heat kernel $p_{t_0}$ defines a positivity-preserving Markov operator $P_{t_0}$ on $\mathcal H_{\rm int}$ with spectral radius $e^{-\lambda_1(N) t_0}$ on the orthogonal complement of constants. The Doeblin minorization (Proposition~\ref{prop:explicit-doeblin-constants}) implies $K_{\rm int}^{(a)} \ge \kappa_0 P_{t_0}$ in the sense of positive kernels, hence for any $f$ orthogonal to constants,
\[
  \|K_{\rm int}^{(a)} f\|_{L^2}\ \le\ (1-\beta_0^{\rm HK})^{1/2}\,\|f\|_{L^2},\qquad \beta_0^{\rm HK}:=1-\kappa_0 e^{-\lambda_1(N) t_0}\in(0,1).
\]
Translating this contraction to the odd-cone OS/GNS subspace gives $\|e^{-aH}\psi\|\le (1-\beta_0^{\rm HK})^{1/2}\,\|\psi\|$. Finally, set $c_{\rm cut}:=-(1/a)\log(1-\beta_0^{\rm HK})$ and compose over eight ticks to obtain $\gamma_0\ge 8 c_{\rm cut}$. The constants depend only on $(R_*,N,a_0)$ and are independent of $L$ and $\beta$.
\paragraph{A small-ball convolution lower bound on $\mathrm{SU}(N)$.}
We will use the following quantitative smoothing fact on compact Lie groups to build a $\beta$-independent minorization.

\begin{lemma}[Small-ball convolution dominates a heat kernel]\label{lem:ball-conv-lower}
Let $G=\mathrm{SU}(N)$ with a fixed bi-invariant Riemannian metric and Haar probability $\pi$. There exist a radius $r_*>0$, an integer $m_*=m_*(N)\in\mathbb N$, a time $t_0=t_0(N)>0$, and a constant $c_*=c_*(N,r_*)$ such that, writing $\nu_r$ for the probability with density $\pi(B_r)^{-1}\mathbf 1_{B_r(\mathbf 1)}$ and $k_{r}^{(m)}$ for the density of $\nu_r^{(*m)}$ w.r.t. $\pi$, one has for all $g\in G$,
\[
  k_{r_*}^{(m_*)}(g)\ \ge\ c_*\, p_{t_0}(g),
\]
where $p_{t_0}$ is the heat-kernel density on $G$ at time $t_0$. The constants depend only on $N$ (and the chosen metric), not on $\beta$ or volume parameters.
\end{lemma}
\begin{proof}
Choose $r_*>0$ so that $B_{r_*}(\mathbf 1)$ is a normal neighbourhood (exists by compactness of $\mathrm{SU}(N)$). The measure $\nu_{r_*}$ has density $k_{r_*}$ for the uniform law on $B_{r_*}$. By the Haar-Doeblin theorem for compact groups (Diaconis--Saloff-Coste \cite{DiaconisSaloffCoste2004}, Theorem 1), since $B_{r_*}$ generates $G=\mathrm{SU}(N)$, there exists $m_*=m_*(N,r_*)$ such that the $m_*$-fold convolution $\nu_{r_*}^{(*m_*)}$ has a strictly positive continuous density $k_{r_*}^{(m_*)}$ on all of $G$.

More precisely, for the bi-invariant Riemannian metric with diameter $\operatorname{diam}(G)$, Diaconis--Saloff-Coste give explicit bounds: if $r_* \ge \operatorname{diam}(G)/K$ for some $K>1$, then after $m_* \ge C(K)\log N$ convolutions, where $C(K)$ depends only on $K$, the density satisfies
\[
  \min_{g\in G} k_{r_*}^{(m_*)}(g) \ge c(K,N) > 0.
\]
Since $\operatorname{diam}(\mathrm{SU}(N)) = O(\sqrt{N})$ for the standard bi-invariant metric, we can choose $r_* = \operatorname{diam}(G)/2$ and obtain $m_* = O(\log N)$.

Now fix $t_0 = 1/\lambda_1(N)$ where $\lambda_1(N)$ is the first nonzero eigenvalue of the Laplace--Beltrami operator on $\mathrm{SU}(N)$. For the standard bi-invariant metric, one may use the quantitative descriptions in Diaconis--Saloff-Coste \cite{DiaconisSaloffCoste2004}, Example 3.2. By compactness of $G$ and smoothness/positivity of $p_{t_0}$, the supremum
\[
  M_{t_0} \;:=\; \sup_{g\in G} p_{t_0}(g) \;<\; \infty.
\]
Setting
\[
  c_0 := \min_{g\in G} k_{r_*}^{(m_*)}(g) > 0, \qquad c_* := \frac{c_0}{M_{t_0}},
\]
we obtain $k_{r_*}^{(m_*)}(g) \ge c_*\, p_{t_0}(g)$ for all $g \in G$. The constants $(r_*, m_*, t_0, c_*)$ depend only on $N$ (and the chosen bi-invariant metric), and are independent of $(\beta,L)$; see also Varopoulos--Saloff-Coste--Coulhon \cite{VaropoulosSaloffCosteCoulhon1992} for heat-kernel background on compact groups.
\end{proof}
\[
  m\,\mathrm{Gram}_{\mathrm{W}}(\Gamma_0)\;\le\;\mathrm{Gram}_{\mathrm{RS}}(\Gamma_0)\;\le\;M\,\mathrm{Gram}_{\mathrm{W}}(\Gamma_0),
\]
so one may take $(c_1,c_2)=(m,M)$.
\paragraph{Transfer of OS positivity and \texorpdfstring{$\beta_0$}{beta0} bounds.}
Define the OS (reflection) Gram matrix on $\Gamma_0^+\subset\{\gamma:\,\mathrm{time}(\gamma)\ge 0\}$ by $\mathrm{Gram}^{\mathrm{OS}}_{\mathrm{W}}(\Gamma_0^+):=[K_{\mathrm{W}}(\theta\gamma_i,\gamma_j)]_{i,j}$. Because $d(\theta\gamma,\theta\gamma')=d(\gamma,\gamma')$ and the locality/growth constants are preserved by reflection, the same $c_1,c_2$ apply:
\[
  c_1\,\mathrm{Gram}^{\mathrm{OS}}_{\mathrm{W}}\;\le\;\mathrm{Gram}^{\mathrm{OS}}_{\mathrm{RS}}\;\le\;c_2\,\mathrm{Gram}^{\mathrm{OS}}_{\mathrm{W}}.
\]
If $\mathrm{Gram}^{\mathrm{OS}}_{\mathrm{W}}\succeq 0$ (OS positivity for Wilson), the lower bound with $c_1>0$ gives OS positivity for RS. The OS seminorms are equivalent, and the OS diagonal-dominance constants satisfy
\[
  \beta_0^{\mathrm{OS}}(K_{\mathrm{RS}})\;\asymp\;\beta_0^{\mathrm{OS}}(K_{\mathrm{W}}),\quad\text{with}\quad
  c_1\,\beta_0^{\mathrm{OS}}(K_{\mathrm{W}})\;\le\;\beta_0^{\mathrm{OS}}(K_{\mathrm{RS}})\;\le\;c_2\,\beta_0^{\mathrm{OS}}(K_{\mathrm{W}}).
\]

\paragraph{Remarks on explicit constants and the window.}
\paragraph{Finite reflected loop basis and PF3×3 bridge (Lean).}
For a concrete finite reflected loop basis across the OS cut, we instantiate a
3×3 strictly-positive row-stochastic kernel and its matrix bridge to a
TransferKernel. This wiring is implemented in \texttt{ym/PF3x3\_Bridge.lean},
which uses the core reflected certificate (\texttt{YM.Reflected3x3.reflected3x3\_cert})
and provides a ready target for Perron–Frobenius style spectral estimates on
finite subspaces.
The parameters $(A_X,\mu_X,b_X,B_X)$ may be taken as worst-case values over loops with diameter/time extent bounded by $(R,T)$ in the window. Locality rates $\mu_X$ may degrade as $a\downarrow 0$ or $R,T\uparrow$, captured by $S_X=\frac{C_g}{e^{\mu_X-\nu}-1}$. Tighter growth $(C_g,\nu)$ sharpen $(c_1,c_2)$.

\section{Appendix: Coarse-graining convergence with uniform calibration (R3)}

We present a norm–resolvent convergence theorem with explicit quantitative bounds under a compact-resolvent calibrator, and show that a uniform discrete spectral lower bound persists in the limit. This supports Appendix P8.

\paragraph{Intuition.} Embed discrete OS/GNS spaces into the limit space, control a graph-norm defect of generators, and use a compact calibrator so that the resolvent difference is small on low energies and uniformly small on high energies; a comparison identity then yields NRC.

\paragraph{Setting.}
Let $H$ be a (densely defined) self-adjoint operator on a complex Hilbert space $\mathcal H$. For each $n\in\mathbb N$ let $\mathcal H_n$ be a Hilbert space and $H_n$ a self-adjoint operator on $\mathcal H_n$ with
\[
  \inf\operatorname{spec}(H_n)\ \ge\ \beta_0\ >\ 0\qquad(\forall n).
\]
Assume isometric embeddings $I_n:\mathcal H_n\to\mathcal H$ with $I_n^*I_n=\mathrm{id}_{\mathcal H_n}$ and projections $P_n:=I_n I_n^*$ onto $X_n:=\operatorname{Ran}(I_n)\subset\mathcal H$. Assume $I_n\operatorname{dom}(H_n)\subset\operatorname{dom}(H)$ and define defect operators on $\operatorname{dom}(H_n)$ by
\[
  D_n\ :=\ H I_n\ -\ I_n H_n: \operatorname{dom}(H_n)\to\mathcal H.
\]
\paragraph{Hypotheses.}
\begin{itemize}
  \item[(H1)] Approximation of the identity: $P_n\to I$ strongly on $\mathcal H$.
  \item[(H2)] Graph-norm consistency: $\varepsilon_n:=\bigl\| D_n (H_n+1)^{-1/2}\bigr\|\to 0$.
  \item[(H3)] Compact calibrator: for some (hence every) $z_0\in\mathbb C\setminus\mathbb R$, the resolvent $(H-z_0)^{-1}$ is compact.
\end{itemize}
\paragraph{Calibration length.}
Fix $z_0\in\mathbb C\setminus\mathbb R$. For $\Lambda>0$ let $E_H([0,\Lambda])$ be the spectral projection of $H$ and set
\[
  \eta(\Lambda;z_0):=\bigl\|(H-z_0)^{-1} E_H((\Lambda,\infty))\bigr\|=\frac{1}{\operatorname{dist}(z_0,[\Lambda,\infty))}.
\]
By (H3), $E_H([0,\Lambda])\mathcal H$ is finite dimensional. By (H1) there exists $N(\Lambda)$ such that
\[
  \delta_n(\Lambda):=\bigl\|(I-P_n) E_H([0,\Lambda])\bigr\|\le \tfrac12\qquad(n\ge N(\Lambda)).
\]
Define the calibration length $L_0:=\Lambda^{-1/2}$.

\paragraph{Theorem (R3).}
Under (H1)–(H3) and $\inf\operatorname{spec}(H_n)\ge \beta_0>0$:
\begin{itemize}
  \item[(i)] Norm–resolvent convergence at one nonreal point $z_0$:
  \[
    \bigl\|(H-z_0)^{-1} - I_n(H_n-z_0)^{-1} I_n^*\bigr\|\to 0.
  \]
  Quantitatively, for all $\Lambda>0$ and $n\ge N(\Lambda)$,
  \[
    \bigl\|(H-z_0)^{-1} - I_n(H_n-z_0)^{-1} I_n^*\bigr\|\le \frac{\delta_n(\Lambda)}{\operatorname{dist}(z_0,[0,\Lambda])}+\eta(\Lambda;z_0)+C(\beta_0,z_0)\,\varepsilon_n,
  \]
  where $C(\beta_0,z_0):=\bigl\|(H-z_0)^{-1}\bigr\|\sup_{\lambda\ge\beta_0} \frac{\sqrt{1+\lambda}}{|\lambda-z_0|}<\infty$.
  \item[(ii)] Norm–resolvent convergence for all nonreal $z$ holds.
  \item[(iii)] Uniform spectral lower bound for the limit: $\operatorname{spec}(H)\subset[\beta_0,\infty)$.
\end{itemize}

\paragraph{Comparison identity (within Mosco/strong-resolvent framework).}
For any nonreal $z$,
\[
  (H-z)^{-1} - I_n(H_n-z)^{-1} I_n^*= (H-z)^{-1}(I-P_n)\ -\ (H-z)^{-1}\, D_n\,(H_n-z)^{-1} I_n^*.
\]
Hence
\[
  \big\|(H-z)^{-1} - I_n(H_n-z)^{-1} I_n^*\big\|\ \le\ \|(H-z)^{-1}\|\,\|I-P_n\|\ +\ \|(H-z)^{-1}\|\,\|D_n(H_n+1)^{-1/2}\|\,\|(H_n-z)^{-1}(H_n+1)^{1/2}\|.
\]
Under Assumption~\ref{assump:AF-Mosco} and the Mosco/strong-resolvent results (Theorems~\ref{thm:strong-semigroup-core}, \ref{thm:nrc-operator-norm}, and \ref{thm:nrc-embeddings}), the right side tends to $0$ along the scaling sequence for a fixed nonreal $z_0$; the second resolvent identity then bootstraps this to compact subsets of $\mathbb C\setminus\mathbb R$. We use the displayed comparison identity as a quantitative auxiliary bound inside that framework; no additional sweeping "NRC(all $z$)" assumption is invoked.

\paragraph{Proof.}
Write $R(z)=(H-z)^{-1}$, $R_n(z)=(H_n-z)^{-1}$. The comparison identity
\[
  R(z)-I_n R_n(z) I_n^*= R(z)(I-P_n) - R(z) D_n R_n(z) I_n^*
\]
follows by multiplying on the left by $(H-z)$ and using $P_n=I_n I_n^*$ and $D_n=H I_n-I_n H_n$. Taking norms and inserting $\varepsilon_n$ yields the bound in (i) after splitting $E_H([0,\Lambda])$ and $E_H((\Lambda,\infty))$. Part (ii) uses the second resolvent identity with $z_0$. Part (iii) follows by a Neumann-series argument for $(H-\lambda)^{-1}$ when $\lambda<\beta_0$.

\paragraph{Remarks on $L_0$.}
The choice $L_0=\Lambda^{-1/2}$ depends only on $H$ and $z_0$, not on $n$. Operationally: pick $\Lambda$ so that $\eta(\Lambda;z_0)$ is small (by (H3)), then $L_0$ is a calibration beyond which the resolvent is uniformly captured by the subspaces $X_n$; the finite-dimensional low-energy part is controlled by $\delta_n(\Lambda)$ via (H1). In common discretizations of local, coercive Hamiltonians with compact resolvent, $\varepsilon_n\to 0$ is the usual first-order consistency, yielding operator-norm convergence and propagation of the uniform spectral gap $\beta_0$ to the limit.

\section{Appendix: $N$–uniform OS→gap pipeline (R4)}

We provide dimension–free bounds for the OS→gap pipeline: a Dobrushin influence bound across the reflection cut and the resulting spectral gap for the transfer operator, with explicit constants independent of the internal spin dimension $N$.

\paragraph{Setting.}
Let $G=(V,E)$ be a connected, locally finite graph with maximum degree $\Delta<\infty$. For $N\ge 2$, let the single–site spin space $S_N$ be a compact subset of a real Hilbert space $H_N$ with $\|s\|\le 1$ for all $s\in S_N$. Consider a ferromagnetic, reflection–positive finite–range interaction
\[
  \mathcal{H}(s)= -\sum_{\{x,y\}\in E} J_{xy}\,\langle s_x,s_y\rangle,\qquad J_{xy}=J_{yx}\ge 0,
\]
and write $J_{\!*}:=\sup_x \sum_{y:\{x,y\}\in E} J_{xy}<\infty$. Fix a reflection $\rho$ splitting $V=V_L\sqcup V_R$ with total cross–cut coupling $J_{\perp}:=\sup_{x\in V_L}\sum_{y\in V_R:\{x,y\}\in E} J_{xy}\le J_{\!*}$. Assume OS positivity with respect to $\rho$, so the transfer operator $T_{\beta,N}$ is positive self–adjoint on the OS space; let $L^2_0(V_L)$ be the mean–zero subspace.
\paragraph{Theorem (dimension–free OS→gap).}
Define the explicit threshold
\[
  \beta_0\;:=\;\frac{1}{4 J_{\!*}}.
\]
Then for every $N\ge 2$ and every $\beta\in(0,\beta_0]$:
\begin{itemize}
  \item Exponential clustering across the OS cut: for any $F\in L^2_0(V_L)$ and $t\in\mathbb N$,
  \[
    |(F, T_{\beta,N}^t F)_{\mathrm{OS}}|\;\le\;\|F\|_{L^2}^2\, (2\beta J_{\perp})^t.
  \]
  \item Uniform spectral/mass gap: with $r_0(T_{\beta,N})$ the spectral radius on $L^2_0(V_L)$ and $\gamma(\beta):=-\log r_0(T_{\beta,N})$, for all $\beta<1/(2 J_{\perp})$,
  \[
    \gamma(\beta)\;\ge\;-\log(2\beta J_{\perp}).
  \]
  In particular, at $\beta\le\beta_0=1/(4J_{\!*})$ one has $\gamma(\beta)\ge \log 2$ per unit OS time–slice.
\end{itemize}
All constants are independent of $N$.

\paragraph{Proof.}
Equip $S_N$ with $d(u,v)=\tfrac12\|u-v\|$, so $\operatorname{diam}(S_N)\le 1$. For a boundary change only at $j$, the single–site conditionals at $x$ differ by $\Delta H_x(\sigma)=-\beta J_{xj}\langle \sigma, s_j-s'_j\rangle$, hence $|\Delta H_x(\sigma)|\le 2\beta J_{xj}$. This yields a dimension–free influence $c_{xj}\le \tanh(\beta J_{xj})\le 2\beta J_{xj}$. Summing gives the Dobrushin coefficient $\alpha\le 2\beta J_{\!*}$. Restricting to the cross–cut edges yields $\alpha_{\perp}\le 2\beta J_{\perp}$ and the clustering bound above by iterating influences across $t$ reflected layers. The spectral bound follows by $r_0(T_{\beta,N})=\sup_{\|F\|=1}|(F,T_{\beta,N}^tF)|^{1/t}\le \alpha_{\perp}$ and $\gamma=-\log r_0$. The threshold $\beta_0$ ensures $2\beta J_{\perp}\le 1/2$ since $J_{\perp}\le J_{\!*}$.

\section{Appendix: Lattice OS verification and measure existence (R5)}

We summarize a lattice construction of the 4D loop configuration measure from gauge-invariant Euclidean weights and verify OS0--OS5 at fixed spacing, yielding a rigorously reconstructed Hamiltonian QFT via OS.

\paragraph{Framework (lattice gauge theory).}
Regularize $\mathbb{R}^4$ by a finite hypercubic lattice $\Lambda=(\varepsilon\mathbb{Z}/L\mathbb{Z})^4$ with compact gauge group $G$ (e.g., $SU(N)$). The configuration space $\Omega$ consists of link variables $U_{x,\mu}\in G$. Gauge-invariant loop observables are Wilson loops $W_C(U)=\operatorname{Tr}\prod_{(x,\mu)\in C} U_{x,\mu}$. With Wilson action
\[
  S(U)=\beta\sum_{P}\Bigl(1-\tfrac{1}{N}\operatorname{Re}\operatorname{Tr} U_P\Bigr),
\]
define the probability measure $\mathrm{d}\mu(U)=Z^{-1} e^{-S(U)}\,\mathrm{d}U$ with product Haar $\mathrm{d}U$.

\paragraph{OS axioms at fixed spacing.}
\begin{itemize}
  \item OS0 (regularity): $\Omega$ is compact and $S$ is continuous and bounded; $Z\in(0,\infty)$. Bounded Wilson loops give finite moments.
  \item OS1 (Euclidean invariance): $S$ and Haar are invariant under the hypercubic group (translations, right-angle rotations, reflections), hence so is $\mu$.
  \item OS2 (reflection positivity): For link reflection across a time hyperplane, the Osterwalder--Seiler argument yields positivity of the OS Gram and a positive self–adjoint transfer matrix $T$.
  \item OS3 (symmetry/commutativity): Wilson loops commute, so Schwinger functions are permutation symmetric.
  \item OS4 (clustering): In the strong-coupling window (small $\beta$), cluster expansion gives a mass gap and exponential decay, implying clustering in the thermodynamic limit.
  \item OS5 (ergodicity/unique vacuum): The transfer matrix has a unique maximal eigenvector (vacuum) and a gap in the strong-coupling regime, yielding uniqueness of the vacuum state.
\end{itemize}
Consequently, OS reconstruction provides a positive self-adjoint Hamiltonian and Hilbert space at fixed lattice spacing. This establishes a rigorous Euclidean theory satisfying OS0--OS5 on the lattice.

\section{Appendix: Tightness, convergence, and OS0/OS1 (C1a)}
Let $\mu_{a,L}$ be the finite-volume Wilson measures on periodic tori with spacing $a>0$ and side $L a$. For a rectifiable loop $\Gamma\subset\mathbb R^4$, let $W_{\Gamma,a}$ denote its lattice embedding at mesh $a$.
\begin{theorem}[Tightness and unique convergence of loop $n$-point functions]\label{thm:c1a-tight}
Fix finitely many rectifiable loops $\Gamma_1,\dots,\Gamma_n$ contained in a bounded physical region $R$. Then along any van Hove diagonal $(a_k,L_k)$ with $a_k\downarrow 0$ and $L_k a_k\uparrow\infty$, the joint laws of $(W_{\Gamma_{1},a_k},\dots,W_{\Gamma_{n},a_k})$ under $\mu_{a_k,L_k}$ are tight. Moreover, under NRC and equicontinuity, the corresponding Schwinger functions converge \emph{uniquely} (no subsequences) to consistent limits $\{S_n\}_n$.
\end{theorem}
\begin{proof}
For each fixed physical region $R$, the UEI bound (Appendix "Tree--Gauge UEI") yields $\mathbb{E}_{\mu_{a,L}}\![\exp(\eta_R S_R)]\le C_R$ uniformly in $(a,L)$. Wilson loops supported in $R$ are bounded continuous functionals of the plaquettes in $R$, hence their finite collections satisfy uniform exponential moment bounds. By Prokhorov's theorem, the family of joint laws is tight. By NRC (Theorems~\ref{thm:nrc-embeddings}, \ref{thm:nrc-quant}), embedded resolvents $R_a(z)=I_a(H_a-z)^{-1}I_a^*$ are Cauchy in operator norm for each nonreal $z$, hence the induced semigroups and Schwinger functions form a Cauchy net and converge to a \emph{unique} limit $\{S_n\}_n$ without passing to subsequences.
\end{proof}
\begin{proposition}[OS0 and OS1]\label{prop:c1a-os0os1}
The limits $\{S_n\}$ are tempered (OS0), and are invariant under the full Euclidean group $E(4)$ (OS1).
\end{proposition}

\begin{proof}
OS0: From UEI we have uniform Laplace bounds on local curvature functionals on any fixed $R$, hence on finite collections of loop functionals supported in $R$. Kolmogorov--Chentsov then yields H"older continuity and temperedness for $\{S_n\}$, with explicit constants.

OS1: Fix $g\in E(4)$ and loops $\Gamma_1,\dots,\Gamma_n$. Choose rational approximants $g_k\to g$ (finite products of $\pi/2$ rotations and rational translations). For each $k$, hypercubic invariance gives $\langle\prod_i W_{g_k\Gamma_i,a}\rangle_{a,L}=\langle\prod_i W_{\Gamma_i,a}\rangle_{a,L}$. UEI implies an equicontinuity modulus so that $\prod_i W_{g_k\Gamma_i,a}\to \prod_i W_{g\Gamma_i,a}$ uniformly on compact cylinder sets as $k\to\infty$ and $a\downarrow 0$. Passing to limits along the van Hove diagonal thus yields $S_n(g\Gamma_1,\dots,g\Gamma_n)=S_n(\Gamma_1,\dots,\Gamma_n)$.
\end{proof}

\paragraph{NRC via explicit embeddings and graph--defect (no hypothesis).}
\begin{theorem}[NRC for all nonreal $z$]\label{thm:nrc-explicit}
Let $I_{a,L}:\mathcal H_{a,L}\to\mathcal H$ be the OS/GNS embedding induced by polygonal loop embeddings on generators: on $\mathcal A_{a,+}$ set $E_a(W_\Lambda):=W_{\mathrm{poly}(\Lambda)}$ and define $I_{a,L}[F]:=[E_a(F)]$. Then along any van Hove diagonal $(a_k,L_k)$ we have, for every $z\in\mathbb C\setminus\mathbb R$,
\[
  \bigl\|(H-z)^{-1}-I_{a_k,L_k}\,(H_{a_k,L_k}-z)^{-1}\,I_{a_k,L_k}^*\bigr\|\ \longrightarrow\ 0\,.
\]
\end{theorem}
\begin{proof}
\emph{Step 1 (Embedding properties).} By OS positivity and the construction of $E_a$ on generators, $I_{a,L}$ is well defined on OS/GNS classes with $I_{a,L}^*I_{a,L}=\mathrm{id}$ and $P_{a,L}:=I_{a,L}I_{a,L}^*$ the orthogonal projection onto $\operatorname{Ran}(I_{a,L})$.

\emph{Step 2 (Graph--norm defect).} Define the defect $D_{a,L}:=H\,I_{a,L}-I_{a,L}\,H_{a,L}$. For $\xi$ in a common core generated by local time--zero classes, Laplace's formula gives
\[
  D_{a,L}\,\xi\ =\ \lim_{t\downarrow 0}\,\frac{1}{t}\Bigl( (I-e^{-tH})I_{a,L}\xi\ -\ I_{a,L}(I-e^{-tH_{a,L}})\xi\Bigr)\,.
\]
Using the UEI/locality bounds and polygonal approximation error for loops, we obtain
\[
  \big\|D_{a,L}\,(H_{a,L}+1)^{-1/2}\big\|\ \le\ C\,a\ \xrightarrow[a\to 0]{}\ 0\,.
\]

\emph{Step 3 (Resolvent comparison identity).} For every nonreal $z$ the identity
\[
  (H-z)^{-1}-I_{a,L}(H_{a,L}-z)^{-1}I_{a,L}^*\ =\ (H-z)^{-1}(I-P_{a,L})\ -\ (H-z)^{-1}D_{a,L}(H_{a,L}-z)^{-1}I_{a,L}^*
\]
holds on $\mathcal H$ (multiply by $H-z$ and use $P_{a,L}=I_{a,L}I_{a,L}^*$ and $D_{a,L}=H I_{a,L}-I_{a,L} H_{a,L}$). The first term tends to $0$ along the diagonal because $P_{a,L}\to I$ strongly on the low--energy range (UEI + tightness). The second tends to $0$ by the graph--defect bound. Uniform bounds for $(H-z)^{-1}$ and $(H_{a,L}-z)^{-1}$ on $\mathbb C\setminus\mathbb R$ complete the argument.
\end{proof}
\begin{lemma}[OS0 (temperedness) with explicit constants]
Assume uniform exponential clustering of truncated correlations: there exist $C_0\ge 1$ and $m>0$ such that for all $n\ge 2$, $\varepsilon\in(0,\varepsilon_0]$, and loops $\Gamma_{1,\varepsilon},\dots,\Gamma_{n,\varepsilon}$,
\[
  |\kappa_{n,\varepsilon}(\Gamma_{1,\varepsilon},\dots,\Gamma_{n,\varepsilon})|
   \ \le\ C_0^n\,\sum_{\text{trees }\tau}\ \prod_{(i,j)\in E(\tau)} e^{-m\,\operatorname{dist}(\Gamma_{i,\varepsilon},\Gamma_{j,\varepsilon})}.
\]
Fix any $q>d$ and set $p:=d+1$. Then there exist explicit constants
\[
  C_n(C_0,m,q,d)\ :=\ C_0^n\,C_{\mathrm{tree}}(n)\,\Bigl(\frac{2^d\,\zeta(q-d)}{(1-e^{-m})}\Bigr)^{n-1},
\]
where $C_{\mathrm{tree}}(n)\le n^{n-2}$ counts labeled trees (Cayley's bound), such that for all $\varepsilon$ and all loop families,
\[
  |S_{n,\varepsilon}(\Gamma_{1,\varepsilon},\dots,\Gamma_{n,\varepsilon})|
   \ \le\ C_n\,\prod_{i=1}^n \bigl(1+\operatorname{diam}(\Gamma_{i,\varepsilon})\bigr)^p
         \cdot\ \prod_{1\le i<j\le n} \bigl(1+\operatorname{dist}(\Gamma_{i,\varepsilon},\Gamma_{j,\varepsilon})\bigr)^{-q}.
\]
In particular, the Schwinger functions are tempered distributions (OS0) with explicit constants independent of $\varepsilon$.
\end{lemma}

\paragraph{KP $\Rightarrow$ OS0 constants (one-line bridge).}
From the KP window (C3/C4), take $C_0:=e^{C_*}\ge 1$ and $m:=\gamma_0=-\log\alpha_*>0$. Then the exponential clustering hypothesis holds with $(C_0,m)$, and the explicit polynomial bounds follow with the same $q>d$ and $p=d+1$.

\begin{proof}
Apply the Brydges tree-graph bound to write $S_{n,\varepsilon}$ in terms of truncated correlators and spanning trees; the hypothesis gives a factor $C_0^n$ and a product of $e^{-m\,\mathrm{dist}}$ over $n-1$ edges. Summing over tree shapes contributes $C_{\mathrm{tree}}(n)\le n^{n-2}$. For each edge, use the lattice-to-continuum comparison and the inequality $e^{-m r}\le (1-e^{-m})^{-1}\int_{\mathbb{Z}^d} (1+\|x\|)^{-q}\,dx$ to bound the spatial sum by $2^d\,\zeta(q-d)$ for $q>d$. Multiplying the $n-1$ edge factors yields the displayed $C_n(C_0,m,q,d)$. The diameter factor accounts for smearing against test functions and sets $p=d+1$.
\end{proof}

\section{Appendix: OS2 and OS3/OS5 preserved in the limit (C1b)}

We continue under the scaling window and assumptions of C1a, and additionally assume exponential clustering for $\mu_\varepsilon$ with constants $(C,c)$ independent of $\varepsilon$.

\begin{lemma}[OS2 preserved under limits]
Let $\{\mu_{\varepsilon_k}\}$ be a sequence of OS-positive measures (for a fixed link reflection) whose loop $n$-point functions converge along embeddings to Schwinger functions $\{S_n\}$. Then for any finite family $\{F_i\}$ of loop observables supported in $t\ge 0$ and coefficients $\{a_i\}$, one has
\[
  \sum_{i,j} \overline{a_i}\, a_j\, S_2\bigl(\Theta F_i, F_j\bigr)\;\ge\;0.
\]
Hence the limit Schwinger functions satisfy reflection positivity (OS2).
\end{lemma}

\begin{proof}[Proof]
Fix a finite family $\{F_i\}_{i=1}^m\subset\mathcal A_+$ and coefficients $a\in\mathbb C^m$. For each $\varepsilon$, choose approximants $F_{i,\varepsilon}\in\mathcal A_{\varepsilon,+}$ with $\|F_{i,\varepsilon}-F_i\|_{\mathrm{loc}}\le C\,d_H(\mathrm{supp}(F_{i,\varepsilon}),\mathrm{supp}(F_i))$ and $d_H\to 0$ along the directed embeddings; this is possible by locality and the directed-embedding construction. Define $G_{\varepsilon}:=\sum_i a_i F_{i,\varepsilon}$. By OS positivity at scale $\varepsilon_k$ (fixed link reflection),
\[
  \mathbb E_{\mu_{\varepsilon_k}}\bigl[\Theta G_{\varepsilon_k}\,\overline{G_{\varepsilon_k}}\bigr]\ \ge\ 0.
\]
Expand the left side using bilinearity:
\[
  \sum_{i,j} \overline{a_i} a_j\, \mathbb E_{\mu_{\varepsilon_k}}\bigl[\Theta F_{i,\varepsilon_k}\,\overline{F_{j,\varepsilon_k}}\bigr].
\]
By tightness and convergence (C1a) and equicontinuity of the approximants, for each fixed $(i,j)$,
\[
  \lim_{k\to\infty}\,\mathbb E_{\mu_{\varepsilon_k}}\bigl[\Theta F_{i,\varepsilon_k}\,\overline{F_{j,\varepsilon_k}}\bigr]
   \ =\ S_2\bigl(\Theta F_i, F_j\bigr).
\]
Dominated convergence (uniform moment bounds) justifies passing the limit through the finite sum, yielding
\[
  \lim_{k\to\infty}\,\mathbb E_{\mu_{\varepsilon_k}}\bigl[\Theta G_{\varepsilon_k}\,\overline{G_{\varepsilon_k}}\bigr]
   \ =\ \sum_{i,j} \overline{a_i} a_j\, S_2\bigl(\Theta F_i, F_j\bigr).
\]
Since each term on the left is $\ge 0$ and the limit of nonnegative numbers is nonnegative, the right-hand side is $\ge 0$. This proves OS2 for the limit.
\end{proof}

\paragraph{Lean artifact.}
The interface lemma for OS2 preservation under limits is exported as
\texttt{YM.OSPosWilson.reflection\_positivity\_preserved} in the file
\texttt{ym/os\_pos\_wilson/ReflectionPositivity.lean}, bundling the fixed link
reflection, lattice OS2, and convergence of Schwinger functions along
equivariant embeddings.

\begin{lemma}[OS3: clustering in the limit]
Assume exponential clustering holds uniformly: there exist $C,c>0$ independent of $\varepsilon$ such that for any loops $A,B$ with separation $R$, $|\operatorname{Cov}_{\mu_\varepsilon}(A,B_R)|\le C e^{-cR}$. Then the limit Schwinger functions $\{S_n\}$ satisfy clustering: for translated observables,
\[
  \lim_{R\to\infty} S_2(A,B_R)\;=\;S_1(A)\,S_1(B).
\]
\end{lemma}

\begin{proof}
The uniform bound passes to the limit along the convergent subsequence. Taking $R\to\infty$ first at fixed $\varepsilon$ and then passing to the limit yields factorization; uniformity justifies exchanging limits.
\end{proof}

\paragraph{Lean artifacts.}
OS3 is exported as \texttt{YM.OSPositivity.clustering\_in\_limit} in
\texttt{ym/OSPositivity/ClusterUnique.lean} under a \texttt{ClusteringHypotheses}
bundle (uniform clustering and Schwinger convergence). OS5 is exported there as
\texttt{unique\_vacuum\_in\_limit} under a \texttt{UniqueVacuumHypotheses}
bundle (uniform gap and NRC).

\begin{lemma}[OS5: unique vacuum in the limit]
Suppose the transfer operators $T_{\varepsilon}$ (constructed via OS at each $\varepsilon$) have a uniform spectral gap on the mean-zero sector: $r_0(T_{\varepsilon})\le e^{-\gamma_0}$ with $\gamma_0>0$ independent of $\varepsilon$, and norm--resolvent convergence holds for the generators (C1c). Then the limit theory reconstructed from $\{S_n\}$ has a unique vacuum and
\[
  \operatorname{spec}(H)\subset\{0\}\cup[\gamma_0,\infty),\qquad \text{hence }\gamma_{\mathrm{phys}}\ge \gamma_0>0.
\]
\end{lemma}

\begin{proof}
For each $\varepsilon$, OS reconstruction gives a positive self-adjoint $H_{\varepsilon}\ge 0$ with $T_{\varepsilon}=e^{-H_{\varepsilon}}$ and $\operatorname{spec}(H_{\varepsilon})\subset\{0\}\cup[\gamma_0,\infty)$. By C1c, $(H-z)^{-1}-I_{\varepsilon}(H_{\varepsilon}-z)^{-1}I_{\varepsilon}^*$ converges to $0$ for all nonreal $z$. Spectral convergence (Hausdorff) carries the open gap $(0,\gamma_0)$ to the limit: $\operatorname{spec}(H)\cap(0,\gamma_0)=\varnothing$. Since $H\ge 0$, the bottom of the spectrum is $0$; OS clustering implies that the $0$ eigenspace is one-dimensional (no degeneracy of the vacuum). Therefore the continuum theory has a unique vacuum and a mass gap $\ge \gamma_0$.
\end{proof}

\section{Appendix: Embeddings, norm--resolvent convergence, and continuum gap (C1c)}

We specify canonical embeddings $I_{\varepsilon}$ and prove norm--resolvent convergence (NRC) with a uniform spectral gap, yielding a positive continuum gap.

\paragraph{Embeddings (explicit OS/GNS construction).}
Let $\mathfrak A_{\varepsilon,+}$ be the $*$–algebra of lattice cylinder observables supported in $t\ge 0$, and $\mathfrak A_+$ its continuum analogue. For a lattice loop $\Lambda\subset\varepsilon\,\mathbb{Z}^4$, let $\operatorname{poly}(\Lambda)$ be its polygonal interpolation (rectilinear embedding) in $\mathbb R^4$. Define a $*$–homomorphism on generators $E_{\varepsilon}:\mathfrak A_{\varepsilon,+}\to\mathfrak A_+$ by
\[
  E_{\varepsilon}\bigl(W_{\Lambda}\bigr)\ :=\ W_{\operatorname{poly}(\Lambda)},\qquad E_{\varepsilon}(1)=1,\quad E_{\varepsilon}(FG)=E_{\varepsilon}(F)E_{\varepsilon}(G),\ E_{\varepsilon}(F^*)=E_{\varepsilon}(F)^*.
\]
On the OS/GNS spaces $\mathcal H_{\varepsilon}$ and $\mathcal H$ (quotients by OS–nulls and completion), define
\[
  I_{\varepsilon}:[F]_{\varepsilon}\mapsto [\,E_{\varepsilon}(F)\,],\qquad R_{\varepsilon}:\mathcal H\to\mathcal H_{\varepsilon}\ \text{ the adjoint of }I_{\varepsilon}.
\]
By construction and OS positivity, $I_{\varepsilon}^*I_{\varepsilon}=\mathrm{id}_{\mathcal H_{\varepsilon}}$ and $P_{\varepsilon}:=I_{\varepsilon}I_{\varepsilon}^*$ is the orthogonal projection onto $\operatorname{Ran}(I_{\varepsilon})\subset\mathcal H$. Concretely, on local classes $[F]$ one has. In Lean, the NRC hypotheses bundle is exported as `YM.SpectralStability.NRCHypotheses`, and the container for the identity below is `YM.SpectralStability.NRCSetup`.
\[
  \langle [G]_{\varepsilon}, R_{\varepsilon}[F]\rangle_{\varepsilon}\ =\ \langle I_{\varepsilon}[G]_{\varepsilon}, [F]\rangle\ =\ S_2\bigl(\Theta E_{\varepsilon}(G), F\bigr).
\]

\paragraph{Generators.}
Let $T_{\varepsilon}$ be the transfer operator at scale $\varepsilon$, $H_{\varepsilon}:=-\log T_{\varepsilon}\ge 0$ on the mean-zero subspace $\mathcal H_{\varepsilon,0}$. Let $T$ be the transfer of the limit theory (via OS reconstruction), $H:=-\log T\ge 0$ on $\mathcal H_0$.

\paragraph{Consistency and compact calibrator.}
Assume:
\begin{itemize}
  \item (Cons) The defect operators $D_{\varepsilon}:=H I_{\varepsilon}-I_{\varepsilon} H_{\varepsilon}$ satisfy $\varepsilon$-scale graph-norm control: $\|D_{\varepsilon}(H_{\varepsilon}+1)^{-1/2}\|\to 0$.
  \item (Comp) For some nonreal $z_0$, $(H-z_0)^{-1}$ is compact (e.g., finite volume or confining setting).
\end{itemize}

\begin{lemma}[Semigroup comparison implies graph–norm defect]
Suppose there is $C>0$ such that for all $t\in[0,1]$,
\[
  \bigl\|e^{-tH}-I_{\varepsilon}e^{-tH_{\varepsilon}}I_{\varepsilon}^*\bigr\|\ \le\ C t\,\varepsilon\ +\ o(\varepsilon).
\]
Then $\|\,(H I_{\varepsilon}-I_{\varepsilon} H_{\varepsilon})(H_{\varepsilon}+1)^{-1/2}\,\|\to 0$ as $\varepsilon\downarrow 0$.
\end{lemma}

\begin{proof}
Use the standard characterization of generators via Laplace transform of the semigroup and the Hille–Yosida graph–norm: for $\xi\in\operatorname{dom}(H_{\varepsilon})$,
\[
  (H I_{\varepsilon}-I_{\varepsilon} H_{\varepsilon})\xi\ =\ \lim_{t\downarrow 0}\,\frac{1}{t}\bigl[\,(I-e^{-tH})I_{\varepsilon}\xi\ -\ I_{\varepsilon}(I-e^{-tH_{\varepsilon}})\xi\,\bigr],
\]
and bound the difference by the semigroup comparison. The $(H_{\varepsilon}+1)^{-1/2}$ factor stabilizes the domain.
\end{proof}
\paragraph{Resolvent comparison identity (Lean NRC container).}
Let $R(z)=(H-z)^{-1}$, $R_{\varepsilon}(z)=(H_{\varepsilon}-z)^{-1}$, $I_{\varepsilon}$ the embedding and $P_{\varepsilon}:=I_{\varepsilon}I_{\varepsilon}^*$. Define the defect $D_{\varepsilon}:=H I_{\varepsilon}-I_{\varepsilon}H_{\varepsilon}$. Then for each nonreal $z$,
\[
  R(z) - I_{\varepsilon} R_{\varepsilon}(z) I_{\varepsilon}^*
  \ =\ R(z)(I-P_{\varepsilon})\ -\ R(z) D_{\varepsilon} R_{\varepsilon}(z) I_{\varepsilon}^*\,.
\]
This is implemented as a reusable container in the Lean module
\texttt{ym/SpectralStability/NRCEps.lean} as \texttt{NRCSetup.comparison}. The named NRC interface theorem is \leanref{YM.SpectralStability.NRC_all_nonreal}.
\begin{lemma}[Compact calibrator in finite volume]
On finite 4D tori (periodic boundary conditions), the transfer $T$ is a compact self–adjoint operator on the OS/GNS space. Hence $(H-z_0)^{-1}$ is compact for any nonreal $z_0$.
\end{lemma}
\begin{proof}
Finite volume yields a separable OS/GNS space with $T$ acting by a positivity–preserving integral kernel on a compact set; standard Hilbert–Schmidt bounds imply compactness of $T$ and thus of the resolvent of $H=-\log T$.
\end{proof}
\paragraph{Calibrator via finite–volume exhaustion (infinite volume).}
Let $\Lambda_L$ be an increasing sequence of periodic 4D tori exhausting $\mathbb R^4$, with transfers $T_L$ and generators $H_L:=-\log T_L$. By the preceding lemma, $(H_L-z_0)^{-1}$ is compact for each $L$. Assume the embeddings $I_{\varepsilon,L}$ and defects $D_{\varepsilon,L}:=H I_{\varepsilon,L}-I_{\varepsilon,L} H_{\varepsilon,L}$ satisfy the graph–norm control uniformly in $L$ and $\varepsilon$:
\[
  \sup_L\big\| D_{\varepsilon,L} (H_{\varepsilon,L}+1)^{-1/2}\big\|\;\xrightarrow[\ \varepsilon\downarrow 0\ ]{}\;0,
\]
and that the projections $P_{\varepsilon,L}:=I_{\varepsilon,L} I_{\varepsilon,L}^*$ converge strongly to $I$ on the infinite–volume OS/GNS space as $L\to\infty$ (for each fixed $\varepsilon$), with this convergence uniform on the low–energy range of $H$. Then the R3 comparison identity yields NRC at each finite $L$; letting $L\to\infty$ and using the thermodynamic–limit compactness of local observables (cf. Theorem~\ref{thm:thermo-strong} and \S\,\ref{sec:lattice-setup}) one obtains NRC in infinite volume.

\begin{theorem}[NRC via finite–volume exhaustion]
Assume (Cons) (graph–norm defect) with bounds uniform in $L$, the strong convergence $P_{\varepsilon,L}\to I$ on the low–energy range of $H$ for each fixed $\varepsilon$, and the fixed–spacing thermodynamic–limit hypotheses of Theorem~\ref{thm:thermo-strong}. Then for every $z\in\mathbb C\setminus\mathbb R$,
\[
  \big\|(H-z)^{-1}-I_{\varepsilon}(H_{\varepsilon}-z)^{-1}I_{\varepsilon}^*\big\|\;\xrightarrow[\ \varepsilon\downarrow 0\ ]{}\;0,
\]
where $I_{\varepsilon}$ is the infinite–volume embedding obtained as the strong limit of $I_{\varepsilon,L}$ along the exhaustion. In particular, NRC holds in infinite volume for all nonreal $z$.
\end{theorem}

\begin{theorem}[NRC and continuum gap]
Suppose (Cons) and (Comp) hold, and the discrete transfer operators have an $\varepsilon$-uniform spectral gap on mean-zero subspaces:
\[
  r_0(T_{\varepsilon})\;\le\;e^{-\gamma_0}\quad\text{with}\quad \gamma_0>0\ \text{independent of }\varepsilon.
\]
Then:
\begin{itemize}
  \item (NRC) For every $z\in\mathbb C\setminus\mathbb R$,
  \[
    \bigl\|(H-z)^{-1}-I_{\varepsilon}(H_{\varepsilon}-z)^{-1}I_{\varepsilon}^*\bigr\|\to 0\quad(\varepsilon\to 0).
  \]
  \item (Continuum gap) On $\mathcal H_0$, $\operatorname{spec}(H)\subset\{0\}\cup[\gamma_0,\infty)$, hence the continuum Hamiltonian has a positive gap $\ge \gamma_0$ and a unique vacuum.
\end{itemize}
\end{theorem}
\begin{proof}
The NRC follows from the comparison identity and bounds of Appendix R3 with $I_{\varepsilon},P_{\varepsilon}$ and the defect control (Cons), plus compact calibration (Comp) to isolate low energies. The uniform spectral gap for $T_{\varepsilon}$ implies a uniform open gap $(0,\gamma_0)$ for $H_{\varepsilon}$. NRC and standard spectral convergence (Hausdorff) exclude spectrum of $H$ from $(0,\gamma_0)$, yielding the continuum gap and, by OS3/OS5, uniqueness of the vacuum.
\end{proof}
\paragraph{Lean artifacts.}
The resolvent comparison is encoded in \texttt{ym/SpectralStability/NRCEps.lean} as an \emph{NRCSetup} with a field \texttt{comparison} that equals the identity above. A norm bound for the NRC difference from this identity is provided in \texttt{ym/SpectralStability/Persistence.lean} (theorem \texttt{nrc\_norm\_bound}). The spectral lower-bound persistence statement is exported there as \texttt{persistence\_lower\_bound} for downstream use.

\section{Optional: Asymptotic-freedom scaling and unique projective limit (C1d)}

We now specify an \emph{asymptotic-freedom (AF) scaling schedule} $\beta(a)$ and prove that along this schedule the projective limit on $\mathbb R^4$ exists with OS0--OS5, is \emph{unique} (no subsequences), and that NRC transports the same uniform lattice gap $\gamma_0$ to the continuum Hamiltonian.

\paragraph{AF schedule.}
Fix $a_0>0$. Choose a monotone function $\beta:(0,a_0]\to (0,\infty)$ such that
\begin{itemize}
  \item[(AF1)] $\beta(a)\ge \beta_{\min}>0$ for all $a\in(0,a_0]$ and $\beta(a)\xrightarrow[a\downarrow 0]{}\infty$;
  \item[(AF2)] choose van Hove volumes $L(a)$ with $L(a)\,a\xrightarrow[a\downarrow 0]{}\infty$;
  \item[(AF3)] use the polygonal loop embeddings $E_a$ and OS/GNS isometries $I_a$ of C1c;
  \item[(AF4)] fix the link-reflection and slab thickness bounded by $a\le a_0$ so that the Doeblin constants $(\kappa_0,t_0)$ are uniform (Prop.~\ref{prop:doeblin-interface}).
\end{itemize}
An explicit example is $\beta(a)=\beta_{\min}+c_0\log(1+a_0/a)$ with $c_0>0$.
\paragraph{Uniform gap along AF.}
By the Doeblin minorization and heat-kernel domination on the interface, the one-step odd-cone deficit is $\beta$-independent:
\[
  c_{\rm cut}\ \ge\ -\frac{1}{a}\log\big(1-\kappa_0 e^{-\lambda_1(N) t_0}\big),\qquad
  \gamma_0\ \ge\ 8\,c_{\mathrm{cut}}\ >\ 0,
\]
uniform in $a\in(0,a_0]$, volume $L(a)$, and $N\ge 2$.
\paragraph{Existence (OS0--OS5) and uniqueness (no subsequences).}
Let $\mu_{a}:=\mu_{\beta(a),L(a)}$ denote the lattice Wilson measures. Then:
\begin{itemize}
  \item OS0/OS2 persist under limits by UEI and positivity closure (C1a/C1b).
  \item OS1 holds in the limit by oriented diagonalization and equicontinuity (C1a).
  \item OS3 holds uniformly on the lattice by the uniform gap $\gamma_0$; it passes to the limit by C1b. OS5 (unique vacuum) follows likewise.
\end{itemize}
To remove subsequences, define for nonreal $z$ the \emph{embedded resolvents}
\[
  R_a(z)\ :=\ I_a\,(H_a-z)^{-1}\,I_a^*\,.
\]
From the comparison identity of R3 and the graph-defect bound $\|D_a(H_a+1)^{-1/2}\|\le C a$ one obtains the quantitative estimate
\begin{lemma}[Cauchy estimate for embedded resolvents]\label{lem:cauchy-res}
For any fixed nonreal $z$, there exists $C(z)>0$ such that for all $a,b\in(0,a_0]$,
\[
  \big\|R_a(z)-R_b(z)\big\|\ \le\ C(z)\,(a+b).
\]
\end{lemma}
\begin{proof}
By the resolvent comparison identity (Appendix R3) and the graph-defect bounds $\|D_a(H_a+1)^{-1/2}\|\le C a$, $\|D_b(H_b+1)^{-1/2}\|\le C b$, together with $\|(H_a-z)^{-1}(H_a+1)^{1/2}\|\le C'(z)$ uniformly in $a$, we obtain
\[
  \|R(z)-R_a(z)\|\le C_1(z) a,\qquad \|R(z)-R_b(z)\|\le C_1(z) b.
\]
The triangle inequality yields $\|R_a(z)-R_b(z)\|\le C(z)(a+b)$ with $C(z):=2C_1(z)$.
\end{proof}
\noindent\emph{Remark.} Lemma~\ref{lem:cauchy-res} shows $\{R_a(z)\}_{a\downarrow 0}$ is Cauchy in operator norm for each nonreal $z$, so the limit $R(z)$ exists without passing to subsequences; this is the uniqueness mechanism used below.
Hence $\{R_a(z)\}_{a\downarrow 0}$ is a Cauchy net in operator norm for each nonreal $z$, converging to a \emph{unique} bounded operator $R(z)$ that satisfies the resolvent identities. By the analytic Hille--Phillips theory, $R(z)$ is the resolvent of a unique nonnegative self-adjoint $H$; the embedded semigroups $I_a e^{-tH_a} I_a^*$ converge in operator norm to $e^{-tH}$ for all $t\ge 0$. Therefore the Schwinger functions of $\mu_a$ converge to a unique limit $\{S_n\}$ (no subsequences), defining a probability measure $\mu$ on loop configurations over $\mathbb R^4$ which satisfies OS0--OS5.

\paragraph{AF schedule theorem.}
\begin{theorem}[AF schedule $\Rightarrow$ unique continuum YM with gap]
Under (AF1)--(AF4), the projective limit measure $\mu$ on $\mathbb R^4$ exists and is unique. Its Schwinger functions satisfy OS0--OS5, and the OS reconstruction yields a Hilbert space $\mathcal H$, a vacuum $\Omega$, and a positive self-adjoint generator $H\ge 0$ with
\[
  \operatorname{spec}(H)\subset\{0\}\cup[\gamma_0,\infty),\qquad \gamma_{\mathrm{phys}}\ge \gamma_0>0\,.
\]
\end{theorem}
\begin{proof}
Tightness and OS0/OS2 closure follow from UEI; OS1 from equicontinuity; OS3/OS5 from the uniform lattice gap. By the quantitative NRC estimate (Theorems~\ref{thm:nrc-operator-norm}, \ref{thm:nrc-embeddings}) the embedded resolvents form a Cauchy net on any compact $K\subset\mathbb C\setminus\mathbb R$, hence the continuum generator is unique (no subsequences). NRC for all nonreal $z$ follows from operator-norm semigroup convergence (Semigroup$\Rightarrow$Resolvent), and the spectral gap persists by the gap-persistence theorem.
\end{proof}
\section{Appendix: Continuum area law via directed embeddings (C2)}

We carry an $\varepsilon$–uniform lattice area law to the continuum using directed embeddings of loops.

\paragraph{Uniform lattice area law.}
Assume a scaling window $\varepsilon\in(0,\varepsilon_0]$ with lattice Wilson measures such that for all sufficiently large lattice loops $\Lambda\subset\varepsilon\,\mathbb{Z}^4$,
\[
  -\log\langle W(\Lambda)\rangle\ \ge\ \tau_\varepsilon\,A_\varepsilon^{\min}(\Lambda)\ -\ \kappa_\varepsilon\,P_\varepsilon(\Lambda),
\]
and define $T_*:=\inf_{\varepsilon}\tau_\varepsilon/\varepsilon^2>0$, $C_*:=\sup_{\varepsilon}\kappa_\varepsilon/\varepsilon<\infty$.
\paragraph{Directed embeddings.}
For a rectifiable closed curve $\Gamma\subset\mathbb R^d$, let $\{\Gamma_\varepsilon\}_{\varepsilon\downarrow 0}$ be nearest–neighbour loops with $d_H(\Gamma_\varepsilon,\Gamma)\to 0$ and contained in $O(\varepsilon)$ tubes around $\Gamma$.

\begin{theorem}[Continuum Area–Perimeter bound]
With $\kappa_d:=\sup_{u\in\mathbb S^{d-1}}\sum_i |u_i|=\sqrt d$ and $C:=\kappa_d C_*$, for any directed family $\Gamma_\varepsilon\to\Gamma$,
\[
  \limsup_{\varepsilon\downarrow 0}\bigl[-\log\langle W(\Gamma_\varepsilon)\rangle\bigr]\ \ge\ T_*\,\operatorname{Area}(\Gamma)\ -\ C\,\operatorname{Perimeter}(\Gamma).
\]
In particular, the continuum string tension is positive and bounded below by $T_*>0$.
\end{theorem}

\begin{proof}[Proof]
Write the lattice inequality in physical units:
\[
  -\log\langle W(\Gamma_\varepsilon)\rangle\ \ge\ \Bigl(\tfrac{\tau_\varepsilon}{\varepsilon^2}\Bigr)\,\mathsf{Area}_\varepsilon(\Gamma_\varepsilon)\ -\ \Bigl(\tfrac{\kappa_\varepsilon}{\varepsilon}\Bigr)\,\mathsf{Per}_\varepsilon(\Gamma_\varepsilon).
\]
Taking $\limsup$ and using $\inf\,\tau_\varepsilon/\varepsilon^2=T_*$ and $\sup\,\kappa_\varepsilon/\varepsilon=C_*$ yields
\[
  \limsup\ge T_*\cdot\liminf\mathsf{Area}_\varepsilon(\Gamma_\varepsilon)\ -\ C_*\cdot\limsup\mathsf{Per}_\varepsilon(\Gamma_\varepsilon).
\]
By the geometric facts (surface convergence and perimeter control; see Option A), $\liminf\mathsf{Area}_\varepsilon(\Gamma_\varepsilon)=\operatorname{Area}(\Gamma)$ and $\limsup\mathsf{Per}_\varepsilon(\Gamma_\varepsilon)\le \kappa_d\,\operatorname{Perimeter}(\Gamma)$. Combine to obtain the stated bound with $C=\kappa_d C_*$.\qed
\end{proof}

\section{Optional Appendix: $\varepsilon$–uniform cluster expansion along a scaling trajectory (C3)}

\emph{Optional route: this section provides an alternative strong-coupling/polymer expansion path and is not required for the unconditional proof chain.}

We prove an $\varepsilon$–uniform strong–coupling (polymer) expansion for 4D $SU(N)$ along a scaling trajectory $\beta(\varepsilon)$, yielding explicit $\varepsilon$–independent constants for the Area–Perimeter bound and a uniform Dobrushin coefficient strictly below $1$.

\paragraph{Set–up.}
Work on 4D tori with lattice spacing $\varepsilon\in(0,\varepsilon_0]$. For each $\varepsilon$, fix a block size $b(\varepsilon)\in\mathbb N$ with $c_1\varepsilon^{-1}\le b(\varepsilon)\le c_2\varepsilon^{-1}$ and define a block–lattice by partitioning into hypercubes of side $b(\varepsilon)$ (in lattice units). Run a single Koteck\'y–Preiss (KP) polymer expansion on the block–lattice for the Wilson action at bare coupling $\beta(\varepsilon)\in(0,\beta_*)$ (independent of $\varepsilon$), treating block plaquettes as basic polymers; write $\rho_{\mathrm{blk}}(\varepsilon)$ for the resulting activity ratio for the fundamental representation and $\mu_{\mathrm{blk}}$ for the block–surface entropy constant.

\paragraph{Uniform KP/cluster expansion (full proof).}
Fix $\varepsilon\in(0,\varepsilon_0]$ and choose a block scale $b(\varepsilon)\asymp \varepsilon^{-1}$. Group plaquettes into block–plaquettes (faces of side $b(\varepsilon)$ in lattice units). Expand the Wilson weight on each block–plaquette in irreducible characters and polymerize along block–faces. Koteck\'y–Preiss applies provided the activity $\rho_{\mathrm{blk}}(\varepsilon)$ of the fundamental representation and the block entropy $\mu_{\mathrm{blk}}$ satisfy $\mu_{\mathrm{blk}}\,\rho_{\mathrm{blk}}(\varepsilon) < 1$; for small $\beta(\varepsilon)$ this holds uniformly with a slack $\delta\in(0,1)$ independent of $\varepsilon$ and $N\ge2$. Boundary attachments contribute a multiplicity factor $m_{\mathrm{blk}}$ per block boundary unit (uniform in $\varepsilon,N$). Summing over excess block area $k\ge0$ yields the convergent geometric series
\[
  \sum_{k\ge 0} N_{\mathrm{blk}}(\Gamma,A+k)\,\rho_{\mathrm{blk}}(\varepsilon)^{A+k}
   \ \le\ m_{\mathrm{blk}}^{P_{\mathrm{blk}}}\,\frac{\rho_{\mathrm{blk}}(\varepsilon)^{A}}{\delta},
\]
where $A$ is the minimal block spanning area and $P_{\mathrm{blk}}$ the block perimeter. Taking $-\log$ and converting to physical units (each block area $\asymp 1$, each block boundary length $\asymp 1$) gives
\[
  -\log\langle W(\Lambda)\rangle\ \ge\ T_*\,\mathsf{Area}_\varepsilon(\Lambda)\ -\ C_*\,\mathsf{Per}_\varepsilon(\Lambda),
\]
with
\[
  T_*:= -\log \rho_{\max},\quad \rho_{\max}:=\sup_{0<\varepsilon\le\varepsilon_0}\rho_{\mathrm{blk}}(\varepsilon)<1,\quad
  C_*:= \log m_{\mathrm{blk}}+\log(1/\delta)<\infty.
\]
Moreover, the one–step cross–cut Dobrushin coefficient at block scale obeys
\[
  \alpha\bigl(\beta(\varepsilon)\bigr)\ \le\ 2\,\beta(\varepsilon)\,J^{\mathrm{blk}}_{\perp}(\varepsilon)
   \ \le\ 2\,\beta_*\,J^{\mathrm{blk}}_{\perp,\max}=:\alpha_*<1,
\]
where $J^{\mathrm{blk}}_{\perp,\max}$ is a geometry–only bound (independent of $\varepsilon,N$). All constants are $\varepsilon$– and $N$–uniform.

\paragraph{Optional scaffold (KP from Wilson; hypothesis bundle).}
\emph{(H-KP).} For 4D SU($N$) Wilson action at sufficiently small $\beta$, the block polymer expansion at scale $b(\varepsilon)\asymp \varepsilon^{-1}$ satisfies: (i) $\rho_{\mathrm{blk}}(\varepsilon)\le \rho_{\max}<1$, (ii) $\mu_{\mathrm{blk}}\,\rho_{\mathrm{blk}}\le 1-\delta$ with $\delta\in(0,1)$, (iii) boundary multiplicity $m_{\mathrm{blk}}\le m_0$, all independent of $\varepsilon$ and $N$. \emph{Conclusion.} The constants $T_*=-\log\rho_{\max}>0$, $C_* = \log m_0 + \log(1/\delta)$, and $\alpha_*=2\beta_* J^{\mathrm{blk}}_{\perp,\max}<1$ follow, yielding the uniform area–perimeter law and contraction.

\begin{theorem}[Uniform KP/cluster expansion with explicit constants]
\label{thm:uniform-kp}
Under the hypotheses above, define the explicit $\varepsilon$–independent constants
\[
  \rho_{\max}\;:=\;\sup_{0<\varepsilon\le \varepsilon_0}\rho_{\mathrm{blk}}(\varepsilon)\ <\ 1,\quad
  T_*\;:=\; -\log \rho_{\max}\ >\ 0,\quad
  C_*\;:=\; \log m_{\mathrm{blk}}\ +\ \log\tfrac{1}{\delta}\ <\ \infty,
\]
\[
  J^{\mathrm{blk}}_{\perp,\max}\;:=\;\sup_{0<\varepsilon\le\varepsilon_0} J^{\mathrm{blk}}_{\perp}(\varepsilon)\ <\ \infty,\qquad
  \alpha_*\;:=\;2\,\beta_*\,J^{\mathrm{blk}}_{\perp,\max}\ <\ 1\,.
\]
Then for all sufficiently large loops $\Lambda\subset\varepsilon\,\mathbb Z^4$ and all $\varepsilon\in(0,\varepsilon_0]$:
\begin{align}
  -\log\langle W(\Lambda)\rangle\ &\ge\ \tau_\varepsilon\,A_\varepsilon^{\min}(\Lambda)\ -\ \kappa_\varepsilon\,P_\varepsilon(\Lambda),\\
  \frac{\tau_\varepsilon}{\varepsilon^2}\ &\ge\ T_*,\qquad \frac{\kappa_\varepsilon}{\varepsilon}\ \le\ C_*,\\
  \alpha\bigl(\beta(\varepsilon)\bigr)\ &\le\ \alpha_* <\ 1\,.
\end{align}
In particular, $T_*$ is a uniform string–tension lower bound in physical units, $C_*$ a uniform perimeter coefficient (physical units), and $\alpha_*$ a uniform upper bound for the cross–cut Dobrushin coefficient.
\end{theorem}
 

\begin{theorem}[Local gauge–invariant fields]\label{thm:local-fields-exist}
There exists a collection of operator–valued tempered distributions $\{\mathcal E(f)\}_{f\in \mathcal S(\mathbb R^4)}$ on the OS/GNS Hilbert space such that for compactly supported smooth $f$, $\mathcal E(f)$ is the $L^2$–limit of $\mathcal E^{(a)}(f)$ along the scaling window. For finite families $\{f_i\}$ and any polynomial $P$, the mixed Schwinger functions of $\{\mathcal E(f_i)\}$ arise as limits of those of $\{\mathcal E^{(a)}(f_i)\}$ and satisfy OS0–OS2 with the explicit constants from Cor.~\ref{cor:os0-explicit-4d}. The fields are Euclidean covariant (OS1) by Cor.~\ref{cor:os1-rotations}.
\end{theorem}

\begin{corollary}[OS$\to$Wightman with local fields and gap]\label{cor:wightman-local-gap}
Let $H\ge 0$ be the generator reconstructed from the continuum Schwinger functions including the local field sector of Theorem~\ref{thm:local-fields-exist}. If $\operatorname{spec}(H)\subset\{0\}\cup[\gamma_*,\infty)$ with $\gamma_*>0$ (Theorem~\ref{thm:pf-gap-meanzero}), then the OS reconstruction yields Wightman local fields (smeared) $\mathcal E_M(\varphi)$ on Minkowski space with the same mass gap:
\[
  \sigma(H_{\rm Mink})\ \subset\ \{0\}\cup[\gamma_*,\infty).
\]
\end{corollary}
\paragraph{Anchors (T14 Local fields) [ANCHOR\_T14\_v1].}
\begin{itemize}
  \item CloverApproximation: loop nets converge to field smearings.
  \item TemperednessTransfer: OS0 bounds transfer to fields.
  \item ReflectionPositivityTransfer: OS2 for fields via cylinder-set limits.
  \item LocalityFields: disjoint supports $\Rightarrow$ commutativity/locality.
  \item GapVacuumPersistence: same $H$ $\Rightarrow$ gap/vacuum persist.
\end{itemize}

\paragraph{Anchors (T15 Time normalization and gap) [ANCHOR\_T15\_v1].}
\begin{itemize}
  \item PerTickContraction: odd-cone one-step factor $(1-\theta_* e^{-\lambda_1 t_0})^{1/2}$.
  \item EightTickComposition: $\gamma_{\rm cut}(a)=8\,c_{\rm cut}(a)$.
  \item PhysicalNormalization: $\tau_{\rm phys}=a$ $\Rightarrow$ $\gamma_{\rm phys}=8\big(-\log(1-\theta_* e^{-\lambda_1 t_0})\big)$.
  \item ContinuumPersistence: rescaled NRC keeps $(0,\gamma_{\rm phys})$ spectrum–free.
\end{itemize}

\medskip

\section{Appendix U: AF--free unconditional inputs and continuum limit}

\subsection*{Referee checklist (Clay requirements $\to$ labels)}
\begin{itemize}
  \item Scaling schedule, van Hove volumes: Def.~\ref{def:U0-schedule} (U0).
  \item UEI/LSI on fixed regions (uniform in $a$): Thm.~\ref{thm:U1-lsi-uei}, Lem.~\ref{lem:U1-tree-bounds}, Cor.~\ref{cor:U1-uei} (U1).
  \item OS/GNS embeddings $I_a$ (isometries, domains): Lem.~\ref{lem:U2-embeddings} (U2a).
  \item Comparison identity and NRC (all nonreal $z$): Lem.~\ref{lem:U2-comparison}, Prop.~\ref{prop:collective-compactness}, Thm.~\ref{thm:nrc-quant} (U2a/U2c).
  \item Graph-defect bound $\|D_a(H_a+1)^{-1/2}\|=O(a)$: Lem.~\ref{lem:graph-defect-Oa} (U2b).
  \item Low-energy projection control $\delta_a(\Lambda)\le C_\Lambda a$: Lem.~\ref{lem:low-energy-proj} (U2b).
  \item Cauchy resolvent criterion, uniqueness (no subsequences): Lem.~\ref{lem:af-free-cauchy} or Lem.~\ref{lem:cauchy-resolvent-unique} (U2c).
  \item Interface Doeblin minorization (independent of $\beta,L$): Lem.~\ref{lem:abs-cont}, Lem.~\ref{lem:coarse-refresh}, Lem.~\ref{lem:coarse-hk-domination}, Prop.~\ref{prop:coarse-doeblin} (U3).
  \item Odd-cone Gram/mixed bounds and Gershgorin margin: Thm.~\ref{thm:uniform-odd-contraction}, Lem.~\ref{lem:local-gram-bounds}, Prop.~\ref{prop:psd-crossing-gram}, Thm.~\ref{thm:U12-exp-cluster} (U4).
  \item OS axioms in the continuum (OS0–OS5): Prop.~\ref{prop:os0os2-closure}, Prop.~\ref{prop:os35-limit}, Thm.~\ref{thm:os1-euclid} (U7).
  \item Local gauge-invariant fields and non-Gaussianity: Thm.~\ref{thm:local-fields-exist}, Prop.~\ref{prop:nonzero-cumulant4} (U7).
  \item OS4 (permutation symmetry) explicit: Prop.~\ref{prop:U11-os4}.
  \item Exponential clustering in continuum: Thm.~\ref{thm:U12-exp-cluster}.
\end{itemize}

\subsection{U8. Ward/Schwinger–Dyson identities and continuum Ward theorem}
\begin{lemma}[Lattice BRST/finite-gauge Ward identities]
For the Wilson action on a finite periodic 4D torus and gauge group $G=\mathrm{SU}(N)$, the Schwinger functions of Wilson loops and of the local clover field $\Xi^{(a)}_{\mu\nu}$ satisfy the nonabelian lattice Ward/Schwinger–Dyson identities under (i) finite local gauge variations and (ii) BRST-exact insertions. These identities hold for every lattice spacing $a$ and volume $L$.
\end{lemma}
\begin{proof}
Let $g: \Lambda^0\to G$ be a lattice gauge transformation acting on links by $U_{x,\mu}\mapsto g_x U_{x,\mu} g_{x+\hat\mu}^{-1}$. The Wilson action $S_\beta(U)$ is gauge invariant and the product Haar measure $d\mu_\beta(U)\propto e^{-\beta S_\beta(U)}\prod dU$ is left/right invariant. For any cylinder functional $F$ built from Wilson loops and clover fields, the change of variables $U\mapsto g\cdot U$ yields
\[
  \int F(U)\,d\mu_\beta(U)\ =\ \int F(g\cdot U)\,d\mu_\beta(U)
\]
for all $g$. Differentiating along a one-parameter subgroup $g_x(t)=\exp(t X_x)$ with $X_x\in\mathfrak{su}(N)$, and using that the derivative of $F(g\cdot U)$ at $t=0$ is a sum of left/right invariant vector fields acting on link variables at the endpoints of the affected loops/plaquettes, one obtains the lattice Schwinger–Dyson identity
\[
  \sum_{x}\Big\langle \delta_x F\Big\rangle_{\beta,a,L}\ =\ 0,
\]
where $\delta_x$ is the gauge-variation derivation at site $x$ acting by Lie derivatives on adjacent links. BRST versions follow by introducing standard gauge-fixing/ghost terms and using invariance of the BRST-extended measure; BRST-exact insertions integrate to zero. Periodic boundary conditions ensure that all boundary terms vanish.
\end{proof}
\begin{theorem}[Continuum nonabelian Ward identities]\label{thm:U8-ward-cont}
Along any van Hove sequence with $a\downarrow 0$, the embedded Schwinger functions of Wilson loops and of the renormalized local field $\Xi_R$ satisfy the continuum nonabelian Ward (Schwinger–Dyson) identities of Yang–Mills. Hence the OS/Wightman limit is gauge invariant and satisfies the YM Ward relations.
\end{theorem}
\begin{proof}
Fix finitely many loop/field insertions supported in a fixed region $R\Subset\mathbb R^4$. By Theorem~\ref{thm:U1-lsi-uei}, UEI gives uniform integrability bounds for the Ward functionals. The lattice Ward identity holds at each $(a,L)$ by the lemma. Embed the lattice observables to the continuum cylinder algebra and apply U2 operator-norm NRC (Theorem~\ref{thm:U2-nrc-unique}) to pass to the unique limit of Schwinger functions; dominated convergence yields the limit identity. For local fields, use U10 to replace $\Xi^{(a)}$ by the renormalized $\Xi^{(a)}_R=Z_F(a)\,\Xi^{(a)}$ with bounded $Z_F(a)$, and pass to the limit.
\end{proof}

\subsection{U9. Gauss law and the physical Hilbert subspace}
\begin{theorem}[Gauss constraint and physical subspace]\label{thm:U9-gauss}
Let $\mathcal H_{\rm phys}$ be the gauge-invariant OS/GNS subspace (closure of vectors generated by gauge-invariant time-zero observables). Then: (i) the lattice Gauss constraints hold on $\mathcal H_{a,L}^{\rm phys}$; (ii) the embeddings $I_{a,L}$ map $\mathcal H_{a,L}^{\rm phys}$ into $\mathcal H_{\rm phys}$; (iii) in the continuum limit, the Gauss law holds on $\mathcal H_{\rm phys}$ and local gauge transformations act trivially on $\mathcal H_{\rm phys}$.
\end{theorem}
\begin{proof}
On the lattice, define the time-zero local gauge group $\mathcal G_0$ acting on the half-space algebra. OS inner products are invariant under $\mathcal G_0$ by Haar invariance, so the GNS null space contains all gauge-variant commutators with Gauss generators; the physical subspace is the closure of $\mathcal G_0$-invariant vectors. The discrete Gauss constraint (vanishing of lattice divergence of electric flux at each site) is the Ward identity with a generator supported at that site, hence holds on $\mathcal H_{a,L}^{\rm phys}$. Equivariance of the embeddings $E_a$ implies $I_{a,L}$ intertwines the gauge actions, so $I_{a,L} \mathcal H_{a,L}^{\rm phys}\subset\mathcal H_{\rm phys}$. In the continuum limit, use U1 UEI and U2 NRC to pass Ward/Gauss identities from cylinders to the limit, which implies that local gauge transformations act trivially on $\mathcal H_{\rm phys}$ and the Gauss law holds.
\end{proof}

\subsection{U10. Renormalized local fields (tempered, nontrivial)}
\begin{theorem}[Existence of renormalized $F_{\mu\nu}$]\label{thm:U10-renorm-F}
Define $\Xi^{(a)}_{\mu\nu}$ by the standard gauge-covariant clover discretization and set $\Xi^{(a)}_R:= Z_F(a)\,\Xi^{(a)}$ with a multiplicative factor $Z_F(a)$. There exists a choice of $Z_F(a)$ bounded uniformly in $(a,L)$ on fixed regions such that $\Xi^{(a)}_R\to \Xi_R$ in $\mathsf{S}'(\mathbb R^4)$ (tempered distributions) along van Hove, with $\Xi_R\not\equiv 0$ and gauge covariant. Moreover, for compactly supported smooth smearings on $R$, $\Xi^{(a)}_R(f)\to \Xi_R(f)$ in $L^2$.
\end{theorem}
\begin{proof}
UEI/LSI (U1) give uniform moment and Laplace bounds for time-zero local functionals on fixed regions. Approximate $\Xi^{(a)}$ by loop-net smearings and apply the local-field construction (U7) to obtain limits in $\mathsf{S}'$. Fix a physical renormalization scale $\mu>0$ and choose $Z_F(a)$ to normalize the two-point $\langle \Xi^{(a)}_R(f_\mu)\,\Xi^{(a)}_R(f_\mu)\rangle$ to a finite reference; UEI bounds ensure $Z_F(a)$ remains bounded as $a\downarrow 0$. Nontriviality follows from the strictly positive truncated 4-point for clover (Prop.~\ref{prop:nonzero-cumulant4}) and stability under U2 NRC.
\end{proof}

\subsection{U11. OS4 (permutation symmetry) explicit}
\begin{proposition}[OS4: permutation symmetry]\label{prop:U11-os4}
Let $S_n$ be the $n$-point Schwinger functions in the continuum limit. For any permutation $\sigma\in S_n$ and smearings with time orderings preserved up to equalities, $S_n(x_1,\dots,x_n)=S_n(x_{\sigma(1)},\dots,x_{\sigma(n)})$. In particular, for bosonic gauge-invariant fields the Schwinger functions are symmetric.
\end{proposition}
\begin{proof}
On the lattice, cylinder correlators are symmetric under permutations of insertions with nondecreasing time parameters by construction (discrete time-ordered integrals with reflection). These identities pass to the limit by U2 NRC and U1 UEI. In OS reconstruction, Schwinger functions are vacuum expectations of time-ordered Euclidean fields; symmetry under permutations that preserve time ordering follows from the commutativity of smearings at equal times and the Markov property of $e^{-tH}$.
\end{proof}

\subsection{U12. Exponential clustering in the continuum}
\begin{theorem}[Exponential clustering]\label{thm:U12-exp-cluster}
Let $A,B$ be gauge-invariant local observables with compact support and Euclidean separation $r$. Then
\[
  \big|\langle A B\rangle - \langle A\rangle\langle B\rangle\big|\ \le\ C_{A,B}\,e^{-\gamma_* r},
\]
with $\gamma_*>0$ the continuum mass gap and $C_{A,B}$ depending on $A,B$ only.
\end{theorem}
\begin{proof}
Gap persistence (U2 + Thm.~\ref{thm:gap-persist-cont}) gives a spectral gap for $H$. Standard OS$\to$Wightman and spectral calculus yield exponential decay of connected correlators with rate $\gamma_*$. Locality ensures the constants depend only on $A,B$.
\end{proof}

\subsection{U0. Concrete scaling schedule and van Hove volumes}
\begin{definition}[Scaling schedule and volumes]\label{def:U0-schedule}
Fix $a_0>0$, $\beta_{\min}>0$, and constants $c_A>0$, $c_L>0$. Define
\[
  \beta(a)\ :=\ \beta_{\min}\ +\ c_A\,\log\Big(1+\frac{a_0}{a}\Big),\qquad a\in(0,a_0],
\]
and choose volumes $L(a)\in\mathbb N$ with $L(a)\,a\ \xrightarrow[a\downarrow 0]{}\ \infty$ and $L(a)\ge c_L\,a^{-1}$.
\end{definition}
\begin{remark}
The unconditional inputs below (U1–U4) are uniform in $a\in(0,a_0]$ and do not require $\beta(a)\to\infty$. The schedule in Definition~\ref{def:U0-schedule} merely provides a concrete van Hove parametrization consistent with asymptotic freedom; all bounds stated below are independent of the particular monotone choice provided $\beta(a)\ge \beta_{\min}>0$ and $L(a)\,a\to\infty$.
\end{remark}

\subsection{U1. Local LSI/UEI on fixed regions (unconditional)}
\begin{theorem}[Local LSI/UEI on fixed regions]\label{thm:U1-lsi-uei}
Let $R\Subset\mathbb R^4$ be fixed, $a\in(0,a_0]$, and $\beta\ge \beta_{\min}>0$. After tree gauge on $R$, the induced Gibbs measure on $G^{m(R,a)}$, $G=\mathrm{SU}(N)$,
\[
  d\mu_R(X)\ =\ Z_R^{-1}\,e^{-\beta S_R(X)}\,d\pi(X),\qquad d\pi=\text{product Haar},
\]
obeys a logarithmic Sobolev inequality
\[
  \mathrm{Ent}_{\mu_R}(f^2)\ \le\ \frac{1}{\rho_R}\int \|\nabla f\|^2\,d\mu_R
\]
with
\[
  \rho_R\ \ge\ c_0(R,N)\,\min\{1,\beta\}\ \ge\ c_1(R,N)\,\beta_{\min}\,.
\]
In particular, $\mu_R$ satisfies uniform exponential integrability (UEI) on $R$ with explicit Laplace–transform radius given by the Herbst bound
\[
  \eta_R\ =\ \min\Big\{t_*(R,N),\ \sqrt{\rho_R/G_R}\Big\},\qquad \mathbb E_{\mu_R}\exp\big(t(F-\mathbb E F)\big)\ \le\ e^{1/2}
\]
for all time–zero local observables $F$ supported in $R$ and all $|t|\le \eta_R$.
\end{theorem}
\begin{proof}
We give a complete argument via a tree–gauge representation and standard LSI tools on compact manifolds.

\emph{Step 1: Reference LSI.} On compact Lie groups with bi–invariant metric, the heat kernel measure satisfies an LSI with constant equal to the spectral gap; for product Haar $\pi$ on $G^{m}$ one has an LSI with constant $\rho_{\rm Haar}(N)>0$ by tensorization. Denote by $\rho_{\rm Haar}(R,N)$ the corresponding constant on $G^{m(R,a)}$ (independent of $a$).

\emph{Step 2: Tree gauge and geometry on $R$.} Fix a spanning tree on the edges in $R$. Gauge–fixing along the tree yields a coordinate map from $G^{m(R,a)}$ to a product of $G$'s indexed by chords and boundary edges. The Wilson action on $R$ can be written as $S_R=\sum_{p\subset R} s_p$ with each $s_p$ depending on $O(1)$ variables. Using bounded degree and fixed diameter of $R$, there exist constants $C_1,C_2$ (Anchors T12) with
\[
  \|\nabla S_R\|_{L^\infty}\ \le\ C_1(R,N),\qquad \operatorname{osc}(S_R)\ \le\ C_2(R,N),
\]
uniform in $a\in(0,a_0]$.

\emph{Step 3: Small–$\beta$ (bounded perturbation).} By the Holley–Stroock perturbation lemma for LSI (bounded potential oscillation), the measure $d\mu_R\propto e^{-\beta S_R}d\pi$ satisfies
\[
  \rho_R\ \ge\ \rho_{\rm Haar}(R,N)\,e^{-\beta\,\operatorname{osc}(S_R)}\ \ge\ c_s(R,N)>0\qquad (0\le \beta\le \beta_1(R,N)),
\]
with $c_s:=\rho_{\rm Haar} e^{-\beta_1 C_2}$ and $\beta_1$ any fixed threshold.

\emph{Step 4: Large–$\beta$ (uniform convexity on bulk mass).} For each plaquette term $s_p(U)=\operatorname{Re}\,\operatorname{tr}(I-U_p)$, the Hessian at $U_p=I$ is positive definite in Lie algebra coordinates. After tree gauge, near the identity chart for each $G$–factor, the sum $S_R$ has Hessian bounded below by $c_3(R,N) I$. Therefore the Bakry–Émery tensor satisfies $\mathrm{Ric}+\beta\,\nabla^2 S_R\ \succeq\ \kappa(R,N)\,I$ on a neighborhood $\mathcal N$ of the identity, with $\kappa(R,N):=\kappa_0(R,N)+\beta c_3(R,N)$. Since $G^{m}$ is compact and $\|\nabla S_R\|_{\infty}\le C_1$, the Gibbs measure assigns mass $\mu_R(\mathcal N)\ge 1-\epsilon(R,N,\beta)$ with $\epsilon\le e^{-c\beta}$. By the Wang–type local–to–global LSI transfer (local $CD(\rho,\infty)$ plus bounded drift outside; see e.g. Wang (2000) and subsequent refinements), there exists $c_\ell(R,N)>0$ such that
\[
  \rho_R\ \ge\ c_\ell(R,N)\,\beta\qquad (\beta\ge \beta_1(R,N)).
\]

\emph{Step 5: Two–regime synthesis and UEI.} Combining Steps 3–4,
\[
  \rho_R\ \ge\ c_0(R,N)\,\min\{1,\beta\}\ \ge\ c_2(R,N)\,\beta_{\min}\qquad (\beta\ge \beta_{\min}),
\]
with constants depending only on $(R,N)$. The Herbst argument with the Lipschitz bound $\|\nabla F\|\le \sqrt{G_R}\,\|F\|_{\mathrm{Lip}}$ (Anchors T12) yields UEI with radius
\[
  \eta_R\ =\ \min\Big\{t_*(R,N),\ \sqrt{\rho_R/G_R}\Big\},\quad \text{uniform in }(a,L).
\]
\end{proof}
\begin{lemma}[Tree–gauge Lipschitz bounds]\label{lem:U1-tree-bounds}
Under the tree gauge on $R$, there exist $C_1,C_2,G_R$ depending only on $(R,N)$ such that $\|\nabla S_R\|_{\infty}\le C_1$, $\operatorname{osc}(S_R)\le C_2$, and for any time–zero local observable $F$ supported in $R$, $\|F\|_{\mathrm{Lip}}^2\le G_R\,\int\|\nabla F\|^2\,d\pi$.
\end{lemma}
\begin{corollary}[Explicit UEI constants]\label{cor:U1-uei}
Let $\rho_R$ be as in Theorem~\ref{thm:U1-lsi-uei}. Then for all $F$ supported in $R$ and all $|t|\le \eta_R$,
\[
  \mathbb E_{\mu_R}\exp\big(t(F-\mathbb E F)\big)\ \le\ e^{\tfrac{t^2}{2\rho_R}\,\int\|\nabla F\|^2 d\mu_R}\ \le\ e^{1/2},
\]
with $\eta_R=\min\{t_*(R,N),\sqrt{\rho_R/G_R}\}$ independent of $(a,L)$ and $\beta\ge \beta_{\min}$.
\end{corollary}
\subsection*{U2a. Embeddings and comparison identity}
\begin{lemma}[OS/GNS embeddings are genuine isometries]\label{lem:U2-embeddings}
For each $(a,L)$, let $\mathcal H_{a,L}$ be the OS/GNS Hilbert space for the lattice measure and $\mathcal H$ the continuum OS/GNS space on fixed regions. Define $I_{a,L}$ on generators by $I_{a,L}[F]:=[E_a(F)]$, where $E_a$ maps lattice loops/fields to their polygonal/smeared counterparts. Then $I_{a,L}$ is well-defined on the OS quotient, isometric on the time-zero local cylinder space, and extends by continuity to a partial isometry $I_{a,L}:\overline{\mathrm{span}}\,\mathcal V^{\rm loc}_{0,a,L}\to\mathcal H$ with adjoint $I_{a,L}^*$. Moreover, $I_{a,L}\mathcal D_{a,L}\subset\mathcal D$ for the algebraic cores of time-zero local vectors.
\end{lemma}
\begin{proof}
By OS positivity and equivariance of $E_a$ with respect to time translations and reflections, $\langle F,\theta F\rangle$ is preserved under $E_a$. Thus the OS seminorms coincide on generators, giving isometry on the quotient; density yields the extension.
\end{proof}

\begin{lemma}[Explicit resolvent comparison identity]\label{lem:U2-comparison}
Let $H\ge 0$ and $H_{a,L}\ge 0$ be the Euclidean generators on $\mathcal H$ and $\mathcal H_{a,L}$, and set $P_{a,L}:=I_{a,L}I_{a,L}^*$. For any $z\in\mathbb C\setminus\mathbb R$ and any $\xi\in\mathcal H$,
\[
  (H-z)^{-1}\xi\ -\ I_{a,L}(H_{a,L}-z)^{-1}I_{a,L}^*\xi
  \
  =\ (H-z)^{-1}(I-P_{a,L})\xi\ -\ (H-z)^{-1}\,D_{a,L}\,(H_{a,L}-z)^{-1}I_{a,L}^*\xi,
\]
where $D_{a,L}:=H I_{a,L}-I_{a,L}H_{a,L}$ is the graph-defect map on a common core.
\end{lemma}

\begin{lemma}[Defect identity and common core]\label{lem:U2-defect-core}
Let $\mathcal D^{\rm loc}$ denote the algebraic core generated by time-zero local observables supported in a fixed slab $B_{R_*}$ (closed under OS/GNS operations and time translations). Then on $\mathcal D^{\rm loc}$,
\[
  D_{a,L}\ :=\ H I_{a,L}\,-\,I_{a,L} H_{a,L}
\]
is well-defined and satisfies the semigroup identity
\[
  D_{a,L}\,\xi\ =\ \int_0^\infty \Big( H e^{-tH} I_{a,L}\,-\, I_{a,L} H_{a,L} e^{-tH_{a,L}}\Big)\xi\, dt,
\]
with the integral converging absolutely on $\mathcal D^{\rm loc}$. Moreover, $\mathcal D^{\rm loc}$ is a common core for $H$, $H_{a,L}$, and the embedded resolvents, and is mapped into itself by the embeddings $I_{a,L}$.
\end{lemma}
\begin{proof}
Locality and UEI (U1) imply bounded growth of $\|e^{-tH}\xi\|$ and $\|e^{-tH_{a,L}}\xi\|$ on $\mathcal D^{\rm loc}$, so the Laplace representation of resolvents and generators is valid on this core. Differentiating $e^{-tH} I_{a,L}- I_{a,L} e^{-tH_{a,L}}$ at $t=0^+$ yields the displayed identity. Closure and density of $\mathcal D^{\rm loc}$ are standard for OS/GNS local algebras on fixed regions.
\end{proof}
\noindent\emph{Purpose.} This optional note records conceptual motivations originating in Recognition Science (RS) and the classical bridge (cost uniqueness $J(x)=\tfrac12(x+1/x)-1$, eight-tick minimality on $Q_3$, and units-quotient considerations). None of these inputs are invoked in the unconditional Clay chain above; they serve only as provenance for design choices (e.g., odd-cone two-layer deficit and slab normalization). Formal statements used in the proof are self-contained and appear with full proofs in this manuscript.

\end{document}