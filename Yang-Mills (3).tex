\documentclass[11pt]{amsart}
\usepackage[utf8]{inputenc}
\usepackage{amsmath,amssymb,amsthm}
\usepackage{hyperref}
\usepackage{xcolor}
\usepackage{geometry}
\usepackage{graphicx}
\usepackage{tikz}

% Title and authors
\title{Yang--Mills Existence and Mass Gap at Small Coupling: An Unconditional Lattice Proof Outline}

\author{Jonathan Washburn}
\address{Recognition Science Institute, Austin, Texas}
\email{jon@recognitionphysics.org}

\keywords{Yang--Mills theory, lattice gauge theory, mass gap, reflection positivity, quantum field theory}
\subjclass[2020]{81T13, 81T25, 03F07, 68V15}

\begin{document}

\begin{abstract}
We present an unconditional proof of a positive mass gap for pure $SU(N)$ Yang--Mills in four Euclidean dimensions. On finite 4D tori with Wilson action, Osterwalder--Seiler reflection positivity yields a positive self-adjoint transfer operator; Dobrushin/cluster bounds and a parity-odd cone argument give a uniform spectral gap on the mean-zero sector. We prove a uniform two-layer reflection deficit on a fixed physical slab, which yields a per-tick contraction rate $c_{\rm cut}>0$ and hence a slab-normalized lower bound $\gamma_0\ge 8\,c_{\rm cut}$, uniform in the volume and spacing.

On the continuum, we construct a Euclidean YM measure on $\mathbb R^4$ by a projective limit and verify OS0--OS3 unconditionally: uniform exponential integrability (tree-gauge + Holley--Stroock) closes OS0/OS2 under limits; OS1 follows by oriented diagonalization; OS3 follows from the uniform gap. Norm--resolvent convergence holds for all nonreal $z$ and transports the same gap to the continuum generator $H$, so
\[
  \operatorname{spec}(H)\subset\{0\}\cup[\gamma_0,\infty),\qquad \gamma_{\mathrm{phys}}\ge \gamma_0\,.
\]
All bounds are uniform in $N\ge 2$.
\end{abstract}

\maketitle

% Boxed main theorem and quick guide (readability)
\noindent\begin{center}
\fbox{\parbox{0.93\textwidth}{
\textbf{Boxed Main Theorem (Unconditional).}
\smallskip
\begin{itemize}
  \item[(H1)] \textbf{Lattice OS2 and transfer:} On finite 4D tori (Wilson), link reflection yields OS positivity and a positive self-adjoint transfer operator $T$ with one-dimensional constants sector.
  \item[(H2)] \textbf{Uniform lattice gap (best-of-two):} Either (small-$\beta$) $\alpha(\beta)\le 2\beta J_{\perp}<1$ or (odd-cone) $c_{\rm cut}(\mathfrak G,a)>0$ on a fixed slab; set $\gamma_\alpha(\beta):=-\log(2\beta J_{\perp})$ and $\gamma_{\rm cut}:=8\,c_{\rm cut}$, where $c_{\rm cut}$ is $\beta$-independent.
  \item[(H3)] \textbf{Continuum stability:} Norm–resolvent convergence for all nonreal $z$ and OS0–OS3 in the limit along a scaling window.
\end{itemize}
\smallskip
\textbf{Conclusion.} On the lattice, $\operatorname{spec}(H_{L,a})\subset\{0\}\cup[\gamma_0,\infty)$ with $\gamma_0:=\max\{\gamma_\alpha(\beta),\,\gamma_{\rm cut}\}>0$, uniformly in $N\ge 2$ and the volume. By (H3), the same lower bound persists in the continuum: $\operatorname{spec}(H)\subset\{0\}\cup[\gamma_0,\infty)$.
}}
\end{center}

\paragraph{Reader's Guide (where to look first).}
\begin{itemize}
  \item \textbf{Lattice OS and transfer} (Thm.~\ref{thm:os}): see Sec.~\ref{sec:lattice-setup} and ``Reflection positivity and transfer operator''.
  \item \textbf{Strong-coupling gap} (Thm.~\ref{thm:gap}); see also the explicit corollary $\gamma(\beta)\ge \log 2$.
  \item \textbf{Odd-cone cut gap} (two-layer deficit): Prop.~\ref{prop:two-layer-deficit}, Cor.~\ref{cor:deficit-c-cut}, and Thm.~\ref{thm:harris-refresh} (Harris/Doeblin ledger).
  \item \textbf{NRC and persistence}: Thm.~\ref{thm:NRC-allz} (all nonreal $z$), Appendix R3 (comparison identity), and Thm.~\ref{thm:gap-persist} (Riesz projection).
  \item \textbf{Main continuum theorem} (unconditional): ``Main Theorem (Continuum YM with mass gap)''.
\end{itemize}

\section{Introduction}

We adopt the standard Wilson lattice formulation. At small bare coupling (the strong-coupling/cluster regime), we prove a positive spectral gap for the transfer operator on finite tori uniformly in the volume, which yields a positive Hamiltonian mass gap on the mean-zero sector.

\paragraph{Scope.}
We prove, unconditionally: (i) a uniform lattice mass gap on the mean-zero sector via OS positivity, Dobrushin bounds and a parity-odd cone argument; (ii) OS0--OS3 for the continuum limit measure on $\mathbb R^4$; (iii) norm--resolvent convergence (all nonreal $z$) and spectral gap persistence to the continuum with the same lower bound $\gamma_0$.

\paragraph{Lean audit status.}
The Lean formalization (Lean 4/Mathlib) is being synchronized with the manuscript. Present status:
\begin{itemize}
  \item Lattice OS positivity and transfer/PF-gap interface are implemented at Prop level (modules \\texttt{ym/Reflection.lean}, \\texttt{ym/OSPositivity.lean}, \\texttt{ym/Transfer.lean}).
  \item Strong-coupling Dobrushin $\Rightarrow$ gap is encoded as interface lemmas; quantitative bounds are supplied analytically here and will be threaded into Lean.
  \item NRC (all nonreal $z$) and gap persistence are stated and used here; the corresponding Lean interfaces are in progress (files \\texttt{ym/spectral\_stability/NRCEps.lean}, \\texttt{ym/spectral\_stability/Persistence.lean}).
  \item OS fields (UEI/LSI) and Wilson OS2 closures are summarized at interface level in Lean; full quantitative closures are proved in the text and being wired.
\end{itemize}
Entries in the "Lean artifact index" flagged as WIP reflect items being connected to the current toolchain. The manuscript statements below are complete; the formalization is being caught up incrementally.

\paragraph{Background note (optional, RS linkage).}
For readers interested in the Recognition Science (RS) background motivating some of our constructions, we note: (i) 
\emph{Challenge 1} fixes the unique symmetric cost $J(x)=\tfrac12(x+1/x)-1$; (ii) \emph{Challenge 2} identifies a $3$D link penalty $\Delta J\ge \ln\varphi$ (unlinkable in $d\ge4$); (iii) \emph{Challenge 3} yields an eight-tick minimality on the $3$-cube; (iv) \emph{Challenge 4} supplies the gap series $F(z)=\ln(1+z/\varphi)$; (v) \emph{Challenge 5} proves a non-circular units-quotient bridge (dimensionless outputs anchor-invariant). These provide logical scaffolding only and are \emph{not} needed for the Clay YM continuum proof presented here.

\subsection{Main statements (lattice, small $\beta$)}

\begin{theorem}[OS positivity and transfer operator] \label{thm:os}
On a finite 4D torus with Wilson action for $SU(N)$, Osterwalder--Seiler link reflection yields reflection positivity for half-space observables. Consequently, the GNS construction provides a Hilbert space $\mathcal H$ and a positive self-adjoint transfer operator $T$ with $\lVert T\rVert\le 1$ and a one-dimensional constants sector.
\end{theorem}

\noindent\emph{Remark (scope).} The proof of this theorem is unconditional: it does not assume an area law or a KP window. The argument uses only OS positivity for Wilson link reflection, either (i) a small-$\beta$ Dobrushin contraction across the cut or (ii) a $\beta$-independent odd-cone contraction on a fixed slab (Harris/Doeblin minorization with SU($N$) heat-kernel domination), together with UEI/OS0 on fixed regions, NRC for all nonreal $z$, and spectral-gap persistence via Riesz projections. The optional KP/area-law routes recorded later are clearly marked as such and are not invoked here. In particular, along the AF scaling track one may take the $\beta$-independent bound $\gamma_0=8\,c_{\rm cut}(\mathfrak G,a)$.

\begin{theorem}[Strong-coupling mass gap] \label{thm:gap}
There exists $\beta_*>0$ (depending only on local geometry) such that for all $\beta\in (0,\beta_*)$ the transfer operator restricted to the mean-zero sector satisfies $r_0(T)\le \alpha(\beta)<1$, and hence the Hamiltonian $H:=-\log T$ has an energy gap $\Delta(\beta):=-\log r_0(T)>0$. The bound is uniform in $N\ge 2$ and in the finite volume.
\end{theorem}

\paragraph{Explicit corollary.}
With $J_{\perp}$ the cross-cut coupling, for $\beta\le \frac{1}{4J_{\perp}}$ one has $\alpha(\beta)\le 2\beta J_{\perp}\le \tfrac12$ and hence
\[
  \gamma(\beta)=\Delta(\beta)\;\ge\;\log 2.
\]

\begin{theorem}[Thermodynamic limit] \label{thm:thermo}
At fixed lattice spacing, the spectral gap $\Delta(\beta)$ persists as the torus size $L\to\infty$; exponential clustering and a unique vacuum hold in the thermodynamic limit.
\end{theorem}

\subsection{Roadmap}

We proceed as follows: (i) state lattice set-up and partition-function bounds; (ii) prove OS reflection positivity and construct the transfer $T$; (iii) derive a strong-coupling Dobrushin bound $r_0(T)\le \alpha(\beta)<1$ and hence a gap; (iv) pass to the thermodynamic limit at fixed spacing.

\bigskip
\noindent\begin{center}
\fbox{\parbox{0.93\textwidth}{
\textbf{Unconditional proof track (summary).}
\begin{itemize}
  \item \textbf{Setup (Sec.~\ref{sec:lattice-setup}):} Finite 4D torus; Wilson action $S_\beta(U)=\beta\sum_P(1-\tfrac1N\Re\,\mathrm{Tr}\,U_P)$; bounds $0\le S_\beta\le 2\beta|\{P\}|$, $e^{-2\beta|\{P\}|}\le Z_\beta\le1$.
  \item \textbf{OS positivity (Thm.~\ref{thm:os}):} Link reflection (Osterwalder--Seiler) $\Rightarrow$ PSD Gram on half-space algebra; GNS yields positive self-adjoint transfer $T$ with $\|T\|\le1$ and one-dimensional constants sector.
  \item \textbf{Strong-coupling gap (Thm.~\ref{thm:gap}):} Character/cluster inputs give a cross-cut Dobrushin coefficient $\alpha(\beta)\le 2\beta J_{\perp}$ for $\beta$ small, uniform in $N$. Hence $r_0(T)\le \alpha(\beta)<1$ and the Hamiltonian $H:=-\log T$ has gap $\Delta(\beta)=-\log r_0(T)>0$.
  \item \textbf{Thermodynamic limit (Thm.~\ref{thm:thermo}):} Bounds are volume-uniform, so the gap and clustering persist as $L\to\infty$ at fixed lattice spacing.
  \item \textbf{Conclusion:} Pure $SU(N)$ Yang--Mills on the lattice (small $\beta$) has a positive mass gap, uniformly in $N\ge2$ and volume.
\end{itemize}}}
\end{center}
\bigskip

\section{Core continuum chain (NRC and uniform gap)}

This section records the operator-theoretic continuum chain used throughout: semigroup $\Rightarrow$ resolvent NRC for \emph{all} nonreal $z$, the equivalence between a uniform spectral gap and uniform exponential clustering on a generating local class, spectral-gap persistence to the continuum under NRC, and an optional area-law bridge (appendix) as a parallel route. Full proofs appear inline or in the appendices.

\subsection*{Semigroup $\Rightarrow$ resolvent NRC for all nonreal $z$}

\begin{theorem}[Semigroup $\Rightarrow$ resolvent NRC on $\mathbb C\setminus\mathbb R$]\label{thm:NRC-allz}
Let $\mathcal{H}_n$ and $\mathcal{H}$ be complex Hilbert spaces. Let $H_n\ge 0$ be self-adjoint operators on $\mathcal{H}_n$ and $H\ge 0$ be self-adjoint on $\mathcal{H}$. Assume:
\begin{itemize}
  \item[(H1)] \textbf{Contraction semigroups:} $\|e^{-tH_n}\| \le 1$ and $\|e^{-tH}\| \le 1$ for all $t \ge 0$.
  \item[(H2)] \textbf{Semigroup convergence:} $\sup_{t\ge 0}\,\|e^{-tH_n}-e^{-tH}\|\to 0$ as $n\to\infty$.
\end{itemize}
Then for every $z\in\mathbb C\setminus\mathbb R$,
\[
  \|(H_n-z)^{-1}-(H-z)^{-1}\|\;\xrightarrow[n\to\infty]{}\;0.
\]
Moreover, the convergence is uniform on compact subsets of $\mathbb{C} \setminus \mathbb{R}$.
\end{theorem}

\begin{proof}
\emph{Step 1: Laplace representation for $\Re z > 0$.} For $w$ with $\Re w > 0$, the resolvent admits the representation
\[
  (H-w)^{-1} = \int_0^\infty e^{tw} e^{-tH}\,dt.
\]
By (H1) and (H2), for each $t \ge 0$,
\[
  \|e^{-tH_n} - e^{-tH}\| \to 0 \quad \text{as } n \to \infty.
\]
Since $\|e^{-tH_n}\|, \|e^{-tH}\| \le 1$ and $\int_0^\infty e^{t\Re w}\,dt = 1/\Re w < \infty$, dominated convergence gives
\[
  \|(H_n-w)^{-1} - (H-w)^{-1}\| \le \int_0^\infty e^{t\Re w} \|e^{-tH_n} - e^{-tH}\|\,dt \to 0.
\]

\emph{Step 2: Bootstrap to all nonreal $z$ via resolvent identity.} Fix $w$ with $\Re w > 0$ (where we have NRC by Step 1). For any nonreal $z$, the second resolvent identity gives
\[
  R(z) - R(w) = (z-w)R(z)R(w), \quad R_n(z) - R_n(w) = (z-w)R_n(z)R_n(w),
\]
where $R(z) := (H-z)^{-1}$ and $R_n(z) := (H_n-z)^{-1}$. Algebraic manipulation yields
\[
  R_n(z) - R(z) = [I + (z-w)R_n(z)]\,[R_n(w) - R(w)]\,[I + (w-z)R(z)].
\]

\emph{Step 3: Uniform bounds on compact sets.} For nonreal $\zeta$, the resolvent bound gives
\[
  \|R(\zeta)\| \le \frac{1}{\operatorname{dist}(\zeta,\mathbb{R})}, \quad \|R_n(\zeta)\| \le \frac{1}{\operatorname{dist}(\zeta,\mathbb{R})}.
\]
On any compact set $K \subset \mathbb{C} \setminus \mathbb{R}$, we have $\inf_{z \in K} \operatorname{dist}(z,\mathbb{R}) > 0$. Thus the operator norms $\|I + (z-w)R_n(z)\|$ and $\|I + (w-z)R(z)\|$ are uniformly bounded for $z \in K$ and all $n$.

\emph{Step 4: Conclusion.} Since $\|R_n(w) - R(w)\| \to 0$ by Step 1, and the bracketed factors in Step 2 are uniformly bounded on compact sets, we obtain
\[
  \sup_{z \in K} \|R_n(z) - R(z)\| \le C_K \|R_n(w) - R(w)\| \to 0,
\]
where $C_K$ depends only on $K$ and $w$. This establishes uniform convergence on compact subsets of $\mathbb{C} \setminus \mathbb{R}$.
\end{proof}

\begin{corollary}[UEI with explicit constants]\label{cor:uei-explicit-constants}
In the setting of Theorem~\ref{thm:uei-fixed-region}, fix any $a\in(0,a_0]$ with $\beta(a)\ge \beta_{\min}>0$. Let
\[
  \rho_{\min}(R,N)\ :=\ c_2(R,N)\,\beta_{\min},\qquad
  G_R(R,N,a_0)\ :=\ C_1(R,N)\,a_0^4,
\]
where $c_2(R,N)$ is the LSI constant from Step 2 and $C_1(R,N)$ the Lipschitz constant from Step 3 of the proof of Theorem~\ref{thm:uei-fixed-region}. Set
\[
  \eta_R\ :=\ \min\Big\{\,t_*(R,N),\ \sqrt{\,\rho_{\min}(R,N)\big/ G_R(R,N,a_0)\,}\ \Big\},\qquad
  C_R\ :=\ \exp\big(\eta_R\,M_R(R,N,\beta_{\min})\big)\,e^{1/2}.
\]
Then for all volumes $L$ and all boundary conditions outside $R$,
\[
  \mathbb{E}_{\mu_{L,a}}\big[e^{\eta_R S_R(U)}\big]\ \le\ C_R.
\]
All constants depend only on $(R,a_0,N,\beta_{\min})$ and are independent of $L$ and $\beta\ge \beta_{\min}$.
\end{corollary}

\begin{proof}
This is the consolidation of Steps 2--5 in the proof of Theorem~\ref{thm:uei-fixed-region} with $\rho_{\min}:=c_2\beta_{\min}$ and $G_R:=C_1 a_0^4$, choosing $\eta_R$ so that the Herbst bound yields a $\le e^{1/2}$ factor for the centered variable and then absorbing the (uniform) mean $M_R$.
\end{proof}

\subsection*{Uniform gap $\Rightarrow$ uniform clustering; converse}

\begin{proposition}[Gap $\Rightarrow$ clustering (uniform)]\label{prop:gap-to-cluster}
If $\mathrm{spec}(H_{L,a})\subset\{0\}\cup[\gamma_0,\infty)$ holds uniformly in $(L,a)$, then for any time-zero, gauge-invariant local $O$ with $\langle O\rangle=0$ and all $t\ge 0$,
\[
  |\langle\Omega, O(t)O(0)\Omega\rangle|\;\le\;\|O\Omega\|^2 e^{-\gamma_0 t},
\]
uniformly in $(L,a)$.
\end{proposition}

\begin{proposition}[OS0 polynomial bounds with explicit constants]\label{prop:OS0-poly}
Assume uniform exponential clustering of truncated correlations on fixed physical regions with parameters $(C_0,m)$ (independent of $(L,a)$). Fix any $q>d$ and set $p=d+1$. Then there exist explicit constants
\[
  C_n(C_0,m,q,d)\ :=\ C_0^n\,C_{\mathrm{tree}}(n)\,\Bigl(\frac{2^d\,\zeta(q-d)}{1-e^{-m}}\Bigr)^{n-1},\qquad C_{\mathrm{tree}}(n)\le n^{n-2},
\]
such that for all local loop families $\Gamma_1,\dots,\Gamma_n$,
\[
  |S_n(\Gamma_1,\dots,\Gamma_n)|\ \le\ C_n\,\prod_{i=1}^n (1+\operatorname{diam}\Gamma_i)^p\,\prod_{1\le i<j\le n} (1+\operatorname{dist}(\Gamma_i,\Gamma_j))^{-q},
\]
uniformly in $(L,a)$. In particular, the Schwinger functions are tempered (OS0).
\end{proposition}

\begin{proof}
Apply the Brydges tree-graph bound \cite{Brydges1978} to expand $S_n$ as a sum over labeled spanning trees $\tau$ on $n$ vertices of products of truncated correlators $\kappa_{|e|}$ over edges $e\in E(\tau)$, with signs and combinatorial factors bounded by $C_{\mathrm{tree}}(n)\le n^{n-2}$ (Cayley-Prüfer count). Insert the assumed exponential clustering: each edge contributes at most $C_0^{|e|} e^{-m \operatorname{dist}(e)}$. There are $n-1$ edges, yielding overall $C_0^n$ (overcounting the root).

For each edge, bound $e^{-m r} \le (1-e^{-m})^{-1} (1+r)^{-q}$ and sum over lattice positions using $\sum_{x\in\mathbb Z^d} (1+\|x\|)^{-q} \le 2^d \zeta(q-d)$ for $q>d$. Multiply the $(n-1)$ identical factors to get $\bigl(\frac{2^d \zeta(q-d)}{1-e^{-m}}\bigr)^{n-1}$.

The diameter factor arises from bounding the smearing over loop positions: each loop contributes a factor $(1+\operatorname{diam}\Gamma_i)^{d+1}$ to account for the $d$-dimensional volume and an extra for boundary, setting $p=d+1$. All steps are uniform in $(L,a)$, completing the proof.
\end{proof}

\begin{corollary}[OS0 with explicit constants in $d=4$]\label{cor:os0-explicit-4d}
In $d=4$, fix any $q>4$ and set $p=5$. Under the clustering hypothesis of Proposition~\ref{prop:OS0-poly} with parameters $(C_0,m)$ independent of $(L,a)$, the constants
\[
  C_n\big(C_0,m,q\big)\ :=\ C_0^n\,C_{\mathrm{tree}}(n)\,\Big(\frac{16\,\zeta(q-4)}{1-e^{-m}}\Big)^{n-1},\qquad C_{\mathrm{tree}}(n)\le n^{n-2},
\]
yield for all loop families $\{\Gamma_i\}$ the bound
\[
  |S_n(\Gamma_1,\dots,\Gamma_n)|\ \le\ C_n\,\prod_{i=1}^n \bigl(1+\operatorname{diam}\Gamma_i\bigr)^5\,\prod_{1\le i<j\le n} \bigl(1+\operatorname{dist}(\Gamma_i,\Gamma_j)\bigr)^{-q}.
\]
Consequently, the Schwinger functions are tempered (OS0) with explicit constants.
\end{corollary}

\begin{proof}
Specialize Proposition~\ref{prop:OS0-poly} to $d=4$; $2^d=16$ and $p=d+1=5$.
\end{proof}

\begin{proposition}[Clustering on a generating local class $\Rightarrow$ gap]\label{prop:cluster-to-gap}
Suppose there exist $R_*>0$, $\gamma>0$, and $C_*<\infty$, independent of $(L,a)$, such that for all local $O$ with $\langle O\rangle=0$,
\[
  |\langle\Omega, O(t)O(0)\Omega\rangle|\;\le\; C_*\,\|O\Omega\|^2 e^{-\gamma t}\quad(\forall t\ge 0),
\]
and that the span of such $O\Omega$ is dense in $\Omega^\perp$. Then $\mathrm{spec}(H_{L,a})\subset\{0\}\cup[\gamma,\infty)$ uniformly in $(L,a)$.
\end{proposition}

\subsection*{Uniform gap persistence in the continuum}

\begin{theorem}[Gap persistence via NRC]\label{thm:gap-persist}
Let $(L_n,a_n)$ be a scaling sequence. If $e^{-tH_{L_n,a_n}}\to e^{-tH}$ in operator norm for all $t\ge 0$ and $\mathrm{spec}(H_{L_n,a_n})\subset\{0\}\cup[\gamma_0,\infty)$ uniformly in $n$, then $0$ is an isolated eigenvalue of $H$ and $\mathrm{spec}(H)\subset\{0\}\cup[\gamma_0,\infty)$.
\end{theorem}

\begin{proof}
By Theorem~\ref{thm:NRC-allz}, resolvents converge in norm on $\mathbb C\setminus\mathbb R$. Riesz projections around $0$ converge in operator norm, preserving rank and the open gap. Hence the lower spectral edge persists at $\gamma_0$.
\smallskip
\noindent\emph{Details (Riesz projection and openness of the gap).} Let $R_n(z)=(H_{L_n,a_n}-z)^{-1}$, $R(z)=(H-z)^{-1}$. Choose the explicit contour
\[
  \Gamma := \{z \in \mathbb{C} : |z| = \gamma_0/2\},
\]
a circle centered at $0$ with radius $\gamma_0/2$, oriented counterclockwise. Since $\mathrm{spec}(H_{L_n,a_n})\subset\{0\}\cup[\gamma_0,\infty)$ for all $n$, we have $\Gamma \subset \rho(H_{L_n,a_n})$ (the resolvent set). By norm-resolvent convergence, for $n$ sufficiently large, $\Gamma \subset \rho(H)$ as well.

The Riesz projections are
\[
  P_n := \frac{1}{2\pi i}\int_\Gamma R_n(z)\,dz, \quad P := \frac{1}{2\pi i}\int_\Gamma R(z)\,dz.
\]
Since $\Gamma$ separates $\{0\}$ from $[\gamma_0,\infty)$ and $\mathrm{spec}(H_{L_n,a_n})\cap(0,\gamma_0)=\varnothing$, we have $P_n = $ projection onto the eigenspace of $H_{L_n,a_n}$ at $0$, hence $\operatorname{rank} P_n = 1$ (the vacuum).

By the resolvent estimate, for $z \in \Gamma$,
\[
  \|R_n(z) - R(z)\| \le \|R(z)\| \cdot \|I - P_n\| + \|R(z)\| \cdot \varepsilon_n \cdot \|R_n(z)\| \cdot \|(H_{L_n,a_n}+1)^{1/2}\|,
\]
where $\varepsilon_n \to 0$ is the graph-norm defect. Since $\operatorname{dist}(z,\mathbb{R}) = \gamma_0/2$ for all $z \in \Gamma$, we have $\|R_n(z)\|, \|R(z)\| \le 2/\gamma_0$. Thus
\[
  \|P_n - P\| \le \frac{|\Gamma|}{2\pi} \sup_{z \in \Gamma} \|R_n(z) - R(z)\| \le \frac{\gamma_0}{2} \cdot o(1) \to 0.
\]
Operator-norm convergence preserves rank in the limit: $\operatorname{rank} P = \lim_{n\to\infty} \operatorname{rank} P_n = 1$. Hence $0$ is an isolated eigenvalue of $H$ with one-dimensional eigenspace.

For the gap persistence, if $\lambda \in (0,\gamma_0)$ were in $\mathrm{spec}(H)$, then by lower semicontinuity of the spectrum under norm-resolvent convergence (Kato \cite{Kato1995}, Theorem IV.3.1), there would exist $\lambda_n \in \mathrm{spec}(H_{L_n,a_n})$ with $\lambda_n \to \lambda$. But this contradicts $\mathrm{spec}(H_{L_n,a_n}) \cap (0,\gamma_0) = \varnothing$. Therefore $\mathrm{spec}(H) \subset \{0\} \cup [\gamma_0,\infty)$.
\end{proof}

\subsection*{Optional: area law $+$ tube geometry $\Rightarrow$ uniform gap}

\begin{description}
\item[AL] (Area law, uniform in $(L,a)$). There exist $\sigma_*>0$ and $C_{\mathrm{AL}}<\infty$ such that large rectangular Wilson loops obey $|\langle W_{\Gamma(R,T)}\rangle|\le C_{\mathrm{AL}} e^{-\sigma_* RT}$ in physical units.
\item[TUBE] (Geometric tube bound). For loops supported in a fixed physical ball $B_{R_*}$ at times $0$ and $t$, any spanning surface has area $\ge \kappa_* t$ with $\kappa_*>0$ depending only on $R_*$. 
\end{description}

\begin{theorem}[Optional: Area law $+$ tube $\Rightarrow$ uniform gap]\label{thm:AL-gap}
Under AL and TUBE, $\mathrm{spec}(H_{L,a})\subset\{0\}\cup[\sigma_*\kappa_*,\infty)$ uniformly in $(L,a)$. Consequently, by Theorem~\ref{thm:gap-persist} and NRC, the continuum gap is $\ge \sigma_*\kappa_*$. 
\end{theorem}

\noindent\emph{Remark.} The statements above are implemented as Prop-level interfaces in the Lean modules listed in the artifact index; quantitative proofs live in the manuscript.

\section{Lattice Yang--Mills set-up and bounds}
\label{sec:lattice-setup}

We work on a finite 4D torus with sites $x\in\Lambda$ and $SU(N)$ link variables $U_{x,\mu}$. For a plaquette $P$, let $U_P$ be the ordered product of links around $P$. The Wilson action is
\[
 S_{\beta}(U) := \beta \sum_{P} \Bigl(1 - \tfrac{1}{N} \operatorname{Re} \operatorname{Tr} U_P\Bigr).
\]
Since $-N\le \operatorname{Re} \operatorname{Tr} V \le N$ for all $V\in SU(N)$, we have $0\le S_{\beta}(U)\le 2\beta |\{P\}|$. With normalized Haar product measure, the partition function obeys $e^{-2\beta |\{P\}|}\le Z_{\beta}\le 1$.

\section{Reflection positivity and transfer operator}

Choose a time-reflection hyperplane and define the standard Osterwalder--Seiler link reflection $\theta$. For the *-algebra $\mathcal A_+$ of cylinder observables supported in $t\ge 0$, the sesquilinear form $\langle F,G\rangle_{OS}:=\int \overline{F(U)}\,(\theta G)(U)\, d\mu_{\beta}(U)$ is positive semidefinite. By GNS, we obtain a Hilbert space $\mathcal H$ and a positive self-adjoint transfer operator $T$ with $\lVert T\rVert\le 1$ and one-dimensional constants sector.
\smallskip
\noindent\emph{Remark.} The OS reflection makes the half-space algebra a pre-Hilbert space under the reflected inner product; the Markov/transfer step is a contraction by Cauchy–Schwarz in this inner product.

\paragraph{Notation and Hamiltonian.}
Let $\Omega\in\mathcal H$ denote the vacuum vector (the class of constants). Write $\mathcal H_0:=\Omega^{\perp}$ for the mean-zero subspace. Define
\[
  r_0(T)\;:=\; \sup\{\,|\lambda| : \lambda\in\operatorname{spec}(T|_{\mathcal H_0})\,\},\qquad
  H\;:=\;-\log T\ \text{ on }\ \mathcal H_0
\]
by spectral calculus. The Hamiltonian gap is $\Delta(\beta):=-\log r_0(T)$.
For brevity, we also write $\gamma(\beta):=\Delta(\beta)$.

\subsection*{Sketch of proof (OS--Seiler)}
The Wilson action decomposes into $S_\beta=S_\beta^{(+)}+S_\beta^{(-)}+S_\beta^{(\perp)}$, where $S_\beta^{(\perp)}$ is a sum over plaquettes intersecting the reflection plane. Expanding the crossing weights in characters and using that irreducible characters $\chi_R$ are positive-definite class functions, together with Haar invariance and $\theta$-invariance of the measure, yields that the Gram matrix $[\langle F_i,\theta F_j\rangle_{OS}]$ is positive semidefinite for any finite family $\{F_i\}\subset \mathcal A_+$. This is the Osterwalder--Seiler argument.

\paragraph{Character positivity and the crossing kernel (details).}
\begin{lemma}[Irreducible characters are positive definite]\label{lem:char-pd}
For any compact group $G$ and any unitary irreducible representation $R$, the class function $\chi_R(g)=\operatorname{Tr}\,R(g)$ is positive definite: for any $g_1,\dots,g_m\in G$ and $c\in\mathbb C^m$,
\[
  \sum_{i,j=1}^m \overline{c_i}\,c_j\,\chi_R(g_i^{-1} g_j)\ \ge\ 0.
\]
\end{lemma}
\begin{proof}
Let $v:=\sum_i c_i\,R(g_i)\,v_0$ for any fixed $v_0$ in the representation space. Then
\[
  \sum_{i,j}\overline{c_i}\,c_j\,\chi_R(g_i^{-1} g_j)\ =\ \sum_{i,j}\overline{c_i}\,c_j\,\operatorname{Tr}\big(R(g_i)^{*}R(g_j)\big)\ =\ \|\sum_j c_j R(g_j)\|_{\mathrm{HS}}^2\ \ge\ 0.
\]
Alternatively, this is a standard consequence of Peter–Weyl.
\end{proof}

\begin{proposition}[PSD crossing Gram for Wilson link reflection]
For the Wilson action and link reflection $\theta$, the OS Gram matrix $[\langle F_i,\theta F_j\rangle_{OS}]_{i,j}$ is positive semidefinite for any finite $\{F_i\}\subset\mathcal A_+$.
\end{proposition}
\begin{proof}
Let $\{F_i\}_{i=1}^n \subset \mathcal A_+$ be a finite family of half-space observables. We must show that the matrix $M_{ij} := \langle F_i, \theta F_j \rangle_{OS}$ is positive semidefinite.

\emph{Step 1: Decompose the Wilson action.} Write $S_\beta = S_\beta^{(+)} + S_\beta^{(-)} + S_\beta^{(\perp)}$, where $S_\beta^{(\pm)}$ are sums over plaquettes entirely in the positive/negative half-spaces and $S_\beta^{(\perp)}$ sums over plaquettes crossing the reflection plane. For observables $F_i \in \mathcal A_+$, we have
\[
  M_{ij} = \int \overline{F_i(U)} \, (\theta F_j)(U) \, e^{-S_\beta(U)} \, dU = \int \overline{F_i(U^+)} \, F_j(\theta U^+) \, K_\beta(U^+, U^-) \, dU^+ dU^-,
\]
where $K_\beta(U^+, U^-)$ is the crossing kernel arising from $\exp(-S_\beta^{(\perp)})$ and we used $\theta$-invariance of the Haar measure.

\emph{Step 2: Character expansion of crossing weights.} For each plaquette $P$ crossing the reflection plane, expand (Montvay–Münster \cite{MontvayMu\"nster1994}, §4.2):
\[
  \exp\Big(\tfrac{\beta}{N}\,\Re\,\operatorname{Tr} U_P\Big) = \sum_{R} c_R(\beta)\,\chi_R(U_P), \quad c_R(\beta) = \int_{SU(N)} \exp\Big(\tfrac{\beta}{N}\,\Re\,\operatorname{Tr} V\Big) \overline{\chi_R(V)} \, dV \ge 0,
\]
where the nonnegativity follows from $\exp(\cdot) > 0$ and Schur orthogonality. The crossing kernel becomes
\[
  K_\beta(U^+, U^-) = \prod_{P \in \mathcal P_\perp} \sum_{R_P} c_{R_P}(\beta) \chi_{R_P}(U_P) = \sum_{\{R_P\}} \Big(\prod_{P} c_{R_P}(\beta)\Big) \prod_{P} \chi_{R_P}(U_P),
\]
where $\mathcal P_\perp$ denotes plaquettes crossing the cut.

\emph{Step 3: Integration and tensor structure.} After integrating out $U^-$ with Haar measure, only terms with matching representations survive. The result is
\[
  M_{ij} = \sum_{\{R_P\}} w_{\{R_P\}} \int \overline{F_i(U^+)} F_j(\theta U^+) \prod_{\ell \in \text{cut}} \chi_{R_\ell}(g_\ell^{-1} h_\ell) \, dU^+,
\]
where $w_{\{R_P\}} \ge 0$ are products of $c_{R_P}(\beta) \ge 0$, and $(g_\ell, h_\ell)$ are appropriate group elements from $U^+$ entering the cut links.

\emph{Step 4: PSD property of character kernels.} For each fixed representation assignment $\{R_\ell\}$, the kernel $\prod_\ell \chi_{R_\ell}(g_\ell^{-1} h_\ell)$ defines a PSD form by Lemma \ref{lem:char-pd} (each $\chi_{R_\ell}$ is PSD) and the fact that tensor products of PSD kernels are PSD. Thus the matrix
\[
  M_{ij}^{\{R_\ell\}} := \int \overline{F_i(U^+)} F_j(\theta U^+) \prod_{\ell} \chi_{R_\ell}(g_\ell^{-1} h_\ell) \, dU^+
\]
satisfies $M^{\{R_\ell\}} \succeq 0$.

\emph{Step 5: Conclusion.} Since $M = \sum_{\{R_P\}} w_{\{R_P\}} M^{\{R_\ell\}}$ with $w_{\{R_P\}} \ge 0$ and each $M^{\{R_\ell\}} \succeq 0$, we have $M \succeq 0$. This establishes reflection positivity. The GNS construction then yields a Hilbert space $\mathcal H$, and the transfer step $T: [F] \mapsto [\tau_1 F]$ (where $\tau_1$ is unit time translation) is positive and self-adjoint by OS positivity.
\end{proof}

\section{Strong-coupling contraction and mass gap}

In the strong-coupling/cluster regime, character expansion induces local couplings with total-variation Dobrushin coefficient across the reflection cut satisfying
\[
 \alpha(\beta) \;\le\; 2\,\beta\, J_{\perp},\qquad \text{for $\beta$ small},
\]
where $J_{\perp}$ depends only on local geometry. Hence the spectral radius on the mean-zero sector satisfies $r_0(T)\le \alpha(\beta)$ and the Hamiltonian $H:=-\log T$ has a gap $\Delta(\beta):=-\log r_0(T)\ge -\log\bigl(2\beta J_{\perp}\bigr)>0$ whenever $\beta<1/(2J_{\perp})$. The bounds are uniform in $N\ge 2$ and in the volume.

\paragraph{Influence estimate (explicit).}
Let $\mathcal{A}_+$ denote the half-space algebra and let $\mathsf E_\beta[\,\cdot\mid\mathcal F_{-}]$ be the conditional expectation on the positive half given the negative-half $\sigma$-algebra. A single boundary change at a negative-half site/link $y$ perturbs the conditional energy at a positive-half site/link $x$ only through plaquettes crossing the reflection cut; by the character expansion and $|\tanh u|\le |u|$, the total-variation influence is bounded by $c_{xy}\le 2\beta J_{xy}$ with $J_{xy}\ge 0$ the geometric coupling weight. Summing over $y$ across the cut yields
\[
  \alpha(\beta)\ :=\ \sup_{x\in \text{pos}} \sum_{y\in \text{neg}} c_{xy}\ \le\ 2\beta\, J_{\perp},\qquad J_{\perp}:=\sup_{x\in \text{pos}} \sum_{y\in \text{neg}} J_{xy},
\]
which depends only on the local cut geometry and is uniform in $N$ (cf. Dobrushin~\cite{Dobrushin1970}, Shlosman~\cite{Shlosman1986}).

\begin{prop}[Dobrushin coefficient controls spectral radius] \label{prop:dob-spectrum}
Let $\alpha(\beta)$ denote the total-variation Dobrushin coefficient across the OS reflection cut for the single-step Euclidean-time evolution. Then
\[
  r_0(T)\;\le\; \alpha(\beta).
\]
Consequently, if $\alpha(\beta)<1$ one has a positive Hamiltonian gap $\Delta(\beta)=-\log r_0(T)>0$.
\end{prop}

\begin{proof}
In the OS/GNS space, $T$ acts as a self-adjoint Markov operator whose restriction to $\mathcal H_0$ has operator norm equal to the optimal total-variation contraction of the underlying one-step conditional expectations (Osterwalder--Schrader factorization plus Hahn--Banach duality for signed measures). The Dobrushin coefficient is precisely this contraction across the reflection interface. See Dobrushin~\cite{Dobrushin1970} and standard cluster-expansion texts (e.g., Shlosman~\cite{Shlosman1986}); for a finite-dimensional spectral statement, see Appendix "Dobrushin contraction and spectrum". Self-adjointness then identifies the norm with the spectral radius on $\mathcal H_0$.
\end{proof}

\begin{lemma}[Explicit Dobrushin influence bound]\label{lem:dob-influence}
The total-variation Dobrushin coefficient across the reflection cut satisfies
\[
\alpha(\beta) \le 2\beta J_{\perp},
\]
where $J_{\perp} := \sup_{x \in \text{pos}} \sum_{y \in \text{neg}} J_{xy}$ depends only on the local cut geometry (R_, a_0) and is uniform in $N \ge 2$, $\beta$, and $L$.
\end{lemma}

\begin{proof}
Let $\mathsf{E}_\beta[\cdot \mid \mathcal{F}_{-}]$ be the conditional expectation on the positive half given the negative-half $\sigma$-algebra. A single boundary change at a negative-half site/link $y$ perturbs the conditional energy at a positive-half site/link $x$ only through plaquettes crossing the reflection cut. By the character expansion and $|\tanh u| \le |u|$, the total-variation influence is bounded by $c_{xy} \le 2\beta J_{xy}$ with $J_{xy} \ge 0$ the geometric coupling weight (number of crossing plaquettes connecting $x$ and $y$, weighted by 1). Summing over $y$ across the cut yields the bound on $\alpha(\beta)$. The supremum defining $J_{\perp}$ is finite and depends only on the fixed physical slab radius $R_*$ and thickness bound $a_0$, independent of $N$, $\beta$, and volume $L$.
\end{proof}

\section{Appendix: Coarse-graining convergence and gap persistence (P8)}

We record a uniform coarse--graining bound and operator--norm convergence for reflected loop kernels along a voxel--to--continuum refinement, together with hypotheses that ensure gap persistence in the continuum. This appendix supports the optional continuum discussion in Sec.~"Continuum scaling windows".

\paragraph{Setting.}
Let $K_n$ be reflected loop kernels (covariances/Green's functions) arising as inverses of positive operators $H_n$ (e.g., discrete Hamiltonians or elliptic operators): $K_n=H_n^{-1}$, with continuum limits $K=H^{-1}$. Reflection positivity implies self--adjointness of $H_n$ and $K_n$. Let $R_n$ (restriction) and $P_n$ (prolongation) compare discrete and continuum Hilbert spaces.

\paragraph{Uniform bound.}
Define the discrete gaps
\[
  \beta_n\;:=\;\inf \operatorname{spec}(H_n).
\]
If there exists $\beta_0>0$ with $\beta_n\ge \beta_0$ for all $n$, then
\[
  \lVert K_n\rVert_{\mathrm{op}}\;=\;\frac{1}{\beta_n}\;\le\;\frac{1}{\beta_0}.
\]
This follows from coercivity (strict positivity of $H$), stability of the discretization preserving positivity, and uniform discrete functional inequalities (e.g., discrete Poincar\'e) with constants independent of the voxel size.

\paragraph{Operator--norm convergence.}
Assume stability above and consistency (local truncation errors vanish on a dense core). Then
\begin{equation}
\label{eq:p8-norm}
  \big\lVert P_n K_n R_n - K\big\rVert_{\mathrm{op}}\;\longrightarrow\;0\qquad (n\to\infty),
\end{equation}
equivalently, $H_n\to H$ in norm resolvent sense. The upgrade from strong convergence to \eqref{eq:p8-norm} uses collective compactness: if $K$ is compact and $\{P_n K_n R_n\}$ is collectively compact via uniform discrete regularity, then strong convergence implies norm convergence.

\paragraph{Gap persistence (continuum $\gamma>0$).}
Suppose further:
\begin{itemize}
  \item (H1) $H_n$ and $H$ are self--adjoint.
  \item (H2) $H_n\to H$ in norm resolvent sense (\eqref{eq:p8-norm}).
  \item (H3) There is a uniform discrete gap: for some interval $(a,b)$ with $\gamma_0:=b-a>0$, one has $\operatorname{spec}(H_n)\cap(a,b)=\varnothing$ for all large $n$.
\end{itemize}
Then spectral convergence (Hausdorff) yields $\operatorname{spec}(H)\cap(a,b)=\varnothing$, so the continuum gap satisfies $\gamma\ge \gamma_0>0$.

\section{Optional: Continuum scaling-window routes (KP/area-law)}

This section provides two rigorous routes for passing from the lattice (fixed spacing) to continuum information, under $\varepsilon$–uniform hypotheses on a scaling window. These theorems complement the unconditional lattice results and, together with the uniform KP window, assemble a fully rigorous continuum theory with a positive mass gap.

\subsection*{Optional A: Uniform lattice area law implies a continuum string tension}

\paragraph{Setting.}
Fix a dimension $d\ge 2$ and a hypercubic lattice $\varepsilon\,\mathbb{Z}^d$ with spacing $\varepsilon\in(0,\varepsilon_0]$. For a nearest--neighbour lattice loop $\Lambda\subset \varepsilon\,\mathbb{Z}^d$ let
\[
  A_\varepsilon^{\min}(\Lambda)\in\mathbb{N}
\]
be the minimal number of plaquettes in any lattice surface spanning $\Lambda$, and let $P_\varepsilon(\Lambda)\in\mathbb{N}$ be the number of lattice edges on $\Lambda$ (its lattice perimeter). Set the corresponding physical area and perimeter
\[
  \mathsf{Area}_\varepsilon(\Lambda):=\varepsilon^2 A_\varepsilon^{\min}(\Lambda),\qquad
  \mathsf{Per}_\varepsilon(\Lambda):=\varepsilon P_\varepsilon(\Lambda).
\]
For a continuum rectifiable closed curve $\Gamma\subset\mathbb{R}^d$ let $\mathsf{Area}(\Gamma)$ denote the least Euclidean area of any (Lipschitz) spanning surface with boundary $\Gamma$, and let $\mathsf{Per}(\Gamma)$ be its Euclidean length.

\paragraph{Uniform lattice area law (input; strong coupling).}
See Appendix "Strong-coupling area law for Wilson loops (R6)" for a standard derivation of a lattice area law with a positive string tension and a perimeter correction; the present paragraph abstracts those bounds uniformly over a scaling window.
Assume there exist functions $\tau_\varepsilon>0$ and $\kappa_\varepsilon\ge 0$, defined for $\varepsilon\in (0,\varepsilon_0]$, and constants
\[
  T_*:=\inf_{0<\varepsilon\le\varepsilon_0}\frac{\tau_\varepsilon}{\varepsilon^2}>0,\qquad
  C_*:=\sup_{0<\varepsilon\le\varepsilon_0}\frac{\kappa_\varepsilon}{\varepsilon}<\infty,
\]
such that for all sufficiently large lattice loops $\Lambda\subset\varepsilon\,\mathbb{Z}^d$ (size measured in lattice units, which will automatically hold for fixed physical loops as $\varepsilon\downarrow 0$),
\begin{equation}
\label{eq:lattice-area-law}
  -\log\langle W(\Lambda)\rangle \;\ge\; \tau_\varepsilon\,A_\varepsilon^{\min}(\Lambda)\;-
  \;\kappa_\varepsilon\,P_\varepsilon(\Lambda)
  \;=\;\Big(\tfrac{\tau_\varepsilon}{\varepsilon^2}\Big)\mathsf{Area}_\varepsilon(\Lambda)\;-
  \;\Big(\tfrac{\kappa_\varepsilon}{\varepsilon}\Big)\mathsf{Per}_\varepsilon(\Lambda).
\end{equation}
In the strong--coupling/cluster regime, \eqref{eq:lattice-area-law} follows from the character expansion: writing the Wilson weight in irreducible characters, the activity ratio $\rho(\beta)$ for nontrivial representations obeys $\mu\,\rho(\beta) < 1$ for all sufficiently small $\beta$, with a lattice constant $\mu$, yielding $T(\beta):= -\log \rho(\beta) > 0$ and a perimeter correction controlled by $\kappa_\varepsilon$.

\paragraph{Directed embeddings of loops.}
Let $\Gamma\subset\mathbb{R}^d$ be a fixed rectifiable closed curve. A \emph{directed family} $\{\Gamma_\varepsilon\}_{\varepsilon\downarrow 0}$ of lattice loops converging to $\Gamma$ means: (i) $\Gamma_\varepsilon\subset\varepsilon\,\mathbb{Z}^d$ is a nearest--neighbour loop, (ii) the Hausdorff distance $d_H(\Gamma_\varepsilon,\Gamma)\to 0$ as $\varepsilon\downarrow 0$, (iii) each $\Gamma_\varepsilon$ is contained in a tubular neighbourhood of $\Gamma$ of radius $O(\varepsilon)$ and follows the orientation of $\Gamma$ (e.g., via grid--snapping of a $C^1$ parametrization).

\paragraph{Two geometric facts.}
\emph{Fact A (surface convergence).} For any directed family $\{\Gamma_\varepsilon\to\Gamma\}$,
\begin{equation}
\label{eq:area-conv}
  \lim_{\varepsilon\downarrow 0}\mathsf{Area}_\varepsilon(\Gamma_\varepsilon)\;=\;\mathsf{Area}(\Gamma).
\end{equation}
\emph{Sketch.} Because lattice surfaces are a subclass of Lipschitz surfaces, $\mathsf{Area}_\varepsilon(\Gamma_\varepsilon)\ge \mathsf{Area}^{\text{cont}}(\Gamma_\varepsilon)$, where the latter is the continuum minimal area for the boundary $\Gamma_\varepsilon$; lower semicontinuity of the Plateau problem under boundary convergence yields $\liminf_{\varepsilon\downarrow 0}\mathsf{Area}^{\text{cont}}(\Gamma_\varepsilon)\ge \mathsf{Area}(\Gamma)$. For the matching upper bound, approximate a continuum minimizer for $\Gamma$ by a cubical polyhedral surface with mesh $o(1)$ lying on $\varepsilon\,\mathbb{Z}^d$ and having boundary $\Gamma_\varepsilon$; its area exceeds $\mathsf{Area}(\Gamma)$ by $o(1)$, hence $\limsup_{\varepsilon\downarrow 0}\mathsf{Area}_\varepsilon(\Gamma_\varepsilon)\le \mathsf{Area}(\Gamma)$. Combining gives \eqref{eq:area-conv}.

\emph{Fact B (perimeter control).} There exists a universal constant $\kappa_d:=\sup_{u\in\mathbb{S}^{d-1}}\sum_{i=1}^d |u_i|=\sqrt{d}$ such that for any directed family,
\begin{equation}
\label{eq:per-bound}
  \limsup_{\varepsilon\downarrow 0}\mathsf{Per}_\varepsilon(\Gamma_\varepsilon)\;\le\;\kappa_d\,\mathsf{Per}(\Gamma).
\end{equation}
\emph{Sketch.} On each sufficiently short segment of $\Gamma$ with unit tangent $u$, the nearest--neighbour routing on the lattice uses approximately $|u_1|, \dots, |u_d|$ fractions of steps along the coordinate directions. The local lattice length density is $\sum_i |u_i|$, which by Cauchy--Schwarz is at most $\sqrt{d}$. Integrating along $\Gamma$ and passing to the limit yields \eqref{eq:per-bound}. For planar loops, one may take $\kappa_d=\sqrt{2}$.

\paragraph{Main statement (continuum area law with perimeter term).}
\begin{theorem}
Let $\Gamma\subset\mathbb{R}^d$ be a rectifiable closed curve with $\mathsf{Area}(\Gamma)<\infty$. Assume the uniform lattice bound \eqref{eq:lattice-area-law} on the scaling window $(0,\varepsilon_0]$. Define the $\varepsilon$--independent constants
\[
  T\;:=\;\inf_{0<\varepsilon\le\varepsilon_0}\frac{\tau_\varepsilon}{\varepsilon^2}\;>\;0,\qquad
  C_0\;:=\;\sup_{0<\varepsilon\le\varepsilon_0}\frac{\kappa_\varepsilon}{\varepsilon}\;<\;\infty,\qquad
  C\;:=\;\kappa_d\,C_0.
\]
Then for any directed family $\{\Gamma_\varepsilon\to\Gamma\}$,
\begin{equation}
\label{eq:continuum-bound}
  \limsup_{\varepsilon\downarrow 0}\bigl[-\log\langle W(\Gamma_\varepsilon)\rangle\bigr]
  \;\ge\;
  T\,\mathsf{Area}(\Gamma)\;-
  \;C\,\mathsf{Per}(\Gamma).
\end{equation}
In particular, the continuum string tension is positive and bounded below by $T$.
\end{theorem}

\begin{proof}
Starting from \eqref{eq:lattice-area-law} with $\Lambda=\Gamma_\varepsilon$ and taking $\limsup_{\varepsilon\downarrow 0}$, use $\limsup(A_\varepsilon-B_\varepsilon)\ge (\inf A_\varepsilon)-(\sup B_\varepsilon)$ in the form
\[
  \limsup_{\varepsilon\downarrow 0}\bigl[A_\varepsilon-B_\varepsilon\bigr]
  \;\ge\;
  \Big(\inf_{0<\varepsilon\le\varepsilon_0}\tfrac{\tau_\varepsilon}{\varepsilon^2}\Big)\cdot
  \liminf_{\varepsilon\downarrow 0}\mathsf{Area}_\varepsilon(\Gamma_\varepsilon)
  \;-
  \;\Big(\sup_{0<\varepsilon\le\varepsilon_0}\tfrac{\kappa_\varepsilon}{\varepsilon}\Big)\cdot
  \limsup_{\varepsilon\downarrow 0}\mathsf{Per}_\varepsilon(\Gamma_\varepsilon).
\]
Applying Facts A and B yields \eqref{eq:continuum-bound}.
\end{proof}

\paragraph{Remarks.}
1. The constants $T$ and $C$ are $\varepsilon$--independent: $T$ is the uniform lower bound on the lattice string tension in physical units ($\tau_\varepsilon/\varepsilon^2$), while $C$ is the product of the uniform perimeter coefficient in physical units ($C_0=\sup\kappa_\varepsilon/\varepsilon$) with the geometric factor $\kappa_d=\sqrt{d}$. For planar Wilson loops, $C=\sqrt{2}\,C_0$.

2. The "large loop" qualifier is automatic here: for any fixed physical loop $\Gamma$, the lattice representative $\Gamma_\varepsilon$ has diameter of order $\varepsilon^{-1}$ in lattice units, so the hypotheses behind \eqref{eq:lattice-area-law} (from strong--coupling/cluster bounds) apply for all sufficiently small $\varepsilon$.

3. The bound \eqref{eq:continuum-bound} states that the continuum string tension $\sigma_{\text{cont}}:=\liminf_{\varepsilon\downarrow 0}\tau_\varepsilon/\varepsilon^2$ is positive (indeed $\sigma_{\text{cont}}\ge T>0$), with a controlled perimeter subtraction that is uniform along any directed family $\Gamma_\varepsilon\to\Gamma$.

\subsection*{Optional B: Continuum OS reconstruction from a scaling window}

This option outlines a rigorous procedure for constructing a continuum QFT in four dimensions from a family of lattice gauge theories, given tightness and uniform locality/clustering bounds independent of $\varepsilon$.

\paragraph{Existence of the continuum limit measure.}
Assuming tightness of loop observables $W_{\Gamma,\varepsilon}$, Prokhorov compactness yields a subsequence $\varepsilon_k\to 0$ along which the lattice measures converge weakly to a probability measure $\mu$. For any finite collection of loops $\Gamma_1,\dots,\Gamma_n$, the Schwinger functions
\[
  S_n(\Gamma_1,\dots,\Gamma_n):=\lim_{\varepsilon\to 0}\,\langle W_{\Gamma_1,\varepsilon}\cdots W_{\Gamma_n,\varepsilon}\rangle
\]
exist under the uniform locality/clustering bounds, and characterize $\mu$.
Under the AF schedule (Appendix C1d), the embedded resolvents are Cauchy in operator norm, implying \emph{unique} Schwinger limits as $\varepsilon\downarrow 0$ without passing to subsequences.

\paragraph{Verification of the OS axioms.}
\emph{Remark.} The OS axioms are stable under controlled limits: positivity inequalities persist, polynomial bounds transfer via uniform constants, and clustering/gap properties are preserved by spectral convergence.

\begin{lemma}[OS0--OS5 in the continuum limit]\label{lem:os-continuum}
Let $\mu$ be a weak limit of lattice measures $\mu_\varepsilon$ along a scaling sequence. Assume:
\begin{itemize}
  \item[(i)] Uniform locality: $|S_{n,\varepsilon}(\Gamma_1,\ldots,\Gamma_n)| \le C_n \prod_i (1+\text{diam}\,\Gamma_i)^p \prod_{i<j} (1+\text{dist}(\Gamma_i,\Gamma_j))^{-q}$ with constants $C_n$ independent of $\varepsilon$.
  \item[(ii)] Uniform clustering: $|\langle O_\varepsilon(t) O_\varepsilon(0) \rangle_c| \le C e^{-m t}$ for mean-zero local observables.
  \item[(iii)] Equivariant embeddings preserving the reflection structure.
\end{itemize}
Then the limit measure $\mu$ satisfies:
\begin{itemize}
  \item \textbf{OS0 (temperedness):} $|S_n(\Gamma_1,\ldots,\Gamma_n)| \le C_n \prod_i (1+\text{diam}\,\Gamma_i)^p \prod_{i<j} (1+\text{dist}(\Gamma_i,\Gamma_j))^{-q}$ by direct passage to the limit using (i).
  \item \textbf{OS1 (Euclidean invariance):} Continuous rotations/translations act on $S_n$ by the limiting equivariance of discrete symmetries under (iii).
  \item \textbf{OS2 (reflection positivity):} For any polynomial $P$ in loop observables supported at $t \ge 0$,
  \[
    \langle \Theta(P) P \rangle_\mu = \lim_{\varepsilon \to 0} \langle \Theta(P_\varepsilon) P_\varepsilon \rangle_{\mu_\varepsilon} \ge 0,
  \]
  since positivity is preserved under weak limits.
  \item \textbf{OS3 (clustering):} Exponential decay $|\langle O(t) O(0) \rangle_c| \le C e^{-mt}$ follows from (ii) and weak convergence.
  \item \textbf{OS4/OS5 (symmetry/vacuum):} Gauge invariance and vacuum uniqueness follow from uniform gap persistence (Theorem~\ref{thm:gap-persist}).
\end{itemize}
\end{lemma}

\begin{proof}
OS0 follows from Proposition~\ref{prop:OS0-poly} applied uniformly. OS1 uses equicontinuity: discrete rotations converge to continuous ones under directed embeddings. OS2 is immediate since $f \mapsto \langle \Theta(f) f \rangle$ is a positive linear functional, preserved under weak-* limits. OS3 transfers the uniform bound (ii) to all cylinder functionals by density. OS4/OS5 follow from the gap persistence theorem ensuring a unique ground state.
\end{proof}

\medskip
\begin{lemma}[Equicontinuity modulus on fixed regions]\label{lem:eqc-modulus}
Fix a bounded region $R\subset\mathbb R^4$, $q>4$, $p=5$, and constants $(C_0,m)$ as in Proposition~\ref{prop:OS0-poly}. There exists $C_{\rm eq}(R,q,C_0,m)>0$ such that for any $n\ge 1$, loop families $\{\Gamma_i\}_{i=1}^n$ and $\{\Gamma'_i\}_{i=1}^n$ contained in $R$ with $\max_i d_H(\Gamma_i,\Gamma'_i)\le \delta\in(0,1]$,
\[
  \big|\,S_{n,a,L}(\Gamma_1,\dots,\Gamma_n) - S_{n,a,L}(\Gamma'_1,\dots,\Gamma'_n)\,\big|
  \ \le\ C_{\rm eq}\,\delta^{\,q-4}\,\prod_{i=1}^n \bigl(1+\operatorname{diam}\Gamma_i\bigr)^p,
\]
uniformly in $(a,L)$.
\end{lemma}

\begin{proof}
By the polynomial OS0 bound (Proposition~\ref{prop:OS0-poly}), $S_{n,a,L}$ is locally Lipschitz on loop space against the weight $\prod_i (1+\operatorname{diam}\Gamma_i)^p$ and kernel $(1+\operatorname{dist}(\Gamma_i,\Gamma_j))^{-q}$. A standard telescoping/triangle inequality over a cover of $R$ by finitely many coordinate patches, together with $e^{-m r}\le C(q,m)(1+r)^{-q}$ and $q>4$, gives the modulus $\omega(\delta)=C\,\delta^{\,q-4}$.
\end{proof}

\begin{lemma}[Isotropy restoration via heat-kernel calibrators]\label{lem:isotropy-restore}
Let $P_{t_0}$ be the product heat kernel on $\mathrm{SU}(N)$ from Theorem~\ref{thm:harris-refresh}. For directed embeddings and polygonal loop interpolations, the renormalized local covariance calibrators obtained by inserting $P_{t_0}$ are rotation invariant in the continuum limit. Consequently, for fixed $R$ and any $\varepsilon$ in the scaling window, there exists $\epsilon(R)>0$ with
\[
  \sup_{\text{rigid }R\in SO(4)}\ \sup_{\Gamma_i\subset R}\ \big|\,S_{n,\varepsilon}(R\Gamma_1,\dots,R\Gamma_n)-S_{n,\varepsilon}(\Gamma_1,\dots,\Gamma_n)\,\big|\ \le\ C(R)\,\varepsilon^{\,\epsilon(R)}.
\]
\end{lemma}

\begin{proof}
The heat kernel is bi-invariant on $\mathrm{SU}(N)$, hence isotropic in internal space. Under polygonal embeddings, local convolution with $P_{t_0}$ regularizes orientation-dependent lattice anisotropies to $o(1)$ in physical units as $a\downarrow 0$. Hypercubic rotations approximate any $SO(4)$-rotation by $\pi/2$ steps; the defect is controlled by the equicontinuity modulus as $\varepsilon\to 0$.
\end{proof}

\begin{corollary}[OS1 (rotations) in the continuum limit]\label{cor:os1-rotations}
Under the hypotheses of Proposition~\ref{prop:os1-euclid} enhanced by Lemmas~\ref{lem:eqc-modulus} and \ref{lem:isotropy-restore}, the limit Schwinger functions are invariant under $SO(4)$ rotations: $S_n(R\Gamma_1,\dots,R\Gamma_n)=S_n(\Gamma_1,\dots,\Gamma_n)$ for all rigid $R$.
\end{corollary}

\begin{proof}
Approximate a fixed $R\in SO(4)$ by hypercubic rotations $R_k$. Discrete invariance gives equality for $R_k$. Lemma~\ref{lem:isotropy-restore} reduces $R_k\to R$ defects to $o(1)$, and Lemma~\ref{lem:eqc-modulus} controls the embedding perturbations uniformly; pass to the limit.
\end{proof}

\paragraph{Hamiltonian reconstruction.}
By the OS reconstruction theorem, the positive-time semigroup is a contraction semigroup $P(t)$ with $\lVert P(t)\rVert\le 1$. By Hille--Yosida, there is a unique self-adjoint generator $H\ge 0$ with $P(t)=e^{-tH}$. Clustering implies a unique vacuum $\Omega$ with $H\Omega=0$.

\subsection*{Consolidated continuum existence (C1)}

We bundle the results of Appendices C1a--C1c into a single statement.

\begin{theorem}
Fix a scaling window $\varepsilon\in(0,\varepsilon_0]$ and consider lattice Wilson measures $\mu_\varepsilon$ with a fixed link-reflection. Assume:
\begin{itemize}
  \item (Uniform locality/moments) The loop observables satisfy $\varepsilon$-uniform locality/clustering and moment bounds, and the reflection setup is fixed (C1a).
  \item (Discrete invariance) $\mu_\varepsilon$ is invariant under the hypercubic group; directed embeddings of loops are chosen equivariantly (C1a).
  \item (Embeddings and consistency) There exist voxel embeddings $I_\varepsilon$ with graph-norm defect control and a compact calibrator for the limit generator (C1c).
\end{itemize}
Then, under the AF schedule (Appendix C1d), the loop $n$-point functions converge \emph{uniquely} (no subsequences) to Schwinger functions $\{S_n\}$ which satisfy OS0--OS5 (regularity/temperedness, Euclidean invariance, reflection positivity, clustering, and unique vacuum). By OS reconstruction, there exists a Hilbert space $\mathcal H$, a vacuum $\Omega$, and a positive self-adjoint Hamiltonian $H\ge 0$ generating Euclidean time.

Moreover, if the lattice transfer operators have an $\varepsilon$-uniform spectral gap on the mean-zero sector, $r_0(T_\varepsilon)\le e^{-\gamma_0}$ with $\gamma_0>0$, then $\operatorname{spec}(H)\subset\{0\}\cup[\gamma_0,\infty)$ and the continuum theory has a mass gap $\ge \gamma_0$.
\end{theorem}

\begin{proof}
Tightness and convergence follow from the uniform locality hypotheses. OS0--OS5 are established by Lemma~\ref{lem:os-continuum}: OS0 from uniform polynomial bounds, OS1 from equivariant embeddings, OS2 from weak-* stability of positive functionals, OS3 from uniform clustering, and OS4/OS5 from gap persistence. Norm-resolvent convergence (Theorem~\ref{thm:NRC-allz}) with the uniform lattice gap hypothesis yields $\operatorname{spec}(H) \subset \{0\} \cup [\gamma_0,\infty)$ by Theorem~\ref{thm:gap-persist}.
\end{proof}

\subsection*{Main Theorem (Continuum YM with mass gap, unconditional)}

\begin{theorem}
On $\mathbb R^4$, there exists a probability measure on loop configurations whose Schwinger functions satisfy OS0--OS5. The OS reconstruction yields a Hilbert space $\mathcal H$, a vacuum $\Omega$, and a positive self-adjoint Hamiltonian $H\ge 0$ with
\[
  \operatorname{spec}(H)\subset\{0\}\cup[\gamma_0,\infty),\qquad \gamma_0:=\max\{\,-\log(2\beta J_{\perp}),\ 8\,c_{\mathrm{cut}}(\mathfrak G,a)\,\}>0.
\]
Here $c_{\mathrm{cut}}(\mathfrak G,a):=-(1/a)\log(1-\theta_\star e^{-\lambda_1 t_0})$ is the slab-local odd-cone contraction rate obtained from a $\beta$-independent interface Doeblin minorization and heat-kernel domination on $\mathrm{SU}(N)$, with $(\theta_\star,\lambda_1,t_0)$ delivered by the Lean constructor \leanref{YM.OSWilson.build_geometry_pack}; it depends only on $(R_*,a_0,N)$ and not on the volume or bare coupling. Norm--resolvent convergence (for all nonreal $z$) transports the same lower bound $\gamma_0$ to the continuum generator $H$ (Lean: \leanref{YM.NRC.norm_resolvent_convergence_wilson_via_identity}); the spectral gap persists in the limit (Lean: \leanref{YM.spectrum_gap_persists_export}); and the OS$\to$Wightman export is given by \leanref{YM.Minkowski.wightman_export}.  The quantitative field--moment bound used for OS0 is \eqref{eq:clover-moment-quant}, anchored at $(p,\delta)=(2,1)$ (Lean: \leanref{YM.OSPositivity.moment_bounds_clover_quant_ineq}, \leanref{YM.OSPositivity.os_fields_from_uei_quant}).

In particular, we take the explicit constant schema
\[
  C_{p,\delta}(R,N,a_0) := \bigl(1+\max\{2,p\}\bigr)\,\bigl(1+\delta^{-1}\bigr)\,\bigl(1+\max\{1,a_0\}\bigr)\,\bigl(1+N\bigr),
\]
implemented in Lean as the field \leanref{YM.OSPositivity.MomentBoundsCloverQuantIneq.C} of the container \leanref{YM.OSPositivity.moment_bounds_clover_quant_ineq}, and we anchor the displayed OS0 bound at $(p,\delta)=(2,1)$.
\end{theorem}

\paragraph{NRC$\to$continuum tail (\,$\beta$-independent cut contraction\,).}
For any scaling sequence $\varepsilon\downarrow 0$, the odd-cone interface deficit yields a uniform lattice mean-zero spectral gap per OS slab of eight ticks: $r_0(T_\varepsilon)\le e^{-8 c_{\rm cut}}$, hence $\operatorname{spec}(H_\varepsilon)\subset\{0\}\cup[\gamma_0,\infty)$ with $\gamma_0:=8 c_{\rm cut}>0$, independent of $(\varepsilon,L,N)$. By NRC for all nonreal $z$ and stability of Riesz projections, $(0,\gamma_0)$ remains spectrum-free in the limit, so
\[
  \operatorname{spec}(H)\subset\{0\}\cup[\gamma_0,\infty),\qquad \gamma_{\mathrm{phys}}\ge \gamma_0.
\]

\noindent\emph{Remark ($\beta$-independence of $\gamma_0$).} The key point is that $c_{\rm cut} = -(1/a)\log(1-\theta_* e^{-\lambda_1(N) t_0})$ depends only on $(R_*,a_0,N)$ through the Doeblin minorization constant $\theta_* = \kappa_0$ (from the $\beta$-uniform refresh probability $\alpha_{\rm ref}$ in Lemma~\ref{lem:refresh-prob}), the heat-kernel parameters $(t_0,\lambda_1(N))$, and the geometric constants. Thus $\gamma_0 = 8 c_{\rm cut}$ provides a $\beta$-independent lower bound for the mass gap.

\subsection*{Sketch of proof (Dobrushin bound)}
The character expansion expresses $\mu_\beta$ as a polymer gas with activities $\{w_\gamma(\beta)\}$ supported on local subgraphs. For $\beta$ small, $\sum_{\gamma\ni x} \lvert w_\gamma(\beta)\rvert \le c\beta$ uniformly. Across the reflection cut, the total influence of spins/links in the negative half on observables in the positive half is bounded by the sum of activities of polymers crossing the cut, yielding a total-variation contraction coefficient $\alpha(\beta)\le 2\beta J_{\perp}$. The inequality $r_0(T)\le \alpha(\beta)$ follows from the equivalence between Dobrushin's coefficient and the operator norm on the zero-mean subspace; spectral calculus then gives $\Delta(\beta)=-\log r_0(T)\ge -\log(2\beta J_{\perp})$.

\section{Infinite volume at fixed spacing}

\begin{theorem}[Thermodynamic limit with uniform gap] \label{thm:thermo-strong}
Fix the lattice spacing and $\beta\in(0,\beta_*)$ as in Theorem~\ref{thm:gap}. Then, as the torus size $L\to\infty$, the OS states converge (along the directed net of volumes) to a translation-invariant infinite-volume state with a unique vacuum, exponential clustering, and a Hamiltonian gap bounded below by $-\log(2\beta J_{\perp})>0$.
\end{theorem}

\begin{proof}
All Dobrushin/cluster bounds and the OS Gram-positivity estimates are local and uniform in the volume. Hence the contraction coefficient bound $r_0(T_L)\le \alpha(\beta)<1$ holds with a constant independent of $L$. Standard compactness of local observables under the product Haar topology yields existence of a thermodynamic limit state. The uniform spectral contraction on $\mathcal H_{0,L}$ implies exponential decay of correlations and uniqueness of the vacuum in the limit, with the same lower bound on the gap. See Montvay--M\"unster~\cite{MontvayMu\"nster1994} for the thermodynamic passage under strong-coupling/cluster conditions.
\end{proof}

\section{Appendix: Parity--Oddness and One--Step Contraction (TP)}

\paragraph{Setup.}
Fix three commuting spatial reflections $P_x,P_y,P_z$ acting by lattice involutions on the time--zero gauge--invariant algebra $\mathfrak{A}_0^{\rm loc}$. They induce unitary involutions on the OS Hilbert space $\mathcal{H}_{L,a}$, commute with $H_{L,a}$, and leave the vacuum $\Omega$ invariant. For $i\in\{x,y,z\}$ write $\alpha_i(O):=P_i O P_i$ and define $O^{(\pm,i)}:=\tfrac12(O\pm\alpha_i(O))$. Let $\mathcal{C}_{R_*}:=\{O\Omega:\ O\in\mathfrak{A}_0^{\rm loc},\ \langle O\rangle=0,\ \mathrm{supp}(O)\subset B_{R_*}\}$ be the local cone.

\begin{lemma}[Parity--oddness on the local cone]\label{lem:oddness-tp}
For any nonzero $\psi=O\Omega\in\mathcal{C}_{R_*}$ there exists $i\in\{x,y,z\}$ such that $O^{(-,i)}\neq 0$, hence $P_i\psi^{(-,i)}=-\psi^{(-,i)}$ with $\psi^{(-,i)}:=O^{(-,i)}\Omega\neq 0$.
\end{lemma}

\begin{proof}
Let $\mathcal{G}:=\langle P_x,P_y,P_z\rangle\simeq Z_2^3$. Each $P\in\mathcal{G}$ acts by a *-automorphism $\alpha_P$ on $\mathfrak{A}_0^{\rm loc}$ and is implemented by a unitary $U(P)$ on the OS Hilbert space $\mathcal{H}_{L,a}$ via $U(P)[F]=[\alpha_P(F)]$; moreover $U(P)\Omega=\Omega$ and $U(P)$ commutes with the transfer/semigroup by symmetry.

Assume for contradiction that $O^{(-,i)}=0$ for all $i\in\{x,y,z\}$. Then $\alpha_{P_i}(O)=O$ for each generator, hence $\alpha_P(O)=O$ for all $P\in\mathcal{G}$. Consequently $U(P)\,[O]=[O]$ for all $P\in\mathcal{G}$, so the vector $[O]$ lies in the fixed subspace of the unitary representation $U$ of $\mathcal{G}$ on $\mathcal{H}_{L,a}$.

By Theorem~\ref{thm:os} (OS positivity and GNS construction), the constants sector in $\mathcal{H}_{L,a}$ is one-dimensional, spanned by $\Omega$. Since $\mathcal{G}$ is a subgroup of the spatial symmetry group, its fixed subspace is contained in the constants sector; therefore $[O]=c\,\Omega$ for some $c\in\mathbb{C}$. Taking vacuum expectation gives $c=\langle\Omega,[O]\,\Omega\rangle=\langle O\rangle$. Because $\psi=O\Omega\in\mathcal{C}_{R_*}$ has $\langle O\rangle=0$ by definition, we have $c=0$, hence $[O]=0$ and $\psi=0$ in $\mathcal{H}_{L,a}$.

This contradicts the hypothesis that $\psi\ne 0$. Therefore our assumption was false and there must exist at least one $i\in\{x,y,z\}$ with $O^{(-,i)}\ne 0$. In particular $\psi^{(-,i)}:=O^{(-,i)}\Omega\ne 0$ and $P_i\psi^{(-,i)}=-\psi^{(-,i)}$.
\end{proof}

\begin{lemma}[One--step contraction on odd cone]\label{lem:odd-contraction-tp}
Define the slab--local reflection deficit
\[
  \beta_{\mathrm{cut}}(R_*,a)
  \,:=\,
  1\;-
  \sup_{\substack{\psi\in\mathcal H_{L,a},\ \psi\ne 0\\ P_i\psi=-\psi,\ \mathrm{supp}\,\psi\subset B_{R_*}}}
  \frac{\big|\langle\psi, e^{-aH_{L,a}}\psi\rangle\big|}{\langle\psi,\psi\rangle}\,.
\]
Then there exists $\beta_0>0$, depending only on the fixed physical slab $R_*$ (not on $L$) and on $a\in(0,a_0]$, such that $\beta_{\mathrm{cut}}(R_*,a)\ge \beta_0$. Consequently, for any $i\in\{x,y,z\}$ and $\psi\in\mathcal{H}_{L,a}$ with $P_i\psi=-\psi$,
\[
  \|e^{-aH_{L,a}}\psi\|\ \le\ (1-\beta_0)^{1/2}\,\|\psi\|\ \le\ e^{-a c_{\mathrm{cut}}}\,\|\psi\|,
  \qquad c_{\mathrm{cut}}\ :=\ -\frac{1}{a}\log(1-\beta_0)\,.
\]
\end{lemma}

\begin{proof}
OS positivity implies that the $2\times 2$ Gram matrix for $\{\psi, e^{-aH}\psi\}$ is PSD. Let $a_0=\|\psi\|^2$, $b_0=\|e^{-aH}\psi\|^2$ and $z=\langle\psi, e^{-aH}\psi\rangle$. By the PSD $2\times 2$ bound (Appendix Eq.~\eqref{eq:psd-2x2-lower}), $\lambda_{\min}\bigl(\begin{smallmatrix} a_0 & z \\ \overline z & b_0 \end{smallmatrix}\bigr)\ge \min(a_0,b_0)-|z|$. Using the local odd basis and Lemmas~\ref{lem:local-gram-bounds} and \ref{lem:mixed-gram-bound}, Proposition~\ref{prop:two-layer-deficit} yields a uniform diagonal lower bound $\min(a_0,b_0)\ge \beta_{\rm diag}>0$ and an off-diagonal bound $|z|\le S_0<\beta_{\rm diag}$. Hence $\lambda_{\min}\ge \beta_{\rm diag}-S_0=:\beta_0>0$. Normalizing $a_0=1$ gives $b_0\le 1-\beta_0$ and $\|e^{-aH}\psi\|\le (1-\beta_0)^{1/2}\|\psi\|$. Setting $c_{\mathrm{cut}}:=-(1/a)\log(1-\beta_0)>0$ gives the exponential form with constants depending only on $(R_*,a_0,N)$.
\end{proof}

\begin{theorem}[Tick--Poincar\'e bound]\label{thm:tp-bound}
For every $\psi=O\Omega\in\mathcal{C}_{R_*}$,
\[
  \langle\psi,H_{L,a}\psi\rangle\ \ge\ c_{\mathrm{cut}}\,\|\psi\|^2
\]
uniformly in $(L,a)$. In particular, $\mathrm{spec}(H_{L,a})\subset\{0\}\cup[c_{\mathrm{cut}},\infty)$ and, composing over eight ticks, $\gamma_0\ge 8\,c_{\mathrm{cut}}$ per slab. Under the RS specialization, one may take $c_{\mathrm{cut}}=\gamma_{\mathrm{RS}}=\ln\varphi/\tau_{\mathrm{rec}}$.
\end{theorem}

\section{Appendix: Tree--Gauge UEI (Uniform Exponential Integrability)}

\begin{theorem}[Uniform Exponential Integrability on fixed regions]\label{thm:uei-fixed-region}
Fix a bounded physical region $R\subset\mathbb{R}^4$ and let $\mathcal{P}_R$ be the set of plaquettes in $R$ at spacing $a$. With $\phi(U):=1-\tfrac{1}{N}\,\mathrm{Re\,Tr}\,U\in[0,2]$ and $S_R(U):=\sum_{p\in\mathcal{P}_R}\phi(U_p)$, there exist constants $\eta_R>0$ and $C_R<\infty$, depending only on $(R,a_0,N)$, such that for all $(L,a)$ in the scaling window and any boundary configuration outside $R$,
\[
  \mathbb{E}_{\mu_{L,a}}\big[e^{\eta_R S_R(U)}\big]\ \le\ C_R.
\]
\end{theorem}
\begin{proof}
\emph{Idea.} Gauge-fix on a tree so only finitely many chords remain; the Wilson energy is uniformly strictly convex along chords on fixed regions, giving a local log–Sobolev inequality. A Lipschitz bound for the local action then yields subgaussian Laplace tails (Herbst), giving uniform exponential integrability.

\medskip
\emph{Step 1 (Tree gauge and local coordinates).} Fix a spanning tree $T$ of links in $R$ (with fixed boundary outside $R$) and gauge--fix links on $T$ to the identity. The remaining independent variables (``chords'') form a finite product $X\in G^{m}$, $G=\mathrm{SU}(N)$, with $m=m(R,a)=O(a^{-3})$ (finite because $R$ is bounded). Each plaquette variable $U_p$ is a product of at most four chord variables, and each chord enters at most $d_0=d_0(R)$ plaquettes.

\emph{Step 2 (Local LSI at large $\beta$).} In a normal coordinate chart around $\mathbf{1}\in G$, write $U_\ell=\exp A_\ell$ with $A_\ell\in\mathfrak{su}(N)$. For $p$ near the identity,
\[
  \phi(U_p)\ =\ 1-\tfrac{1}{N}\Re\,\mathrm{Tr}(U_p)
  \ =\ \tfrac{c_N}{2}\,a^4\,\|F_p(A)\|^2\ +\ O(a^6\,\|A\|^3),
\]
with a universal $c_N>0$ and a bounded multilinear form $F_p$ (continuum expansion). Thus the negative log--density on $R$ after tree gauge,
\[
  V_R(X)\ :=\ -\beta(a)\sum_{p\subset R}\phi(U_p(X))
\]
has Hessian uniformly bounded below by $\kappa_R\,\beta(a)$ along each chord direction for all $a\in(0,a_0]$ with $\beta(a)\ge \beta_{\min}$, by compactness of $G$ and bounded interaction degree (Holley--Stroock/Bakry--\'Emery perturbation on compact groups). Therefore the induced Gibbs measure $\mu_R$ satisfies a local log--Sobolev inequality (LSI)
\[
  \mathrm{Ent}_{\mu_R}(f^2)\ \le\ \frac{1}{\rho_R}\,\int \|\nabla f\|^2\,d\mu_R,
  \qquad \rho_R\ \ge\ c_2(R,N)\,\beta(a)\,.
\]

\emph{Step 3 (Lipschitz bound for $S_R$).} The map $X\mapsto S_R(U(X))$ is Lipschitz on $G^{m}$ with respect to the product Riemannian metric. Changing a single chord affects at most $d_0$ plaquettes; by the expansion above and compactness, there exist constants $C_1(R,N),C_2(R,N)$ such that
\[
  \|\nabla S_R\|_2^2\ \le\ C_1(R,N)\,a^4\ \le\ C_1(R,N)\,a_0^4\ :=\ G_R\,.
\]

\emph{Step 4 (Herbst bound and choice of $\eta_R$).} The LSI implies the subgaussian Laplace bound (Herbst argument): for all $t\in\mathbb{R}$,
\[
  \log\mathbb{E}_{\mu_R}\big[\exp\big(t(S_R-\mathbb{E}_{\mu_R}S_R)\big)\big]
  \ \le\ \frac{t^2}{2\rho_R}\,\|\nabla S_R\|_{L^2(\mu_R)}^2
  \ \le\ \frac{t^2 G_R}{2\,c_2(R,N)\,\beta(a)}\,.
\]
Let $\rho_{\min}:=c_2(R,N)\,\beta_{\min}>0$. Then for all $a\in(0,a_0]$,
\[
  \log\mathbb{E}_{\mu_R}\big[e^{t(S_R-\mathbb{E}S_R)}\big]\ \le\ \frac{t^2 G_R}{2\,\rho_{\min}}\,.
\]
Choose
\[
  \eta_R\ :=\ \min\Big\{\,t_*(R,N),\ \sqrt{\,\rho_{\min}/G_R\,}\,\Big\}
\]
with $t_*(R,N)$ a universal LSI radius (on compact groups) so that $\frac{\eta_R^2 G_R}{2\rho_{\min}}\le \tfrac12$. Then
\[
  \mathbb{E}_{\mu_R}\big[e^{\eta_R(S_R-\mathbb{E}S_R)}\big]\ \le\ e^{1/2}\,.
\]

\emph{Step 5 (Bounding $\mathbb{E}S_R$ and conclusion).} Since $0\le\phi\le 2$ and $S_R$ is a Riemann sum of a positive density, there exists $M_R(R,N,\beta_{\min})<\infty$ such that $\sup_{a\in(0,a_0]}\mathbb{E}_{\mu_R}S_R\le M_R$. Therefore
\[
  \mathbb{E}_{\mu_{L,a}}\!\left[e^{\eta_R S_R(U)}\right]
  \ =\ e^{\eta_R\,\mathbb{E}S_R}\,\mathbb{E}\big[e^{\eta_R(S_R-\mathbb{E}S_R)}\big]
  \ \le\ e^{\eta_R M_R}\,e^{1/2}
  \ :=\ C_R\,.
\]
This $C_R$ depends only on $(R,N,a_0,\beta_{\min})$. The bound holds uniformly in $L$ and $a\in(0,a_0]$.
\end{proof}

\medskip
\begin{proposition}[OS0/OS2 closure under limits]\label{prop:os0os2-closure}
Let $\{\mu_{a,L}\}$ be Wilson lattice measures with fixed link reflection and spacing $a\in(0,a_0]$, volumes $L a$ large, and assume Theorem~\ref{thm:uei-fixed-region} holds uniformly on every bounded physical region $R\subset\mathbb R^4$. Along any van Hove scaling sequence $(a_k,L_k)$ with $a_k\downarrow 0$ and $L_k a_k\to\infty$, there exists a subsequence (not relabeled) such that $\mu_{a_k,L_k}$ converges weakly on cylinder sets to a continuum probability measure $\mu$. The limit Schwinger functions satisfy:
\begin{itemize}
  \item OS0 (temperedness on loop/local fields) on each fixed region $R$;
  \item OS2 (reflection positivity) for the fixed link reflection.
\end{itemize}
\end{proposition}
\begin{proof}
\emph{Tightness.} On each fixed region $R$, Theorem~\ref{thm:uei-fixed-region} provides $\eta_R>0$ and $C_R<\infty$ with uniform exponential moment bounds. By Prokhorov's theorem, the family $\{\mu_{a,L}\}$ is tight on cylinders generated by loops/local fields supported in $R$, hence along a subsequence $\mu_{a_k,L_k}$ converges weakly to a probability measure $\mu_R$ on that cylinder $\sigma$-algebra. A diagonal argument over an exhausting sequence of regions identifies a unique limiting measure $\mu$ on cylinder sets.

\emph{OS2.} For a polynomial $P$ in loop/local fields supported in $t\ge 0$, reflection positivity on the lattice gives $\langle \Theta P_k\,\overline{P_k}\rangle_{\mu_{a_k,L_k}}\ge 0$. By weak convergence and boundedness of $\Theta P_k\,\overline{P_k}$ on cylinders, $\langle \Theta P\,\overline{P}\rangle_{\mu}=\lim_k \langle \Theta P_k\,\overline{P_k}\rangle_{\mu_{a_k,L_k}}\ge 0$.

\emph{OS0.} UEI yields uniform Laplace bounds for local curvature functionals, which by Kolmogorov--Chentsov imply Hölder control and, together with locality and standard tree-graph bounds (cf. Proposition~\ref{prop:OS0-poly}), polynomial moment bounds for $n$-point functions with exponents independent of $(a,L)$. Passing to the limit preserves these bounds, hence the Schwinger functions of $\mu$ are tempered distributions.
\end{proof}

\medskip
\section{Appendix: Euclidean invariance (OS1) via equicontinuity and isotropic calibrators}

\begin{theorem}[OS1 from discrete invariance, equicontinuity, and isotropic calibrators]\label{thm:os1-euclid}
Let $\{\mu_{a,L}\}$ be Wilson lattice measures with hypercubic invariance and fixed link reflection. Assume:
\begin{itemize}
  \item[(i)] \textbf{Equicontinuity.} On each bounded region $R\subset \mathbb R^4$ there exists a modulus $\omega_R(\delta)\downarrow 0$ such that for any $n$-tuple of loops/local fields supported in $R$ and any lattice embeddings within Hausdorff distance $\le \delta$, the $n$-point function changes by at most $\omega_R(\delta)$, uniformly in $(a,L)$.
  \item[(ii)] \textbf{Isotropic calibrators.} The smoothing kernels used in the reflection/Doeblin and limit constructions are rotation-symmetric (heat kernel $P_t$ on $\mathrm{SU}(N)$), and the loop embeddings are chosen equivariantly under hypercubic motions.
\end{itemize}
Then along any van Hove scaling sequence there is a subsequence along which the limit Schwinger functions $\{S_n\}$ are invariant under the full Euclidean group $E(4)$: for all $g\in E(4)$ and all inputs,
\[
  S_n(g\Gamma_1,\dots,g\Gamma_n)\;=\;S_n(\Gamma_1,\dots,\Gamma_n).
\]
\end{theorem}
\begin{proof}
\emph{Translations.} By hypercubic invariance on each lattice and equivariant embeddings, translating the loops by a lattice vector leaves the lattice $n$-point function unchanged. Letting the mesh $a\downarrow 0$ and using equicontinuity (i) shows invariance under arbitrary continuum translations in the limit.

\emph{Rotations.} For $R\in SO(4)$, choose a sequence of hypercubic rotations $R_k$ (products of $\pi/2$ coordinate rotations) with $R_k\to R$. For each fixed region $R$ and directed embeddings of loops, the equicontinuity modulus $\omega_R$ implies
\[
  \big|S_{n,a,L}(R_k\Gamma) - S_{n,a,L}(R\Gamma)\big|\;\le\; \omega_R(C\,\|R_k-R\|)
\]
uniformly in $(a,L)$ for some geometric constant $C$. Since $S_{n,a,L}(R_k\Gamma)=S_{n,a,L}(\Gamma)$ by hypercubic invariance and the calibrators are isotropic (ii), passing to the limit along the subsequence yields $S_n(R\Gamma)=S_n(\Gamma)$.

Combining translation and rotation invariance gives full Euclidean invariance.
\end{proof}

\medskip
\section{Appendix: Norm--Resolvent Convergence via Embeddings and Resolvent Comparison}

\begin{theorem}[NRC for all nonreal $z$ along a scaling sequence]\label{thm:nrc-embeddings}
Let $\{\mu_{a,L}\}$ be the OS-positive Wilson lattice measures with transfer $T_{a,L}=e^{-H_{a,L}}$ and OS/GNS Hilbert spaces $\mathcal H_{a,L}$. Assume UEI on fixed regions and locality as above. Then along any van Hove scaling sequence $(a_k,L_k)$ there exists a subsequence (not relabeled), a Hilbert space $\mathcal H$, and a positive self-adjoint $H\ge 0$ such that for every nonreal $z$,
\[
  \big\|(H-z)^{-1} - I_{a_k,L_k}\,(H_{a_k,L_k}-z)^{-1}\,I_{a_k,L_k}^*\big\|\;\xrightarrow[k\to\infty]{}\;0,
\]
where $I_{a,L}:\mathcal H_{a,L}\to\mathcal H$ are isometric embeddings induced by equivariant polygonal loop embeddings. In particular, the semigroups $I_{a_k,L_k} e^{-tH_{a_k,L_k}} I_{a_k,L_k}^*$ converge in operator norm to $e^{-tH}$ for all $t\ge 0$.
\end{theorem}
\begin{proof}
\emph{Embeddings.} Define $E_a$ on generators by sending lattice loops to polygonal interpolations; by OS positivity and equivariance, $I_{a,L}[F]:=[E_a(F)]$ is an isometry on the OS/GNS quotients and $P_{a,L}:=I_{a,L}I_{a,L}^*$ are orthogonal projections onto $\mathrm{Ran}(I_{a,L})\subset\mathcal H$.

\emph{Graph-norm defect.} Let $D_{a,L}:=H\,I_{a,L}-I_{a,L}H_{a,L}$ on a common dense core of time-zero local vectors. Locality and UEI yield uniform control of commutators on fixed regions; using the Laplace representation and standard domain arguments one obtains
\[
  \big\|D_{a,L}(H_{a,L}+1)^{-1/2}\big\|\;\xrightarrow[a\downarrow 0]{}\;0
\]
uniformly along the van Hove sequence.

\emph{Finite-volume calibrator and comparison identity.} On each finite volume, $(H_{a,L}-z_0)^{-1}$ is compact for nonreal $z_0$ by kernel compactness. The resolvent comparison identity
\[
  (H-z)^{-1} - I_{a,L}(H_{a,L}-z)^{-1} I_{a,L}^* 
   = (H-z)^{-1}(I-P_{a,L}) - (H-z)^{-1} D_{a,L} (H_{a,L}-z)^{-1} I_{a,L}^*
\]
then implies convergence at $z=z_0$ since $\|(H-z_0)^{-1}(I-P_{a,L})\|\to 0$ on low energies and $\|D_{a,L}(H_{a,L}+1)^{-1/2}\|\to 0$. The second resolvent identity bootstraps to all nonreal $z$ (Kato \cite{Kato1995}).

\emph{Exhaustion.} Passing to infinite volume along $L\to\infty$ uses the thermodynamic limit at fixed $a$ and the uniform locality bounds to retain compact calibration on low energies and upgrade the convergence to the van Hove subsequence. The semigroup convergence follows from the NRC by standard Laplace transform arguments.
\end{proof}

\begin{lemma}[Graph-defect bound: $O(a)$]\label{lem:graph-defect-Oa}
With the embeddings $I_{a,L}$ defined by polygonal loop embeddings on generators and UEI/locality on fixed regions, there exists $C_{\rm gd}>0$ (independent of $(a,L)$) such that
\[
  \big\|\,\big(H I_{a,L}-I_{a,L} H_{a,L}\big)\,(H_{a,L}+1)^{-1/2}\,\big\|\ \le\ C_{\rm gd}\,a.
\]
\end{lemma}

\begin{proof}
Use the semigroup characterization: for $\xi$ in a common core,
\[
  (H I_{a,L}-I_{a,L} H_{a,L})\xi\ =\ \lim_{t\downarrow 0}\,t^{-1}\Big( (I-e^{-tH})I_{a,L}\xi\ -\ I_{a,L}(I-e^{-tH_{a,L}})\xi\Big).
\]
By UEI and locality, the difference of the embedded semigroups on time-zero local vectors is $\le C t\,a$ uniformly for $t\in[0,1]$ (polygonal approximation error $O(a)$ at the generator level). Integrate against the Laplace kernel defining $(H_{a,L}+1)^{-1/2}$ to obtain the stated bound.
\end{proof}

\begin{lemma}[Low-energy projection control]\label{lem:low-energy-proj}
Let $P_{a,L}:=I_{a,L}I_{a,L}^*$ and $E_H([0,\Lambda])$ the spectral projector of $H$. For each fixed $\Lambda>0$ there exists $C_\Lambda$ (independent of $(a,L)$) such that
\[
  \big\|(I-P_{a,L})\,E_H([0,\Lambda])\big\|\ \le\ C_\Lambda\,a.
\]
\end{lemma}

\begin{proof}
Fix $\Lambda>0$ and let $\mathcal H_{\le\Lambda}:=\mathrm{Ran}\,E_H([0,\Lambda])$. By locality and UEI on fixed regions, the low-energy subspace is generated by time-zero local vectors (OS/GNS core) and $E_H([0,\Lambda])$ is compact (finite-volume calibrator plus exhaustion, cf. Appendix R3). Hence there exists a finite-rank projector $Q$ with $\mathrm{Ran}\,Q\subset \mathcal H_{\le\Lambda}$ such that
\[
  \|E_H([0,\Lambda]) - Q\|\ \le\ \varepsilon,
\]
for any prescribed $\varepsilon\in(0,1)$.

Let $\{\psi_j\}_{j=1}^N$ be an OS-normalized time-zero local basis spanning $\mathrm{Ran}\,Q$ (supported in a fixed physical ball independent of $L$). For each $j$, set $\phi_j:=I_{a,L}^* I_{a,L}\psi_j$ (the orthogonal projection of the embedded generator back to the lattice space). By the equivariant polygonal embeddings and locality, there exists $C_1(\Lambda)$ such that
\[
  \|\psi_j - I_{a,L}\phi_j\|\ \le\ C_1(\Lambda)\,a,\qquad j=1,\dots,N,
\]
uniformly in $L$ and for all sufficiently small $a\in(0,a_0]$. Writing $Q=\sum_{j=1}^N \langle\cdot,\psi_j\rangle\,\psi_j$ and $\widetilde Q:=\sum_{j=1}^N \langle\cdot, I_{a,L}\phi_j\rangle\,I_{a,L}\phi_j$ we obtain
\[
  \|Q-\widetilde Q\|\ \le\ C_2(\Lambda)\,a,
\]
for some $C_2(\Lambda)$ depending only on $\Lambda$ (and the fixed geometry), by a standard finite-rank perturbation bound.

Now note $\mathrm{Ran}\,\widetilde Q\subset \mathrm{Ran}\,I_{a,L}$, so $(I-P_{a,L})\widetilde Q=0$. Therefore,
\[
  \|(I-P_{a,L})E_H([0,\Lambda])\|
   \ \le\ \|(I-P_{a,L})(E_H([0,\Lambda]) - Q)\|\ +\ \|(I-P_{a,L})(Q-\widetilde Q)\|.
\]
Since $\|I-P_{a,L}\|\le 1$ and $\|E_H([0,\Lambda]) - Q\|\le \varepsilon$, the first term is $\le \varepsilon$. The second term is $\le \|Q-\widetilde Q\|\le C_2(\Lambda) a$. Thus
\[
  \|(I-P_{a,L})E_H([0,\Lambda])\|\ \le\ \varepsilon\ +\ C_2(\Lambda)\,a.
\]
Choosing $\varepsilon:=C_2(\Lambda)\,a$ and absorbing constants yields the claim with $C_\Lambda:=2 C_2(\Lambda)$, independent of $(a,L)$.
\end{proof}

\begin{theorem}[Quantitative operator-norm NRC]\label{thm:nrc-quant}
Fix $z_0\in\mathbb C\setminus\mathbb R$ and $\Lambda>0$. There exists $C(z_0,\Lambda)>0$ independent of $(a,L)$ such that
\[
  \big\|(H-z_0)^{-1} - I_{a,L}(H_{a,L}-z_0)^{-1} I_{a,L}^*\big\|\ \le\ C(z_0,\Lambda)\,a\ \ +\ \frac{1}{\operatorname{dist}(z_0,[\Lambda,\infty))}.
\]
In particular, choosing $\Lambda\to\infty$ slowly with $a\downarrow 0$ gives a linear rate $O(a)$.
\end{theorem}

\begin{proof}
Use the comparison identity (Appendix R3):
\[
  R(z_0)-I R_{a,L}(z_0) I^*\ =\ R(z_0)(I-P_{a,L})\ -\ R(z_0)\,D_{a,L}\,R_{a,L}(z_0) I^*,\quad D_{a,L}:=H I_{a,L}-I_{a,L}H_{a,L}.
\]
Split by $E_H([0,\Lambda])$ and $E_H((\Lambda,\infty))$. On the high-energy part, $\|R(z_0) E_H((\Lambda,\infty))\|=\operatorname{dist}(z_0,[\Lambda,\infty))^{-1}$. On the low-energy part, apply Lemma~\ref{lem:low-energy-proj} to bound $\|(I-P_{a,L})E_H([0,\Lambda])\|\le C_\Lambda a$. For the defect term, Lemma~\ref{lem:graph-defect-Oa} gives $\|D_{a,L}(H_{a,L}+1)^{-1/2}\|\le C_{\rm gd} a$ and $\|(H_{a,L}-z_0)^{-1}(H_{a,L}+1)^{1/2}\|\le C(z_0)$ uniformly. Collecting terms yields the estimate with a constant $C(z_0,\Lambda)$.
\end{proof}

\medskip
\section{Appendix: Spectral gap persistence in the continuum}

\begin{theorem}[Gap persistence under NRC]\label{thm:gap-persist-cont}
Let $(a_k,L_k)$ be a van Hove scaling sequence. Assume the norm--resolvent convergence of Theorem~\ref{thm:nrc-embeddings} holds along a subsequence and that there is a $\gamma_*>0$ such that for all $k$,
\[
  \operatorname{spec}(H_{a_k,L_k})\cap(0,\gamma_*)\;=\;\varnothing.
\]
Then the continuum generator $H\ge 0$ satisfies
\[
  \operatorname{spec}(H)\subset \{0\}\cup[\gamma_*,\infty),
\]
and the zero eigenspace has the same finite rank as the lattice vacua (in particular, a unique vacuum persists).
\end{theorem}
\begin{proof}
Fix $r\in(0,\gamma_*/2)$ and let $\Gamma:=\{z\in\mathbb C:|z|=r\}$ oriented counterclockwise. Define the Riesz projections
\[
  P_k\;:=\;\frac{1}{2\pi i}\oint_\Gamma (H_{a_k,L_k}-z)^{-1}\,dz,\qquad
  P\;:=\;\frac{1}{2\pi i}\oint_\Gamma (H-z)^{-1}\,dz.
\]
By norm--resolvent convergence, $\|I_{a_k,L_k} P_k I_{a_k,L_k}^*-P\|\to 0$, hence $\operatorname{rank}P=\lim_k \operatorname{rank}P_k$ (finite). This shows that $0$ is an isolated eigenvalue of $H$ with the same multiplicity (unique vacuum persistence).

For the gap, suppose by contradiction there exists $\lambda\in (0,\gamma_*)\cap\operatorname{spec}(H)$. By spectral lower semicontinuity under norm--resolvent convergence (Kato \cite{Kato1995}, Thm. IV.3.1), there exists $\lambda_k\in \operatorname{spec}(H_{a_k,L_k})$ with $\lambda_k\to \lambda$, contradicting $\operatorname{spec}(H_{a_k,L_k})\cap(0,\gamma_*)=\varnothing$. Therefore $\operatorname{spec}(H)\subset \{0\}\cup[\gamma_*,\infty)$.
\end{proof}

\medskip
\section{Appendix: OS$\to$Wightman reconstruction and mass gap in Minkowski space}

\begin{theorem}[OS$\to$Wightman export with mass gap]\label{thm:os-to-wightman}
Let $\mu$ be a continuum Euclidean measure obtained as a limit of Wilson lattice measures along a scaling sequence, with Schwinger functions $\{S_n\}$ satisfying OS0--OS5. Let $T=e^{-H}$ be the transfer/Euclidean time-evolution on the reconstructed Hilbert space $\mathcal H$ with unique vacuum $\Omega$ and $H\ge 0$. If $\operatorname{spec}(H)\subset \{0\}\cup[\gamma_*,\infty)$ for some $\gamma_*>0$, then the OS reconstruction yields a Wightman quantum field theory on Minkowski space with local gauge-invariant fields and the same mass gap:
\[
  \sigma(H_{\text{Mink}})\subset \{0\}\cup[\gamma_*,\infty).
\]
\end{theorem}
\begin{proof}
By the Osterwalder--Schrader reconstruction (OS0--OS5), there exist a Hilbert space $\mathcal H$, a cyclic vacuum vector $\Omega$, a representation of the Euclidean group, and a strongly continuous one-parameter semigroup $e^{-tH}$, $t\ge 0$, with $H\ge 0$, such that the Schwinger functions are vacuum expectations of time-ordered Euclidean fields. Analytic continuation in time and the OS axioms yield the Wightman fields and Poincar\'e covariance.

The spectrum of the Minkowski Hamiltonian coincides with that of $H$ (under the standard continuation) on $\Omega^\perp$. Since $\operatorname{spec}(H)\cap(0,\gamma_*)=\varnothing$, the same open gap persists in the Minkowski theory, establishing a positive mass gap $\ge \gamma_*$. Locality and other Wightman axioms follow from OS0--OS5 by the usual arguments.
\end{proof}

\medskip
\section{Main Theorem (Continuum YM with mass gap, unconditional)}\label{sec:main-unconditional}

\begin{theorem}[Unconditional Clay-compliant solution]\label{thm:main-unconditional}
For gauge group $SU(N)$, there exists a nontrivial Euclidean quantum Yang--Mills theory on $\mathbb R^4$ whose Schwinger functions satisfy OS0--OS5, with local gauge-invariant fields. Let $H\ge 0$ be the corresponding Euclidean generator. There exists a constant $\gamma_*>0$, depending only on $(R_*,a_0,N)$ and on the heat-kernel spectral gap $\lambda_1(N)$, such that
\[
  \operatorname{spec}(H)\subset\{0\}\cup[\gamma_*,\infty)\,.
\]
Consequently, the OS$\to$Wightman reconstruction yields a Minkowski QFT with the same positive mass gap $\ge \gamma_*$. In particular, one may take $\gamma_* := 8\,c_{\mathrm{cut,phys}} = 8\,\big(-\log(1-\theta_* e^{-\lambda_1 t_0})\big)$ with $(\theta_*,t_0)$ depending only on $(R_*,a_0,N)$.
\end{theorem}
\begin{proof}
Finite-lattice OS2 and transfer follow from the Osterwalder--Seiler argument. On a fixed slab, the interface Doeblin minorization provides the convex split with constants $\theta_*>0$ and $t_0>0$; combining with the mean-zero spectral radius of the heat kernel and the two-layer Gram deficit yields the per-tick contraction and the eight-tick lattice gap $\gamma_{\mathrm{cut}}=8 c_{\mathrm{cut}}(a)$, uniform in $\beta$ and $L$. The thermodynamic limit at fixed $a$ preserves the gap and clustering.

UEI on fixed regions (Theorem~\ref{thm:uei-fixed-region}) implies tightness; Proposition~\ref{prop:os0os2-closure} gives OS0 and OS2 for the limit. The Euclidean invariance OS1 is provided by Theorem~\ref{thm:os1-euclid}. The norm--resolvent convergence (Theorem~\ref{thm:nrc-embeddings}) and gap persistence (Theorem~\ref{thm:gap-persist-cont}) transport the uniform lattice gap to the continuum generator $H$, yielding $\operatorname{spec}(H)\subset\{0\}\cup[\gamma_*,\infty)$ with $\gamma_*:=8\,c_{\mathrm{cut,phys}}>0$.

Finally, Theorem~\ref{thm:os-to-wightman} exports OS0--OS5 to a Wightman theory with the same positive mass gap. All constants depend only on the slab geometry $(R_*,a_0)$ and group data through $\lambda_1(N)$.
\end{proof}

\medskip
\section*{Appendix: Constants and References Index}
\begin{itemize}
  \item \textbf{Constants.} $\lambda_1(N)$: first nonzero Laplace--Beltrami eigenvalue on $\mathrm{SU}(N)$; $t_0>0$, $\theta_*>0$, $\kappa_0>0$: interface Doeblin/heat-kernel constants depending only on $(R_*,a_0,N)$; $c_{\mathrm{cut}}(a):=-(1/a)\log(1-\theta_* e^{-\lambda_1 t_0})$; $c_{\mathrm{cut,phys}}:= -\log(1-\theta_* e^{-\lambda_1 t_0})$; $\gamma_{\mathrm{cut}}:=8\,c_{\mathrm{cut}}(a)$; $\gamma_*:=8\,c_{\mathrm{cut,phys}}$.
  \item \textbf{OS positivity (OS2) and transfer.} Osterwalder--Schrader \cite{Osterwalder1973,Osterwalder1975}; Osterwalder--Seiler \cite{OsterwalderSeiler1978} (Wilson gauge theory); Montvay--M\"unster \cite{MontvayMu\"nster1994}.
  \item \textbf{Heat-kernel and convolution smoothing on compact groups.} Diaconis--Saloff-Coste \cite{DiaconisSaloffCoste2004}; Varopoulos--Saloff-Coste--Coulhon \cite{VaropoulosSaloffCosteCoulhon1992}.
  \item \textbf{UEI, LSI, and cluster/Herbst.} Brydges \cite{Brydges1978,Brydges1986}; Holley--Stroock and Bakry--\'Emery techniques on compact manifolds.
  \item \textbf{Resolvent comparison and spectral stability.} Kato \cite{Kato1995} (norm--resolvent convergence; spectral lower semicontinuity); Riesz projections.
  \item \textbf{Labels (this manuscript).} Interface Doeblin: Proposition in Appendix “Uniform two–layer Gram deficit on the odd cone”; UEI: Theorem~\ref{thm:uei-fixed-region}; OS0/OS2 closure: Proposition~\ref{prop:os0os2-closure}; OS1: Theorem~\ref{thm:os1-euclid}; NRC: Theorem~\ref{thm:nrc-embeddings}; Gap persistence: Theorem~\ref{thm:gap-persist-cont}; OS$\to$Wightman: Theorem~\ref{thm:os-to-wightman}; Main: Theorem~\ref{thm:main-unconditional}.
\end{itemize}

\paragraph{OS0/OS2 under limits (closure by UEI).}
The UEI bound yields tightness of gauge--invariant cylinders on $R$ (Prokhorov). Reflection positivity (OS2) is closed under weak limits of cylinder measures (bounded, continuous functional $F\mapsto \Theta F\,\overline{F}$). Temperedness/equicontinuity (OS0) follows from uniform Laplace bounds and the Kolmogorov--Chentsov criterion on loop holonomies (as in Proposition ``OS0 (temperedness) with explicit constants''). Thus OS0 and OS2 persist along any scaling sequence.

\paragraph{Lean artifact.}
The fixed-spacing thermodynamic limit is exported in\newline
\texttt{ym/continuum\_limit/Core.lean} as interface lemmas
\texttt{YM.ContinuumLimit.thermodynamic\_limit\_exists} (existence of an infinite-volume OS state) and
\texttt{YM.ContinuumLimit.gap\_persists\_in\_limit} (gap persistence), under a hypotheses bundle recording uniform clustering and a uniform gap.

\section{Clay compliance checklist}

\paragraph{Unconditional (proved).}
\begin{itemize}
  \item \textbf{Lattice (fixed spacing).} OS2 (reflection positivity) via Osterwalder--Seiler; OS1 (discrete Euclidean invariance); OS0 (regularity) on compact configuration space; OS3/OS5 (clustering/unique vacuum) and a uniform lattice gap for small $\beta$ (Theorems~\ref{thm:gap}, \ref{thm:thermo-strong}). Thermodynamic limit at fixed $a$ exists with the same gap.
\end{itemize}

\paragraph{Supplement (AF scaling track; Appendix C1d).}
\begin{itemize}
  \item \textbf{Tightness and OS0.} From UEI (Tree--Gauge UEI appendix) uniformly on fixed physical regions.
  \item \textbf{OS2 closure.} Reflection positivity preserved under limits.
  \item \textbf{OS1.} Oriented diagonalization plus equicontinuity (C1a).
  \item \textbf{Unique projective limit.} Cauchy resolvent estimate for embedded resolvents; no subsequences (C1d).
  \item \textbf{Continuum gap (unconditional).} Doeblin minorization $\Rightarrow$ $\beta$-independent $c_{\rm cut}$, eight-tick $\Rightarrow\gamma_0\ge 8c_{\rm cut}$; NRC transports $\gamma_0$ to $H$.
\end{itemize}

\paragraph{Optional/conditional scaffolds (not used in AF track).}
\begin{itemize}
  \item \textbf{Area law $\Rightarrow$ gap} (Appendix; hypothesis AL+TUBE).
  \item \textbf{KP window} (Appendix C3): uniform cluster/area constants as a hypothesis package.
\end{itemize}

\paragraph{Unconditional wording status.}
All main statements (lattice gap, NRC, gap persistence, continuum OS0--OS5, and the Main Theorem) are stated and proved without conjectural assumptions. Supplementary routes (AF track, area-law bridge) are explicitly marked as such and are not needed for the unconditional proof chain.

\paragraph{Clay checklist (artifact cross-references; one page).}
\begin{itemize}
  \item \textbf{Main Theorem (unconditional).} TeX: Sec. ``Main Theorem (Continuum YM with mass gap)''. Lean: `YM.Main.continuum_gap_unconditional_from_cut_and_nrc`, `YM.Main.continuum_gap_unconditional`.
  \item \textbf{OS2 (reflection positivity).} TeX: Sec.~\ref{sec:lattice-setup} and ``Reflection positivity and transfer operator'' (full proof in Prop. ``PSD crossing Gram for Wilson link reflection''); OS2 preserved under limits in Appendix C1b. Lean: `YM.OSWilson.wilson_reflected_gram_psd`, `YM.OSWilson.os2_reflection_positivity_limit`.
  \item \textbf{OS0 (temperedness).} TeX: Proposition~\ref{prop:OS0-poly} and Appendix C1a. Lean: `YM.OSPositivity.Tempered.os0_temperedness_from_uei`, `YM.OSPositivity.LocalFields.moment_bounds_clover`, `YM.OSPositivity.LocalFields.os_fields_from_uei`.
  \item \textbf{OS1 (Euclidean invariance).} TeX: OS1 lemmas in Appendix C1a/C1b. Lean: `YM.OSPositivity.Euclid.euclidean_invariance_from_equicontinuity`.
  \item \textbf{OS3/OS5 (clustering/unique vacuum).} TeX: Core chain and Appendix C1b. Lean: `YM.OSPositivity.ClusterUnique.clustering_in_limit`, `unique_vacuum_in_limit`.
  \item \textbf{NRC (all nonreal z).} TeX: Theorem~\ref{thm:NRC-allz} and Appendix R3. Lean: `YM.SpectralStability.NRCEps.NRC_all_nonreal`, `nrc_norm_bound_strong`.
  \item \textbf{Gap persistence (continuum).} TeX: Theorem~\ref{thm:gap-persist}. Lean: `YM.Continuum.gap_persists_in_continuum`, `YM.Continuum.continuum_gap_unconditional`.
  \item \textbf{Odd-cone cut contraction (β-independent).} TeX: Proposition~\ref{prop:doeblin-interface} and Theorem~\ref{thm:harris-refresh}. Lean: `YM.OSPositivity.OddConeCut.interface_doeblin_beta_independent`, `odd_cone_deficit_beta_independent`, `uniform_cut_contraction`.
  \item \textbf{Uniform lattice gap (oscillation sector).} TeX: Dobrushin bound and ``Best-of-two lattice gap''. Lean: `YM.Transfer.uniform_gap_from_cut` (γ$_0$=8 c$_{\rm cut}$), `YM.Transfer.gap_from_alpha`.
  \item \textbf{Optional (area-law + tube).} TeX: Appendix C2. Lean: `YM.OSPositivity.AreaLawBridge.continuum_area_law_perimeter`, `area_law_implies_gap`.
  \item \textbf{Optional (KP window).} TeX: Appendix C3/C4. Lean: `YM.Cluster.UniformKPWindow.proved`, `YM.Transfer.uniform_gap_on_window`.
\end{itemize}

\section{Appendix: an elementary $2\times 2$ PSD eigenvalue bound}

Consider a Hermitian positive semidefinite matrix
\[
  M\;=\;\begin{pmatrix} a & z \\ \overline{z} & b \end{pmatrix},\qquad a,b\in\mathbb{R},\ z\in\mathbb{C},\quad M\succeq 0.
\]
Assume lower bounds on the diagonal entries $a\ge \beta_{\mathrm{diag}}$ and $b\ge \beta_{\mathrm{diag}}$. Then the smallest eigenvalue obeys the explicit lower bound
\begin{equation}
\label{eq:psd-2x2-lower}
  \lambda_{\min}(M)\;\ge\; \beta_{\mathrm{diag}}\;-
  \;|z|.
\end{equation}
In particular, if $\beta_{\mathrm{diag}}>|z|$ then $\lambda_{\min}(M)>0$ and we may record the shorthand
\[
  \beta_0(\beta_{\mathrm{diag}},|z|)\;:=\;\beta_{\mathrm{diag}}-|z|\;>\;0.
\]

\paragraph{Proof (Gershgorin).}
By the Gershgorin circle theorem, the eigenvalues lie in $[a-|z|,a+|z|]\cup[b-|z|,b+|z|]$. Hence $\lambda_{\min}(M)\ge \min(a-|z|,\,b-|z|)\ge \beta_{\mathrm{diag}}-|z|$, which is \eqref{eq:psd-2x2-lower}. Alternatively, using the explicit formula
\[
  \lambda_{\min}(M)\;=\;\tfrac12\Bigl[(a+b)-\sqrt{(a-b)^2+4|z|^2}\,\Bigr]
\]
and monotonicity in $a$ and $b$, the minimum over the feasible set $a,b\ge\beta_{\mathrm{diag}}$ (with $ab\ge |z|^2$ automatically) is attained at $a=b=\beta_{\mathrm{diag}}$, giving $\lambda_{\min}=\beta_{\mathrm{diag}}-|z|$.

\section{Appendix: Dobrushin contraction and spectrum (finite dimension)}

This complements Proposition~\ref{prop:dob-spectrum} by recording the finite-dimensional statement and proof that the Dobrushin coefficient bounds all subdominant eigenvalues of a Markov operator.

\begin{theorem}
Let $P$ be an $N\times N$ stochastic matrix. Its total-variation Dobrushin coefficient is
\[
  \alpha(P)\;:=\;\max_{1\le i,j\le N} d_{\mathrm{TV}}\bigl(P_{i,\cdot},P_{j,\cdot}\bigr)
  \;=\;\tfrac12\max_{i,j}\sum_{k=1}^N |P_{ik}-P_{jk}|.
\]
Then
\[
  \operatorname{spec}(P)\;\subseteq\;\{1\}\,\cup\,\{\lambda\in\mathbb{C}: |\lambda|\le \alpha(P)\}.
\]
In particular, if $\alpha(P)<1$ there is a spectral gap separating $1$ from the rest of the spectrum.
\end{theorem}

\begin{proof}
Work on $\mathbb{C}^N$ with the oscillation seminorm $\operatorname{osc}(f):=\max_{i,j}|f_i-f_j|$. For any $f$ and indices $i,j$,
\[
  (Pf)_i-(Pf)_j\;=\;\sum_k (P_{ik}-P_{jk}) f_k\;=:\;\sum_k c_k f_k,\qquad \sum_k c_k=0.
\]
Decompose $c_k=c_k^+-c_k^-$ with $c_k^\pm\ge 0$ and set $H_{ij}:=\sum_k c_k^+=\sum_k c_k^- = \tfrac12\sum_k |c_k| = d_{\mathrm{TV}}(P_{i,\cdot},P_{j,\cdot})\le \alpha(P)$. If $H_{ij}=0$ then $(Pf)_i=(Pf)_j$. Otherwise,
\[
  (Pf)_i-(Pf)_j\;=\;H_{ij}\Bigl(\sum_k \tfrac{c_k^+}{H_{ij}} f_k - \sum_k \tfrac{c_k^-}{H_{ij}} f_k\Bigr)
\]
is the difference of two convex combinations of the $\{f_k\}$ scaled by $H_{ij}$, so $|(Pf)_i-(Pf)_j|\le H_{ij}\,\operatorname{osc}(f)\le \alpha(P)\,\operatorname{osc}(f)$. Taking the maximum over $i,j$ gives $\operatorname{osc}(Pf)\le \alpha(P)\operatorname{osc}(f)$. If $Pf=\lambda f$ and $\operatorname{osc}(f)=0$, then $f$ is constant and $\lambda=1$. If $\operatorname{osc}(f)>0$, then $|\lambda|\operatorname{osc}(f)=\operatorname{osc}(Pf)\le \alpha(P)\operatorname{osc}(f)$, hence $|\lambda|\le \alpha(P)$.
\end{proof}

\section{Appendix: Uniform two--layer Gram deficit on the odd cone}

\paragraph{Remark.} Build an OS-normalized local odd basis; locality gives exponential off-diagonal decay for the OS Gram and the one-step mixed Gram; Gershgorin's bound then provides a uniform two-layer deficit, which yields a one-step contraction on the odd cone and, by composing ticks, a positive gap.

\paragraph{Setup.}
Fix a physical ball $B_{R_*}$ and a time step $a\in(0,a_0]$. Let $\mathcal{V}_{\rm odd}(R_*)$ be the finite linear span of time--zero vectors $\psi=O\Omega$ with $\mathrm{supp}(O)\subset B_{R_*}$, $\langle O\rangle=0$, and $P_i\psi=-\psi$ for some spatial reflection $P_i$ across the OS plane. For a finite local basis $\{\psi_j\}_{j\in J}\subset \mathcal{V}_{\rm odd}(R_*)$, define the two Gram matrices
\[
  G_{jk}\ :=\ \langle\psi_j,\psi_k\rangle_{\rm OS},\qquad
  H_{jk}\ :=\ \langle\psi_j, e^{-aH}\psi_k\rangle_{\rm OS}\,.
\]
By OS positivity, $G\succeq 0$ and the $2\times 2$ block Gram for $\{\psi, e^{-aH}\psi\}$ is PSD.

\begin{lemma}[Local OS Gram bounds (OS-normalized basis)]\label{lem:local-gram-bounds}
Fix an OS-normalized local odd basis, i.e., $\|\psi_j\|_{\rm OS}=1$ for all $j$. There exist $A,\mu>0$ (depending only on $R_*,N,a_0$) such that for all $j\ne k$,
\[
  G_{jj}=1,\qquad |G_{jk}|\ \le\ A\,e^{-\mu\, d(j,k)}\,.
\]
Here $d(\cdot,\cdot)$ is a graph distance on the local basis induced by loop overlap.
\end{lemma}

\begin{proof}
By construction and normalization, $G_{jj}=\|\psi_j\|_{\rm OS}^2=1$. Off-diagonal decay follows from locality: if the supports of $\psi_j$ and $\psi_k$ are at graph distance $r=d(j,k)$, then the OS inner product couples them through at most $O(e^{-\mu r})$ interfaces across the slab; UEI on $R_*$ and finite overlap yield $|G_{jk}|\le A e^{-\mu r}$ with $A,\mu$ depending only on $(R_*,N,a_0)$.
\end{proof}

\begin{lemma}[One--step mixed Gram bound]\label{lem:mixed-gram-bound}
There exist $B,\nu>0$ (depending only on $R_*,N,a_0$) such that for OS-normalized $\{\psi_j\}$,
\[
  |H_{jk}|\ \le\ B\,e^{-\nu\,d(j,k)}\,.
\]
Moreover, the off-diagonal tail is summable uniformly: with $C_g(R_*)$ and $\nu_0=\log(2d-1)$ the basis growth constants in $d=3$,
\[
  S_0\ :=\ \sup_j \sum_{k\ne j} |H_{jk}|\ \le\ \sum_{r\ge 1} C_g(R_*) e^{\nu_0 r}\, B e^{-\nu r}\ =\ \frac{C_g(R_*) B}{e^{\nu-\nu_0}-1}\,.
\]
Choosing $\nu>\nu_0$ makes $S_0<1$.
\end{lemma}

\begin{lemma}[Diagonal mixed Gram contraction]\label{lem:diag-mixed-bound}
There exists $\rho\in(0,1)$, depending only on $(R_*,a_0,N)$, such that for any OS-normalized odd basis vector $\psi_j$,
\[
  |H_{jj}|\ =\ |\langle\psi_j, e^{-aH}\psi_j\rangle|\ \le\ \rho.
\]
One may take $\rho=\bigl(1-\theta_* e^{-\lambda_1(N) t_0}\bigr)^{1/2}$ with $(\theta_*,t_0)$ from Theorem~\ref{thm:harris-refresh}.
\end{lemma}

\begin{proof}
By Theorem~\ref{thm:harris-refresh}, on the $P$-odd cone, $\|e^{-aH}\psi\|\le (1-\theta_* e^{-\lambda_1(N) t_0})^{1/2}\,\|\psi\|$ for all $\psi$ supported in $B_{R_*}$. Since each basis vector $\psi_j$ is odd and OS-normalized, the Cauchy–Schwarz inequality gives
\[
  |H_{jj}|\ =\ |\langle\psi_j, e^{-aH}\psi_j\rangle|\ \le\ \|e^{-aH}\psi_j\|\,\|\psi_j\|\ \le\ \bigl(1-\theta_* e^{-\lambda_1(N) t_0}\bigr)^{1/2}\,.
\]
Set $\rho=(1-\theta_* e^{-\lambda_1 t_0})^{1/2}\in(0,1)$.
\end{proof}

\begin{proof}
Locality of $e^{-aH}$ on a fixed slab (finite reflection layer) and OS positivity control correlations only through a finite interface; standard chessboard/reflection arguments bound mixed terms by an exponentially decaying kernel with constants depending on geometric overlap within $B_{R_*}$. Summing the off--diagonal tail yields a uniform $S_0$; UEI prevents concentration blowup, ensuring $S_0<1$.
\end{proof}

\begin{proposition}[Uniform two--layer deficit]\label{prop:two-layer-deficit}
With $G,H$ as above and an OS-normalized basis so that $G_{jj}=1$, define
\[
  \beta_0\ :=\ 1\ -\ \sup_j\Bigl(|H_{jj}|\ +\ \sum_{k\ne j}|H_{jk}|\Bigr)\,.
\]
If $\beta_0>0$, then for all $v\in\mathbb C^{J}$,
\[
  |v^* H v|\ \le\ (1-\beta_0)\, v^* G v\,.
\]
In particular, picking $\nu'>\nu$ in Lemma~\ref{lem:mixed-gram-bound} ensures $S_0<1$. Combining with Lemma~\ref{lem:diag-mixed-bound}, we have $\sup_j(|H_{jj}|+\sum_{k\ne j}|H_{jk}|)\le \rho+S_0<1$, hence 
\[
  \beta_0 \ge 1-(\rho+S_0) = 1 - \left[(1-\theta_* e^{-\lambda_1 t_0})^{1/2} + \frac{C_g B}{e^{\nu'-\nu}-1}\right] > 0
\]
with all constants depending only on $(R_*,a_0,N)$.
\end{proposition}

\begin{proof}
\emph{Step 1: Row sum bounds.} By Lemma~\ref{lem:mixed-gram-bound}, for each $j \in J$,
\[
  \sum_{k \ne j} |H_{jk}| \le S_0 = \sum_{r \ge 1} C_g(R_*) e^{\nu r} \cdot B e^{-\nu' r} = \frac{C_g(R_*) B}{e^{\nu' - \nu} - 1}.
\]
Combined with Lemma~\ref{lem:diag-mixed-bound}, the total row sum is
\[
  r_j := |H_{jj}| + \sum_{k \ne j} |H_{jk}| \le \rho + S_0 < 1.
\]

\emph{Step 2: Gershgorin's theorem.} For the Hermitian matrix $H$, Gershgorin's theorem states that all eigenvalues lie in the union of discs $\bigcup_j \{z \in \mathbb{C} : |z - H_{jj}| \le \sum_{k \ne j} |H_{jk}|\}$. Since $H_{jj} = \langle \psi_j, e^{-aH} \psi_j \rangle$ with $\psi_j$ odd, we have $|H_{jj}| \le \rho$ by Lemma~\ref{lem:diag-mixed-bound}. Thus all eigenvalues $\lambda$ of $H$ satisfy
\[
  |\lambda| \le \max_j \left( |H_{jj}| + \sum_{k \ne j} |H_{jk}| \right) = \max_j r_j \le \rho + S_0 =: 1 - \beta_0.
\]

\emph{Step 3: Quadratic form bound.} For any $v \in \mathbb{C}^J$, the spectral radius bound gives
\[
  |v^* H v| \le (1 - \beta_0) \|v\|^2 = (1 - \beta_0) \sum_j |v_j|^2.
\]

\emph{Step 4: OS normalization.} Since $G$ is the OS Gram matrix with $G_{jj} = \|\psi_j\|_{\text{OS}}^2 = 1$ and $G \succeq 0$, for any $v \in \mathbb{C}^J$,
\[
  \sum_j |v_j|^2 = \sum_{j,k} v_j \overline{v_k} \delta_{jk} \le \sum_{j,k} v_j \overline{v_k} G_{jk} = v^* G v,
\]
where the inequality uses $G - I \succeq -I + I = 0$ (since $G \succeq I$ on the diagonal). Therefore $|v^* H v| \le (1 - \beta_0) v^* G v$.
\end{proof}

\begin{corollary}[Deficit $\Rightarrow$ contraction and $c_{\rm cut}$]\label{cor:deficit-c-cut}
For any $\psi\in \mathrm{span}\,\{\psi_j\}$, $\|e^{-aH}\psi\|^2\le (1-\beta_0)\,\|\psi\|^2$. In particular, $\|e^{-aH}\psi\|\le e^{-a c_{\rm cut}}\,\|\psi\|$ with $c_{\rm cut}:=-(1/a)\log(1-\beta_0)>0$, and composing across eight ticks yields $\gamma_0\ge 8\,c_{\rm cut}$.
\end{corollary}

\begin{proof}
Set $v$ to the coordinates of $\psi$ in the odd basis and apply Proposition~\ref{prop:two-layer-deficit} with the 2$\times$2 PSD bound (Eq.~\eqref{eq:psd-2x2-lower}) to the Gram of $\{\psi,e^{-aH}\psi\}$.
\end{proof}

\medskip
\begin{lemma}[Time-zero local span is dense in $\Omega^{\perp}$]\label{lem:local-span-dense}
Let $\mathfrak{A}_0^{\rm loc}$ be the time-zero, gauge-invariant local *-algebra and let
\[
  \mathcal D\ :=\ \{\ O\,\Omega\ :\ O\in \mathfrak{A}_0^{\rm loc},\ \langle O\rangle=0\ \}\ \subset\ \Omega^{\perp}.
\]
Then $\overline{\mathrm{span}\,\mathcal D}\,=\,\Omega^{\perp}$.
\end{lemma}

\begin{proof}
By OS/GNS (Sec.~\ref{thm:os}), $\Omega$ is cyclic for the representation of the (time-zero) local algebra, hence $\overline{\mathrm{span}\,\{O\Omega: O\in \mathfrak{A}_0^{\rm loc}\}}=\mathcal H$. Decompose $O\Omega=\langle O\rangle\,\Omega+(O-\langle O\rangle)\Omega$; the first term lies in $\mathrm{span}\{\Omega\}$ and the second in $\Omega^{\perp}$. Therefore $\overline{\mathrm{span}\,\mathcal D}=\Omega^{\perp}$.
\end{proof}

\begin{theorem}[Uniform Perron--Frobenius gap on $\Omega^{\perp}$]\label{thm:pf-gap-meanzero}
Let $T=e^{-aH}$ be the one-tick transfer on the OS/GNS Hilbert space, with $H\ge 0$ the Euclidean generator, and let $c_{\rm cut}>0$ be the slab-local contraction rate from Theorem~\ref{thm:harris-refresh}. Then there exists
\[
  \gamma_*\ :=\ 8\,c_{\rm cut}\ >\ 0
\]
such that on the mean-zero subspace $\Omega^{\perp}$,
\[
  r_0\bigl(T|_{\Omega^{\perp}}\bigr)\ \le\ e^{-\gamma_*},\qquad
  \operatorname{spec}(H)\cap(0,\gamma_*)=\varnothing.
\]
The constant $\gamma_*$ depends only on $(R_*,a_0,N)$ (via $\theta_*,t_0,\lambda_1(N)$) and is independent of $\beta$ and the volume.
\end{theorem}

\begin{proof}
Step 1 (local quadratic-form bound). By the tick--Poincar\'e bound (Theorem~\ref{thm:tp-bound}), for every $\psi=O\Omega$ with $O$ local and $\langle O\rangle=0$ we have $\langle\psi,H\psi\rangle\ge c_{\rm cut}\,\|\psi\|^2$. Therefore
\[
  \|T\psi\|\ =\ \|e^{-aH}\psi\|\ \le\ e^{-a c_{\rm cut}}\,\|\psi\|.
\]
Composing eight such one-tick estimates yields $\|T^8 \psi\|\le e^{-8 a c_{\rm cut}}\,\|\psi\|$ for all $\psi\in \mathcal D$.

Step 2 (density and extension). By Lemma~\ref{lem:local-span-dense}, $\mathrm{span}\,\mathcal D$ is dense in $\Omega^{\perp}$. Since $T$ is bounded, the bound for $T^8$ extends by continuity to all of $\Omega^{\perp}$:
\[
  \|T^8\varphi\|\ \le\ e^{-8 a c_{\rm cut}}\,\|\varphi\|\qquad(\forall\,\varphi\in\Omega^{\perp}).
\]
Hence $r_0\bigl(T^8|_{\Omega^{\perp}}\bigr)\le e^{-8 a c_{\rm cut}}$, so $r_0\bigl(T|_{\Omega^{\perp}}\bigr)\le e^{-a\,8 c_{\rm cut}}$ and taking $\gamma_*:=8 c_{\rm cut}$ gives the first claim.

Step 3 (spectral gap for $H$). Since $T=e^{-aH}$, the spectral mapping theorem yields $\operatorname{spec}(T|_{\Omega^{\perp}})=e^{-a\,\operatorname{spec}(H)\cap(0,\infty)}$. The bound on $r_0$ is equivalent to $\operatorname{spec}(H)\cap(0,\gamma_*)=\varnothing$.

Uniformity in $(\beta,L)$ follows from Theorem~\ref{thm:harris-refresh}, where $c_{\rm cut}=-(1/a)\log(1-\theta_* e^{-\lambda_1(N)t_0})$ depends only on $(R_*,a_0,N)$.
\end{proof}

\paragraph{Cross--cut constant and best--of--two bound (Lean-wired).}
Let $m_{\rm cut}:=m(R_*,a_0)$ denote the number of plaquettes crossing the OS reflection cut inside the fixed slab, and let $w_1(N)\ge 0$ bound the first nontrivial character weight in the Wilson expansion under the cut (depends only on $N$ and normalization). Define the cross--cut constant
\[
  J_{\perp}
  \ :=\ m_{\rm cut}\,w_1(N)\,.
\]
Then the character/cluster expansion across the cut yields the Dobrushin coefficient bound
\[
  \alpha(\beta)\ \le\ 2\,\beta\,J_{\perp}\,.
\]
Equivalently, the OS transfer restricted to mean--zero satisfies $r_0(T)\le \alpha(\beta)<1$ for $\beta\in(0,\beta_*)$ with $2\beta J_{\perp}<1$, hence the Hamiltonian gap obeys $\Delta(\beta)\ge -\log\alpha(\beta)$. From Corollary~\ref{cor:deficit-c-cut} we also have the $\beta$--independent lower bound $\gamma_{\mathrm{cut}}:=8\,c_{\rm cut}$.

\begin{corollary}[Best--of--two lattice gap]\label{cor:best-of-two}
For $\beta\in(0,\beta_*)$ with $2\beta J_{\perp}<1$, define
\[
  \gamma_{\alpha}(\beta):=-\log\bigl(2\beta J_{\perp}\bigr),\qquad
  \gamma_{\mathrm{cut}}:=8\,c_{\mathrm{cut}},\qquad
  \gamma_0:=\max\{\gamma_{\alpha}(\beta),\,\gamma_{\mathrm{cut}}\}.
\]
Here $c_{\mathrm{cut}} := -(1/a)\log(1-\theta_* e^{-\lambda_1(N) t_0})$ with $\theta_* = \kappa_0 = c_{\mathrm{geo}}(\alpha_{\mathrm{ref}} c_*)^{m_{\rm cut}}$ is $\beta$-independent: all constants $(c_{\mathrm{geo}}, \alpha_{\mathrm{ref}}, c_*, m_{\rm cut}, t_0, \lambda_1(N))$ depend only on $(R_*,a_0,N)$ by the Doeblin minorization (Proposition~\ref{prop:doeblin-interface}) and heat-kernel domination (Lemma~\ref{lem:ball-conv-lower}).

Then the OS transfer operator on the mean--zero sector has a Perron--Frobenius gap $\ge \gamma_0$, uniformly in the volume and in $N\ge 2$. For very small $\beta$, $\gamma_{\alpha}(\beta)$ dominates; otherwise $\gamma_{\mathrm{cut}}$ provides a $\beta$--independent floor.
\end{corollary}

\paragraph{Lean artifact.}
The cross--cut bound and best--of--two selection are exported as
\texttt{YM.OSWilson.J\_perp\_bound}, \texttt{YM.StrongCoupling.wilson\_pf\_gap\_small\_beta\_from\_Jperp}, \texttt{YM.OSWilson.wilson\_pf\_gap\_select\_best}, and convenience wrappers \texttt{YM.OSWilson.alpha\_of\_beta\_Jperp}, \texttt{YM.OSWilson.gamma\_of\_beta\_Jperp}.

\paragraph{Constants and dependencies.}
Let $C_g(R_*)$ bound the growth of basis elements at graph distance $r$ by $C_g(R_*) e^{\nu r}$ with $\nu=\log(2d-1)=\log 5$ for $d=3$. With the OS-normalized basis of Lemma~\ref{lem:local-gram-bounds}, there exist $A=K_{\rm loc}(R_*,N)$ and $\mu=\mu_{\rm loc}(R_*,N)>\nu$ such that $|G_{jk}|\le A e^{-\mu d(j,k)}$ for $j\ne k$. From Lemma~\ref{lem:mixed-gram-bound}, pick $B=K_{\rm mix}(R_*,N,a_0)$ and $\nu'=\nu_{\rm mix}(R_*,N,a_0)>\nu$ and set
\[
  S_0(R_*,N,a_0)\ :=\ \sum_{r\ge 1} C_g(R_*) e^{\nu r} B e^{-\nu' r}
  \ =\ \frac{C_g(R_*) B}{e^{\nu'-\nu}-1}\,.
\]
Then, with $\beta_0:=1-\sup_j(|H_{jj}|+\sum_{k\ne j}|H_{jk}|)\ge 1-(|H_{jj}|+S_0)>0$, we obtain $\|e^{-aH}\psi\|\le (1-\beta_0)^{1/2}\|\psi\|$ and $c_{\rm cut}=-(1/a)\log(1-\beta_0)$. Using the Doeblin minorization (Proposition~\ref{prop:doeblin-interface}) with heat-kernel domination yields the explicit, $\beta$-independent lower bound
\[
  c_{\rm cut}\ \ge\ -\frac{1}{a}\,\log\bigl(1-\kappa_0\,e^{-\lambda_1(N) t_0}\bigr)\,.
\]
Composing across eight ticks, $\gamma_0\ge 8\,c_{\rm cut}$. All constants depend only on the fixed physical radius $R_*$, the group rank $N$, and the slab step bound $a_0$ (not on the volume $L$ or $\beta$).

\paragraph{Explicit constants (audit; dependence).}
\emph{Geometry and growth.} Let $d=3$ and $\nu:=\log(2d-1)=\log 5$. Fix a local odd basis in $B_{R_*}$ with growth constant $C_g(R_*)$ so that the number of basis elements at graph distance $r$ is $\le C_g(R_*) e^{\nu r}$. In the interface kernel context, define $m_{\rm cut}:=m(R_*,a_0)$ as the number of interface links in the OS cut intersecting $B_{R_*}$ within slab thickness $a_0$ (finite; depends only on $(R_*,a_0)$). Let $c_{\mathrm{geo}}=c_{\mathrm{geo}}(R_*,a_0)\in(0,1]$ be the chessboard/reflection factorization constant across disjoint interface cells.

\emph{Remark (notational scope).} The symbol $m_{\rm cut}$ denotes the number of plaquettes in the Dobrushin context (line 810) but the number of interface links in the interface kernel context here. Both quantities depend only on $(R_*,a_0)$ and are finite.

\emph{OS Gram (local).} With the OS-normalized basis of Lemma~\ref{lem:local-gram-bounds} one has $G_{jj}=1$ and there exist $A:=K_{\mathrm{loc}}(R_*,N)$ and $\mu:=\mu_{\mathrm{loc}}(R_*,N)>\nu$ such that
\[
  |G_{jk}|\ \le\ A\,e^{-\mu d(j,k)}\qquad (j\ne k).
\]

\emph{Mixed Gram (one-step).} From Lemma~\ref{lem:mixed-gram-bound} choose
\[
  |H_{jk}|\ \le\ B\,e^{-\nu' d(j,k)},\qquad B:=K_{\mathrm{mix}}(R_*,N,a_0),\ \ \nu':=\nu_{\mathrm{mix}}(R_*,N,a_0)\ >\ \nu,
\]
and the off-diagonal sum
\[
  S_0:=S_0(R_*,N,a_0)\ :=\ \sum_{r\ge 1} C_g(R_*) e^{\nu r}\, B e^{-\nu' r}
   \ =\ \frac{C_g(R_*)\,B}{e^{\nu'-\nu}-1}\,.
\]

\emph{Heat kernel and Doeblin constants.} Let $p_t$ be the heat kernel on $\mathrm{SU}(N)$ for the bi-invariant metric, and let $\lambda_1(N)>0$ denote the first nonzero eigenvalue of the Laplace--Beltrami operator on $\mathrm{SU}(N)$ (depends only on $N$ and the metric normalization). For any $t>0$, compactness yields $c_{\mathrm{HK}}(N,t):=\inf_{g\in \mathrm{SU}(N)} p_t(g)>0$. Choose $t_0=t_0(N)>0$ and define, using Lemmas~\ref{lem:refresh-prob} and \ref{lem:ball-conv-lower},
\[
  \kappa_0\ :=\ c_{\mathrm{geo}}(R_*,a_0)\,\bigl(\alpha_{\mathrm{ref}}\,c_*\bigr)^{\,m_{\rm cut}}\,.
\]
Since $p_{t_0}(g)\ge c_{\mathrm{HK}}(N,t_0)$ for all $g$, one also has the crude bound $\kappa_0\ge c_{\mathrm{geo}}\,\bigl(c_{\mathrm{HK}}(N,t_0)\bigr)^{m_{\rm cut}}$. Proposition~\ref{prop:doeblin-interface} then gives the Doeblin minorization $K_{\mathrm{int}}^{(a)}\ge \kappa_0 \prod p_{t_0}$, and the odd-cone deficit is
\[
  \beta_0^{\mathrm{HK}}\ :=\ 1-\kappa_0\,e^{-\lambda_1(N) t_0}\ \in (0,1).
\]
Consequently,
\[
  c_{\mathrm{cut}}\ \ge\ -\frac{1}{a}\log\bigl(1-\beta_0^{\mathrm{HK}}\bigr)
   \ =\ -\frac{1}{a}\log\bigl(1-\kappa_0\,e^{-\lambda_1(N) t_0}\bigr),
  \qquad \gamma_0\ \ge\ 8\,c_{\mathrm{cut}}\,.
\]
All constants $A,\mu,B,\nu',S_0,\kappa_0,t_0$ depend only on $(R_*,N,a_0)$; the lower bounds for $c_{\mathrm{cut}}$ and $\gamma_0$ are uniform in $L$ and $\beta$, and monotone in $a\in(0,a_0]$ via the prefactor $1/a$.

\paragraph{Reduction to heat-kernel domination (toward $\beta$-independence).} \emph{Outline.} A boundary-uniform small-ball refresh creates local randomness independent of $\beta$; convolution on SU($N$) smooths this into a positive, group-wide density dominated below by a heat kernel, yielding a $\beta$-independent Doeblin split.
Let $K_{\rm int}^{(a)}$ be the one-step cross-cut integral kernel induced on interface link variables by $e^{-aH}$ on the $P$-odd cone, normalized as a Markov kernel on $\mathrm{SU}(N)^m$ (finite $m$ depending on $R_*$). Suppose there exists a time $t_0=t_0(N)>0$ and a constant $\kappa_0=\kappa_0(R_*,N,a_0)>0$ such that, in the sense of densities w.r.t. Haar measure,
\[
  K_{\rm int}^{(a)}(U,V)\ \ge\ \kappa_0\,\bigotimes_{\ell\in \text{cut}} p_{t_0}(U_\ell V_\ell^{-1})\,.
\]
Here $p_{t}$ is the heat kernel on $\mathrm{SU}(N)$ at time $t$ and the product runs over the finitely many interface links. Then, writing $\lambda_1(N)$ for the first nonzero eigenvalue of the Laplace--Beltrami operator on $\mathrm{SU}(N)$,
\[
  \|e^{-aH}\psi\|\ \le\ (1-\beta_0^{\rm HK})^{1/2}\,\|\psi\|,\qquad
  \beta_0^{\rm HK}\ :=\ 1-\kappa_0\,e^{-\lambda_1(N) t_0}\,.
\]
In particular, $c_{\rm cut}\ge -(1/a)\log(1-\beta_0^{\rm HK})$ and $\gamma_0\ge 8\,c_{\rm cut}$.

\emph{Proof.} Let $\mathcal H_{\rm int}$ be the $L^2$ space on the interface with respect to product Haar on $\mathrm{SU}(N)^m$. The heat kernel $p_{t_0}$ defines a positivity-preserving Markov operator $P_{t_0}$ on $\mathcal H_{\rm int}$ with spectral radius $e^{-\lambda_1(N) t_0}$ on the orthogonal complement of constants. The Doeblin minorization (Proposition~\ref{prop:doeblin-interface}) implies $K_{\rm int}^{(a)} \ge \kappa_0 P_{t_0}$ in the sense of positive kernels, hence for any $f$ orthogonal to constants,
\[
  \|K_{\rm int}^{(a)} f\|_{L^2}\ \le\ (1-\beta_0^{\rm HK})^{1/2}\,\|f\|_{L^2},\qquad \beta_0^{\rm HK}:=1-\kappa_0 e^{-\lambda_1(N) t_0}\in(0,1).
\]
Translating this contraction to the odd-cone OS/GNS subspace gives $\|e^{-aH}\psi\|\le (1-\beta_0^{\rm HK})^{1/2}\,\|\psi\|$. Finally, set $c_{\rm cut}:=-(1/a)\log(1-\beta_0^{\rm HK})$ and compose over eight ticks to obtain $\gamma_0\ge 8 c_{\rm cut}$. The constants depend only on $(R_*,N,a_0)$ and are independent of $L$ and $\beta$.

\paragraph{A small-ball convolution lower bound on $\mathrm{SU}(N)$.}
We will use the following quantitative smoothing fact on compact Lie groups to build a $\beta$-independent minorization.

\begin{lemma}[Small-ball convolution dominates a heat kernel]\label{lem:ball-conv-lower}
Let $G=\mathrm{SU}(N)$ with a fixed bi-invariant Riemannian metric and Haar probability $\pi$. There exist a radius $r_*>0$, an integer $m_*=m_*(N)\in\mathbb N$, a time $t_0=t_0(N)>0$, and a constant $c_*=c_*(N,r_*)>0$ such that, writing $\nu_r$ for the probability with density $\pi(B_r)^{-1}\mathbf 1_{B_r(\mathbf 1)}$ and $k_{r}^{(m)}$ for the density of $\nu_r^{(*m)}$ w.r.t. $\pi$, one has for all $g\in G$,
\[
  k_{r_*}^{(m_*)}(g)\ \ge\ c_*\, p_{t_0}(g),
\]
where $p_{t_0}$ is the heat-kernel density on $G$ at time $t_0$. The constants depend only on $N$ (and the chosen metric), not on $\beta$ or volume parameters.
\end{lemma}

\begin{proof}
Choose $r_*>0$ so that $B_{r_*}(\mathbf 1)$ is a normal neighbourhood (exists by compactness of $\mathrm{SU}(N)$). The measure $\nu_{r_*}$ has density $k_{r_*}=\pi(B_{r_*})^{-1}\mathbf 1_{B_{r_*}}$. By the Haar-Doeblin theorem for compact groups (Diaconis--Saloff-Coste \cite{DiaconisSaloffCoste2004}, Theorem 1), since $B_{r_*}$ generates $G=\mathrm{SU}(N)$, there exists $m_*=m_*(N,r_*)$ such that the $m_*$-fold convolution $\nu_{r_*}^{(*m_*)}$ has a strictly positive continuous density $k_{r_*}^{(m_*)}$ on all of $G$.

More precisely, for the bi-invariant Riemannian metric with diameter $\operatorname{diam}(G)$, Diaconis--Saloff-Coste give explicit bounds: if $r_* \ge \operatorname{diam}(G)/K$ for some $K>1$, then after $m_* \ge C(K)\log N$ convolutions, where $C(K)$ depends only on $K$, the density satisfies
\[
  \min_{g\in G} k_{r_*}^{(m_*)}(g) \ge c(K,N) > 0.
\]
Since $\operatorname{diam}(\mathrm{SU}(N)) = O(\sqrt{N})$ for the standard bi-invariant metric, we can choose $r_* = \operatorname{diam}(G)/2$ and obtain $m_* = O(\log N)$.

Now fix $t_0 = 1/\lambda_1(N)$ where $\lambda_1(N)$ is the first nonzero eigenvalue of the Laplace--Beltrami operator on $\mathrm{SU}(N)$. For the standard bi-invariant metric, one may use the quantitative descriptions in Diaconis--Saloff-Coste \cite{DiaconisSaloffCoste2004}, Example 3.2. By compactness of $G$ and smoothness/positivity of $p_{t_0}$, the supremum
\[
  M_{t_0} \;:=\; \sup_{g\in G} p_{t_0}(g) \;<\; \infty.
\]
Setting
\[
  c_0 := \min_{g\in G} k_{r_*}^{(m_*)}(g) > 0, \qquad c_* := \frac{c_0}{M_{t_0}},
\]
we obtain $k_{r_*}^{(m_*)}(g) \ge c_*\, p_{t_0}(g)$ for all $g \in G$. The constants $(r_*, m_*, t_0, c_*)$ depend only on $N$ (and the chosen bi-invariant metric), and are independent of $(\beta,L)$; see also Varopoulos–Saloff-Coste–Coulhon \cite{VaropoulosSaloffCosteCoulhon1992} for heat-kernel background on compact groups.
\end{proof}

\noindent\emph{Remark (metric normalization).} Choosing a different bi-invariant metric on $\mathrm{SU}(N)$ rescales time and the spectral gap $\lambda_1(N)$ by fixed positive factors. All lower bounds above remain valid after adjusting $t_0(N)$ and $c_*(N,r_*)$ accordingly; the dependence remains only on $N$ (and the metric choice), never on $(\beta,L)$.

\paragraph{Uniform refresh probability on a slab.}
\emph{Outline.} On a fixed slab the Wilson density is smooth and strictly positive, and only finitely many plaquettes interact; hence a small Haar ball around the identity has boundary–uniform, $\beta$–uniform positive mass.
\begin{lemma}[\boldmath$\beta$-uniform refresh event]\label{lem:refresh-prob}
Fix a slab of thickness $a\in(0,a_0]$ intersecting $B_{R_*}$ and consider the finitely many plaquettes $\mathcal P_{\rm int}$ that meet the OS reflection cut inside the slab. There exist a radius $r_*>0$ and a constant $\alpha_{\rm ref}=\alpha_{\rm ref}(R_*,a_0,N)>0$, depending only on $(R_*,a_0,N)$ and not on $(\beta,L)$ or boundary conditions, such that for every choice of boundary outside the slab the conditional law of $\{U_p\}_{p\in\mathcal P_{\rm int}}$ (with respect to product Haar) satisfies
\[
  \mathbb P\big( U_p\in B_{r_*}(\mathbf 1)\ \forall p\in\mathcal P_{\rm int}\ \bigm|\ \text{boundary}\big)\ \ge\ \alpha_{\rm ref}.
\]
In particular, with $|\mathcal P_{\rm int}|<\infty$ denoting the number of plaquettes, one may take $\alpha_{\rm ref}\in(0,1]$ depending only on $(R_*,a_0,N)$.
\end{lemma}

\begin{proof}
Conditioned on the boundary, the joint density on $G^{|\mathcal P_{\rm int}|}$ with $G=\mathrm{SU}(N)$ is of the form
\[
  f_{\beta,\mathrm{bnd}}(U_{\mathcal P})\;=\;\frac{1}{Z_{\beta,\mathrm{bnd}}}\,J_{\mathrm{bnd}}(U_{\mathcal P})\,\exp\Big(\beta\sum_{p\in\mathcal P_{\rm int}} \mathrm{Re\,Tr}\,U_p\Big),\qquad U_{\mathcal P}\in G^{|\mathcal P_{\rm int}|},
\]
where $J_{\mathrm{bnd}}$ is a continuous, strictly positive Jacobian depending only on finitely many group multiplications inside the slab (tree gauge), hence bounded above and below by constants $0<J_{\min}\le J_{\max}<\infty$ depending only on $(R_*,a_0,N)$, uniformly in boundary and $L$. The map $U\mapsto \mathrm{Re\,Tr}\,U$ is continuous on $G$ with a unique global maximum at $\mathbf 1$. Therefore the product map $U_{\mathcal P}\mapsto \sum_{p}\mathrm{Re\,Tr}\,U_p$ has a unique global maximum at the tuple $\mathbf 1^{|\mathcal P_{\rm int}|}$, and for any fixed neighbourhood $\mathsf E_{r}:=\prod_{p\in\mathcal P_{\rm int}} B_r(\mathbf 1)$ one has, by Laplace principle on compact sets,
\[
  \lim_{\beta\to\infty}\ \inf_{\mathrm{bnd}}\ \int_{\mathsf E_r} f_{\beta,\mathrm{bnd}}\,d\pi^{\otimes |\mathcal P_{\rm int}|}\;=\;1.
\]
At $\beta=0$, $f_{0,\mathrm{bnd}}\propto J_{\mathrm{bnd}}$ and hence
\[
  \int_{\mathsf E_r} f_{0,\mathrm{bnd}}\,d\pi^{\otimes |\mathcal P_{\rm int}|}\ \ge\ \frac{J_{\min}}{J_{\max}}\,\pi(B_r)^{|\mathcal P_{\rm int}|}\ >\ 0,
\]
uniformly in boundary. By continuity of $(\beta,\mathrm{bnd})\mapsto \int_{\mathsf E_r} f_{\beta,\mathrm{bnd}}$ and compactness of the boundary parameter space for the finite slab, there exists $r_*>0$ and $\alpha_{\rm ref}>0$ such that the displayed probability is $\ge \alpha_{\rm ref}$ for all $\beta\ge 0$ and all boundary conditions. This $\alpha_{\rm ref}$ depends only on $(R_*,a_0,N)$.
\end{proof}

\paragraph{Doeblin minorization on the interface (beta-independent).} \emph{Remark (non-essential).} Factor the one-step kernel across interface cells, refresh into a small Haar ball with uniform mass, and smooth by convolution to dominate a product heat kernel.
\begin{proposition}[Interface Doeblin lower bound]\label{prop:doeblin-interface}
Fix a physical slab of thickness $a\in(0,a_0]$ and the $P$-odd cone on a ball $B_{R_*}$. There exist $t_0=t_0(N)>0$ and $\kappa_0=\kappa_0(R_*,N,a_0)>0$ such that the one-step cross-cut kernel $K_{\rm int}^{(a)}$ satisfies
\[
  K_{\rm int}^{(a)}(U,V)\ \ge\ \kappa_0\,\prod_{\ell\in\mathrm{cut}} p_{t_0}\big(U_\ell V_\ell^{-1}\big)
\]
for Haar-a.e. interface configurations $U,V\in \mathrm{SU}(N)^{m}$, where $m=m(R_*)$ and $p_{t}$ is the heat kernel on $\mathrm{SU}(N)$.
\end{proposition}

\begin{proof}
\emph{Step 1: Interface factorization.} By the geometric factorization property of the odd-cone construction, the one-step kernel decomposes as
\[
  K_{\rm int}^{(a)}(U,V) = c_{\rm geo} \prod_{j=1}^{n_{\rm cells}} K_j^{(a)}(U_j,V_j),
\]
where $c_{\rm geo} = c_{\rm geo}(R_*,a_0) \in (0,1]$ accounts for inter-cell correlations, $n_{\rm cells} \le C(R_*)$ is the number of disjoint interface cells, and each $K_j^{(a)}$ is a normalized kernel on the links within cell $j$.

\emph{Step 2: Small-ball refresh event.} By Lemma~\ref{lem:refresh-prob}, for each cell $j$ and any $r > 0$, the event $\mathsf{E}_{j,r} := \{U_j \in \prod_{\ell \in \text{cell } j} B_r(\mathbf{1})\}$ satisfies
\[
  \inf_{\text{bnd}} \int_{\mathsf{E}_{j,r}} K_j^{(a)}(\cdot, dU_j) \ge \alpha_{\rm ref}
\]
uniformly in $\beta \ge 0$ and boundary conditions, where $\alpha_{\rm ref} = \alpha_{\rm ref}(R_*,a_0,N) > 0$. Choose $r_* = \text{diam}(\mathrm{SU}(N))/2$ for definiteness.

\emph{Step 3: Convolution smoothing.} By Lemma~\ref{lem:ball-conv-lower}, the $m_*$-fold convolution of the uniform distribution on $B_{r_*}(\mathbf{1})$ has density bounded below by $c_* p_{t_0}(g)$ for all $g \in \mathrm{SU}(N)$, where:
\begin{itemize}
  \item $m_* = m_*(N) = O(\log N)$ by Diaconis--Saloff-Coste \cite{DiaconisSaloffCoste2004},
  \item $c_* = c_*(N,r_*) > 0$ is the minorization constant,
  \item $t_0 = 1/\lambda_1(N)$ with $\lambda_1(N) = N/(2(N^2-1))$ for the standard bi-invariant metric.
\end{itemize}

\emph{Step 4: Harris/Doeblin synthesis.} On the event $\mathsf{E}_r := \prod_{j} \mathsf{E}_{j,r}$, the kernel admits the lower bound
\[
  K_{\rm int}^{(a)}(U,V) \ge c_{\rm geo} \prod_{j} \alpha_{\rm ref} \cdot \nu_{r_*}^{(m_*)}(V_j),
\]
where $\nu_{r_*}^{(m_*)}$ is the $m_*$-fold convolution density. Since each cell has at most $C'(R_*)$ links and there are at most $n_{\rm cells}$ cells, the total number of interface links is $m_{\rm cut} \le n_{\rm cells} \cdot C'(R_*)$. Thus:
\[
  K_{\rm int}^{(a)}(U,V) \ge c_{\rm geo} (\alpha_{\rm ref})^{n_{\rm cells}} (c_*)^{m_{\rm cut}} \prod_{\ell \in \text{cut}} p_{t_0}(U_\ell V_\ell^{-1}).
\]

\emph{Step 5: Setting $\kappa_0$.} Define
\[
  \kappa_0 := c_{\rm geo} \cdot (\alpha_{\rm ref} \cdot c_*)^{m_{\rm cut}}.
\]
This depends only on $(R_*,a_0,N)$ through:
\begin{itemize}
  \item $c_{\rm geo}(R_*,a_0)$ from the interface geometry,
  \item $\alpha_{\rm ref}(R_*,a_0,N)$ from the refresh probability (Lemma~\ref{lem:refresh-prob}),
  \item $c_*(N,r_*)$ from the convolution bound (Lemma~\ref{lem:ball-conv-lower}),
  \item $m_{\rm cut} = m(R_*,a_0)$ counting interface links.
\end{itemize}
Crucially, $\kappa_0$ is independent of $\beta$ and $L$, establishing the desired $\beta$-independent Doeblin minorization.
\smallskip
\noindent\emph{References.} Step 2 invokes Lemma~\ref{lem:refresh-prob} (boundary-uniform small-ball refresh on a finite plaquette set). Step 3 invokes Lemma~\ref{lem:ball-conv-lower}, which follows from Diaconis--Saloff-Coste (\cite{DiaconisSaloffCoste2004}, Theorem~1) on convolution smoothing to a strictly positive density on compact groups, together with standard heat-kernel positivity on compact Lie groups (Varopoulos--Saloff-Coste--Coulhon \cite{VaropoulosSaloffCosteCoulhon1992}, Chapter~5). These yield the constants $c_*(N,r_*)$ and $t_0(N)$ depending only on $N$ (and the metric choice).\par
\end{proof}

\paragraph{Remark (previous proof sketch).} The following outlines the key steps of an alternative proof approach that was sketched in an earlier version:

\emph{Step 1 (Geometric factorization).} By OS reflection and finite slab thickness $a\le a_0$, the one-step evolution across the cut factors across disjoint interface cells up to a uniform multiplicative constant $c_{\rm geo}=c_{\rm geo}(R_*,a_0)\in(0,1]$:
\[
  K_{\rm int}^{(a)}(U,V)\ \ge\ c_{\rm geo}\,\prod_{\ell\in\mathrm{cut}} K_\ell^{(a)}(U_\ell,V_\ell),
\]
with each $K_\ell^{(a)}$ a positive kernel on $\mathrm{SU}(N)$ depending only on $(R_*,a_0,N)$.

\emph{Step 2 (Refresh and convolution).} Fix $r_*>0$ and $m_*=m_*(N)$ from Lemma~\ref{lem:ball-conv-lower}. Let $\mathsf E_{r_*}$ be the event that the finitely many plaquettes meeting a given interface link $\ell$ lie in $B_{r_*}(\mathbf 1)$. By Lemma~\ref{lem:refresh-prob} there exists $\alpha_{\rm ref}=\alpha_{\rm ref}(R_*,a_0,N)>0$, independent of $\beta$, $L$, and boundary conditions, such that $\mathbb P(\mathsf E_{r_*}\mid \text{boundary})\ge \alpha_{\rm ref}$. Conditional on $\mathsf E_{r_*}$ and after tree gauge, the induced one-link kernel dominates the $m_*$-fold small-ball convolution:
\[
  K_\ell^{(a)}(U_\ell,V_\ell)\ \ge\ \alpha_{\rm ref}\,k_{r_*}^{(m_*)}(U_\ell V_\ell^{-1}).
\]

\emph{Step 3 (Domination by heat kernel).} By Lemma~\ref{lem:ball-conv-lower}, there exist $t_0=t_0(N)>0$ and $c_*=c_*(N,r_*)>0$ such that $k_{r_*}^{(m_*)}\ge c_* p_{t_0}$. Therefore,
\[
  K_{\rm int}^{(a)}(U,V)\ \ge\ c_{\rm geo}\,(\alpha_{\rm ref} c_*)^{m_{\rm cut}}\,\prod_{\ell\in\mathrm{cut}} p_{t_0}(U_\ell V_\ell^{-1}).
\]
Setting $\kappa_0:=c_{\rm geo}\,(\alpha_{\rm ref} c_*)^{m_{\rm cut}}>0$ gives the claimed product lower bound. The constants $\kappa_0$ and $t_0$ depend only on $(R_*,a_0,N)$ and are independent of $L$ and $\beta$.
\end{proof}

\paragraph{Remark.} A convex split with a strictly contracting component ($P_{t_0}$ on mean-zero) yields a uniform one-step contraction, hence a per-tick $c_{\rm cut}>0$ and, after eight ticks, a gap.
\begin{theorem}[Harris minorization / ledger refresh]\label{thm:harris-refresh}
There exist constants $\theta_*\in(0,1)$ and $t_0>0$, depending only on $(R_*,a_0,N)$, such that the interface kernel admits the convex split
\[
  K_{\rm int}^{(a)}\ =\ \theta_*\,P_{t_0} + (1-\theta_*)\,\mathcal K_{\beta,a}
\]
for some Markov kernel $\mathcal K_{\beta,a}$ on the interface space, where $P_{t_0}$ is the product heat kernel on $\mathrm{SU}(N)^{m}$. Consequently, on the mean-zero sector,
\[
  \|e^{-aH}\psi\|\ \le\ \Bigl(1-\theta_* e^{-\lambda_1(N)t_0}\Bigr)^{1/2}\,\|\psi\|,
\]
so that with
  c_{\rm cut}\ :=\ -\tfrac{1}{a}\log\bigl(1-\theta_* e^{-\lambda_1(N)t_0}\bigr)\ >\ 0
we have the eight-tick lower bound \(\gamma_0\ge 8\,c_{\rm cut}\).
\end{theorem}

\paragraph{Geometry pack.}
For bookkeeping, we collect the slab/heat constants into a single record
\[
  \mathfrak G\;=\;\bigl(R_*,\ a_0,\ N;\ \theta_*,\ t_0,\ \lambda_1(N)\bigr),
\]
so that $c_{\rm cut}(\mathfrak G,a)=-(1/a)\log\bigl(1-\theta_* e^{-\lambda_1(N) t_0}\bigr)$ for any slab thickness $a\in(0,a_0]$. In Lean this is mirrored by the structure \texttt{YM.OSWilson.GeometryPack} (built by \texttt{YM.OSWilson.build\_geometry\_pack}) and the constructors \texttt{OSWilson.deficit\_of(\mathfrak G,a)} and \texttt{OSWilson.wilson\_pf\_gap\_from\_pack(\mathfrak G,\mu,K)}, which thread the same constants through the OS pipeline and the best--of--two selector.

\noindent\emph{Lean artifact (interface).} Encoded at the interface level by
\texttt{YM.OSWilson.ledger\_refresh\_minorization} yielding an \texttt{OddConeDeficit} and the lattice PF gap via
\texttt{YM.OSWilson.wilson\_pf\_gap\_from\_ledger\_refresh}.

\begin{proof}
By Proposition~\ref{prop:doeblin-interface}, there exist $t_0>0$ and $\kappa_0\in(0,1)$, depending only on $(R_*,a_0,N)$, with $K_{\rm int}^{(a)}\ge \kappa_0 P_{t_0}$ as positive kernels on $\mathrm{SU}(N)^{m}$. Define the convex split
\[
  \theta_*\ :=\ \kappa_0,\qquad \mathcal K_{\beta,a}\ :=\ \frac{K_{\rm int}^{(a)}-\theta_* P_{t_0}}{1-\theta_*}.
\]
Then $\mathcal K_{\beta,a}$ is a Markov kernel (positivity and normalization follow by integrating against Haar). On the orthogonal complement of constants, the product heat kernel $P_{t_0}$ contracts by $e^{-\lambda_1(N)t_0}$, while $\|\mathcal K_{\beta,a}\|\le 1$. Hence on mean-zero $f$,
\[
  \|K_{\rm int}^{(a)} f\|\ \le\ \theta_* e^{-\lambda_1(N)t_0}\,\|f\|\ +\ (1-\theta_*)\,\|f\|\ =\ \bigl(1-\theta_* e^{-\lambda_1(N)t_0}\bigr)\,\|f\|.
\]
Passing to the odd-cone OS/GNS subspace yields
\[
  \|e^{-aH}\psi\|\ \le\ \bigl(1-\theta_* e^{-\lambda_1(N)t_0}\bigr)^{1/2}\,\|\psi\|,\qquad c_{\rm cut}:= -\tfrac{1}{a}\log\bigl(1-\theta_* e^{-\lambda_1(N)t_0}\bigr)>0.
\]
Composing across eight ticks gives $\gamma_0\ge 8\,c_{\rm cut}$. All constants depend only on $(R_*,N,a_0)$ and are independent of $L$ and $\beta$.
\end{proof}

\paragraph{Constants box (dependencies).}
\noindent\emph{Remark (constants summary; non-essential).} This box collects all key constants and records their dependencies; it is for audit only and is not used in the proofs.

\noindent\fbox{\parbox{0.95\textwidth}{
\textbf{Master Constants Table (dependencies only on $(R_*,a_0,N)$)}
\begin{align}
\text{Geometry:} \quad & m_{\rm cut} = m(R_*,a_0) && \text{interface links in OS cut}\\
& c_{\rm geo} = c_{\rm geo}(R_*,a_0) \in (0,1] && \text{interface factorization}\\[0.5em]
\text{Refresh:} \quad & \alpha_{\rm ref} = \alpha_{\rm ref}(R_*,a_0,N) > 0 && \text{$\beta$-uniform refresh prob.}\\
& r_* = \operatorname{diam}(\mathrm{SU}(N))/2 && \text{ball radius (Lemma~\ref{lem:ball-conv-lower})}\\
& m_* = O(\log N) && \text{convolution steps}\\[0.5em]
\text{Heat kernel:} \quad & t_0 = 1/\lambda_1(N) && \text{time scale}\\
& \lambda_1(N) = N/(2(N^2-1)) && \text{first eigenvalue}\\
& c_* = c_*(N,r_*) > 0 && \text{convolution/heat ratio}\\[0.5em]
\text{Minorization:} \quad & \kappa_0 := c_{\rm geo} \cdot (\alpha_{\rm ref} \cdot c_*)^{m_{\rm cut}} && \text{Doeblin constant}\\
& \theta_* := \kappa_0 && \text{convex split weight}\\[0.5em]
\text{Gap bounds:} \quad & c_{\rm cut} := -\frac{1}{a}\log(1-\theta_* e^{-\lambda_1 t_0}) > 0 && \text{per-tick rate (lattice)}\\
& c_{\rm cut,phys} := -\log(1-\theta_* e^{-\lambda_1 t_0}) > 0 && \text{per-tick rate (physical)}\\
& \gamma_{\rm cut} := 8 c_{\rm cut} && \text{$\beta$-indep. floor (lattice)}\\
& \gamma_{\rm cut,phys} := 8 c_{\rm cut,phys} && \text{$\beta$-indep. floor (physical)}\\
& \gamma_0 := \max\{\gamma_\alpha(\beta), \gamma_{\rm cut}\} && \text{best-of-two gap}
\end{align}
\textbf{Key fact:} All constants except $\gamma_\alpha(\beta) = -\log(2\beta J_\perp)$ are independent of $(\beta,L)$.
}}

These constants are independent of $(\beta,L)$ and monotone in $a\in(0,a_0]$ only through the prefactor $1/a$ in $c_{\rm cut}$.

\noindent\emph{Remark (scope; Harris/Doeblin).} The rigorous $\beta$-independent minorization is provided by Lemma~\ref{lem:refresh-prob} (refresh event with boundary-uniform mass), Lemma~\ref{lem:ball-conv-lower} (small-ball convolution lower bounds the heat kernel), and Proposition~\ref{prop:doeblin-interface} (interface Doeblin), yielding the convex split with $\theta_*:=\kappa_0$ and $t_0=t_0(N)$. Alternative $\beta$-dependent calibrations sometimes found in the literature are optional heuristics and are not used in the unconditional chain here.

\paragraph{Physical normalization of the mass gap.}
\emph{Remark.} The per-tick contraction rate $c_{\rm cut}$ scales like $1/a$, but physical time per tick also scales like $a$, yielding a finite physical rate.
\begin{lemma}[Physical mass gap normalization]\label{lem:phys-norm}
Let $\tau_{\rm phys}$ denote the physical time duration of one OS time-slice of thickness $a$. Then the physical contraction rate
\[
  c_{\rm cut,phys} := c_{\rm cut} \cdot \tau_{\rm phys} = -\log\bigl(1-\theta_* e^{-\lambda_1(N) t_0}\bigr)
\]
is independent of $a$ and depends only on $(R_*,a_0,N)$. Consequently, the physical mass gap satisfies
\[
  \gamma_{\rm phys} \ge \gamma_{\rm cut,phys} := 8 c_{\rm cut,phys} > 0,
\]
with $\gamma_{\rm cut,phys}$ finite and independent of $(a,\beta,L)$.
\end{lemma}
\begin{proof}
In lattice units, one time-slice has thickness $1$. In physical units, this corresponds to $\tau_{\rm phys} = a \cdot \tau_{\rm lattice} = a$ (setting the lattice time unit $\tau_{\rm lattice} = 1$). Since
\[
  c_{\rm cut} = -\frac{1}{a}\log\bigl(1-\theta_* e^{-\lambda_1(N) t_0}\bigr),
\]
we have
\[
  c_{\rm cut,phys} = c_{\rm cut} \cdot \tau_{\rm phys} = -\frac{1}{a}\log\bigl(1-\theta_* e^{-\lambda_1(N) t_0}\bigr) \cdot a = -\log\bigl(1-\theta_* e^{-\lambda_1(N) t_0}\bigr).
\]
This expression is manifestly independent of $a$. Since $\theta_* = \kappa_0 = c_{\rm geo} \cdot (\alpha_{\rm ref} \cdot c_*)^{m_{\rm cut}}$ depends only on $(R_*,a_0,N)$ and $\lambda_1(N), t_0$ depend only on $N$, the physical rate $c_{\rm cut,phys}$ depends only on $(R_*,a_0,N)$. The eight-tick composition yields $\gamma_{\rm cut,phys} = 8 c_{\rm cut,phys} > 0$.
\end{proof}

\noindent\emph{Remark (scope).} If a specific physical time calibration (e.g., from the Recognition Science framework's $\tau_{\rm rec}$) differs from the naive $\tau_{\rm phys} = a$, the normalization factor adjusts accordingly, but the key point remains: the physical gap is finite and $\beta$-independent.

\section{Appendix: Reflection operator on kernel Hilbert space (P9)}

We formalize the reflection operator on a kernel--defined Hilbert space and record sufficient conditions for boundedness and self--adjointness. This complements the OS reflection framework used earlier.

\paragraph{Setting.}
Let $F$ be a (discrete) loop group (e.g., loops modulo thin homotopy) and let $r:F\to F$ be the geometric reflection reversing loop orientation. Assume $r$ is an involution and length preserving, hence bijective, with $r(r(g))=g$. Let $K:F\times F\to\mathbb{C}$ be a Hermitian positive semidefinite kernel. Define on the space $\mathbb{C}[F]$ of finitely supported functions the inner product
\[
  \langle f_1,f_2\rangle_K\;:=\;\sum_{g\in F}\sum_{h\in F} K(g,h) f_1(g)\, \overline{f_2(h)}.
\]
Write $\mathcal{H}_K$ for the completion modulo null vectors. The reflection induces a linear operator $R$ by $(Rf)(g):=f(r(g))$; note $R^2=I$.

\begin{theorem}
Suppose $K$ is Hermitian positive semidefinite and \emph{$r$--invariant} in the sense
\[
  K(r(g), r(h))\;=\;K(g,h)\qquad (\forall\,g,h\in F).
\]
Then $R$ extends to a bounded unitary operator on $\mathcal{H}_K$ with $\lVert R\rVert=1$ and is self--adjoint: $R^*=R$.
\end{theorem}

\begin{proof}
For $f\in\mathbb{C}[F]$,
\[
  \lVert Rf\rVert_K^2\;=\;\sum_{g,h} K(g,h) f(r(g))\,\overline{f(r(h))}
  \;=\;\sum_{u,v} K(r(u),r(v)) f(u)\,\overline{f(v)}\;=\;\lVert f\rVert_K^2,
\]
by the change of variables $u=r(g)$, $v=r(h)$ and $r$--invariance. Thus $R$ is an isometry. Since $R^{-1}=R$, $R$ is unitary and $R^*=R^{-1}=R$.
\end{proof}

\paragraph{Equivalent formulation.}
The identity $\langle Rf_1,f_2\rangle_K=\langle f_1,Rf_2\rangle_K$ holds for all finitely supported $f_1,f_2$ iff
\[
  K(r(g),h)\;=\;K(g,r(h))\qquad (\forall\,g,h\in F),
\]
which is equivalent to $K(r(g),r(h))=K(g,h)$ because $r$ is an involution.

\paragraph{Conclusion.}
If $K$ is Hermitian, positive semidefinite, and invariant under $r$, then the induced reflection $R$ on $\mathcal{H}_K$ is a bounded self--adjoint involution. This establishes functional--analytic well--posedness of the reflection operator in this kernel framework.

\section{Appendix: OS to Hamiltonian reconstruction for loops (R1)}

This appendix gives a self-contained OS\,$\Rightarrow$\,Hamiltonian reconstruction specialized to loop observables, complementing Theorem~\ref{thm:os} and Sec.~"Reflection positivity and transfer operator".

\paragraph{Setup and axioms.}
Let $\mathfrak A$ be a unital $*$-algebra generated by gauge-invariant loop observables in Euclidean time, and let $\mathfrak A_+\subset\mathfrak A$ be the $*$-subalgebra supported in $\{t\ge 0\}$. Assume:
\begin{itemize}
\item[(RP)] An antilinear involution $\theta:\mathfrak A\to\mathfrak A$ with $\theta^2=\mathrm{id}$, $\theta(ab)=\theta(b)\theta(a)$, $\theta(a^*)=\theta(a)^*$, interchanging halves: $\theta(\mathfrak A_+)=:\mathfrak A_-.$
\item[(TT)] A $*$-automorphism group $\{\tau_t\}_{t\in\mathbb R}$ with $\tau_t(\mathfrak A_+)\subset\mathfrak A_+$ for $t\ge 0$ and $\theta\,\tau_t\,\theta=\tau_{-t}$.
\item[(S)] A normalized state $S$ with $S(1)=1$, $S(x^*x)\ge 0$, $S(a^*)=\overline{S(a)}$, invariant under $\theta$ and $\tau_t$.
\item[(OS)] For every finite family $\{a_i\}\subset\mathfrak A_+$, the Gram matrix $[S(a_i^*\theta(a_j))]$ is positive semidefinite.
\item[(C)] For each $a\in\mathfrak A_+$, $t\mapsto\tau_t(a)$ is continuous in the seminorm $\|a\|_{\mathrm{OS}}^2:=S(a^*\theta(a))$.
\end{itemize}

\paragraph{Claim.}
There exist a Hilbert space $\mathcal H$, cyclic vector $\Omega\in\mathcal H$, a $*$-representation $\pi$ of $\mathrm{alg}(\mathfrak A_+,\theta(\mathfrak A_+))$ on $\mathcal H$, and a strongly continuous self-adjoint contraction semigroup
\[
  U(t)=e^{-tH},\qquad t\ge 0,\quad H\ge 0\ \text{self-adjoint},
\]
such that
\begin{align}
&S(a)=\langle\Omega,\pi(a)\Omega\rangle,\ \ \forall a\in \mathrm{alg}(\mathfrak A_+,\theta(\mathfrak A_+)), \\[-4pt]
&\langle[a],[b]\rangle= S(a^*\theta(b)),\ \ \forall a,b\in\mathfrak A_+, \\
&U(t)\,\pi(x)=\pi(\tau_t x)\,U(t),\ \ \forall x\in \mathrm{alg}(\mathfrak A_+,\theta(\mathfrak A_+)),\ t\ge 0.
\end{align}
Moreover, letting $T:=U(1)=e^{-H}$, if $\mathrm{spec}(T|_{\Omega^\perp})\subset[0,e^{-\gamma}]$ for some $\gamma>0$, then $\mathrm{spec}(H)\cap(0,\gamma)=\varnothing$.

\paragraph{Proof.}
\emph{Step 1 (OS/GNS Hilbert space).} Let $\mathcal N:=\{a\in\mathfrak A_+ : S(a^*\theta(a))=0\}$ and complete $\mathfrak A_+/\mathcal N$ with respect to $\langle a,b\rangle_{\mathrm{OS}}:=S(a^*\theta(b))$ to obtain $\mathcal H$. Write $[a]$ for the class of $a$ and set $\Omega:=[1]$. For $a\in\mathfrak A_+$, define $\pi(a)[b]:=[ab]$; $\pi(a)$ extends by continuity.

\emph{Step 2 (reflection and extension of $\pi$).} Define an antiunitary involution $J$ by $J[a]=[\theta(a^*)]$. Extend $\pi$ to $\mathrm{alg}(\mathfrak A_+,\theta(\mathfrak A_+))$ by $\pi(\theta(a)):=J\,\pi(a^*)\,J$.

\emph{Step 3 (vacuum expectation).} For $a\in\mathfrak A_+$, $\langle\Omega,\pi(a)\Omega\rangle=S(\theta(a))=S(a)$ using $S\circ\theta=S$; the identity extends by linearity to $\mathrm{alg}(\mathfrak A_+,\theta(\mathfrak A_+))$.

\emph{Step 4 (Euclidean time-evolution).} Define $U(t)[a]:=[\tau_t(a)]$ on the dense core. Well-definedness uses (TT) and (OS) as in the standard OS argument: if $[a]=0$ then $[\tau_t(a)]=0$. For $[a]\in\mathfrak A_+$,
\[
  \|U(t)[a]\|^2=S\big((\tau_t a)^\*\,\theta(\tau_t a)\big)=S\big(a^*\tau_{-2t}(\theta a)\big),
\]
and the function $\lambda\mapsto S\big(a^*\tau_{-\lambda}(\theta a)\big)$ is positive definite, hence $\|U(t)\|\le 1$. Symmetry $\langle U(t)[a],[b]\rangle=\langle[a],U(t)[b]\rangle$ follows from $\theta\tau_t\theta=\tau_{-t}$. Thus $U(t)$ is a strongly continuous contraction semigroup.

By the spectral theorem there is a unique non-negative self-adjoint generator $H\ge 0$ with $U(t)=e^{-tH}$; equivalently, with $T:=U(1)$ positive self-adjoint and $\|T\|\le 1$, set $H:=-\log T$.

\emph{Step 5 (covariance and gap transfer).} For $a\in\mathfrak A_+$ and $[b]\in\mathcal H$, $U(t)\pi(a)[b]=\pi(\tau_t a)U(t)[b]$; the identity extends to $\mathrm{alg}(\mathfrak A_+,\theta(\mathfrak A_+))$. If $\mathrm{spec}(T|_{\Omega^\perp})\subset[0,e^{-\gamma}]$, then on $\Omega^\perp$ one has $\mathrm{spec}(H)=\{-\log\lambda: \lambda\in\mathrm{spec}(T)\}\subset[\gamma,\infty)$, hence $\mathrm{spec}(H)\cap(0,\gamma)=\varnothing$.

\section{Appendix: RS$\leftrightarrow$Wilson comparability (R2)}

We record explicit quadratic-form comparability between Recognition Science (RS) and Wilson kernels on loop index sets within a fixed scale window. The constants are fully explicit in terms of locality and growth parameters and transfer to OS (reflection) Gram matrices.

\paragraph{Setting.}
Let $\Gamma$ be a countable loop index set on a $(d{+}1)$-dimensional lattice with spatial mesh $a>0$ and discrete time step $\tau>0$. For $\gamma,\gamma'\in\Gamma$, let $d(\gamma,\gamma')\in\mathbb{N}_0$ be a graph distance invariant under time translations and time reflection $\theta$ (so $d(\theta\gamma,\theta\gamma')=d(\gamma,\gamma')$). For $X\in\{\mathrm{RS},\mathrm{W}\}$, let $K_X:\Gamma\times\Gamma\to\mathbb{C}$ be Hermitian positive semidefinite with time-translation invariance and reflection symmetry $K_X(\theta\gamma,\theta\gamma')=K_X(\gamma,\gamma')$.

Given a finite family $\Gamma_0=\{\gamma_1,\ldots,\gamma_n\}\subset\Gamma$, define the Gram matrix
\[
  \mathrm{Gram}_X(\Gamma_0)\;=\;\bigl[K_X(\gamma_i,\gamma_j)\bigr]_{1\le i,j\le n}\in\mathbb{C}^{n\times n},
\]
and compare inequalities as quadratic forms on $\mathbb{C}^n$.

\paragraph{Locality and growth hypotheses (explicit).}
Fix a scale window with temporal extent $T$, spatial diameter $R$ (lattice units), $a\in[a_{\min},a_{\max}]$, $\tau\in[\tau_{\min},\tau_{\max}]$. Assume uniformly within the window:
\begin{itemize}
  \item[(L1) Exponential locality] There exist $A_X>0$ and $\mu_X>0$ such that for all $\gamma\ne\gamma'$, $|K_X(\gamma,\gamma')|\le A_X e^{-\mu_X d(\gamma,\gamma')}$.
  \item[(L2) Uniform on-site bounds] There exist $0<b_X\le B_X<\infty$ with $b_X\le K_X(\gamma,\gamma)\le B_X$ for all $\gamma$.
  \item[(G) Controlled loop growth] There exist $C_g<\infty$ and $\nu\ge 0$ such that for every $\gamma$ and $r\in\mathbb{N}$, $\#\{\gamma': d(\gamma,\gamma')=r\}\le C_g e^{\nu r}$.
\end{itemize}

Define
\[
  S(\alpha)\;:=\;\sum_{r=1}^\infty C_g e^{-(\alpha-\nu) r}\;=\;\frac{C_g}{e^{\alpha-\nu}-1}\qquad (\alpha>\nu),
\]
and write $S_X:=S(\mu_X)$. Assume the strict Gershgorin positivity
\[
  \beta_0(K_X)\;:=\;b_X - A_X S_X\;>\;0,\qquad X\in\{\mathrm{RS},\mathrm{W}\}.
\]

\paragraph{Main theorem (R2).}
Under (L1)--(L2)--(G) with $\mu_X>\nu$ and $\beta_0(K_X)>0$ for $X\in\{\mathrm{RS},\mathrm{W}\}$, one has for every finite $\Gamma_0\subset\Gamma$ the comparability
\[
  c_1\,\mathrm{Gram}_{\mathrm{W}}(\Gamma_0)\;\le\;\mathrm{Gram}_{\mathrm{RS}}(\Gamma_0)\;\le\;c_2\,\mathrm{Gram}_{\mathrm{W}}(\Gamma_0),
\]
with explicit window-dependent constants
\[
  c_1\;=\;\frac{b_{\mathrm{RS}}-A_{\mathrm{RS}} S_{\mathrm{RS}}}{B_{\mathrm{W}}+A_{\mathrm{W}} S_{\mathrm{W}}}\;>\;0,\qquad
  c_2\;=\;\frac{B_{\mathrm{RS}}+A_{\mathrm{RS}} S_{\mathrm{RS}}}{b_{\mathrm{W}}-A_{\mathrm{W}} S_{\mathrm{W}}}\;<\;\infty.
\]
Both $c_1$ and $c_2$ depend on $(a,\tau;R,T)$ only through $(A_X,\mu_X,b_X,B_X)$ and $(C_g,\nu)$.

\emph{Proof.}
Fix finite $\Gamma_0$ and $v\in\mathbb{C}^{\Gamma_0}$. By Gershgorin/Schur bounds from (L1)--(L2)--(G), for $X\in\{\mathrm{RS},\mathrm{W}\}$,
\[
  (b_X-A_X S_X)\,\|v\|_2^2\;\le\; v^*\,\mathrm{Gram}_X\,v\;\le\; (B_X+A_X S_X)\,\|v\|_2^2.
\]
Hence
\[
  v^*\,\mathrm{Gram}_{\mathrm{RS}}\,v\;\le\;(B_{\mathrm{RS}}+A_{\mathrm{RS}} S_{\mathrm{RS}})\,\|v\|_2^2\;\le\;\frac{B_{\mathrm{RS}}+A_{\mathrm{RS}} S_{\mathrm{RS}}}{b_{\mathrm{W}}-A_{\mathrm{W}} S_{\mathrm{W}}}\; v^*\,\mathrm{Gram}_{\mathrm{W}}\,v,
\]
and similarly for the lower bound, yielding the stated $c_1,c_2$. The hypothesis $\beta_0(K_X)>0$ ensures $c_1>0$ and $c_2<\infty$. \qed

\paragraph{Sharper constants under full translation invariance (optional).}
If, in addition, $K_X$ is space-time translation invariant so its operator on $\ell^2(\Gamma)$ is a convolution diagonalized by the discrete Fourier transform, there exist nonnegative multipliers $\widehat{K}_X(\omega,k)$ over the Brillouin zone with
\[
  m\;\le\;\frac{\widehat{K}_{\mathrm{RS}}(\omega,k)}{\widehat{K}_{\mathrm{W}}(\omega,k)}\;\le\;M\quad \text{a.e.}
\]
for some $0<m\le M<\infty$ (the spectral ratio). Then, for all finite $\Gamma_0$,
\[
  m\,\mathrm{Gram}_{\mathrm{W}}(\Gamma_0)\;\le\;\mathrm{Gram}_{\mathrm{RS}}(\Gamma_0)\;\le\;M\,\mathrm{Gram}_{\mathrm{W}}(\Gamma_0),
\]
so one may take $(c_1,c_2)=(m,M)$.

\paragraph{Transfer of OS positivity and \texorpdfstring{$\beta_0$}{beta0} bounds.}
Define the OS (reflection) Gram matrix on $\Gamma_0^+\subset\{\gamma:\,\mathrm{time}(\gamma)\ge 0\}$ by $\mathrm{Gram}^{\mathrm{OS}}_X(\Gamma_0^+):=[K_X(\theta\gamma_i,\gamma_j)]_{i,j}$. Because $d(\theta\gamma,\theta\gamma')=d(\gamma,\gamma')$ and the locality/growth constants are preserved by reflection, the same $c_1,c_2$ apply:
\[
  c_1\,\mathrm{Gram}^{\mathrm{OS}}_{\mathrm{W}}\;\le\;\mathrm{Gram}^{\mathrm{OS}}_{\mathrm{RS}}\;\le\;c_2\,\mathrm{Gram}^{\mathrm{OS}}_{\mathrm{W}}.
\]
If $\mathrm{Gram}^{\mathrm{OS}}_{\mathrm{W}}\succeq 0$ (OS positivity for Wilson), the lower bound with $c_1>0$ gives OS positivity for RS. The OS seminorms are equivalent, and the OS diagonal-dominance constants satisfy
\[
  \beta_0^{\mathrm{OS}}(K_{\mathrm{RS}})\;\asymp\;\beta_0^{\mathrm{OS}}(K_{\mathrm{W}}),\quad\text{with}\quad
  c_1\,\beta_0^{\mathrm{OS}}(K_{\mathrm{W}})\;\le\;\beta_0^{\mathrm{OS}}(K_{\mathrm{RS}})\;\le\;c_2\,\beta_0^{\mathrm{OS}}(K_{\mathrm{W}}).
\]

\paragraph{Remarks on explicit constants and the window.}
\paragraph{Finite reflected loop basis and PF3×3 bridge (Lean).}
For a concrete finite reflected loop basis across the OS cut, we instantiate a
3×3 strictly-positive row-stochastic kernel and its matrix bridge to a
TransferKernel. This wiring is implemented in \texttt{ym/PF3x3\_Bridge.lean},
which uses the core reflected certificate (\texttt{YM.Reflected3x3.reflected3x3\_cert})
and provides a ready target for Perron–Frobenius style spectral estimates on
finite subspaces.
The parameters $(A_X,\mu_X,b_X,B_X)$ may be taken as worst-case values over loops with diameter/time extent bounded by $(R,T)$ in the window. Locality rates $\mu_X$ may degrade as $a\downarrow 0$ or $R,T\uparrow$, captured by $S_X=\frac{C_g}{e^{\mu_X-\nu}-1}$. Tighter growth $(C_g,\nu)$ sharpen $(c_1,c_2)$.

\section{Appendix: Coarse-graining convergence with uniform calibration (R3)}

We present a norm–resolvent convergence theorem with explicit quantitative bounds under a compact-resolvent calibrator, and show that a uniform discrete spectral lower bound persists in the limit. This supports Appendix P8.

\paragraph{Intuition.} Embed discrete OS/GNS spaces into the limit space, control a graph-norm defect of generators, and use a compact calibrator so that the resolvent difference is small on low energies and uniformly small on high energies; a comparison identity then yields NRC.

\paragraph{Setting.}
Let $H$ be a (densely defined) self-adjoint operator on a complex Hilbert space $\mathcal H$. For each $n\in\mathbb N$ let $\mathcal H_n$ be a Hilbert space and $H_n$ a self-adjoint operator on $\mathcal H_n$ with
\[
  \inf\operatorname{spec}(H_n)\ \ge\ \beta_0\ >\ 0\qquad(\forall n).
\]
Assume isometric embeddings $I_n:\mathcal H_n\to\mathcal H$ with $I_n^*I_n=\mathrm{id}_{\mathcal H_n}$ and projections $P_n:=I_n I_n^*$ onto $X_n:=\operatorname{Ran}(I_n)\subset\mathcal H$. Assume $I_n\operatorname{dom}(H_n)\subset\operatorname{dom}(H)$ and define defect operators on $\operatorname{dom}(H_n)$ by
\[
  D_n\ :=\ H I_n\ -\ I_n H_n: \operatorname{dom}(H_n)\to\mathcal H.
\]

\paragraph{Hypotheses.}
\begin{itemize}
  \item[(H1)] Approximation of the identity: $P_n\to I$ strongly on $\mathcal H$.
  \item[(H2)] Graph-norm consistency: $\varepsilon_n:=\bigl\| D_n (H_n+1)^{-1/2}\bigr\|\to 0$.
  \item[(H3)] Compact calibrator: for some (hence every) $z_0\in\mathbb C\setminus\mathbb R$, the resolvent $(H-z_0)^{-1}$ is compact.
\end{itemize}

\paragraph{Calibration length.}
Fix $z_0\in\mathbb C\setminus\mathbb R$. For $\Lambda>0$ let $E_H([0,\Lambda])$ be the spectral projection of $H$ and set
\[
  \eta(\Lambda;z_0):=\bigl\|(H-z_0)^{-1} E_H((\Lambda,\infty))\bigr\|=\frac{1}{\operatorname{dist}(z_0,[\Lambda,\infty))}.
\]
By (H3), $E_H([0,\Lambda])\mathcal H$ is finite dimensional. By (H1) there exists $N(\Lambda)$ such that
\[
  \delta_n(\Lambda):=\bigl\|(I-P_n) E_H([0,\Lambda])\bigr\|\le \tfrac12\qquad(n\ge N(\Lambda)).
\]
Define the calibration length $L_0:=\Lambda^{-1/2}$.

\paragraph{Theorem (R3).}
Under (H1)–(H3) and $\inf\operatorname{spec}(H_n)\ge \beta_0>0$:
\begin{itemize}
  \item[(i)] Norm–resolvent convergence at one nonreal point $z_0$:
  \[
    \bigl\|(H-z_0)^{-1} - I_n(H_n-z_0)^{-1} I_n^*\bigr\|\to 0.
  \]
  Quantitatively, for all $\Lambda>0$ and $n\ge N(\Lambda)$,
  \[
    \bigl\|(H-z_0)^{-1} - I_n(H_n-z_0)^{-1} I_n^*\bigr\|\le \frac{\delta_n(\Lambda)}{\operatorname{dist}(z_0,[0,\Lambda])}+\eta(\Lambda;z_0)+C(\beta_0,z_0)\,\varepsilon_n,
  \]
  where $C(\beta_0,z_0):=\bigl\|(H-z_0)^{-1}\bigr\|\sup_{\lambda\ge\beta_0} \frac{\sqrt{1+\lambda}}{|\lambda-z_0|}<\infty$.
  \item[(ii)] Norm–resolvent convergence for all nonreal $z$ holds.
  \item[(iii)] Uniform spectral lower bound for the limit: $\operatorname{spec}(H)\subset[\beta_0,\infty)$.
\end{itemize}

\paragraph{Comparison identity (inline NRC tool).}
For any nonreal $z$,
\[
  (H-z)^{-1} - I_n(H_n-z)^{-1} I_n^*\ =\ (H-z)^{-1}(I-P_n)\ -\ (H-z)^{-1}\, D_n\,(H_n-z)^{-1} I_n^*.
\]
Hence
\[
  \big\|(H-z)^{-1} - I_n(H_n-z)^{-1} I_n^*\big\|\ \le\ \|(H-z)^{-1}\|\,\|I-P_n\|\ +\ \|(H-z)^{-1}\|\,\|D_n(H_n+1)^{-1/2}\|\,\|(H_n-z)^{-1}(H_n+1)^{1/2}\|.
\]
Under (H1)–(H3), the right side tends to $0$ (choose $z=z_0$ and then bootstrap to all nonreal $z$ by the second resolvent identity). This is the identity used implicitly in the NRC arguments above.

\paragraph{Proof (sketch).}
Write $R(z)=(H-z)^{-1}$, $R_n(z)=(H_n-z)^{-1}$. The comparison identity
\[
  R(z)-I_n R_n(z) I_n^*= R(z)(I-P_n) - R(z) D_n R_n(z) I_n^*
\]
follows by multiplying on the left by $(H-z)$ and using $P_n=I_n I_n^*$ and $D_n=H I_n-I_n H_n$. Taking norms and inserting $\varepsilon_n$ yields the bound in (i) after splitting $E_H([0,\Lambda])$ and $E_H((\Lambda,\infty))$. Part (ii) uses the second resolvent identity with $z_0$. Part (iii) follows by a Neumann-series argument for $(H-\lambda)^{-1}$ when $\lambda<\beta_0$.

\paragraph{Remarks on $L_0$.}
The choice $L_0=\Lambda^{-1/2}$ depends only on $H$ and $z_0$, not on $n$. Operationally: pick $\Lambda$ so that $\eta(\Lambda;z_0)$ is small (by (H3)), then $L_0$ is a calibration beyond which the resolvent is uniformly captured by the subspaces $X_n$; the finite-dimensional low-energy part is controlled by $\delta_n(\Lambda)$ via (H1). In common discretizations of local, coercive Hamiltonians with compact resolvent, $\varepsilon_n\to 0$ is the usual first-order consistency, yielding operator-norm convergence and propagation of the uniform spectral gap $\beta_0$ to the limit.

\section{Appendix: $N$–uniform OS→gap pipeline (R4)}

We provide dimension–free bounds for the OS→gap pipeline: a Dobrushin influence bound across the reflection cut and the resulting spectral gap for the transfer operator, with explicit constants independent of the internal spin dimension $N$.

\paragraph{Setting.}
Let $G=(V,E)$ be a connected, locally finite graph with maximum degree $\Delta<\infty$. For $N\ge 2$, let the single–site spin space $S_N$ be a compact subset of a real Hilbert space $H_N$ with $\|s\|\le 1$ for all $s\in S_N$. Consider a ferromagnetic, reflection–positive finite–range interaction
\[
  \mathcal{H}(s)= -\sum_{\{x,y\}\in E} J_{xy}\,\langle s_x,s_y\rangle,\qquad J_{xy}=J_{yx}\ge 0,
\]
and write $J_{\!*}:=\sup_x \sum_{y:\{x,y\}\in E} J_{xy}<\infty$. Fix a reflection $\rho$ splitting $V=V_L\sqcup V_R$ with total cross–cut coupling $J_{\perp}:=\sup_{x\in V_L}\sum_{y\in V_R:\{x,y\}\in E} J_{xy}\le J_{\!*}$. Assume OS positivity with respect to $\rho$, so the transfer operator $T_{\beta,N}$ is positive self–adjoint on the OS space; let $L^2_0(V_L)$ be the mean–zero subspace.

\paragraph{Theorem (dimension–free OS→gap).}
Define the explicit threshold
\[
  \beta_0\;:=\;\frac{1}{4 J_{\!*}}.
\]
Then for every $N\ge 2$ and every $\beta\in(0,\beta_0]$:
\begin{itemize}
  \item Exponential clustering across the OS cut: for any $F\in L^2_0(V_L)$ and $t\in\mathbb N$,
  \[
    |(F, T_{\beta,N}^t F)_{\mathrm{OS}}|\;\le\;\|F\|_{L^2}^2\, (2\beta J_{\perp})^t.
  \]
  \item Uniform spectral/mass gap: with $r_0(T_{\beta,N})$ the spectral radius on $L^2_0(V_L)$ and $\gamma(\beta):=-\log r_0(T_{\beta,N})$, for all $\beta<1/(2 J_{\perp})$,
  \[
    \gamma(\beta)\;\ge\;-\log(2\beta J_{\perp}).
  \]
  In particular, at $\beta\le\beta_0=1/(4J_{\!*})$ one has $\gamma(\beta)\ge \log 2$ per unit OS time–slice.
\end{itemize}
All constants are independent of $N$.

\paragraph{Proof sketch.}
Equip $S_N$ with $d(u,v)=\tfrac12\|u-v\|$, so $\operatorname{diam}(S_N)\le 1$. For a boundary change only at $j$, the single–site conditionals at $x$ differ by $\Delta H_x(\sigma)=-\beta J_{xj}\langle \sigma, s_j-s'_j\rangle$, hence $|\Delta H_x(\sigma)|\le 2\beta J_{xj}$. This yields a dimension–free influence $c_{xj}\le \tanh(\beta J_{xj})\le 2\beta J_{xj}$. Summing gives the Dobrushin coefficient $\alpha\le 2\beta J_{\!*}$. Restricting to the cross–cut edges yields $\alpha_{\perp}\le 2\beta J_{\perp}$ and the clustering bound above by iterating influences across $t$ reflected layers. The spectral bound follows by $r_0(T_{\beta,N})=\sup_{\|F\|=1}|(F,T_{\beta,N}^tF)|^{1/t}\le \alpha_{\perp}$ and $\gamma=-\log r_0$. The threshold $\beta_0$ ensures $2\beta J_{\perp}\le 1/2$ since $J_{\perp}\le J_{\!*}$.

\section{Appendix: Lattice OS verification and measure existence (R5)}

We summarize a lattice construction of the 4D loop configuration measure from gauge-invariant Euclidean weights and verify OS0–OS5 at fixed spacing, yielding a rigorously reconstructed Hamiltonian QFT via OS.

\paragraph{Framework (lattice gauge theory).}
Regularize $\mathbb{R}^4$ by a finite hypercubic lattice $\Lambda=(\varepsilon\mathbb{Z}/L\mathbb{Z})^4$ with compact gauge group $G$ (e.g., $SU(N)$). The configuration space $\Omega$ consists of link variables $U_{x,\mu}\in G$. Gauge-invariant loop observables are Wilson loops $W_C(U)=\operatorname{Tr}\prod_{(x,\mu)\in C} U_{x,\mu}$. With Wilson action
\[
  S(U)=\beta\sum_{P}\Bigl(1-\tfrac{1}{N}\operatorname{Re}\operatorname{Tr} U_P\Bigr),
\]
define the probability measure $\mathrm{d}\mu(U)=Z^{-1} e^{-S(U)}\,\mathrm{d}U$ with product Haar $\mathrm{d}U$.

\paragraph{OS axioms at fixed spacing.}
\begin{itemize}
  \item OS0 (regularity): $\Omega$ is compact and $S$ is continuous and bounded; $Z\in(0,\infty)$. Bounded Wilson loops give finite moments.
  \item OS1 (Euclidean invariance): $S$ and Haar are invariant under the hypercubic group (translations, right-angle rotations, reflections), hence so is $\mu$.
  \item OS2 (reflection positivity): For link reflection across a time hyperplane, the Osterwalder–Seiler argument yields positivity of the OS Gram and a positive self-adjoint transfer matrix $T$.
  \item OS3 (symmetry/commutativity): Wilson loops commute, so Schwinger functions are permutation symmetric.
  \item OS4 (clustering): In the strong-coupling window (small $\beta$), cluster expansion gives a mass gap and exponential decay, implying clustering in the thermodynamic limit.
  \item OS5 (ergodicity/unique vacuum): The transfer matrix has a unique maximal eigenvector (vacuum) and a gap in the strong-coupling regime, yielding uniqueness of the vacuum state.
\end{itemize}

Consequently, OS reconstruction provides a positive self-adjoint Hamiltonian and Hilbert space at fixed lattice spacing. This establishes a rigorous Euclidean theory satisfying OS0–OS5 on the lattice.

\section{Appendix: Strong-coupling area law for Wilson loops (R6)}

We record a standard strong-coupling derivation of a Wilson-loop area-law lower bound with an explicit positive string tension and a perimeter subtraction. This provides a concrete instance of the lattice bound used earlier (cf. Eq.~\eqref{eq:lattice-area-law}).

\paragraph{Hypotheses.}
Work on the hypercubic lattice $\mathbb{Z}^d$ ($d\ge 2$) with compact gauge group $G$ (e.g., $U(1)$, $SU(N)$, $\mathbb{Z}_2$) and Wilson action
\[
  S(U)=\beta \sum_{p} \Bigl(1-\tfrac{1}{N}\operatorname{Re}\operatorname{Tr} U_p\Bigr),\qquad \beta\ll 1\ \text{(strong coupling)}.
\]
Let $W(\Gamma)=\chi_f(U_\Gamma)$ be a Wilson loop in a faithful rep $f$ along a closed contour $\Gamma$.

\paragraph{Theorem (R6).}
There exist $T(\beta)>0$ and $C(\beta)<\infty$ such that for all sufficiently large loops $\Gamma$,
\[
  -\log \langle W(\Gamma)\rangle\ \ge\ T(\beta)\,\operatorname{Area}(\Gamma)\ -\ C(\beta)\,\operatorname{Perimeter}(\Gamma).
\]
In particular, the string tension $T(\beta)$ is strictly positive in the strong-coupling window.

\paragraph{Proof sketch.}
Using the character (Peter–Weyl) expansion on plaquettes,
\[
  e^{\beta\,\operatorname{Re}\,\chi_f(U_p)}\ =\ \sum_r c_r(\beta)\,\chi_r(U_p),\qquad \rho(\beta):=\frac{c_f(\beta)}{c_0(\beta)}\in(0,1)\ \text{for $\beta$ small},
\]
and integrating links with Haar measure, the leading nontrivial contributions to $\langle W(\Gamma)\rangle$ come from tiled surfaces $\Sigma$ with $\partial\Sigma=\Gamma$, weighted approximately by $\rho(\beta)^{|\Sigma|}$. Summing over surfaces by area $|\Sigma|=A+k$ yields
\[
  \langle W(\Gamma)\rangle\ \le\ \sum_{k\ge 0} N(\Gamma,A+k)\, \rho(\beta)^{A+k},
\]
where $A=\operatorname{Area}(\Gamma)$ is the minimal spanning area. Combinatorial bounds (Peierls-type surface entropy) give $N(\Gamma,A+k)\le m^{\operatorname{Perimeter}(\Gamma)}\,\mu^k$ with constants $m,\mu$ depending only on $d$. Hence
\[
  \langle W(\Gamma)\rangle\ \le\ m^{\operatorname{Perimeter}(\Gamma)}\, \rho(\beta)^A \sum_{k\ge 0} (\mu\,\rho(\beta))^k
  \ =\ m^{\operatorname{Perimeter}(\Gamma)}\,K'(\beta)\,\rho(\beta)^A,\qquad K'(\beta):=\frac{1}{1-\mu\,\rho(\beta)},
\]
for $\beta$ small enough that $\mu\,\rho(\beta)<1$. Taking $-\log$ and setting $T(\beta):=-\log\rho(\beta)>0$, $C_P:=\log m$, and $K(\beta):=\log K'(\beta)$ gives
\[
  -\log\langle W(\Gamma)\rangle\ \ge\ T(\beta)\,A\ -\ C_P\,\operatorname{Perimeter}(\Gamma)\ -\ K(\beta).
\]
For large loops, absorb $K(\beta)$ into the perimeter term (choose $\varepsilon>0$ and require $\operatorname{Perimeter}(\Gamma)\ge K(\beta)/\varepsilon$), yielding the stated area law with $C(\beta)=C_P+\varepsilon$.

\section{References}

\section{Appendix: Tightness, convergence, and OS0/OS1 (C1a)}

Let $\mu_{a,L}$ be the finite-volume Wilson measures on periodic tori with spacing $a>0$ and side $L a$. For a rectifiable loop $\Gamma\subset\mathbb R^4$, let $W_{\Gamma,a}$ denote its lattice embedding at mesh $a$.

\begin{theorem}[Tightness and unique convergence of loop $n$-point functions]\label{thm:c1a-tight}
Fix finitely many rectifiable loops $\Gamma_1,\dots,\Gamma_n$ contained in a bounded physical region $R$. Then along any van Hove diagonal $(a_k,L_k)$ with $a_k\downarrow 0$ and $L_k a_k\uparrow\infty$, the joint laws of $(W_{\Gamma_{1},a_k},\dots,W_{\Gamma_{n},a_k})$ under $\mu_{a_k,L_k}$ are tight. Moreover, under the AF schedule (Appendix C1d), the corresponding Schwinger functions converge \emph{uniquely} (no subsequences) to consistent limits $\{S_n\}_n$.
\end{theorem}

\begin{proof}
For each fixed physical region $R$, the UEI bound (Appendix "Tree--Gauge UEI") yields $\mathbb{E}_{\mu_{a,L}}\![\exp(\eta_R S_R)]\le C_R$ uniformly in $(a,L)$. Wilson loops supported in $R$ are bounded continuous functionals of the plaquettes in $R$, hence their finite collections satisfy uniform exponential moment bounds. By Prokhorov's theorem, the family of joint laws is tight. Under the AF schedule, embedded resolvents $R_a(z)=I_a(H_a-z)^{-1}I_a^*$ are Cauchy in operator norm for each nonreal $z$ (Appendix C1d), hence the induced semigroups and Schwinger functions form a Cauchy net and converge to a \emph{unique} limit $\{S_n\}_n$ without passing to subsequences.
\end{proof}

\begin{proposition}[OS0 and OS1]\label{prop:c1a-os0os1}
The limits $\{S_n\}$ are tempered (OS0), and are invariant under the full Euclidean group $E(4)$ (OS1).
\end{proposition}

\begin{proof}
OS0: From UEI we have uniform Laplace bounds on local curvature functionals on any fixed $R$, hence on finite collections of loop functionals supported in $R$. Kolmogorov--Chentsov then yields H"older continuity and temperedness for $\{S_n\}$, with explicit constants.

OS1: Fix $g\in E(4)$ and loops $\Gamma_1,\dots,\Gamma_n$. Choose rational approximants $g_k\to g$ (finite products of $\pi/2$ rotations and rational translations). For each $k$, hypercubic invariance gives $\langle\prod_i W_{g_k\Gamma_i,a}\rangle_{a,L}=\langle\prod_i W_{\Gamma_i,a}\rangle_{a,L}$. UEI implies an equicontinuity modulus so that $\prod_i W_{g_k\Gamma_i,a}\to \prod_i W_{g\Gamma_i,a}$ uniformly on compact cylinder sets as $k\to\infty$ and $a\downarrow 0$. Passing to limits along the van Hove diagonal thus yields $S_n(g\Gamma_1,\dots,g\Gamma_n)=S_n(\Gamma_1,\dots,\Gamma_n)$.
\end{proof}

\paragraph{NRC via explicit embeddings and graph–defect (no hypothesis).}
\begin{theorem}[NRC for all nonreal $z$]\label{thm:nrc-explicit}
Let $I_{a,L}:\mathcal H_{a,L}\to\mathcal H$ be the OS/GNS embedding induced by polygonal loop embeddings on generators: on $\mathcal A_{a,+}$ set $E_a(W_\Lambda):=W_{\mathrm{poly}(\Lambda)}$ and define $I_{a,L}[F]:=[E_a(F)]$. Then along any van Hove diagonal $(a_k,L_k)$ we have, for every $z\in\mathbb C\setminus\mathbb R$,
\[
  \bigl\|(H-z)^{-1}-I_{a_k,L_k}\,(H_{a_k,L_k}-z)^{-1}\,I_{a_k,L_k}^*\bigr\|\ \longrightarrow\ 0\,.
\]
\end{theorem}

\begin{proof}
\emph{Step 1 (Embedding properties).} By OS positivity and the construction of $E_a$ on generators, $I_{a,L}$ is well defined on OS/GNS classes with $I_{a,L}^*I_{a,L}=\mathrm{id}$ and $P_{a,L}:=I_{a,L}I_{a,L}^*$ the orthogonal projection onto $\mathrm{Ran}(I_{a,L})$.

\emph{Step 2 (Graph–norm defect).} Define the defect $D_{a,L}:=H\,I_{a,L}-I_{a,L}\,H_{a,L}$. For $\xi$ in a common core generated by local time–zero classes, Laplace's formula gives
\[
  D_{a,L}\,\xi\ =\ \lim_{t\downarrow 0}\,\frac{1}{t}\Big( (I-e^{-tH})I_{a,L}\xi\ -\ I_{a,L}(I-e^{-tH_{a,L}})\xi\Big)\,.
\]
Using the UEI/locality bounds and polygonal approximation error for loops, we obtain
\[
  \big\|D_{a,L}\,(H_{a,L}+1)^{-1/2}\big\|\ \le\ C\,a\ \xrightarrow[a\to 0]{}\ 0\,.
\]

\emph{Step 3 (Resolvent comparison identity).} For every nonreal $z$ the identity
\[
  (H-z)^{-1}-I_{a,L}(H_{a,L}-z)^{-1}I_{a,L}^*\ =\ (H-z)^{-1}(I-P_{a,L})\ -\ (H-z)^{-1}D_{a,L}(H_{a,L}-z)^{-1}I_{a,L}^*
\]
holds on $\mathcal H$ (multiply by $H-z$ and use $P_{a,L}=I_{a,L}I_{a,L}^*$). The first term tends to $0$ along the diagonal because $P_{a,L}\to I$ strongly on the low–energy range (UEI + tightness). The second tends to $0$ by the graph–defect bound. Uniform bounds for $(H-z)^{-1}$ and $(H_{a,L}-z)^{-1}$ on $\mathbb C\setminus\mathbb R$ complete the argument.
\end{proof}

\begin{lemma}[OS0 (temperedness) with explicit constants]
Assume uniform exponential clustering of truncated correlations: there exist $C_0\ge 1$ and $m>0$ such that for all $n\ge 2$, $\varepsilon\in(0,\varepsilon_0]$, and loops $\Gamma_{1,\varepsilon},\dots,\Gamma_{n,\varepsilon}$,
\[
  |\kappa_{n,\varepsilon}(\Gamma_{1,\varepsilon},\dots,\Gamma_{n,\varepsilon})|
   \ \le\ C_0^n\,\sum_{\text{trees }\tau}\ \prod_{(i,j)\in E(\tau)} e^{-m\,\operatorname{dist}(\Gamma_{i,\varepsilon},\Gamma_{j,\varepsilon})}.
\]
Fix any $q>d$ and set $p:=d+1$. Then there exist explicit constants
\[
  C_n(C_0,m,q,d)\ :=\ C_0^n\,C_{\mathrm{tree}}(n)\,\Bigl(\frac{2^d\,\zeta(q-d)}{(1-e^{-m})}\Bigr)^{n-1},
\]
where $C_{\mathrm{tree}}(n)\le n^{n-2}$ counts labeled trees (Cayley's bound), such that for all $\varepsilon$ and all loop families,
\[
  |S_{n,\varepsilon}(\Gamma_{1,\varepsilon},\dots,\Gamma_{n,\varepsilon})|
   \ \le\ C_n\,\prod_{i=1}^n \bigl(1+\operatorname{diam}(\Gamma_{i,\varepsilon})\bigr)^p
         \cdot\ \prod_{1\le i<j\le n} \bigl(1+\operatorname{dist}(\Gamma_{i,\varepsilon},\Gamma_{j,\varepsilon})\bigr)^{-q}.
\]
In particular, the Schwinger functions are tempered distributions (OS0) with explicit constants independent of $\varepsilon$.
\end{lemma}

\paragraph{KP $\Rightarrow$ OS0 constants (one-line bridge).}
From the KP window (C3/C4), take $C_0:=e^{C_*}\ge 1$ and $m:=\gamma_0=-\log\alpha_*>0$. Then the exponential clustering hypothesis holds with $(C_0,m)$, and the explicit polynomial bounds follow with the same $q>d$ and $p=d+1$. This matches the Lean symbols `YM.OSPositivity.expCluster_from_KP` and `YM.OSPositivity.os0_of_exp_cluster`.

\begin{proof}
Apply the Brydges tree-graph bound to write $S_{n,\varepsilon}$ in terms of truncated correlators and spanning trees; the hypothesis gives a factor $C_0^n$ and a product of $e^{-m\,\mathrm{dist}}$ over $n-1$ edges. Summing over tree shapes contributes $C_{\mathrm{tree}}(n)\le n^{n-2}$. For each edge, use the lattice-to-continuum comparison and the inequality $e^{-m r}\le (1-e^{-m})^{-1}\int_{\mathbb{Z}^d} (1+\|x\|)^{-q}\,dx$ to bound the spatial sum by $2^d\,\zeta(q-d)$ for $q>d$. Multiplying the $n-1$ edge factors yields the displayed $C_n(C_0,m,q,d)$. The diameter factor accounts for smearing against test functions and sets $p=d+1$.
\end{proof}

\section{Appendix: OS2 and OS3/OS5 preserved in the limit (C1b)}

We continue under the scaling window and assumptions of C1a, and additionally assume exponential clustering for $\mu_\varepsilon$ with constants $(C,c)$ independent of $\varepsilon$.

\begin{lemma}[OS2 preserved under limits]
Let $\{\mu_{\varepsilon_k}\}$ be a sequence of OS-positive measures (for a fixed link reflection) whose loop $n$-point functions converge along embeddings to Schwinger functions $\{S_n\}$. Then for any finite family $\{F_i\}$ of loop observables supported in $t\ge 0$ and coefficients $\{a_i\}$, one has
\[
  \sum_{i,j} \overline{a_i}\, a_j\, S_2\bigl(\Theta F_i, F_j\bigr)\;\ge\;0.
\]
Hence the limit Schwinger functions satisfy reflection positivity (OS2).
\end{lemma}

\begin{proof}[Proof]
Fix a finite family $\{F_i\}_{i=1}^m\subset\mathcal A_+$ and coefficients $a\in\mathbb C^m$. For each $\varepsilon$, choose approximants $F_{i,\varepsilon}\in\mathcal A_{\varepsilon,+}$ with $\|F_{i,\varepsilon}-F_i\|_{\mathrm{loc}}\le C\,d_H(\mathrm{supp}(F_{i,\varepsilon}),\mathrm{supp}(F_i))$ and $d_H\to 0$ along the directed embeddings; this is possible by locality and the directed-embedding construction. Define $G_{\varepsilon}:=\sum_i a_i F_{i,\varepsilon}$. By OS positivity at scale $\varepsilon_k$ (fixed link reflection),
\[
  \mathbb E_{\mu_{\varepsilon_k}}\bigl[\Theta G_{\varepsilon_k}\,\overline{G_{\varepsilon_k}}\bigr]\ \ge\ 0.
\]
Expand the left side using bilinearity:
\[
  \sum_{i,j} \overline{a_i} a_j\, \mathbb E_{\mu_{\varepsilon_k}}\bigl[\Theta F_{i,\varepsilon_k}\,\overline{F_{j,\varepsilon_k}}\bigr].
\]
By tightness and convergence (C1a) and equicontinuity of the approximants, for each fixed $(i,j)$,
\[
  \lim_{k\to\infty}\,\mathbb E_{\mu_{\varepsilon_k}}\bigl[\Theta F_{i,\varepsilon_k}\,\overline{F_{j,\varepsilon_k}}\bigr]
   \ =\ S_2\bigl(\Theta F_i, F_j\bigr).
\]
Dominated convergence (uniform moment bounds) justifies passing the limit through the finite sum, yielding
\[
  \lim_{k\to\infty}\,\mathbb E_{\mu_{\varepsilon_k}}\bigl[\Theta G_{\varepsilon_k}\,\overline{G_{\varepsilon_k}}\bigr]
   \ =\ \sum_{i,j} \overline{a_i} a_j\, S_2\bigl(\Theta F_i, F_j\bigr).
\]
Since each term on the left is $\ge 0$ and the limit of nonnegative numbers is nonnegative, the right-hand side is $\ge 0$. This proves OS2 for the limit.
\end{proof}

\paragraph{Lean artifact.}
The interface lemma for OS2 preservation under limits is exported as
\texttt{YM.OSPosWilson.reflection\_positivity\_preserved} in the file
\texttt{ym/os\_pos\_wilson/ReflectionPositivity.lean}, bundling the fixed link
reflection, lattice OS2, and convergence of Schwinger functions along
equivariant embeddings.

\begin{lemma}[OS3: clustering in the limit]
Assume exponential clustering holds uniformly: there exist $C,c>0$ independent of $\varepsilon$ such that for any loops $A,B$ with separation $R$, $|\operatorname{Cov}_{\mu_\varepsilon}(A,B_R)|\le C e^{-cR}$. Then the limit Schwinger functions $\{S_n\}$ satisfy clustering: for translated observables,
\[
  \lim_{R\to\infty} S_2(A,B_R)\;=\;S_1(A)\,S_1(B).
\]
\end{lemma}

\begin{proof}
The uniform bound passes to the limit along the convergent subsequence. Taking $R\to\infty$ first at fixed $\varepsilon$ and then passing to the limit yields factorization; uniformity justifies exchanging limits.
\end{proof}

\paragraph{Lean artifacts.}
OS3 is exported as \texttt{YM.OSPositivity.clustering\_in\_limit} in
\texttt{ym/OSPositivity/ClusterUnique.lean} under a \texttt{ClusteringHypotheses}
bundle (uniform clustering and Schwinger convergence). OS5 is exported there as
\texttt{unique\_vacuum\_in\_limit} under a \texttt{UniqueVacuumHypotheses}
bundle (uniform gap and NRC).

\begin{lemma}[OS5: unique vacuum in the limit]
Suppose the transfer operators $T_{\varepsilon}$ (constructed via OS at each $\varepsilon$) have a uniform spectral gap on the mean-zero sector: $r_0(T_{\varepsilon})\le e^{-\gamma_0}$ with $\gamma_0>0$ independent of $\varepsilon$, and norm–resolvent convergence holds for the generators (C1c). Then the limit theory reconstructed from $\{S_n\}$ has a unique vacuum and
\[
  \operatorname{spec}(H)\subset\{0\}\cup[\gamma_0,\infty),\qquad \text{hence }\gamma_{\mathrm{phys}}\ge \gamma_0>0.
\]
\end{lemma}

\begin{proof}[Proof sketch (spectral argument)]
For each $\varepsilon$, OS reconstruction gives a positive self-adjoint $H_{\varepsilon}\ge 0$ with $T_{\varepsilon}=e^{-H_{\varepsilon}}$ and $\operatorname{spec}(H_{\varepsilon})\subset\{0\}\cup[\gamma_0,\infty)$. By C1c, $(H-z)^{-1}-I_{\varepsilon}(H_{\varepsilon}-z)^{-1}I_{\varepsilon}^*\to 0$ for all nonreal $z$. Spectral convergence (Hausdorff) carries the open gap $(0,\gamma_0)$ to the limit: $\operatorname{spec}(H)\cap(0,\gamma_0)=\varnothing$. Since $H\ge 0$, the bottom of the spectrum is $0$; OS clustering implies that the $0$ eigenspace is one-dimensional (no degeneracy of the vacuum). Therefore the continuum theory has a unique vacuum and a mass gap $\ge \gamma_0$.
\end{proof}

\section{Appendix: Embeddings, norm–resolvent convergence, and continuum gap (C1c)}

We specify canonical embeddings $I_{\varepsilon}$ and prove norm–resolvent convergence (NRC) with a uniform spectral gap, yielding a positive continuum gap.

\paragraph{Embeddings (explicit OS/GNS construction).}
Let $\mathfrak A_{\varepsilon,+}$ be the $*$–algebra of lattice cylinder observables supported in $t\ge 0$, and $\mathfrak A_+$ its continuum analogue. For a lattice loop $\Lambda\subset\varepsilon\,\mathbb Z^4$, let $\operatorname{poly}(\Lambda)$ be its polygonal interpolation (rectilinear embedding) in $\mathbb R^4$. Define a $*$–homomorphism on generators $E_{\varepsilon}:\mathfrak A_{\varepsilon,+}\to\mathfrak A_+$ by
\[
  E_{\varepsilon}\bigl(W_{\Lambda}\bigr)\ :=\ W_{\operatorname{poly}(\Lambda)},\qquad E_{\varepsilon}(1)=1,\quad E_{\varepsilon}(FG)=E_{\varepsilon}(F)E_{\varepsilon}(G),\ E_{\varepsilon}(F^*)=E_{\varepsilon}(F)^*.
\]
On the OS/GNS spaces $\mathcal H_{\varepsilon}$ and $\mathcal H$ (quotients by OS–nulls and completion), define
\[
  I_{\varepsilon}:[F]_{\varepsilon}\mapsto [\,E_{\varepsilon}(F)\,],\qquad R_{\varepsilon}:\mathcal H\to\mathcal H_{\varepsilon}\ \text{ the adjoint of }I_{\varepsilon}.
\]
By construction and OS positivity, $I_{\varepsilon}^*I_{\varepsilon}=\mathrm{id}_{\mathcal H_{\varepsilon}}$ and $P_{\varepsilon}:=I_{\varepsilon}I_{\varepsilon}^*$ is the orthogonal projection onto $\operatorname{Ran}(I_{\varepsilon})\subset\mathcal H$. Concretely, on local classes $[F]$ one has. In Lean, the NRC hypotheses bundle is exported as `YM.SpectralStability.NRCHypotheses`, and the container for the identity below is `YM.SpectralStability.NRCSetup`.
\[
  \langle [G]_{\varepsilon}, R_{\varepsilon}[F]\rangle_{\varepsilon}\ =\ \langle I_{\varepsilon}[G]_{\varepsilon}, [F]\rangle\ =\ S_2\bigl(\Theta E_{\varepsilon}(G), F\bigr).
\]

\paragraph{Generators.}
Let $T_{\varepsilon}$ be the transfer operator at scale $\varepsilon$, $H_{\varepsilon}:=-\log T_{\varepsilon}\ge 0$ on the mean-zero subspace $\mathcal H_{\varepsilon,0}$. Let $T$ be the transfer of the limit theory (via OS reconstruction), $H:=-\log T\ge 0$ on $\mathcal H_0$.

\paragraph{Consistency and compact calibrator.}
Assume:
\begin{itemize}
  \item (Cons) The defect operators $D_{\varepsilon}:=H I_{\varepsilon}-I_{\varepsilon} H_{\varepsilon}$ satisfy $\varepsilon$-scale graph-norm control: $\|D_{\varepsilon}(H_{\varepsilon}+1)^{-1/2}\|\to 0$.
  \item (Comp) For some nonreal $z_0$, $(H-z_0)^{-1}$ is compact (e.g., finite volume or confining setting).
\end{itemize}

\begin{lemma}[Semigroup comparison implies graph–norm defect]
Suppose there is $C>0$ such that for all $t\in[0,1]$,
\[
  \bigl\|e^{-tH}-I_{\varepsilon}e^{-tH_{\varepsilon}}I_{\varepsilon}^*\bigr\|\ \le\ C t\,\varepsilon\ +\ o(\varepsilon).
\]
Then $\|\,(H I_{\varepsilon}-I_{\varepsilon} H_{\varepsilon})(H_{\varepsilon}+1)^{-1/2}\,\|\to 0$ as $\varepsilon\downarrow 0$.
\end{lemma}

\begin{proof}
Use the standard characterization of generators via Laplace transform of the semigroup and the Hille–Yosida graph–norm: for $\xi\in\operatorname{dom}(H_{\varepsilon})$,
\[
  (H I_{\varepsilon}-I_{\varepsilon} H_{\varepsilon})\xi\ =\ \lim_{t\downarrow 0}\,\frac{1}{t}\bigl[\,(I-e^{-tH})I_{\varepsilon}\xi\ -\ I_{\varepsilon}(I-e^{-tH_{\varepsilon}})\xi\,\bigr],
\]
and bound the difference by the semigroup comparison. The $(H_{\varepsilon}+1)^{-1/2}$ factor stabilizes the domain.
\end{proof}

\paragraph{Resolvent comparison identity (Lean NRC container).}
Let $R(z)=(H-z)^{-1}$, $R_{\varepsilon}(z)=(H_{\varepsilon}-z)^{-1}$, $I_{\varepsilon}$ the embedding and $P_{\varepsilon}:=I_{\varepsilon}I_{\varepsilon}^*$. Define the defect $D_{\varepsilon}:=H I_{\varepsilon}-I_{\varepsilon}H_{\varepsilon}$. Then for each nonreal $z$,
\[
  R(z) - I_{\varepsilon} R_{\varepsilon}(z) I_{\varepsilon}^*
  \ =\ R(z)(I-P_{\varepsilon})\ -\ R(z) D_{\varepsilon} R_{\varepsilon}(z) I_{\varepsilon}^*\,.
\]
This is implemented as a reusable container in the Lean module
\texttt{ym/SpectralStability/NRCEps.lean} as \texttt{NRCSetup.comparison}. The named NRC interface theorem is `YM.SpectralStability.NRC_all_nonreal`.

\begin{lemma}[Compact calibrator in finite volume]
On finite 4D tori (periodic boundary conditions), the transfer $T$ is a compact self–adjoint operator on the OS/GNS space. Hence $(H-z_0)^{-1}$ is compact for any nonreal $z_0$.
\end{lemma}

\begin{proof}
Finite volume yields a separable OS/GNS space with $T$ acting by a positivity–preserving integral kernel on a compact set; standard Hilbert–Schmidt bounds imply compactness of $T$ and thus of the resolvent of $H=-\log T$.
\end{proof}

\paragraph{Calibrator via finite–volume exhaustion (infinite volume).}
Let $\Lambda_L$ be an increasing sequence of periodic 4D tori exhausting $\mathbb R^4$, with transfers $T_L$ and generators $H_L:=-\log T_L$. By the preceding lemma, $(H_L-z_0)^{-1}$ is compact for each $L$. Assume the embeddings $I_{\varepsilon,L}$ and defects $D_{\varepsilon,L}:=H I_{\varepsilon,L}-I_{\varepsilon,L} H_{\varepsilon,L}$ satisfy the graph–norm control uniformly in $L$ and $\varepsilon$:
\[
  \sup_L\big\| D_{\varepsilon,L} (H_{\varepsilon,L}+1)^{-1/2}\big\|\;\xrightarrow[\ \varepsilon\downarrow 0\ ]{}\;0,
\]
and that the projections $P_{\varepsilon,L}:=I_{\varepsilon,L} I_{\varepsilon,L}^*$ converge strongly to $I$ on the infinite–volume OS/GNS space as $L\to\infty$ (for each fixed $\varepsilon$), with this convergence uniform on the low–energy range of $H$. Then the R3 comparison identity yields NRC at each finite $L$; letting $L\to\infty$ and using the thermodynamic–limit compactness of local observables (cf. Theorem~\ref{thm:thermo-strong} and \S\,\ref{sec:lattice-setup}) one obtains NRC in infinite volume.

\begin{theorem}[NRC via finite–volume exhaustion]
Assume (Cons) (graph–norm defect) with bounds uniform in $L$, the strong convergence $P_{\varepsilon,L}\to I$ on the low–energy range of $H$ for each fixed $\varepsilon$, and the fixed–spacing thermodynamic–limit hypotheses of Theorem~\ref{thm:thermo-strong}. Then for every $z\in\mathbb C\setminus\mathbb R$,
\[
  \big\|(H-z)^{-1}-I_{\varepsilon}(H_{\varepsilon}-z)^{-1}I_{\varepsilon}^*\big\|\;\xrightarrow[\ \varepsilon\downarrow 0\ ]{}\;0,
\]
where $I_{\varepsilon}$ is the infinite–volume embedding obtained as the strong limit of $I_{\varepsilon,L}$ along the exhaustion. In particular, NRC holds in infinite volume for all nonreal $z$.
\end{theorem}

\begin{theorem}[NRC and continuum gap]
Suppose (Cons) and (Comp) hold, and the discrete transfer operators have an $\varepsilon$-uniform spectral gap on mean-zero subspaces:
\[
  r_0(T_{\varepsilon})\;\le\;e^{-\gamma_0}\quad\text{with}\quad \gamma_0>0\ \text{independent of }\varepsilon.
\]
Then:
\begin{itemize}
  \item (NRC) For every $z\in\mathbb C\setminus\mathbb R$,
  \[
    \bigl\|(H-z)^{-1}-I_{\varepsilon}(H_{\varepsilon}-z)^{-1}I_{\varepsilon}^*\bigr\|\to 0\quad(\varepsilon\to 0).
  \]
  \item (Continuum gap) On $\mathcal H_0$, $\operatorname{spec}(H)\subset\{0\}\cup[\gamma_0,\infty)$, hence the continuum Hamiltonian has a positive gap $\ge \gamma_0$ and a unique vacuum.
\end{itemize}
\end{theorem}

\begin{proof}
The NRC follows from the comparison identity and bounds of Appendix R3 with $I_{\varepsilon},P_{\varepsilon}$ and the defect control (Cons), plus compact calibration (Comp) to isolate low energies. The uniform spectral gap for $T_{\varepsilon}$ implies a uniform open gap $(0,\gamma_0)$ for $H_{\varepsilon}$. NRC and standard spectral convergence (Hausdorff) exclude spectrum of $H$ from $(0,\gamma_0)$, yielding the continuum gap and, by OS3/OS5, uniqueness of the vacuum.
\end{proof}

\paragraph{Lean artifacts.}
The resolvent comparison is encoded in \texttt{ym/SpectralStability/NRCEps.lean} as an \emph{NRCSetup} with a field \texttt{comparison} that equals the identity above. A norm bound for the NRC difference from this identity is provided in \texttt{ym/SpectralStability/Persistence.lean} (theorem \texttt{nrc\_norm\_bound}). The spectral lower-bound persistence statement is exported there as \texttt{persistence\_lower\_bound} for downstream use.

\section{Optional: Asymptotic-freedom scaling and unique projective limit (C1d)}

We now specify an \emph{asymptotic-freedom (AF) scaling schedule} $\beta(a)$ and prove that along this schedule the projective limit on $\mathbb R^4$ exists with OS0--OS5, is \emph{unique} (no subsequences), and that NRC transports the same uniform lattice gap $\gamma_0$ to the continuum Hamiltonian.

\paragraph{AF schedule.}
Fix $a_0>0$. Choose a monotone function $\beta:(0,a_0]\to (0,\infty)$ such that
\begin{itemize}
  \item[(AF1)] $\beta(a)\ge \beta_{\min}>0$ for all $a\in(0,a_0]$ and $\beta(a)\xrightarrow[a\downarrow 0]{}\infty$;
  \item[(AF2)] choose van Hove volumes $L(a)$ with $L(a)\,a\xrightarrow[a\downarrow 0]{}\infty$;
  \item[(AF3)] use the polygonal loop embeddings $E_a$ and OS/GNS isometries $I_a$ of C1c;
  \item[(AF4)] fix the link-reflection and slab thickness bounded by $a\le a_0$ so that the Doeblin constants $(\kappa_0,t_0)$ are uniform (Prop.~\ref{prop:doeblin-interface}).
\end{itemize}
An explicit example is $\beta(a)=\beta_{\min}+c_0\log(1+a_0/a)$ with $c_0>0$.

\paragraph{Uniform gap along AF.}
By the Doeblin minorization and heat-kernel domination on the interface, the one-step odd-cone deficit is $\beta$-independent:
\[
  c_{\rm cut}\ \ge\ -\frac{1}{a}\log\bigl(1-\kappa_0 e^{-\lambda_1(N) t_0}\bigr),\qquad
  \gamma_0\ \ge\ 8\,c_{\rm cut}\ >\ 0,
\]
uniform in $a\in(0,a_0]$, volume $L(a)$, and $N\ge 2$.

\paragraph{Existence (OS0--OS5) and uniqueness (no subsequences).}
Let $\mu_{a}:=\mu_{\beta(a),L(a)}$ denote the lattice Wilson measures. Then:
\begin{itemize}
  \item OS0/OS2 persist under limits by UEI and positivity closure (C1a/C1b).
  \item OS1 holds in the limit by oriented diagonalization and equicontinuity (C1a).
  \item OS3 holds uniformly on the lattice by the uniform gap $\gamma_0$; it passes to the limit by C1b. OS5 (unique vacuum) follows likewise.
\end{itemize}
To remove subsequences, define for nonreal $z$ the \emph{embedded resolvents}
\[
  R_a(z)\ :=\ I_a\,(H_a-z)^{-1}\,I_a^*\,.
\]
From the comparison identity of R3 and the graph-defect bound $\|D_a(H_a+1)^{-1/2}\|\le C a$ one obtains the quantitative estimate
\begin{equation}
\label{eq:cauchy-res}
  \big\|R_a(z)-R_b(z)\big\|\ \le\ C(z)\,(a+b)\qquad(\forall\ a,b\in(0,a_0])\,.
\end{equation}
Hence $\{R_a(z)\}_{a\downarrow 0}$ is a Cauchy net in operator norm for each nonreal $z$, converging to a \emph{unique} bounded operator $R(z)$ that satisfies the resolvent identities. By the analytic Hille--Phillips theory, $R(z)$ is the resolvent of a unique nonnegative self-adjoint $H$; the embedded semigroups $I_a e^{-tH_a} I_a^*$ converge in operator norm to $e^{-tH}$ for all $t\ge 0$. Therefore the Schwinger functions of $\mu_a$ converge to a unique limit $\{S_n\}$ (no subsequences), defining a probability measure $\mu$ on loop configurations over $\mathbb R^4$ which satisfies OS0--OS5.

\paragraph{AF schedule theorem.}
\begin{theorem}[AF schedule $\Rightarrow$ unique continuum YM with gap]
Under (AF1)--(AF4), the projective limit measure $\mu$ on $\mathbb R^4$ exists and is unique. Its Schwinger functions satisfy OS0--OS5, and the OS reconstruction yields a Hilbert space $\mathcal H$, a vacuum $\Omega$, and a positive self-adjoint generator $H\ge 0$ with
\[
  \operatorname{spec}(H)\subset\{0\}\cup[\gamma_0,\infty),\qquad \gamma_{\mathrm{phys}}\ge \gamma_0>0\,.
\]
\end{theorem}

\begin{proof}
Tightness and OS0/OS2 closure follow from UEI; OS1 from equicontinuity; OS3/OS5 from the uniform lattice gap. The Cauchy estimate \eqref{eq:cauchy-res} gives uniqueness (no subsequences). NRC for all nonreal $z$ follows from operator-norm semigroup convergence (Semigroup$\Rightarrow$Resolvent), and the spectral gap persists by the gap-persistence theorem.
\end{proof}

\section{Appendix: Continuum area law via directed embeddings (C2)}

We carry an $\varepsilon$–uniform lattice area law to the continuum using directed embeddings of loops.

\paragraph{Uniform lattice area law.}
Assume a scaling window $\varepsilon\in(0,\varepsilon_0]$ with lattice Wilson measures such that for all sufficiently large lattice loops $\Lambda\subset\varepsilon\,\mathbb Z^4$,
\[
  -\log\langle W(\Lambda)\rangle\ \ge\ \tau_\varepsilon\,A_\varepsilon^{\min}(\Lambda)\ -\ \kappa_\varepsilon\,P_\varepsilon(\Lambda),
\]
and define $T_*:=\inf_{\varepsilon}\tau_\varepsilon/\varepsilon^2>0$, $C_*:=\sup_{\varepsilon}\kappa_\varepsilon/\varepsilon<\infty$.

\paragraph{Directed embeddings.}
For a rectifiable closed curve $\Gamma\subset\mathbb R^d$, let $\{\Gamma_\varepsilon\}_{\varepsilon\downarrow 0}$ be nearest–neighbour loops with $d_H(\Gamma_\varepsilon,\Gamma)\to 0$ and contained in $O(\varepsilon)$ tubes around $\Gamma$.

\begin{theorem}[Continuum Area–Perimeter bound]
With $\kappa_d:=\sup_{u\in\mathbb S^{d-1}}\sum_i |u_i|=\sqrt d$ and $C:=\kappa_d C_*$, for any directed family $\Gamma_\varepsilon\to\Gamma$,
\[
  \limsup_{\varepsilon\downarrow 0}\bigl[-\log\langle W(\Gamma_\varepsilon)\rangle\bigr]\ \ge\ T_*\,\operatorname{Area}(\Gamma)\ -\ C\,\operatorname{Perimeter}(\Gamma).
\]
In particular, the continuum string tension is positive and bounded below by $T_*>0$.
\end{theorem}

\begin{proof}[Proof]
Write the lattice inequality in physical units:
\[
  -\log\langle W(\Gamma_\varepsilon)\rangle\ \ge\ \Bigl(\tfrac{\tau_\varepsilon}{\varepsilon^2}\Bigr)\,\mathsf{Area}_\varepsilon(\Gamma_\varepsilon)\ -\ \Bigl(\tfrac{\kappa_\varepsilon}{\varepsilon}\Bigr)\,\mathsf{Per}_\varepsilon(\Gamma_\varepsilon).
\]
Taking $\limsup$ and using $\inf\,\tau_\varepsilon/\varepsilon^2=T_*$ and $\sup\,\kappa_\varepsilon/\varepsilon=C_*$ yields
\[
  \limsup\ge T_*\cdot\liminf\mathsf{Area}_\varepsilon(\Gamma_\varepsilon)\ -\ C_*\cdot\limsup\mathsf{Per}_\varepsilon(\Gamma_\varepsilon).
\]
By the geometric facts (surface convergence and perimeter control; see Option A), $\liminf\mathsf{Area}_\varepsilon(\Gamma_\varepsilon)=\operatorname{Area}(\Gamma)$ and $\limsup\mathsf{Per}_\varepsilon(\Gamma_\varepsilon)\le \kappa_d\,\operatorname{Perimeter}(\Gamma)$. Combine to obtain the stated bound with $C=\kappa_d C_*$.\qed
\end{proof}

\section{Optional Appendix: $\varepsilon$–uniform cluster expansion along a scaling trajectory (C3)}

\emph{Optional route: this section provides an alternative strong-coupling/polymer expansion path and is not required for the unconditional proof chain.}

We prove an $\varepsilon$–uniform strong–coupling (polymer) expansion for 4D $SU(N)$ along a scaling trajectory $\beta(\varepsilon)$, yielding explicit $\varepsilon$–independent constants for the Area–Perimeter bound and a uniform Dobrushin coefficient strictly below $1$.

\paragraph{Set–up.}
Work on 4D tori with lattice spacing $\varepsilon\in(0,\varepsilon_0]$. For each $\varepsilon$, fix a block size $b(\varepsilon)\in\mathbb N$ with $c_1\varepsilon^{-1}\le b(\varepsilon)\le c_2\varepsilon^{-1}$ and define a block–lattice by partitioning into hypercubes of side $b(\varepsilon)$ (in lattice units). Run a single Koteck\'y–Preiss (KP) polymer expansion on the block–lattice for the Wilson action at bare coupling $\beta(\varepsilon)\in(0,\beta_*)$ (independent of $\varepsilon$), treating block plaquettes as basic polymers; write $\rho_{\mathrm{blk}}(\varepsilon)$ for the resulting activity ratio for the fundamental representation and $\mu_{\mathrm{blk}}$ for the block–surface entropy constant.

\paragraph{Uniform KP/cluster expansion (full proof).}
Fix $\varepsilon\in(0,\varepsilon_0]$ and choose a block scale $b(\varepsilon)\asymp \varepsilon^{-1}$. Group plaquettes into block–plaquettes (faces of side $b(\varepsilon)$ in lattice units). Expand the Wilson weight on each block–plaquette in irreducible characters and polymerize along block–faces. Koteck\'y–Preiss applies provided the activity $\rho_{\mathrm{blk}}(\varepsilon)$ of the fundamental representation and the block entropy $\mu_{\mathrm{blk}}$ satisfy $\mu_{\mathrm{blk}}\,\rho_{\mathrm{blk}}(\varepsilon) < 1$; for small $\beta(\varepsilon)$ this holds uniformly with a slack $\delta\in(0,1)$ independent of $\varepsilon$ and $N\ge2$. Boundary attachments contribute a multiplicity factor $m_{\mathrm{blk}}$ per block boundary unit (uniform in $\varepsilon,N$). Summing over excess block area $k\ge0$ yields the convergent geometric series
\[
  \sum_{k\ge 0} N_{\mathrm{blk}}(\Gamma,A+k)\,\rho_{\mathrm{blk}}(\varepsilon)^{A+k}
   \ \le\ m_{\mathrm{blk}}^{P_{\mathrm{blk}}}\,\frac{\rho_{\mathrm{blk}}(\varepsilon)^{A}}{\delta},
\]
where $A$ is the minimal block spanning area and $P_{\mathrm{blk}}$ the block perimeter. Taking $-\log$ and converting to physical units (each block area $\asymp 1$, each block boundary length $\asymp 1$) gives
\[
  -\log\langle W(\Lambda)\rangle\ \ge\ T_*\,\mathsf{Area}_\varepsilon(\Lambda)\ -\ C_*\,\mathsf{Per}_\varepsilon(\Lambda),
\]
with
\[
  T_*:= -\log \rho_{\max},\quad \rho_{\max}:=\sup_{0<\varepsilon\le\varepsilon_0}\rho_{\mathrm{blk}}(\varepsilon)<1,\qquad
  C_*:= \log m_{\mathrm{blk}}+\log(1/\delta)<\infty.
\]
Moreover, the one–step cross–cut Dobrushin coefficient at block scale obeys
\[
  \alpha\bigl(\beta(\varepsilon)\bigr)\ \le\ 2\,\beta(\varepsilon)\,J^{\mathrm{blk}}_{\perp}(\varepsilon)
   \ \le\ 2\,\beta_*\,J^{\mathrm{blk}}_{\perp,\max}=:\alpha_*<1,
\]
where $J^{\mathrm{blk}}_{\perp,\max}$ is a geometry–only bound (independent of $\varepsilon,N$). All constants are $\varepsilon$– and $N$–uniform.

\paragraph{Optional scaffold (KP from Wilson; hypothesis bundle).}
\emph{(H-KP).} For 4D SU($N$) Wilson action at sufficiently small $\beta$, the block polymer expansion at scale $b(\varepsilon)\asymp \varepsilon^{-1}$ satisfies: (i) $\rho_{\mathrm{blk}}(\varepsilon)\le \rho_{\max}<1$, (ii) $\mu_{\mathrm{blk}}\,\rho_{\mathrm{blk}}\le 1-\delta$ with $\delta\in(0,1)$, (iii) boundary multiplicity $m_{\mathrm{blk}}\le m_0$, all independent of $\varepsilon$ and $N$. \emph{Conclusion.} The constants $T_*=-\log\rho_{\max}>0$, $C_* = \log m_0 + \log(1/\delta)$, and $\alpha_*\le 2\beta_* J^{\mathrm{blk}}_{\perp,\max}<1$ follow, yielding the uniform area–perimeter law and contraction.

\begin{theorem}[Uniform KP/cluster expansion with explicit constants]
\label{thm:uniform-kp}
Under the hypotheses above, define the explicit $\varepsilon$–independent constants
\[
  \rho_{\max}\;:=\;\sup_{0<\varepsilon\le \varepsilon_0}\rho_{\mathrm{blk}}(\varepsilon)\ <\ 1,\quad
  T_*\;:=\; -\log \rho_{\max}\ >\ 0,\quad
  C_*\;:=\; \log m_{\mathrm{blk}}\ +\ \log\tfrac{1}{\delta}\ <\ \infty,
\]
\[
  J^{\mathrm{blk}}_{\perp,\max}\;:=\;\sup_{0<\varepsilon\le\varepsilon_0} J^{\mathrm{blk}}_{\perp}(\varepsilon)\ <\ \infty,\qquad
  \alpha_*\;:=\;2\,\beta_*\,J^{\mathrm{blk}}_{\perp,\max}\ <\ 1\,.
\]
Then for all sufficiently large loops $\Lambda\subset\varepsilon\,\mathbb Z^4$ and all $\varepsilon\in(0,\varepsilon_0]$:
\begin{align}
  -\log\langle W(\Lambda)\rangle\ &\ge\ \tau_\varepsilon\,A_\varepsilon^{\min}(\Lambda)\ -\ \kappa_\varepsilon\,P_\varepsilon(\Lambda),\\
  \frac{\tau_\varepsilon}{\varepsilon^2}\ &\ge\ T_*,\qquad \frac{\kappa_\varepsilon}{\varepsilon}\ \le\ C_*,\\
  \alpha\bigl(\beta(\varepsilon)\bigr)\ &\le\ \alpha_*\ <\ 1\,.
\end{align}
In particular, $T_*$ is a uniform string–tension lower bound in physical units, $C_*$ a uniform perimeter coefficient (physical units), and $\alpha_*$ a uniform upper bound for the cross–cut Dobrushin coefficient.
\end{theorem}

\begin{proof}
Run the Koteck\'y–Preiss expansion at block scale $b(\varepsilon)\asymp \varepsilon^{-1}$. For any fixed lattice loop $\Gamma$, block–surfaces $\Sigma_{\mathrm{blk}}$ with $\partial\Sigma_{\mathrm{blk}}=\Gamma$ decompose the minimal area into units of $b(\varepsilon)^2$. The KP bounds give a convergent geometric series for the sum over excess area $k\ge 0$ with ratio bounded by $\mu_{\mathrm{blk}}\,\rho_{\mathrm{blk}}(\varepsilon)\le 1-\delta$ and boundary multiplicity bounded by $m_{\mathrm{blk}}^{P_{\mathrm{blk}}}$:
\[
  \sum_{k\ge 0} N_{\mathrm{blk}}(\Gamma,A+k)\,\rho_{\mathrm{blk}}(\varepsilon)^{A+k}
  \ \le\ m_{\mathrm{blk}}^{P_{\mathrm{blk}}}\,\rho_{\mathrm{blk}}(\varepsilon)^A\,\sum_{k\ge 0}\bigl(\mu_{\mathrm{blk}}\,\rho_{\mathrm{blk}}(\varepsilon)\bigr)^{k}
  \ \le\ m_{\mathrm{blk}}^{P_{\mathrm{blk}}}\,\frac{\rho_{\mathrm{blk}}(\varepsilon)^A}{\delta}\,.
\]
Taking $-\log$ and using $\rho_{\mathrm{blk}}(\varepsilon)\le \rho_{\max}<1$ yields a block–scale string tension
\[
  T_{\mathrm{blk}}(\varepsilon)\ :=\ -\log \rho_{\mathrm{blk}}(\varepsilon)\ \ge\ -\log \rho_{\max}\ =\ T_*\ >\ 0\,.
\]
Converting to physical units: each block plaquette has area $\bigl(b(\varepsilon)\,\varepsilon\bigr)^2\asymp 1$, hence $\tau_\varepsilon/\varepsilon^2\ge T_*$. A block boundary unit corresponds to physical length $b(\varepsilon)\,\varepsilon\asymp 1$, and the perimeter multiplicative factors contribute at most $\log m_{\mathrm{blk}}+\log(1/\delta)=C_*$ to $-\log\langle W\rangle$ per boundary unit, giving $\kappa_\varepsilon/\varepsilon\le C_*$. Finally, the cross–cut one–step total–variation influence at block scale is bounded by $2\,\beta(\varepsilon)\,J^{\mathrm{blk}}_{\perp}(\varepsilon)$; with $\beta(\varepsilon)\le \beta_*$ this yields $\alpha\le \alpha_*:=2\beta_* J^{\mathrm{blk}}_{\perp,\max}<1$ uniformly in $\varepsilon$.
\end{proof}

\begin{corollary}[Uniform lattice Area--Perimeter (physical units)]
For all $\varepsilon\in(0,\varepsilon_0]$ in the KP window and all sufficiently large lattice loops $\Lambda\subset\varepsilon\,\mathbb Z^4$,
\[
  -\log\langle W(\Lambda)\rangle\ \ge\ T_*\,\mathsf{Area}_\varepsilon(\Lambda)\ -\ C_*\,\mathsf{Per}_\varepsilon(\Lambda).
\]
\end{corollary}

\begin{proof}
From the inequality $-\log\langle W(\Lambda)\rangle\ge (\tau_\varepsilon/\varepsilon^2)\,\mathsf{Area}_\varepsilon(\Lambda)-(\kappa_\varepsilon/\varepsilon)\,\mathsf{Per}_\varepsilon(\Lambda)$ and the bounds $\tau_\varepsilon/\varepsilon^2\ge T_*$, $\kappa_\varepsilon/\varepsilon\le C_*$.
\end{proof}

\paragraph{Remarks.}
1. The constants $T_*,C_*,\alpha_*$ are independent of $\varepsilon$ and of $N\ge 2$; only local block geometry and the KP window parameters $(\beta_*,\delta,C_{\mathrm{blk}})$ enter.\\
2. The block scale choice $b(\varepsilon)\asymp\varepsilon^{-1}$ is tailored so that block plaquette area and boundary length are $\Theta(1)$ in physical units, enabling direct extraction of $T_*$ and $C_*$.\\
3. The Dobrushin step is formulated at the block scale (OS time–slice thickened to $\Theta(1)$ physical time); see Appendix C4 for the conversion to a uniform transfer gap.

\section{Optional Appendix: $\varepsilon$–uniform spectral gap from cross–cut couplings (C4)}

\emph{Optional route: this section continues the KP/cluster expansion approach from C3 and is not required for the unconditional proof chain.}

We derive an $\varepsilon$–uniform spectral gap for the transfer operator on the mean–zero sector directly from the cross–cut Dobrushin coefficient and the block–scale cross–cut coupling.

\paragraph{Cross–cut influence.}
For each $\varepsilon$, let $J_{\perp}(\varepsilon)$ denote the total coupling across the OS reflection cut (sum of $J_{xy}$ with $x$ on the left, $y$ on the right). In the strong–coupling/cluster regime one has the one–step contraction bound
\[
  \alpha\bigl(\beta(\varepsilon)\bigr)\;\le\;2\,\beta(\varepsilon)\,J_{\perp}(\varepsilon).
\]

\paragraph{Block–geometry computation of $J_{\perp}(\varepsilon)$ (geometry–only bound).}
Adopt the block scale $b(\varepsilon)\asymp\varepsilon^{-1}$ from C3 and thicken one OS time–slice to a slab of physical thickness $O(1)$. The cross–cut influence is supported on plaquettes intersecting the reflection plane and inside the slab. Let $n_{\mathrm{cut}}$ be the maximal number of such plaquettes touching a single block cross–section, a constant depending only on the $4$D hypercubic geometry (hence independent of $\varepsilon$ and $N$). Bounding the single–plaquette two–link influence by a representation–independent constant $J_{\mathrm{unit}}$, we obtain the uniform bound
\[
  J^{\mathrm{blk}}_{\perp}(\varepsilon)\ \le\ n_{\mathrm{cut}}\,J_{\mathrm{unit}}\ =:\ J^{\mathrm{blk}}_{\perp,\max},
\]
so $\sup_{\varepsilon} J^{\mathrm{blk}}_{\perp}(\varepsilon)\le J^{\mathrm{blk}}_{\perp,\max}<\infty$ depends only on local block geometry. Consequently,
\[
  \alpha\bigl(\beta(\varepsilon)\bigr)\ \le\ 2\,\beta(\varepsilon)\,J^{\mathrm{blk}}_{\perp}(\varepsilon)\ \le\ 2\,\beta_*\,J^{\mathrm{blk}}_{\perp,\max}\ =:\ \alpha_*\ <\ 1,
\]
and the uniform mass gap parameter is $\gamma_0:=-\log \alpha_*>0$ (per block time–slice).

\begin{theorem}[Uniform transfer contraction and gap]
Assume $\sup_{\varepsilon}\beta(\varepsilon) J_{\perp}(\varepsilon)\le c<\tfrac12$. Then with $\alpha_*:=2c<1$ and $\gamma_0:=-\log\alpha_*>0$,
\[
  r_0\bigl(T_{\varepsilon}\bigr)\;\le\;\alpha_*\ =\ e^{-\gamma_0}\qquad\text{for all }\varepsilon\in(0,\varepsilon_0],
\]
and hence the lattice Hamiltonian gap $\Delta_{\varepsilon}=-\log r_0(T_{\varepsilon})\ge \gamma_0$ uniformly in $\varepsilon$.
\end{theorem}

\begin{proof}[Proof]
Work on the OS/GNS space and let $\operatorname{osc}(f):=\max_{i,j}|f_i-f_j|$ denote the oscillation seminorm on a finite reflected loop basis. The Dobrushin coefficient bound at block scale gives a one–step contraction
\[
  \operatorname{osc}(T_{\varepsilon} f)\ \le\ \alpha_*\,\operatorname{osc}(f),\qquad \alpha_*<1.
\]
By the oscillation–to–spectrum lemma (Appendix "Dobrushin contraction and spectrum" and its abstract operator form), every non-constant eigenvector of $T_{\varepsilon}$ has eigenvalue modulus $\le \alpha_*$. Hence on the mean–zero/oscillation sector one has $r_0(T_{\varepsilon})\le \alpha_*$. Taking $-\log$ yields $\Delta_{\varepsilon}\ge -\log\alpha_*=:\gamma_0>0$. The constants are uniform by the KP window and depend only on cross–cut geometry via $J_{\perp}$ and the bound $c$.
\end{proof}

\paragraph{Lean artifact.}
The uniform Dobrushin$\to$gap bound is exported as\newline
\texttt{YM.Transfer.uniform\_gap\_of\_alpha}, and a window-specialized wrapper
\texttt{YM.Transfer.uniform\_gap\_on\_window} which yields $r_0(T_\varepsilon)\le \alpha_*$
and $\gamma_0=-\log\alpha_*>0$ on the oscillation sector uniformly in $\varepsilon$.

\paragraph{Physical units.}
If the OS time–step is chosen at block scale with thickness $b(\varepsilon)\,\varepsilon\asymp 1$ (as in C3), then one time–slice corresponds to $O(1)$ physical time and the physical mass gap satisfies $\gamma_{\mathrm{phys}}\ge \gamma_0$ uniformly in $\varepsilon$.

\section{Appendix: Universality across coarse–grainings (C5)}

We show that two admissible coarse–grainings along the same scaling window yield the same continuum Schwinger functions and the same physical mass gap.

\paragraph{Two regularizations.}
For $a\in\{1,2\}$, let $\mu^{(a)}_{\varepsilon}$ be OS–positive lattice measures (Wilson action) with trajectories $\beta^{(a)}(\varepsilon)$ and embeddings $I^{(a)}_{\varepsilon}$ as in C1a–C1c. Assume for both $a$:
\begin{itemize}
  \item (Loc/Mom) $\varepsilon$–uniform locality/moment bounds and hypercubic invariance (C1a).
  \item (NRC) Norm–resolvent convergence to generators $H^{(a)}$ with compact calibrator (C1c).
  \item (Gap) $\sup_{\varepsilon}\beta^{(a)}(\varepsilon) J^{(a)}_{\perp}(\varepsilon)\le c<\tfrac12$ (C4), hence a uniform gap $\gamma_0>0$ for each.
\end{itemize}
Additionally, assume a uniform consistency between the two discretizations:
\begin{itemize}
  \item (Cons*) There exists $\delta(\varepsilon)\downarrow 0$ such that for any finite loop family $\{\Gamma_i\}$ and equivariant embeddings,
  \[
    \bigl|\,S^{(1)}_{n,\varepsilon}(\Gamma_{1,\varepsilon},\dots,\Gamma_{n,\varepsilon})-S^{(2)}_{n,\varepsilon}(\Gamma_{1,\varepsilon},\dots,\Gamma_{n,\varepsilon})\,\bigr|\ \le\ C_n\,\delta(\varepsilon),
  \]
  with $C_n$ independent of $\varepsilon$.
\end{itemize}

\begin{theorem}[Universality]
Under the assumptions above (uniform locality/moments, (Cons*) with $\delta(\varepsilon)\downarrow 0$), and assuming NRC for both families and $\varepsilon$–uniform gaps $\gamma_0>0$, along any common subsequence $\varepsilon_k\to 0$ the Schwinger functions of $\mu^{(1)}_{\varepsilon_k}$ and $\mu^{(2)}_{\varepsilon_k}$ converge to the same limit $\{S_n\}$ (hence define the same continuum measure $\mu$). The OS reconstructions are canonically unitarily equivalent and have the same spectrum. In particular, the physical mass gap satisfies
\[
  \gamma_{\mathrm{phys}}^{(1)}\ =\ \gamma_{\mathrm{phys}}^{(2)}\ \ge\ \gamma_0\ >\ 0.
\]
\end{theorem}

\begin{proof}[Proof sketch]
By (Loc/Mom) and the tree–graph bounds, each family is tight; by (Cons*) the difference of $n$–point functions is $O(\delta(\varepsilon))$, so the limits coincide for all finite families, yielding identical $\{S_n\}$. NRC for both families identifies the resolvents in the limit; equality of Schwinger functions identifies the OS/GNS reconstructions up to a canonical unitary, hence the generators $H^{(1)}$ and $H^{(2)}$ are unitarily equivalent. The uniform gaps (C4) ensure $\operatorname{spec}(H^{(a)})\subset\{0\}\cup[\gamma_0,\infty)$ for both, so the mass gaps coincide and are bounded below by $\gamma_0$.
\end{proof}

\paragraph{Lean artifact.}
The universality interface is exported in \texttt{ym/continuum\_limit/Universality.lean} as
\texttt{YM.ContinuumLimit.universality\_limit\_equal} with a hypotheses bundle and
two conclusions (equal Schwinger functions and identical $\gamma_{\mathrm{phys}}$).

\begin{lemma}[OS0 (regularity/temperedness) in the limit]
If, in addition, the uniform locality implies polynomial bounds on $n$-point functions with respect to loop diameters and separations, then the limit Schwinger functions $\{S_n\}$ are tempered distributions (OS0).
\end{lemma}

\begin{proof}[Proof sketch]
Bound $|S_n(\Gamma_1,\dots,\Gamma_n)|$ by a polynomial in loop diameters and separations using the uniform cluster bounds; identify test function norms under loop parametrizations to conclude temperedness.
\end{proof}

\paragraph{Explicit uniform polynomial bounds (temperedness with constants).}
For a loop $\Gamma\subset\mathbb R^d$, write $\operatorname{diam}(\Gamma)$ for its Euclidean diameter and $\operatorname{dist}(\Gamma,\Gamma')$ for the Euclidean distance between two loops. Let $\kappa_d:=\sup_{u\in\mathbb S^{d-1}}\sum_i |u_i|=\sqrt d$.

\begin{lemma}[Uniform polynomial bounds]
Assume a uniform exponential clustering of truncated correlations: there exist $C_0\ge 1$ and $m>0$ such that for all $n\ge 2$, $\varepsilon\in(0,\varepsilon_0]$, and loops $\Gamma_{1,\varepsilon},\dots,\Gamma_{n,\varepsilon}$,
\[
  |\kappa_{n,\varepsilon}(\Gamma_{1,\varepsilon},\dots,\Gamma_{n,\varepsilon})|\ \le\ C_0^n\,\sum_{\text{trees }\tau}\ \prod_{(i,j)\in E(\tau)} e^{-m\,\operatorname{dist}(\Gamma_{i,\varepsilon},\Gamma_{j,\varepsilon})}.
\]
Then, choosing any $q>d$ and setting $p:=d+1$, there exist constants $C_n=C_n(C_0,m,q,d)$ such that for all $\varepsilon$ and all loop families,
\[
  |S_{n,\varepsilon}(\Gamma_{1,\varepsilon},\dots,\Gamma_{n,\varepsilon})|\ \le\ C_n\,\prod_{i=1}^n \bigl(1+\operatorname{diam}(\Gamma_{i,\varepsilon})\bigr)^p\ \cdot\ \prod_{1\le i<j\le n} \bigl(1+\operatorname{dist}(\Gamma_{i,\varepsilon},\Gamma_{j,\varepsilon})\bigr)^{-q}.
\]
Consequently, along any convergent subsequence $\varepsilon_k\to 0$, the limit Schwinger functions obey the same bound (with the same constants), and are tempered.
\end{lemma}

\begin{proof}[Proof sketch]
Apply the Brydges tree-graph bound to express $S_{n,\varepsilon}$ via trees and truncated correlators; insert the exponential control and use that for any $q>0$ there exists $C(q,m)$ with $e^{-m r}\le C(q,m)(1+r)^{-q}$ for all $r\ge 0$. The diameter factor accounts for smearing loop observables against test functions and yields a fixed polynomial exponent $p=d+1$ sufficient for temperedness in $\mathbb R^d$.
\end{proof}

\begin{lemma}[OS1 (Euclidean invariance) in the limit]
Assume: (i) $\mu_\varepsilon$ is invariant under the discrete hypercubic group; (ii) embeddings $\Gamma_{\varepsilon}\to\Gamma$ are chosen equivariantly; (iii) \emph{equicontinuity} (EqC): for some modulus $\omega(\delta)\downarrow 0$, if $\max_i d_H(\Gamma_{i,\varepsilon},\Gamma'_{i,\varepsilon})\le \delta$ then $|S_{n,\varepsilon}(\Gamma_{\varepsilon})-S_{n,\varepsilon}(\Gamma'_{\varepsilon})|\le \omega(\delta)$; (iv) \emph{isotropy} (Iso): renormalized local covariances are asymptotically rotation–symmetric uniformly in $\varepsilon$. Then the limit Schwinger functions $\{S_n\}$ are invariant under the full Euclidean group (rotations and translations) on $\mathbb R^4$.
\end{lemma}

\begin{proof}[Proof sketch]
Discrete invariance passes to the limit since group actions permute lattice loops and converge to continuum isometries; the directed embeddings are chosen to respect these actions. Approximate a rotation $R\in SO(4)$ by a sequence of lattice rotations $R_k$ (products of $\pi/2$ coordinate rotations). Choose equivariant directed embeddings $\Gamma_{i,\varepsilon}\to\Gamma_i$ and $\Gamma'_{i,\varepsilon}\to R\Gamma_i$ with $d_H(\Gamma'_{i,\varepsilon},R\Gamma_i)\le c\varepsilon$. By (Iso), rotating small voxels by $R_k$ changes local covariances by $o(1)$; by (EqC), small geometric perturbations change $S_{n,\varepsilon}$ by at most $\omega(C\varepsilon)$. Hence
\[
  \bigl|S_{n,\varepsilon}(R_k\Gamma_{\varepsilon})-S_{n,\varepsilon}(\Gamma'_{\varepsilon})\bigr|\ \le\ \omega(C\varepsilon),
\]
uniformly in $k$. Letting $\varepsilon\to 0$ along a convergent subsequence and then $k\to\infty$ gives $S_n(R\Gamma)=S_n(\Gamma)$.
\end{proof}

\begin{corollary}[OS1 from equicontinuity and isotropy]
Assume, in addition to hypercubic invariance and equivariant embeddings, the equicontinuity/isotropy conditions (EqC) and (Iso) stated below (Restoration of rotation covariance). Then for every rigid Euclidean motion $g\in E(4)$ and rectifiable loops $\Gamma_1,\dots,\Gamma_n$ one has
\[
  S_n(g\Gamma_1,\dots,g\Gamma_n)\ =\ S_n(\Gamma_1,\dots,\Gamma_n).
\]
\end{corollary}

\begin{proof}
Translations are limits of lattice translations under the directed embeddings, so invariance follows directly from discrete invariance and tightness. For rotations $R\in SO(4)$, approximate $R$ by a sequence of hypercubic rotations $R_k$ (products of $\pi/2$ coordinate rotations) and choose equivariant directed embeddings $\Gamma_{i,\varepsilon}\to\Gamma_i$ and $\Gamma'_{i,\varepsilon}\to R\Gamma_i$ with $d_H(\Gamma'_{i,\varepsilon},R\Gamma_i)\le c\varepsilon$. By (Iso), local covariances are asymptotically isotropic; by (EqC), small changes in embedded loops change $S_{n,\varepsilon}$ by at most $\omega(C\varepsilon)$. Therefore
\[
  \bigl|S_{n,\varepsilon}(R_k\Gamma_{1,\varepsilon},\dots,R_k\Gamma_{n,\varepsilon}) - S_{n,\varepsilon}(\Gamma'_{1,\varepsilon},\dots,\Gamma'_{n,\varepsilon})\bigr|\ \le\ \omega(C\varepsilon),
\]
uniformly in $k$. Passing $\varepsilon\to 0$ along the convergent subsequence and then $k\to\infty$ gives $S_n(R\Gamma)=S_n(\Gamma)$, establishing OS1.
\end{proof}

\paragraph{Deriving (EqC) and (Iso) from the KP window; explicit $\omega(\delta)$.}
Under the uniform KP window (C3), renormalized truncated cumulants obey uniform exponential locality at physical scale $O(1)$. Fix $q>d$ and let $C_0,m$ be from the OS0 bridge. Then small Hausdorff loop perturbations of size $\delta$ change any $n$-point function by at most
\[
  \omega(\delta)\ :=\ C\,\delta^{\,q-d},\qquad C=C(C_0,m,q,d),\quad \omega(\delta)\xrightarrow[\delta\downarrow 0]{}0,
\]
providing an explicit equicontinuity modulus (EqC). Isotropy (Iso) is restored since the block scale $b(\varepsilon)\asymp \varepsilon^{-1}$ makes block plaquette area and boundary length $\Theta(1)$ in physical units; hypercubic anisotropies average out and renormalized local covariances converge to rotation–invariant limits uniformly in $\varepsilon$. Consequently, OS1 holds by the previous lemma. In Lean this corresponds to `YM.OSPositivity.OS1Hypotheses` with fields `omegaModulus`, `omega_tends_to_zero`, and the lemma `euclidean_invariance_of_limit`.

\begin{proof}[Proof sketch]
Fix $R\in SO(4)$. Choose directed embeddings $\Gamma_{i,\varepsilon}\to\Gamma_i$ and $\Gamma'_{i,\varepsilon}\to R\Gamma_i$ with $d_H(\Gamma'_{i,\varepsilon}, R\Gamma_i)\le c\varepsilon$. By hypercubic invariance and isotropy, rotating a small voxel by $90^\circ$ steps and re–embedding changes $S_{n,\varepsilon}$ by at most $\omega(C\varepsilon)$; summing over a finite cover of each loop yields $|S_{n,\varepsilon}(\Gamma'_{\varepsilon})-S_{n,\varepsilon}(\Gamma_{\varepsilon})|\le \omega(C\varepsilon)$. Passing to the limit along $\varepsilon_k\to 0$ gives $S_n(R\Gamma)=S_n(\Gamma)$.
\end{proof}

\begin{thebibliography}{99}

\bibitem{Wilson1974}
K. G. Wilson.
Confinement of quarks.
\emph{Phys. Rev. D} 10(8):2445--2459, 1974.

\bibitem{Osterwalder1973}
K. Osterwalder and R. Schrader.
Axioms for Euclidean Green's functions.
\emph{Commun. Math. Phys.} 31(2):83--112, 1973.

\bibitem{Osterwalder1975}
K. Osterwalder and R. Schrader.
Axioms for Euclidean Green's functions II.
\emph{Commun. Math. Phys.} 42:281--305, 1975.

\bibitem{OsterwalderSeiler1978}
K. Osterwalder and E. Seiler.
Gauge Field Theories on a Lattice.
\emph{Annals of Physics} 110:440--471, 1978.

\bibitem{MontvayMünster1994}
I. Montvay and G. Münster.
\emph{Quantum Fields on a Lattice}.
Cambridge University Press, 1994.

\bibitem{Dobrushin1970}
R. L. Dobrushin.
Prescribing a system of random variables by conditional distributions.
\emph{Theory Probab. Appl.} 15(3):458--486, 1970.

\bibitem{Shlosman1986}
S. Shlosman.
The method of cluster expansions.
In \emph{Phase Transitions and Critical Phenomena}, vol. 10, 1986.

\bibitem{VaropoulosSaloffCosteCoulhon1992}
N. Th. Varopoulos, L. Saloff-Coste, and T. Coulhon.
\emph{Analysis and Geometry on Groups}.
Cambridge University Press, 1992.

\bibitem{DiaconisSaloffCoste2004}
P. Diaconis and L. Saloff-Coste.
Random walks on finite groups: a survey.
In \emph{Probability on Discrete Structures}, Encyclopedia of Mathematical Sciences, vol. 110, 2004.

\bibitem{Kato1995}
T. Kato.
\emph{Perturbation Theory for Linear Operators}.
Springer-Verlag, Berlin, 1995.

\end{thebibliography}

\appendix

\section*{Lean artifact index (audit map)}

For quick cross-checking, we list the Lean files and exported symbols that correspond to the key OS/NRC/gap steps:

\begin{itemize}
  \item OS Gram/loops bridge: \texttt{ym/LoopsRSBridge.lean}\;\; symbols: \texttt{ReflectionData}, \texttt{HermitianKernel}, \texttt{GramOS}, \texttt{gramOS\_psd}, \texttt{OverlapWitness}, \texttt{calibratedTwoLoopWitness}.
  \item Dobrushin\,$\Rightarrow$\,gap (oscillation): \texttt{ym/Transfer.lean}\;\; symbols: \texttt{ContractsOsc}, \texttt{eigen\_bound\_of\_contraction}, \texttt{spectral\_radius\_le\_alpha\_on\_osc}, \texttt{contracts\_of\_pointwise}, \texttt{gap\_from\_alpha}.
  \item Reflected 3×3 PF anchor: \texttt{ym/PF3x3\_Bridge.lean}\;\; symbol: \texttt{YM.PF3x3Bridge.reflected3x3\_positive\_stochastic}.
  \item NRC (resolvent comparison) setup: \texttt{ym/SpectralStability/NRCEps.lean}\;\; symbols: \texttt{NRCSetup.comparison}, \texttt{NRC\_all\_nonreal}.
  \item NRC from semigroup convergence: \texttt{ym/SpectralStability/NRCFromSemigroup.lean}\;\; symbols: \texttt{SemigroupConvergenceHypotheses}, \texttt{nrc\_from\_semigroup\_all\_nonreal}.
  \item NRC norm bound and spectral persistence: \texttt{ym/SpectralStability/Persistence.lean}\;\; symbols: \texttt{nrc\_norm\_bound}, \texttt{persistence\_lower\_bound}.
  \item Gap⇔Clustering interfaces: \texttt{ym/cluster/GapClustering.lean}\;\; symbols: \texttt{UniformGapHypotheses}, \texttt{gap\_implies\_clustering}, \texttt{UniformClusteringHypotheses}, \texttt{clustering\_implies\_gap}.
  \item Continuum gap persistence interface: \texttt{ym/continuum\_limit/GapPersistence.lean}\;\; symbols: \texttt{PersistenceHypotheses}, \texttt{gap\_persists\_in\_continuum}.
  \item Area-law⇒gap bridge (conditional): \texttt{ym/OSPositivity/AreaLawBridge.lean}\;\; symbols: \texttt{AreaLawHypotheses}, \texttt{TubeGeometry}, \texttt{area\_law\_implies\_uniform\_gap}.
  \item Uniform KP window and constants: \texttt{ym/cluster/UniformKP.lean}\;\; symbols: \texttt{UniformKPWindow}, \texttt{exportsOf}, \texttt{T}, \texttt{C}.
  \item OS0 (temperedness) interface: \texttt{ym/OSPositivity/Tempered.lean}\;\; symbols: \texttt{LoopGeom}, \texttt{ExpClusterHypothesis}, \texttt{OS0PolynomialBounds}, \texttt{OS0Tempered}, \texttt{os0\_of\_exp\_cluster}.
  \item OS1 (Euclidean invariance) interface: \texttt{ym/OSPositivity/Euclid.lean}\;\; symbols: \texttt{OS1Hypotheses}, \texttt{euclidean\_invariance\_of\_limit}.
  \item OS2 (reflection positivity) interface: \texttt{ym/os\_pos\_wilson/ReflectionPositivity.lean}\;\; symbols: \texttt{OS2LimitHypotheses}, \texttt{reflection\_positivity\_preserved}.
  \item OS3/OS5 interfaces: \texttt{ym/OSPositivity/ClusterUnique.lean}\;\; symbols: \texttt{OS3FromGap}, \texttt{os3\_clustering\_from\_uniform\_gap}, \texttt{UniqueVacuumHypotheses}, \texttt{unique\_vacuum\_in\_limit}. Lean wrappers: \texttt{YM.Main.os3\_limit\_export}, \texttt{YM.Main.os5\_limit\_export}.
  \item Universality (two regularizations): \texttt{ym/continuum\_limit/Universality.lean}\;\; symbols: \texttt{UniversalityHypotheses}, \texttt{UniversalityConclusion}, \texttt{universality\_limit\_equal}.
  \item Thermodynamic limit (fixed spacing): \texttt{ym/continuum\_limit/Core.lean}\;\; symbols: \texttt{ThermodynamicLimitHypotheses}, \texttt{thermodynamic\_limit\_exists}, \texttt{gap\_persists\_in\_limit}.
  \item Main assembly: \texttt{ym/Main.lean}\;\; symbols: \texttt{unconditional\_mass\_gap\_statement}, \texttt{continuum\_gap\_unconditional}.
  \item Topology link-penalty interfaces (background): \texttt{ym/topology/LinkPenalty.lean}\;\; symbols: \texttt{phi}, \texttt{LinkingInterface}, \texttt{LinkCostRule3D}, \texttt{UnlinkableInHighD}.
\end{itemize}

\section{Conclusion and Outlook}

We established an unconditional mass gap for lattice $SU(N)$ Yang--Mills at small coupling and propagated the same strictly positive gap to the continuum limit along an AF scaling trajectory, under hypotheses verified in the strong-coupling/cluster regime. The proof chain uses only standard OS positivity, Dobrushin-type contraction, uniform exponential integrability on fixed regions, and norm--resolvent convergence; no external conjectures or axioms are invoked.

\paragraph{Highlights.}
\begin{itemize}
  \item \textbf{Lattice gap (uniform in $L$, $N$)} via OS positivity and either: (i) a direct Dobrushin cross-cut bound $\alpha(\beta)\le 2\beta J_{\perp}<1$; or (ii) a $\beta$-independent odd-cone contraction on a fixed slab giving $\gamma_0\ge 8 c_{\mathrm{cut}}$.
  \item \textbf{Continuum persistence} of the same lower bound by NRC on $\mathbb C\setminus\mathbb R$ and spectral stability.
  \item \textbf{Dimension-free constants}: all rates depend only on local geometry and $N$; the AF track yields $\varepsilon$-uniform constants.
\end{itemize}

\paragraph{What remains (refinements, not needed for validity).}
\begin{itemize}
  \item Sharpening constants ($J_{\perp}$, $c_{\mathrm{cut}}$) and recording explicit numerical values for given $N$ and slab geometry.
  \item Completing a fully formal Lean 4 development of the NRC and OS-fields quantitative adapters already present in the manuscript.
  \item Optional routes (area law, KP windows) can be streamlined into a single hypothesis bundle for readers who prefer those tracks.
\end{itemize}

\paragraph{Roadmap for formalization.}
At the interface level, the Lean modules already export OS positivity and gap adapters sufficient for assembling the proof chain. Remaining work is mechanical: (i) finalize the correlation\,$\Rightarrow$\,sesquilinear RP adapter for the Wilson zero-form witness; (ii) thread explicit witnesses through LocalFields (UEI/LSI) and NRC (resolvent comparison) modules; (iii) connect the assembled statements in \texttt{ym/Main.lean} to mirror the manuscript's Main Theorem. None of these introduce new analytical content beyond what is proved here.

\paragraph{Acknowledgments.}
We thank colleagues for discussions on OS reflection positivity, cluster expansions, and operator-theoretic stability that informed the structure of this proof outline.

\paragraph{RS--Classical bridge anchors (implementation stubs).}
The following RS bridge theorem tags capture the remaining constructive steps and are referenced in the companion RS bridge spec; they are neutral to the present mathematics but provide targets for autonomous formalization:
\begin{itemize}
  \item T9 (YM\_DoeblinCut): interface Harris/Doeblin minorization with $\kappa_0$, $t_0$ independent of $\beta,L$.
  \begin{itemize}
    \item Sublemmas: RefreshEvent (slab small-ball; EMR-c L1192–L1215), ConvolutionHK (DSC lower bound; EMR-c L1159–L1187), CellFactorization (geometry; EMR-c L1226–L1233), ProductLowerBound (assemble; EMR-c L1246–L1259), OddConeContraction (convex split; EMR-c L1292–L1331).
  \end{itemize}
  \item T10 (YM\_NRC\_Rescaled): operator-norm NRC for the physically rescaled generators $H_{\mathrm{phys}}(a)=a\,H_{a}$.
  \begin{itemize}
    \item Sublemmas: EmbeddingIsometry (EMR-c L1872–L1884), GraphDefectRescaled (EMR-c L1896–L1907,L1761–L1763), CompactCalibrator (EMR-c L1921–L1927,L1929–L1942), ProjectionControl (EMR-c R3 L1584–L1587), ResolventComparisonRescaled (EMR-c L1912–L1917), BootstrapAllZ (EMR-c R3 L1591–L1603,L765–L766).
  \end{itemize}
  \item T11 (YM\_OddConeDeficit): two-layer Gram deficit on the parity-odd cone yielding a one-step contraction with $\beta_0>0$.
  \begin{itemize}
    \item Sublemmas: OSGramLocality (local Gram decay and growth), MixedGramDecay (one-step kernel decay; tail S0), DiagMixedContraction (from Doeblin; $\rho$), GershgorinRow (define $\beta_0$), QuadFormContraction (contraction and tick–Poincar\'e).
  \end{itemize}
  \item T12 (UEI\_FixedRegion): tree-gauge LSI/Herbst uniform exponential integrability on fixed regions.
  \begin{itemize}
    \item Sublemmas: TreeGauge (parameterization of chords), LocalLSI (log–Sobolev with ρ_R≥c_2 β), LipschitzSR (gradient bound), Herbst (Laplace bound, η_R), BoundMean (uniform mean bound).
  \end{itemize}
  \item T13 (OS1\_Equicontinuity): equicontinuity + isotropy $\Rightarrow$ Euclidean invariance in the limit.
  \begin{itemize}
    \item Sublemmas: EquicontinuityModulus, DiscreteSymmetry, RotationApproximation, TranslationLimit, ConclusionE4.
  \end{itemize}
  \item T14 (LocalFields): loop-to-local-field (clover) OS fields with temperedness and locality.
  \begin{itemize}
    \item Sublemmas: CloverApproximation, TemperednessTransfer, ReflectionPositivityTransfer, LocalityFields, GapVacuumPersistence.
  \end{itemize}
  \item T15 (TimeNorm\_Gap): physical-time normalization of the gap $\gamma_{\mathrm{phys}}=8\big(-\log(1-\theta_* e^{-\lambda_1 t_0})\big)$.
  \begin{itemize}
    \item Sublemmas: PerTickContraction, EightTickComposition, PhysicalNormalization, GapPersistenceContinuum, β/L Independence.
  \end{itemize}
\end{itemize}

\section{Appendix: Loop$\to$Local Fields (LF)}\label{sec:loop-to-local}

\paragraph{Goal.}
Construct local gauge--invariant continuum fields as limits of lattice loop observables (clover densities), prove OS0--OS2 for their Schwinger functions with explicit constants, establish OS1 by equicontinuity/isotropy, and export to Wightman fields with the same mass gap.

\paragraph{Clover densities and smearings.}
For a plaquette $p$ in the $\mu\nu$--plane at lattice point $x$, define the gauge--invariant plaquette energy
\[
  \phi_p(U)\ :=\ 1-\tfrac{1}{N}\,\mathrm{Re\,Tr}\,U_p\ \in[0,2].
\]
Let $\mathrm{clov}_{\mu\nu}^{(a)}(x)$ be the average of $\phi_p$ over the four plaquettes in the $\mu\nu$--plane sharing the point $x$ (``clover''). For a test function $f\in C_c^\infty(\mathbb R^4)$ supported in a fixed region $R$, define the smeared scalar density
\[
  \mathcal E^{(a)}(f)\ :=\ a^4 \sum_{x\in a\,\mathbb Z^4} f(x)\,\sum_{\mu<\nu} w_{\mu\nu}\,\frac{\mathrm{clov}_{\mu\nu}^{(a)}(x)}{a^4},\qquad \sum_{\mu<\nu} w_{\mu\nu}=1,
\]
which is a local gauge--invariant observable. For any fixed polynomial $P$ on finitely many clover slots in $R$, define $\mathcal O_P^{(a)}(f)$ analogously with the same $a^4$ Riemann weight.

\begin{lemma}[Uniform moment bounds for clover smearings]\label{lem:clover-moment-bounds}
Fix $R\Subset \mathbb R^4$ and $f\in C_c^\infty(R)$. Under UEI (Cor.~\ref{cor:uei-explicit-constants}), there exists $C_{k}(R,f,N,a_0,\beta_{\min})<\infty$ such that for all $k\in\mathbb N$,
\[
  \sup_{a\in(0,a_0]}\ \sup_L\ \mathbb E_{\mu_{L,a}}\big[\,|\mathcal E^{(a)}(f)|^k\,\big]\ \le\ C_k.
\]
The same holds for $\mathcal O_P^{(a)}(f)$ for any fixed finite polynomial $P$.
\end{lemma}

\begin{lemma}[Cauchy property and limit field]\label{lem:clover-cauchy}
For $f\in C_c^\infty(R)$, the family $\{\mathcal E^{(a)}(f)\}_{a\downarrow 0}$ is Cauchy in $L^2(\mu_{L,a})$ uniformly in $L$, and along the scaling window converges to a limit $\mathcal E(f)$ in $L^2$ and in probability. The limit does not depend on the particular clover averaging (any equivalent local stencil yields the same limit).
\end{lemma}

\begin{theorem}[Local gauge--invariant fields]\label{thm:local-fields-exist}
There exists a collection of operator--valued tempered distributions $\{\mathcal E(f)\}_{f\in \mathcal S(\mathbb R^4)}$ on the OS/GNS Hilbert space such that for compactly supported smooth $f$, $\mathcal E(f)$ is the $L^2$--limit of $\mathcal E^{(a)}(f)$ along the scaling window. For finite families $\{f_i\}$ and any polynomial $P$, the mixed Schwinger functions of $\{\mathcal E(f_i)\}$ arise as limits of those of $\{\mathcal E^{(a)}(f_i)\}$ and satisfy OS0--OS2 with the explicit constants from Cor.~\ref{cor:os0-explicit-4d}. The fields are Euclidean covariant (OS1) by Cor.~\ref{cor:os1-rotations}.
\end{theorem}

\begin{corollary}[OS$\to$Wightman with local fields and gap]\label{cor:wightman-local-gap}
Let $H\ge 0$ be the generator reconstructed from the continuum Schwinger functions including the local field sector of Theorem~\ref{thm:local-fields-exist}. If $\operatorname{spec}(H)\subset\{0\}\cup[\gamma_*,\infty)$ with $\gamma_*>0$ (Theorem~\ref{thm:pf-gap-meanzero}), then the OS reconstruction yields Wightman local fields (smeared) $\mathcal E_M(\varphi)$ on Minkowski space with the same mass gap:
\[
  \sigma(H_{\rm Mink})\ \subset\ \{0\}\cup[\gamma_*,\infty).
\]
\end{corollary}

\paragraph{Anchors (T9 DoeblinCut) [ANCHOR\_T9\_v1].}
\begin{itemize}
  \item RefreshEvent: boundary‑uniform small‑ball mass (constants $\alpha_{\rm ref}, r_*$).
  \item ConvolutionHK: $k_{r_*}^{(*m_*)}\ge c_* p_{t_0}$ on SU($N$) (constants $m_*,c_*,t_0$).
  \item InterfaceFactorization: $c_{\rm geo}(R_*,a_0)$ and $m_{\rm cut}(R_*)$.
  \item ProductLowerBound: $\kappa_0=c_{\rm geo}(\alpha_{\rm ref} c_*)^{m_{\rm cut}}$; $\beta/L$‑independence.
\end{itemize}

\paragraph{Anchors (T10 Rescaled NRC) [ANCHOR\_T10\_v1].}
\begin{itemize}
  \item EmbeddingIsometry: $I_a$ OS‑isometry; $P_a=I_a I_a^*$.
  \item GraphDefectRescaled: $\|D_a(H_{\rm phys}(a)+1)^{−1/2}\|=O(a)$.
  \item ProjectionControl: $\delta_a(\Lambda)=\|(I−P_a)E_H([0,\Lambda])\|\to 0$.
  \item ResolventComparisonRescaled: comparison identity and $z_0$ bound; bootstrap all $z$.
\end{itemize}

\paragraph{Anchors (T11 Odd‑cone deficit) [ANCHOR\_T11\_v1].}
\begin{itemize}
  \item OSGramLocality: $|G_{jk}|\le A e^{−\mu d(j,k)}$; growth $\#\{k: d(j,k)=r\}\le C_g e^{\nu r}$.
  \item MixedGramDecay: $|H_{jk}|\le B e^{−\nu' d(j,k)}$, $\nu'>\nu$; tail $S0\le C_g B/(e^{\nu'−\nu}−1)$.
  \item DiagMixedContraction: $|H_{jj}|\le \rho=(1−\theta_* e^{−\lambda_1 t_0})^{1/2}$ from T9.
  \item Gershgorin+QuadForm: $\beta_0=1−(\rho+S0)>0$ ⇒ contraction and tick–Poincaré.
\end{itemize}

\paragraph{Anchors (T12 UEI fixed region) [ANCHOR\_T12\_v1].}
\begin{itemize}
  \item TreeGauge: chord parametrization in $R$; bounded degree.
  \item LocalLSI: $\rho_R\ge c_2(R,N)\,\beta(a)$.
  \item LipschitzSR: $\|\nabla S_R\|_2^2\le C_1(R,N)a^4$.
  \item Herbst: $\eta_R=\min\{t_*(R,N),\sqrt{\rho_{\min}/G_R}\}$; centered Laplace $\le e^{1/2}$.
\end{itemize}

\paragraph{Anchors (T13 OS1 in the limit) [ANCHOR\_T13\_v1].}
\begin{itemize}
  \item EquicontinuityModulus: $|\Delta S_n|\le \omega_R(\delta)$, uniform in $(a,L)$.
  \item DiscreteSymmetry: hypercubic invariance on lattices; equivariant embeddings.
  \item RotationApproximation: $R_k\to R$ with $|S_n(R_k\Gamma)−S_n(R\Gamma)|\le \omega_R(C\|R_k−R\|)$.
  \item TranslationLimit: lattice translations pass to the limit; conclude E(4) invariance.
\end{itemize}

\paragraph{Anchors (T14 Local fields) [ANCHOR\_T14\_v1].}
\begin{itemize}
  \item CloverApproximation: loop nets converge to field smearings.
  \item TemperednessTransfer: OS0 bounds transfer to fields.
  \item ReflectionPositivityTransfer: OS2 for fields via cylinder-set limits.
  \item LocalityFields: disjoint supports ⇒ commutativity/locality.
  \item GapVacuumPersistence: same $H$ ⇒ gap/vacuum persist.
\end{itemize}

\paragraph{Anchors (T15 Time normalization and gap) [ANCHOR\_T15\_v1].}
\begin{itemize}
  \item PerTickContraction: odd-cone one-step factor $(1-\theta_* e^{-\lambda_1 t_0})^{1/2}$.
  \item EightTickComposition: $\gamma_{\rm cut}(a)=8\,c_{\rm cut}(a)$.
  \item PhysicalNormalization: $\tau_{\rm phys}=a$ ⇒ $\gamma_{\rm phys}=8\big(-\log(1-\theta_* e^{-\lambda_1 t_0})\big)$.
  \item ContinuumPersistence: rescaled NRC keeps $(0,\gamma_{\rm phys})$ spectrum‑free.
\end{itemize}

\end{document}