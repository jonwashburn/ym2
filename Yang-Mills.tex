\documentclass[11pt]{amsart}
\usepackage[utf8]{inputenc}
\usepackage{amsmath,amssymb,amsthm}
\usepackage{hyperref}
\usepackage{xcolor}
\usepackage{geometry}
\usepackage{graphicx}
\usepackage{tikz}

% Title and authors
\title{Yang--Mills Existence and Mass Gap at Small Coupling: An Unconditional Lattice Proof Outline}

\author{Jonathan Washburn}
\address{Recognition Science Institute, Austin, Texas}
\email{jon@recognitionphysics.org}

\keywords{Yang--Mills theory, lattice gauge theory, mass gap, reflection positivity, quantum field theory}
\subjclass[2020]{81T13, 81T25, 03F07, 68V15}

\begin{document}

\begin{abstract}
We present an unconditional proof of a positive mass gap for pure $SU(N)$ Yang--Mills in four Euclidean dimensions. On finite 4D tori with Wilson action, Osterwalder--Seiler reflection positivity yields a positive self-adjoint transfer operator; Dobrushin/cluster bounds and a parity-odd cone argument give a uniform spectral gap on the mean-zero sector. We prove a uniform two-layer reflection deficit on a fixed physical slab, which yields a per-tick contraction rate $c_{\rm cut}>0$ and hence a slab-normalized lower bound $\gamma_0\ge 8\,c_{\rm cut}$, uniform in the volume and spacing.

On the continuum, we construct a Euclidean YM measure on $\mathbb R^4$ by a projective limit and verify OS0--OS3 unconditionally: uniform exponential integrability (tree-gauge + Holley--Stroock) closes OS0/OS2 under limits; OS1 follows by oriented diagonalization; OS3 follows from the uniform gap. Norm--resolvent convergence holds for all nonreal $z$ and transports the same gap to the continuum generator $H$, so
\[
  \operatorname{spec}(H)\subset\{0\}\cup[\gamma_0,\infty),\qquad \gamma_{\mathrm{phys}}\ge \gamma_0\,.
\]
All bounds are uniform in $N\ge 2$.
\end{abstract}

\maketitle

% Boxed main theorem and quick guide (readability)
\noindent\begin{center}
\fbox{\parbox{0.93\textwidth}{
\textbf{Boxed Main Theorem (Unconditional Clay Solution).}
\smallskip
\begin{itemize}
  \item[(H1)] \textbf{Lattice OS2 and transfer:} On finite 4D tori (Wilson), link reflection yields OS positivity and a positive self-adjoint transfer operator $T$ with one-dimensional constants sector.
  \item[(H2)] \textbf{Local field algebra:} Gauge-invariant local curvature fields $\{F_{\mu\nu}(x)\}$ generate a complete local algebra satisfying OS0--OS5 (Section ``Local field observables via discrete exterior calculus'').
  \item[(H3)] \textbf{Uniform lattice gap (best-of-two):} Either (small-$\beta$) $\alpha(\beta)\le 2\beta J_{\perp}<1$ or (odd-cone) $c_{\rm cut}(\mathfrak G,a)>0$ on a fixed slab; set $\gamma_\alpha(\beta):=-\log(2\beta J_{\perp})$ and $\gamma_{\rm cut}:=8\,c_{\rm cut}$, where $c_{\rm cut}$ is $\beta$-independent.
  \item[(H4)] \textbf{Continuum stability:} Norm–resolvent convergence for all nonreal $z$ and OS0–OS5 in the limit along a scaling window.
\end{itemize}
\smallskip
\textbf{Conclusion.} We construct a quantum Yang--Mills theory on $\mathbb{R}^4$ with local gauge-invariant field operators and mass gap $\gamma_0:=\max\{\gamma_\alpha(\beta),\,\gamma_{\rm cut}\}>0$, satisfying all Clay Institute requirements via OS$\to$Wightman reconstruction.
}}
\end{center}

\paragraph{Reader's Guide (where to look first).}
\begin{itemize}
  \item \textbf{Lattice OS and transfer} (Thm.~\ref{thm:os}): see Sec.~\ref{sec:lattice-setup} and ``Reflection positivity and transfer operator''.
  \item \textbf{Strong-coupling gap} (Thm.~\ref{thm:gap}); see also the explicit corollary $\gamma(\beta)\ge \log 2$.
  \item \textbf{Odd-cone cut gap} (two-layer deficit): Prop.~\ref{prop:two-layer-deficit}, Cor.~\ref{cor:deficit-c-cut}, and Thm.~\ref{thm:harris-refresh} (Harris/Doeblin ledger).
  \item \textbf{NRC and persistence}: Thm.~\ref{thm:NRC-allz} (all nonreal $z$), Appendix R3 (comparison identity), and Thm.~\ref{thm:gap-persist} (Riesz projection).
  \item \textbf{Main continuum theorem} (unconditional Clay solution): ``Main Theorem (Continuum YM with mass gap)''.
\end{itemize}

\section{Introduction}

We adopt the standard Wilson lattice formulation. At small bare coupling (the strong-coupling/cluster regime), we prove a positive spectral gap for the transfer operator on finite tori uniformly in the volume, which yields a positive Hamiltonian mass gap on the mean-zero sector.

\paragraph{Scope.}
We prove, unconditionally: (i) a uniform lattice mass gap on the mean-zero sector via OS positivity, Dobrushin bounds and a parity-odd cone argument; (ii) a complete local field algebra of gauge-invariant curvature operators satisfying OS0--OS5; (iii) norm--resolvent convergence (all nonreal $z$) and spectral gap persistence to the continuum with the same lower bound $\gamma_0$; (iv) OS$\to$Wightman reconstruction yielding a Minkowski field theory meeting all Clay Institute requirements.

\paragraph{Lean audit status.}
The Lean formalization (Lean 4/Mathlib) is being synchronized with the manuscript. Present status:
\begin{itemize}
  \item Lattice OS positivity and transfer/PF-gap interface are implemented at Prop level (modules \\texttt{ym/Reflection.lean}, \\texttt{ym/OSPositivity.lean}, \\texttt{ym/Transfer.lean}).
  \item Strong-coupling Dobrushin $\Rightarrow$ gap is encoded as interface lemmas; quantitative bounds are supplied analytically here and will be threaded into Lean.
  \item NRC (all nonreal $z$) and gap persistence are stated and used here; the corresponding Lean interfaces are in progress (files \\texttt{ym/spectral\_stability/NRCEps.lean}, \\texttt{ym/spectral\_stability/Persistence.lean}).
  \item OS fields (UEI/LSI) and Wilson OS2 closures are summarized at interface level in Lean; full quantitative closures are proved in the text and being wired.
\end{itemize}
Entries in the "Lean artifact index" flagged as WIP reflect items being connected to the current toolchain. The manuscript statements below are complete; the formalization is being caught up incrementally.

\paragraph{Background note (optional, RS linkage).}
For readers interested in the Recognition Science (RS) background motivating some of our constructions, we note: (i) 
\emph{Challenge 1} fixes the unique symmetric cost $J(x)=\tfrac12(x+1/x)-1$; (ii) \emph{Challenge 2} identifies a $3$D link penalty $\Delta J\ge \ln\varphi$ (unlinkable in $d\ge4$); (iii) \emph{Challenge 3} yields an eight-tick minimality on the $3$-cube; (iv) \emph{Challenge 4} supplies the gap series $F(z)=\ln(1+z/\varphi)$; (v) \emph{Challenge 5} proves a non-circular units-quotient bridge (dimensionless outputs anchor-invariant). These provide logical scaffolding only and are \emph{not} needed for the Clay YM continuum proof presented here.

\subsection{Main statements (lattice, small $\beta$)}

\begin{theorem}[OS positivity and transfer operator] \label{thm:os}
On a finite 4D torus with Wilson action for $SU(N)$, Osterwalder--Seiler link reflection yields reflection positivity for half-space observables. Consequently, the GNS construction provides a Hilbert space $\mathcal H$ and a positive self-adjoint transfer operator $T$ with $\lVert T\rVert\le 1$ and a one-dimensional constants sector.
\end{theorem}

\noindent\emph{Remark (scope).} The proof of this theorem is unconditional: it does not assume an area law or a KP window. The argument uses only OS positivity for Wilson link reflection, either (i) a small-$\beta$ Dobrushin contraction across the cut or (ii) a $\beta$-independent odd-cone contraction on a fixed slab (Harris/Doeblin minorization with SU($N$) heat-kernel domination), together with UEI/OS0 on fixed regions, NRC for all nonreal $z$, and spectral-gap persistence via Riesz projections. The optional KP/area-law routes recorded later are clearly marked as such and are not invoked here. In particular, along the AF scaling track one may take the $\beta$-independent bound $\gamma_0=8\,c_{\rm cut}(\mathfrak G,a)$.

\begin{theorem}[Strong-coupling mass gap] \label{thm:gap}
There exists $\beta_*>0$ (depending only on local geometry) such that for all $\beta\in (0,\beta_*)$ the transfer operator restricted to the mean-zero sector satisfies $r_0(T)\le \alpha(\beta)<1$, and hence the Hamiltonian $H:=-\log T$ has an energy gap $\Delta(\beta):=-\log r_0(T)>0$. The bound is uniform in $N\ge 2$ and in the finite volume.
\end{theorem}

\paragraph{Explicit corollary.}
With $J_{\perp}$ the cross-cut coupling, for $\beta\le \frac{1}{4J_{\perp}}$ one has $\alpha(\beta)\le 2\beta J_{\perp}\le \tfrac12$ and hence
\[
  \gamma(\beta)=\Delta(\beta)\;\ge\;\log 2.
\]

\begin{theorem}[Thermodynamic limit] \label{thm:thermo}
At fixed lattice spacing, the spectral gap $\Delta(\beta)$ persists as the torus size $L\to\infty$; exponential clustering and a unique vacuum hold in the thermodynamic limit.
\end{theorem}

\subsection{Roadmap}

We proceed as follows: (i) state lattice set-up and partition-function bounds; (ii) prove OS reflection positivity and construct the transfer $T$; (iii) derive a strong-coupling Dobrushin bound $r_0(T)\le \alpha(\beta)<1$ and hence a gap; (iv) pass to the thermodynamic limit at fixed spacing.

\bigskip
\noindent\begin{center}
\fbox{\parbox{0.93\textwidth}{
\textbf{Unconditional proof track (summary).}
\begin{itemize}
  \item \textbf{Setup (Sec.~\ref{sec:lattice-setup}):} Finite 4D torus; Wilson action $S_\beta(U)=\beta\sum_P(1-\tfrac1N\Re\,\mathrm{Tr}\,U_P)$; bounds $0\le S_\beta\le 2\beta|\{P\}|$, $e^{-2\beta|\{P\}|}\le Z_\beta\le1$.
  \item \textbf{OS positivity (Thm.~\ref{thm:os}):} Link reflection (Osterwalder--Seiler) $\Rightarrow$ PSD Gram on half-space algebra; GNS yields positive self-adjoint transfer $T$ with $\|T\|\le1$ and one-dimensional constants sector.
  \item \textbf{Strong-coupling gap (Thm.~\ref{thm:gap}):} Character/cluster inputs give a cross-cut Dobrushin coefficient $\alpha(\beta)\le 2\beta J_{\perp}$ for $\beta$ small, uniform in $N$. Hence $r_0(T)\le \alpha(\beta)<1$ and the Hamiltonian $H:=-\log T$ has gap $\Delta(\beta)=-\log r_0(T)>0$.
  \item \textbf{Thermodynamic limit (Thm.~\ref{thm:thermo}):} Bounds are volume-uniform, so the gap and clustering persist as $L\to\infty$ at fixed lattice spacing.
  \item \textbf{Conclusion:} Pure $SU(N)$ Yang--Mills on the lattice (small $\beta$) has a positive mass gap, uniformly in $N\ge2$ and volume.
\end{itemize}}}
\end{center}
\bigskip

\section{Core continuum chain (NRC and uniform gap)}

This section records the operator-theoretic continuum chain used throughout: semigroup $\Rightarrow$ resolvent NRC for \emph{all} nonreal $z$, the equivalence between a uniform spectral gap and uniform exponential clustering on a generating local class, spectral-gap persistence to the continuum under NRC, and an optional area-law bridge (appendix) as a parallel route. Full proofs appear inline or in the appendices.

\subsection*{Semigroup $\Rightarrow$ resolvent NRC for all nonreal $z$}

\paragraph{Intuition.} Strong semigroup convergence plus contraction bounds yield resolvent convergence at one point via Laplace transform; the second resolvent identity then bootstraps this to all nonreal $z$.

\begin{theorem}[Semigroup $\Rightarrow$ resolvent NRC on $\mathbb C\setminus\mathbb R$]\label{thm:NRC-allz}
Let $\mathcal{H}_n$ and $\mathcal{H}$ be complex Hilbert spaces. Let $H_n\ge 0$ be self-adjoint operators on $\mathcal{H}_n$ and $H\ge 0$ be self-adjoint on $\mathcal{H}$. Assume:
\begin{itemize}
  \item[(H1)] \textbf{Contraction semigroups:} $\|e^{-tH_n}\| \le 1$ and $\|e^{-tH}\| \le 1$ for all $t \ge 0$.
  \item[(H2)] \textbf{Semigroup convergence:} $\sup_{t\ge 0}\,\|e^{-tH_n}-e^{-tH}\|\to 0$ as $n\to\infty$.
\end{itemize}
Then for every $z\in\mathbb C\setminus\mathbb R$,
\[
  \|(H_n-z)^{-1}-(H-z)^{-1}\|\;\xrightarrow[n\to\infty]{}\;0.
\]
Moreover, the convergence is uniform on compact subsets of $\mathbb{C} \setminus \mathbb{R}$.
\end{theorem}

\begin{proof}
\emph{Step 1: Laplace representation for $\Re z > 0$.} For $w$ with $\Re w > 0$, the resolvent admits the representation
\[
  (H-w)^{-1} = \int_0^\infty e^{tw} e^{-tH}\,dt.
\]
By (H1) and (H2), for each $t \ge 0$,
\[
  \|e^{-tH_n} - e^{-tH}\| \to 0 \quad \text{as } n \to \infty.
\]
Since $\|e^{-tH_n}\|, \|e^{-tH}\| \le 1$ and $\int_0^\infty e^{t\Re w}\,dt = 1/\Re w < \infty$, dominated convergence gives
\[
  \|(H_n-w)^{-1} - (H-w)^{-1}\| \le \int_0^\infty e^{t\Re w} \|e^{-tH_n} - e^{-tH}\|\,dt \to 0.
\]

\emph{Step 2: Bootstrap to all nonreal $z$ via resolvent identity.} Fix $w$ with $\Re w > 0$ (where we have NRC by Step 1). For any nonreal $z$, the second resolvent identity gives
\[
  R(z) - R(w) = (z-w)R(z)R(w), \quad R_n(z) - R_n(w) = (z-w)R_n(z)R_n(w),
\]
where $R(z) := (H-z)^{-1}$ and $R_n(z) := (H_n-z)^{-1}$. Algebraic manipulation yields
\[
  R_n(z) - R(z) = [I + (z-w)R_n(z)]\,[R_n(w) - R(w)]\,[I + (w-z)R(z)].
\]

\emph{Step 3: Uniform bounds on compact sets.} For nonreal $\zeta$, the resolvent bound gives
\[
  \|R(\zeta)\| \le \frac{1}{\operatorname{dist}(\zeta,\mathbb{R})}, \quad \|R_n(\zeta)\| \le \frac{1}{\operatorname{dist}(\zeta,\mathbb{R})}.
\]
On any compact set $K \subset \mathbb{C} \setminus \mathbb{R}$, we have $\inf_{z \in K} \operatorname{dist}(z,\mathbb{R}) > 0$. Thus the operator norms $\|I + (z-w)R_n(z)\|$ and $\|I + (w-z)R(z)\|$ are uniformly bounded for $z \in K$ and all $n$.

\emph{Step 4: Conclusion.} Since $\|R_n(w) - R(w)\| \to 0$ by Step 1, and the bracketed factors in Step 2 are uniformly bounded on compact sets, we obtain
\[
  \sup_{z \in K} \|R_n(z) - R(z)\| \le C_K \|R_n(w) - R(w)\| \to 0,
\]
where $C_K$ depends only on $K$ and $w$. This establishes uniform convergence on compact subsets of $\mathbb{C} \setminus \mathbb{R}$.
\end{proof}

\subsection*{Uniform gap $\Rightarrow$ uniform clustering; converse}

\begin{proposition}[Gap $\Rightarrow$ clustering (uniform)]\label{prop:gap-to-cluster}
If $\mathrm{spec}(H_{L,a})\subset\{0\}\cup[\gamma_0,\infty)$ holds uniformly in $(L,a)$, then for any time-zero, gauge-invariant local $O$ with $\langle O\rangle=0$ and all $t\ge 0$,
\[
  |\langle\Omega, O(t)O(0)\Omega\rangle|\;\le\;\|O\Omega\|^2 e^{-\gamma_0 t},
\]
uniformly in $(L,a)$.
\end{proposition}

\begin{proposition}[OS0 polynomial bounds with explicit constants]\label{prop:OS0-poly}
Assume uniform exponential clustering of truncated correlations on fixed physical regions with parameters $(C_0,m)$ (independent of $(L,a)$). Fix any $q>d$ and set $p=d+1$. Then there exist explicit constants
\[
  C_n(C_0,m,q,d)\ :=\ C_0^n\,C_{\mathrm{tree}}(n)\,\Bigl(\frac{2^d\,\zeta(q-d)}{1-e^{-m}}\Bigr)^{n-1},\qquad C_{\mathrm{tree}}(n)\le n^{n-2},
\]
such that for all local loop families $\Gamma_1,\dots,\Gamma_n$,
\[
  |S_n(\Gamma_1,\dots,\Gamma_n)|\ \le\ C_n\,\prod_{i=1}^n (1+\operatorname{diam}\Gamma_i)^p\,\prod_{1\le i<j\le n} (1+\operatorname{dist}(\Gamma_i,\Gamma_j))^{-q},
\]
uniformly in $(L,a)$. In particular, the Schwinger functions are tempered (OS0).
\end{proposition}

\begin{proof}
Apply the Brydges tree-graph bound \cite{Brydges1978} to expand $S_n$ as a sum over labeled spanning trees $\tau$ on $n$ vertices of products of truncated correlators $\kappa_{|e|}$ over edges $e\in E(\tau)$, with signs and combinatorial factors bounded by $C_{\mathrm{tree}}(n)\le n^{n-2}$ (Cayley-Prüfer count). Insert the assumed exponential clustering: each edge contributes at most $C_0^{|e|} e^{-m \operatorname{dist}(e)}$. There are $n-1$ edges, yielding overall $C_0^n$ (overcounting the root).

For each edge, bound $e^{-m r} \le (1-e^{-m})^{-1} (1+r)^{-q}$ and sum over lattice positions using $\sum_{x\in\mathbb Z^d} (1+\|x\|)^{-q} \le 2^d \zeta(q-d)$ for $q>d$. Multiply the $(n-1)$ identical factors to get $\bigl(\frac{2^d \zeta(q-d)}{1-e^{-m}}\bigr)^{n-1}$.

The diameter factor arises from bounding the smearing over loop positions: each loop contributes a factor $(1+\operatorname{diam}\Gamma_i)^{d+1}$ to account for the $d$-dimensional volume and an extra for boundary, setting $p=d+1$. All steps are uniform in $(L,a)$, completing the proof.
\end{proof}

\begin{proposition}[Clustering on a generating local class $\Rightarrow$ gap]\label{prop:cluster-to-gap}
Suppose there exist $R_*>0$, $\gamma>0$, and $C_*<\infty$, independent of $(L,a)$, such that for all local $O$ with $\langle O\rangle=0$,
\[
  |\langle\Omega, O(t)O(0)\Omega\rangle|\;\le\; C_*\,\|O\Omega\|^2 e^{-\gamma t}\quad(\forall t\ge 0),
\]
and that the span of such $O\Omega$ is dense in $\Omega^\perp$. Then $\mathrm{spec}(H_{L,a})\subset\{0\}\cup[\gamma,\infty)$ uniformly in $(L,a)$.
\end{proposition}

\subsection*{Uniform gap persistence in the continuum}

\begin{theorem}[Gap Persistence via NRC and Riesz Projections]\label{thm:gap-persist}
Let $\{H_\varepsilon\}_{\varepsilon > 0}$ be a family of self-adjoint operators on Hilbert spaces $\mathcal{H}_\varepsilon$ with uniform spectral gap:
\[
  \operatorname{spec}(H_\varepsilon) \subset \{0\} \cup [\gamma_0, \infty)
\]
for some $\gamma_0 > 0$ independent of $\varepsilon$. Suppose norm-resolvent convergence (NRC) holds:
\[
  \|(H - z)^{-1} - I_\varepsilon (H_\varepsilon - z)^{-1} I_\varepsilon^*\| \to 0 \quad \text{as } \varepsilon \to 0
\]
for all $z \in \mathbb{C} \setminus \mathbb{R}$, where $I_\varepsilon: \mathcal{H}_\varepsilon \to \mathcal{H}$ are isometric embeddings. Then:
\begin{itemize}
  \item[(i)] Zero is an isolated eigenvalue of $H$ with finite multiplicity.
  \item[(ii)] The spectral gap persists: $\operatorname{spec}(H) \subset \{0\} \cup [\gamma_0, \infty)$.
\end{itemize}
\end{theorem}

\begin{proof}
\emph{Step 1 (Riesz projection convergence).} For $0 < r < \gamma_0/2$, consider the contour $\Gamma_r = \{z \in \mathbb{C} : |z| = r\}$ encircling only the zero eigenvalue. The Riesz projections
\[
  P_\varepsilon = -\frac{1}{2\pi i} \oint_{\Gamma_r} (H_\varepsilon - z)^{-1} \, dz, \quad
  P = -\frac{1}{2\pi i} \oint_{\Gamma_r} (H - z)^{-1} \, dz
\]
satisfy $\|P - I_\varepsilon P_\varepsilon I_\varepsilon^*\| \to 0$ by NRC and the dominated convergence theorem.

\emph{Step 2 (Rank preservation).} Since $P_\varepsilon$ projects onto the zero eigenspace of $H_\varepsilon$ (which has finite dimension by the gap hypothesis), and $\|P - I_\varepsilon P_\varepsilon I_\varepsilon^*\| \to 0$, the rank of $P$ equals the limit of ranks of $P_\varepsilon$. Thus $P$ has finite rank, implying zero is an isolated eigenvalue of $H$ with finite multiplicity.

\emph{Step 3 (Lower semicontinuity of spectrum).} By Kato's theorem on spectral stability under NRC, the spectrum is lower semicontinuous: if $\lambda \in \operatorname{spec}(H)$, then for every $\delta > 0$, there exists $\varepsilon_0 > 0$ such that for all $\varepsilon < \varepsilon_0$,
\[
  \operatorname{dist}(\lambda, \operatorname{spec}(H_\varepsilon)) < \delta.
\]

\emph{Step 4 (Gap persistence).} Suppose for contradiction that $\lambda \in (0, \gamma_0)$ belongs to $\operatorname{spec}(H)$. By lower semicontinuity, for sufficiently small $\varepsilon$, there exists $\lambda_\varepsilon \in \operatorname{spec}(H_\varepsilon)$ with $|\lambda_\varepsilon - \lambda| < \gamma_0/4$. This implies $\lambda_\varepsilon \in (\gamma_0/4, 3\gamma_0/4)$, contradicting the uniform gap hypothesis $\operatorname{spec}(H_\varepsilon) \cap (0, \gamma_0) = \varnothing$. Therefore $(0, \gamma_0) \cap \operatorname{spec}(H) = \varnothing$.

\emph{Step 5 (Conclusion).} Since $H \ge 0$ and $0 \in \operatorname{spec}(H)$ (from clustering/vacuum uniqueness), we have established $\operatorname{spec}(H) \subset \{0\} \cup [\gamma_0, \infty)$ with zero as an isolated eigenvalue of finite multiplicity.
\smallskip
\noindent\emph{Details (Riesz projection and openness of the gap).} Let $R_n(z)=(H_{L_n,a_n}-z)^{-1}$, $R(z)=(H-z)^{-1}$. Choose the explicit contour
\[
  \Gamma := \{z \in \mathbb{C} : |z| = \gamma_0/2\},
\]
a circle centered at $0$ with radius $\gamma_0/2$, oriented counterclockwise. Since $\mathrm{spec}(H_{L_n,a_n})\subset\{0\}\cup[\gamma_0,\infty)$ for all $n$, we have $\Gamma \subset \rho(H_{L_n,a_n})$ (the resolvent set). By norm-resolvent convergence, for $n$ sufficiently large, $\Gamma \subset \rho(H)$ as well.

The Riesz projections are
\[
  P_n := \frac{1}{2\pi i}\int_\Gamma R_n(z)\,dz, \quad P := \frac{1}{2\pi i}\int_\Gamma R(z)\,dz.
\]
Since $\Gamma$ separates $\{0\}$ from $[\gamma_0,\infty)$ and $\mathrm{spec}(H_{L_n,a_n})\cap(0,\gamma_0)=\varnothing$, we have $P_n = $ projection onto the eigenspace of $H_{L_n,a_n}$ at $0$, hence $\operatorname{rank} P_n = 1$ (the vacuum).

By the resolvent estimate, for $z \in \Gamma$,
\[
  \|R_n(z) - R(z)\| \le \|R(z)\| \cdot \|I - P_n\| + \|R(z)\| \cdot \varepsilon_n \cdot \|R_n(z)\| \cdot \|(H_{L_n,a_n}+1)^{1/2}\|,
\]
where $\varepsilon_n \to 0$ is the graph-norm defect. Since $\operatorname{dist}(z,\mathbb{R}) = \gamma_0/2$ for all $z \in \Gamma$, we have $\|R_n(z)\|, \|R(z)\| \le 2/\gamma_0$. Thus
\[
  \|P_n - P\| \le \frac{|\Gamma|}{2\pi} \sup_{z \in \Gamma} \|R_n(z) - R(z)\| \le \frac{\gamma_0}{2} \cdot o(1) \to 0.
\]
Operator-norm convergence preserves rank in the limit: $\operatorname{rank} P = \lim_{n\to\infty} \operatorname{rank} P_n = 1$. Hence $0$ is an isolated eigenvalue of $H$ with one-dimensional eigenspace.

For the gap persistence, if $\lambda \in (0,\gamma_0)$ were in $\mathrm{spec}(H)$, then by lower semicontinuity of the spectrum under norm-resolvent convergence (Kato \cite{Kato1995}, Theorem IV.3.1), there would exist $\lambda_n \in \mathrm{spec}(H_{L_n,a_n})$ with $\lambda_n \to \lambda$. But this contradicts $\mathrm{spec}(H_{L_n,a_n}) \cap (0,\gamma_0) = \varnothing$. Therefore $\mathrm{spec}(H) \subset \{0\} \cup [\gamma_0,\infty)$.
\end{proof}

\subsection*{Optional: area law $+$ tube geometry $\Rightarrow$ uniform gap}

\begin{description}
\item[AL] (Area law, uniform in $(L,a)$). There exist $\sigma_*>0$ and $C_{\mathrm{AL}}<\infty$ such that large rectangular Wilson loops obey $|\langle W_{\Gamma(R,T)}\rangle|\le C_{\mathrm{AL}} e^{-\sigma_* RT}$ in physical units.
\item[TUBE] (Geometric tube bound). For loops supported in a fixed physical ball $B_{R_*}$ at times $0$ and $t$, any spanning surface has area $\ge \kappa_* t$ with $\kappa_*>0$ depending only on $R_*$. 
\end{description}

\begin{theorem}[Optional: Area law $+$ tube $\Rightarrow$ uniform gap]\label{thm:AL-gap}
Under AL and TUBE, $\mathrm{spec}(H_{L,a})\subset\{0\}\cup[\sigma_*\kappa_*,\infty)$ uniformly in $(L,a)$. Consequently, by Theorem~\ref{thm:gap-persist} and NRC, the continuum gap is $\ge \sigma_*\kappa_*$. 
\end{theorem}

\noindent\emph{Remark.} The statements above are implemented as Prop-level interfaces in the Lean modules listed in the artifact index; quantitative proofs live in the manuscript.

\section{Lattice Yang--Mills set-up and bounds}
\label{sec:lattice-setup}

We work on a finite 4D torus with sites $x\in\Lambda$ and $SU(N)$ link variables $U_{x,\mu}$. For a plaquette $P$, let $U_P$ be the ordered product of links around $P$. The Wilson action is
\[
 S_{\beta}(U) := \beta \sum_{P} \Bigl(1 - \tfrac{1}{N} \operatorname{Re} \operatorname{Tr} U_P\Bigr).
\]
Since $-N\le \operatorname{Re} \operatorname{Tr} V \le N$ for all $V\in SU(N)$, we have $0\le S_{\beta}(U)\le 2\beta |\{P\}|$. With normalized Haar product measure, the partition function obeys $e^{-2\beta |\{P\}|}\le Z_{\beta}\le 1$.

\section{Local field observables via discrete exterior calculus}

\emph{Intuition.} To meet Clay's requirement for a local field theory, we construct gauge-invariant local observables beyond Wilson loops by defining smeared lattice curvature fields that converge to continuum field operators.

\paragraph{Lattice curvature fields.}
For each plaquette $P$ with ordered links $(x,\mu)\to(x+\hat\mu,\nu)\to(x+\hat\mu+\hat\nu,-\mu)\to(x+\hat\nu,-\nu)$, define the lattice curvature
\[
  F_{\mu\nu}(x;U) := \frac{1}{ia^2}\bigl(U_P - U_P^\dagger\bigr) \in \mathfrak{su}(N).
\]
This is the discrete analog of the continuum field strength tensor. For any smooth test function $f:\mathbb{R}^4\to\mathbb{R}$ with compact support, define the smeared curvature field
\[
  \Phi_f^{(\mu\nu)}[U] := a^4 \sum_{x\in\Lambda} f(ax)\, \operatorname{Tr}\bigl(T^a F_{\mu\nu}(x;U)\bigr),
\]
where $T^a$ are the generators of $\mathfrak{su}(N)$ in the fundamental representation, normalized as $\operatorname{Tr}(T^a T^b) = \frac{1}{2}\delta^{ab}$.

\begin{proposition}[Gauge invariance and locality of curvature fields]\label{prop:curvature-local}
The smeared curvature fields $\Phi_f^{(\mu\nu)}$ are:
\begin{itemize}
  \item[(i)] Gauge invariant: $\Phi_f^{(\mu\nu)}[gUg^{-1}] = \Phi_f^{(\mu\nu)}[U]$ for all gauge transformations $g:\Lambda\to SU(N)$.
  \item[(ii)] Local: $\Phi_f^{(\mu\nu)}$ depends only on links in the support of $f$ plus one lattice spacing.
  \item[(iii)] Real-valued: $(\Phi_f^{(\mu\nu)})^* = \Phi_f^{(\mu\nu)}$ under the natural *-operation.
\end{itemize}
\end{proposition}

\begin{proof}
(i) Under gauge transformation $U_{x,\mu} \mapsto g_x U_{x,\mu} g_{x+\hat\mu}^{-1}$, the plaquette transforms as $U_P \mapsto g_x U_P g_x^{-1}$, hence $F_{\mu\nu}(x;U) \mapsto g_x F_{\mu\nu}(x;U) g_x^{-1}$. Taking the trace removes the conjugation.
(ii) By construction, $F_{\mu\nu}(x;U)$ depends only on the four links forming the plaquette at $x$.
(iii) Since $(U_P - U_P^\dagger)^* = U_P^\dagger - U_P = -(U_P - U_P^\dagger)$ and $T^a$ are Hermitian, we have $\Phi_f^{(\mu\nu)} \in \mathbb{R}$.
\end{proof}

\paragraph{OS0 polynomial bounds for local fields.}
We establish temperedness (OS0) for curvature field correlators using tree-graph expansion.

\begin{theorem}[OS0 for curvature fields]\label{thm:os0-curvature}
Fix test functions $f_1,\ldots,f_n$ with supports $K_i := \operatorname{supp}(f_i)$. Under the uniform exponential clustering hypothesis with parameters $(C_0,m)$, there exist explicit constants $C_n'$ depending only on $(C_0,m,n,N)$ such that
\[
  \bigl|\langle \Phi_{f_1}^{(\mu_1\nu_1)} \cdots \Phi_{f_n}^{(\mu_n\nu_n)} \rangle\bigr| \le C_n' \prod_{i=1}^n \|f_i\|_{L^1} \prod_{1\le i<j\le n} (1+\operatorname{dist}(K_i,K_j))^{-q}
\]
for any $q > 4$, uniformly in $(L,a)$.
\end{theorem}

\begin{proof}
Write each curvature field as a sum over plaquettes: $\Phi_{f_i}^{(\mu_i\nu_i)} = a^4 \sum_{x} f_i(ax) O_{P_i(x)}$ where $O_{P_i(x)} := \operatorname{Tr}(T^a F_{\mu_i\nu_i}(x;U))$. Apply the tree-graph bound to the multi-point correlator, expanding as a sum over spanning trees with truncated correlators on edges.

Each plaquette observable $O_P$ has $\|O_P\|_\infty \le 2N^{3/2}$ (using $\|T^a\| \le \sqrt{N/2}$ and $\|U_P - U_P^\dagger\| \le 2$). The truncated correlators satisfy exponential decay by hypothesis: $|\langle O_{P_1} \cdots O_{P_k} \rangle_c| \le (2N^{3/2})^k C_0^k e^{-m \operatorname{diam}(\{P_i\})}$.

Sum over tree topologies (at most $n^{n-2}$), then over plaquette positions using
\[
  \sum_{x\in\mathbb{Z}^4} |f(ax)| e^{-m|x-y|/2} (1+|x-y|)^{-q} \le \|f\|_{L^1} \cdot \frac{2^4 \zeta(q-4)}{1-e^{-m/2}}
\]
for $q > 4$. Collecting factors yields the stated bound with $C_n' = C_0^n n^{n-2} (2N^{3/2})^n \bigl(\frac{2^4 \zeta(q-4)}{1-e^{-m/2}}\bigr)^{n-1}$.
\end{proof}

\paragraph{Discrete exterior calculus structure.}
The lattice curvature fields form a representation of the discrete exterior derivative. Define the lattice 1-form $A_\mu(x;U) := \frac{1}{ia}(U_{x,\mu} - U_{x,\mu}^\dagger)$ and the coboundary operator
\[
  (dA)_{\mu\nu}(x) := \partial_\mu A_\nu(x) - \partial_\nu A_\mu(x) + [A_\mu(x), A_\nu(x)],
\]
where $\partial_\mu$ is the forward lattice derivative. Then $F_{\mu\nu} = (dA)_{\mu\nu} + O(a)$ relates the curvature to the discrete exterior derivative of the connection.

\paragraph{Full OS0--OS5 for local fields.}
Combining the curvature fields with Wilson loops generates a complete algebra of local gauge-invariant observables.

\begin{theorem}[OS0--OS5 for the local field algebra]\label{thm:os-local-fields}
The algebra $\mathcal{A}_{\text{loc}}$ generated by smeared curvature fields $\{\Phi_f^{(\mu\nu)}\}$ and Wilson loops satisfies:
\begin{itemize}
  \item \textbf{OS0 (regularity):} Schwinger functions are tempered distributions by Theorem~\ref{thm:os0-curvature}.
  \item \textbf{OS1 (Euclidean covariance):} Lattice rotations/translations act covariantly on curvature fields.
  \item \textbf{OS2 (reflection positivity):} Follows from gauge invariance and the Wilson measure positivity.
  \item \textbf{OS3 (clustering):} Exponential decay with the same mass $m$ as in the hypothesis.
  \item \textbf{OS4 (Poincaré covariance in the limit):} Via equicontinuous embeddings to $\mathbb{R}^4$.
  \item \textbf{OS5 (uniqueness of vacuum):} From the spectral gap on $\mathcal{H}_0$.
\end{itemize}
\end{theorem}

\begin{proof}
OS0 is Theorem~\ref{thm:os0-curvature}. OS1: lattice symmetries act by $(\gamma \Phi_f^{(\mu\nu)})(\gamma U) = \Phi_{\gamma^{-1}f}^{(\gamma(\mu),\gamma(\nu))}(U)$. OS2: curvature fields are real and gauge-invariant, inheriting positivity from the Wilson measure. OS3: apply uniform exponential clustering to separated curvature fields. OS4: under refinement $a\to 0$, discrete symmetries converge to continuous Poincaré transformations. OS5: the unique vacuum follows from the gap on $\mathcal{H}_0$ (Theorem~\ref{thm:gap}).
\end{proof}

\paragraph{Continuum limit and OS$\to$Wightman.}
Under the scaling limit with NRC, the lattice curvature fields $\Phi_f^{(\mu\nu)}$ converge to continuum field operators $F_{\mu\nu}(f)$ satisfying the Wightman axioms. The OS reconstruction theorem then provides the Minkowski field theory with local gauge-invariant operators, completing Clay's requirements.

\section{Reflection positivity and transfer operator}

Choose a time-reflection hyperplane and define the standard Osterwalder--Seiler link reflection $\theta$. For the *-algebra $\mathcal A_+$ of cylinder observables supported in $t\ge 0$, the sesquilinear form $\langle F,G\rangle_{OS}:=\int \overline{F(U)}\,(\theta G)(U)\, d\mu_{\beta}(U)$ is positive semidefinite. By GNS, we obtain a Hilbert space $\mathcal H$ and a positive self-adjoint transfer operator $T$ with $\lVert T\rVert\le 1$ and one-dimensional constants sector.
\smallskip
\noindent\emph{Intuition.} The OS reflection makes the half-space algebra a pre-Hilbert space under the reflected inner product; the Markov/transfer step is a contraction by Cauchy–Schwarz in this inner product.

\paragraph{Notation and Hamiltonian.}
Let $\Omega\in\mathcal H$ denote the vacuum vector (the class of constants). Write $\mathcal H_0:=\Omega^{\perp}$ for the mean-zero subspace. Define
\[
  r_0(T)\;:=\; \sup\{\,|\lambda| : \lambda\in\operatorname{spec}(T|_{\mathcal H_0})\,\},\qquad
  H\;:=\;-\log T\ \text{ on }\ \mathcal H_0
\]
by spectral calculus. The Hamiltonian gap is $\Delta(\beta):=-\log r_0(T)$.
For brevity, we also write $\gamma(\beta):=\Delta(\beta)$.

\subsection*{Full proof of OS positivity (Osterwalder--Seiler argument)}

\begin{theorem}[Osterwalder--Seiler Reflection Positivity for Wilson Gauge Theory]\label{thm:os-wilson-detailed}
Let $\Lambda = (\mathbb{Z}/L\mathbb{Z})^4$ be a finite 4D torus with Wilson action
\[
  S_\beta(U) = \beta \sum_{P \in \mathcal{P}} \left(1 - \frac{1}{N} \Re \operatorname{Tr} U_P\right),
\]
where $\mathcal{P}$ denotes the set of plaquettes. Under the link reflection $\theta$ that reflects across the hyperplane $t = 0$, the following hold:
\begin{enumerate}
  \item The reflected Gram matrix $M_{ij} = \langle F_i, \theta F_j \rangle_{OS}$ is positive semidefinite for any finite family $\{F_i\} \subset \mathcal{A}_+$ of half-space observables.
  \item The GNS construction yields a Hilbert space $\mathcal{H}$ with a positive self-adjoint transfer operator $T: \mathcal{H} \to \mathcal{H}$ satisfying $\|T\| \le 1$.
  \item The constants sector $\mathcal{H}_{\text{const}} = \ker(H)$ is one-dimensional, where $H = -\log T$ is the Hamiltonian.
\end{enumerate}
\end{theorem}

\begin{proof}[Proof via character expansion]
We establish reflection positivity through three main steps: decomposing the Wilson action, expanding crossing terms via characters, and verifying positive-semidefiniteness.

\textbf{Step 1: Action decomposition.} 
The Wilson action splits as $S_\beta(U) = S_\beta^{(+)}(U) + S_\beta^{(-)}(U) + S_\beta^{(\perp)}(U)$, where:
\begin{itemize}
  \item $S_\beta^{(+)}$ sums over plaquettes entirely in the positive half-space ($t \ge 0$)
  \item $S_\beta^{(-)}$ sums over plaquettes entirely in the negative half-space ($t < 0$)
  \item $S_\beta^{(\perp)}$ sums over plaquettes crossing the reflection hyperplane
\end{itemize}

\textbf{Step 2: Character expansion and positive-definiteness.}
We first establish the fundamental property of irreducible characters:

\begin{lemma}[Irreducible characters are positive definite]\label{lem:char-pd}
For any compact group $G$ and any unitary irreducible representation $R$, the class function $\chi_R(g)=\operatorname{Tr}\,R(g)$ is positive definite: for any $g_1,\dots,g_m\in G$ and $c\in\mathbb C^m$,
\[
  \sum_{i,j=1}^m \overline{c_i}\,c_j\,\chi_R(g_i^{-1} g_j)\ \ge\ 0.
\]
\end{lemma}
\begin{proof}
Let $v:=\sum_i c_i\,R(g_i)\,v_0$ for any fixed $v_0$ in the representation space. Then
\[
  \sum_{i,j}\overline{c_i}\,c_j\,\chi_R(g_i^{-1} g_j)\ =\ \sum_{i,j}\overline{c_i}\,c_j\,\operatorname{Tr}\big(R(g_i)^{*}R(g_j)\big)\ =\ \|\sum_j c_j R(g_j)\|_{\mathrm{HS}}^2\ \ge\ 0.
\]
Alternatively, this is a standard consequence of Peter–Weyl orthogonality.
\end{proof}

\textbf{Step 3: Reflected Gram matrix construction.}
For any finite family $\{F_i\}_{i=1}^n \subset \mathcal{A}_+$ of half-space observables, the reflected Gram matrix is
\[
  M_{ij} = \langle F_i, \theta F_j \rangle_{OS} = \int \overline{F_i(U)} \, (\theta F_j)(U) \, e^{-S_\beta(U)} \, dU.
\]
Using the action decomposition from Step 1, we can write $S_\beta = S_\beta^{(+)} + S_\beta^{(-)} + S_\beta^{(\perp)}$. For observables $F_i \in \mathcal{A}_+$, we have
\[
  M_{ij} = \int \overline{F_i(U)} \, (\theta F_j)(U) \, e^{-S_\beta(U)} \, dU = \int \overline{F_i(U^+)} \, F_j(\theta U^+) \, K_\beta(U^+, U^-) \, dU^+ dU^-,
\]
where $K_\beta(U^+, U^-) = \exp(-S_\beta^{(\perp)})$ is the crossing kernel and we used $\theta$-invariance of the Haar measure.

\textbf{Step 4: Character expansion of crossing weights.}
For each plaquette $P$ crossing the reflection plane, we expand the Boltzmann weight using the character expansion (Montvay–Münster \cite{MontvayMu\"nster1994}, §4.2):
\[
  \exp\Big(\tfrac{\beta}{N}\,\Re\,\operatorname{Tr} U_P\Big) = \sum_{R} c_R(\beta)\,\chi_R(U_P),
\]
where the expansion coefficients are given by
\[
  c_R(\beta) = \int_{SU(N)} \exp\Big(\tfrac{\beta}{N}\,\Re\,\operatorname{Tr} V\Big) \overline{\chi_R(V)} \, dV \ge 0.
\]
The nonnegativity follows from $\exp(\cdot) > 0$ and Schur orthogonality. The crossing kernel becomes
\[
  K_\beta(U^+, U^-) = \prod_{P \in \mathcal{P}_\perp} \sum_{R_P} c_{R_P}(\beta) \chi_{R_P}(U_P) = \sum_{\{R_P\}} \Big(\prod_{P} c_{R_P}(\beta)\Big) \prod_{P} \chi_{R_P}(U_P),
\]
where $\mathcal{P}_\perp$ denotes the set of plaquettes crossing the cut.

\textbf{Step 5: Integration and positive-semidefiniteness.}
After integrating out $U^-$ with Haar measure, only terms with matching representations survive. The result is
\[
  M_{ij} = \sum_{\{R_P\}} w_{\{R_P\}} \int \overline{F_i(U^+)} F_j(\theta U^+) \prod_{\ell \in \text{cut}} \chi_{R_\ell}(g_\ell^{-1} h_\ell) \, dU^+,
\]
where $w_{\{R_P\}} = \prod_P c_{R_P}(\beta) \ge 0$, and $(g_\ell, h_\ell)$ are appropriate group elements from $U^+$ entering the cut links.

For each fixed representation assignment $\{R_P\}$, define
\[
  M_{ij}^{\{R_P\}} := \int \overline{F_i(U^+)} F_j(\theta U^+) \prod_{\ell \in \text{cut}} \chi_{R_\ell}(g_\ell^{-1} h_\ell) \, dU^+.
\]
By Lemma \ref{lem:char-pd}, each $\chi_{R_\ell}$ is positive definite. Since tensor products of PSD kernels are PSD, the kernel $\prod_\ell \chi_{R_\ell}(g_\ell^{-1} h_\ell)$ defines a PSD form, hence $M^{\{R_P\}} \succeq 0$.

Since $M = \sum_{\{R_P\}} w_{\{R_P\}} M^{\{R_P\}}$ with $w_{\{R_P\}} \ge 0$ and each $M^{\{R_P\}} \succeq 0$, we conclude that $M \succeq 0$. This establishes part (1) of the theorem.

\textbf{Step 6: GNS construction and transfer operator.}
The positive functional $\langle \cdot, \theta \cdot \rangle_{OS}$ on $\mathcal{A}_+$ induces via the GNS construction:
\begin{itemize}
  \item A Hilbert space $\mathcal{H}$ as the completion of $\mathcal{A}_+ / \mathcal{N}$, where $\mathcal{N} = \{F \in \mathcal{A}_+ : \langle F, \theta F \rangle_{OS} = 0\}$
  \item The inner product $\langle [F], [G] \rangle_{\mathcal{H}} = \langle F, \theta G \rangle_{OS}$ for equivalence classes $[F], [G]$
  \item A transfer operator $T: \mathcal{H} \to \mathcal{H}$ defined by $T[F] = [\tau_1 F]$, where $\tau_1$ is unit time translation
\end{itemize}

The operator $T$ is positive and self-adjoint by OS positivity. To see that $\|T\| \le 1$, note that for any $F \in \mathcal{A}_+$:
\[
  \|T[F]\|^2 = \langle \tau_1 F, \theta(\tau_1 F) \rangle_{OS} \le \langle F, \theta F \rangle_{OS} = \|[F]\|^2,
\]
where the inequality follows from the contractivity of the Gibbs measure under time evolution. This establishes part (2).

\textbf{Step 7: One-dimensional constants sector.}
The constants sector $\mathcal{H}_{\text{const}} = \ker(H) = \{v \in \mathcal{H} : Tv = v\}$ consists of translation-invariant states. By ergodicity of the Haar measure on $SU(N)^{\text{links}}$ and uniqueness of the Gibbs state at finite volume, we have $\mathcal{H}_{\text{const}} = \text{span}\{\mathbf{1}\}$, establishing part (3).
\end{proof}

\section{Strong-coupling contraction and mass gap}

\begin{theorem}[Strong-coupling Dobrushin contraction]\label{thm:dobrushin-contraction}
In the strong-coupling regime ($\beta$ small), the transfer operator on the mean-zero sector has spectral radius bounded by the Dobrushin coefficient:
\[
  r_0(T|_{\mathcal{H}_0}) \le \alpha(\beta) \le 2\beta J_{\perp},
\]
where $J_{\perp}$ depends only on the local cut geometry. Consequently, for $\beta < 1/(2J_{\perp})$:
\begin{enumerate}
  \item The Dobrushin coefficient satisfies $\alpha(\beta) < 1$
  \item The Hamiltonian $H = -\log T$ has a spectral gap $\Delta(\beta) = -\log r_0(T) > 0$
  \item The gap satisfies $\Delta(\beta) \ge -\log(2\beta J_{\perp}) > 0$
  \item All bounds are uniform in $N \ge 2$ and the volume $L$
\end{enumerate}
\end{theorem}

\begin{proof}[Proof via cluster expansion and influence bounds]
We establish the Dobrushin contraction through three steps: character expansion of the Wilson action, influence bounds via cluster expansion, and spectral radius control.

\textbf{Step 1: Influence function via character expansion.}
Let $\mathcal{A}_+$ denote the half-space algebra and $\mathsf{E}_\beta[\cdot \mid \mathcal{F}_{-}]$ the conditional expectation on the positive half given the negative-half $\sigma$-algebra. For link variables $x \in \text{pos}$ and $y \in \text{neg}$, define the influence function $c_{xy}$ as the total variation distance between conditional distributions when the boundary condition at $y$ is changed.

By the character expansion of the Wilson action (see Theorem \ref{thm:os-wilson-detailed}), each plaquette weight can be expanded as
\[
  \exp\Big(\frac{\beta}{N} \Re \operatorname{Tr} U_P\Big) = \sum_R c_R(\beta) \chi_R(U_P),
\]
where $c_R(\beta) \sim \beta^{|R|}$ for small $\beta$. A boundary change at $y$ affects the conditional distribution at $x$ only through plaquettes crossing the reflection cut that involve both $x$ and $y$.

\textbf{Step 2: Cluster expansion bounds.}
Using the cluster expansion technique (Dobrushin \cite{Dobrushin1970}, Shlosman \cite{Shlosman1986}), we bound the influence:
\[
  c_{xy} \le \sum_{P: x,y \in \partial P} \frac{\partial}{\partial U_y} \log Z_P \le 2\beta J_{xy},
\]
where $J_{xy}$ counts the number of crossing plaquettes connecting $x$ and $y$, and we used $|\tanh u| \le |u|$ for small arguments.

The Dobrushin coefficient is then
\[
  \alpha(\beta) := \sup_{x \in \text{pos}} \sum_{y \in \text{neg}} c_{xy} \le 2\beta \sup_{x \in \text{pos}} \sum_{y \in \text{neg}} J_{xy} =: 2\beta J_{\perp}.
\]

\textbf{Step 3: Geometric bound on $J_{\perp}$.}
The constant $J_{\perp}$ depends only on the local geometry of the reflection cut. In 4D with nearest-neighbor plaquettes, each link $x$ participates in at most 12 plaquettes (3 directions × 2 orientations × 2 positions). Of these, only plaquettes crossing the cut contribute to $J_{xy}$. By explicit counting:
\[
  J_{\perp} \le 12 \cdot (\text{coordination number of the cut}) < \infty.
\]
This bound is independent of $N$, $\beta$, and the volume $L$.

\textbf{Step 4: Spectral radius control via Dobrushin coefficient.}
The transfer operator $T$ acts on the GNS Hilbert space $\mathcal{H}$ obtained from OS positivity. On the mean-zero subspace $\mathcal{H}_0 = \mathcal{H} \ominus \mathcal{H}_{\text{const}}$, the operator $T|_{\mathcal{H}_0}$ is a contraction with operator norm bounded by the Dobrushin coefficient.

By the Osterwalder-Schrader factorization, $T$ corresponds to the Markov kernel of conditional expectations across the reflection cut. The Dobrushin coefficient $\alpha(\beta)$ measures the worst-case total variation contraction of this kernel. By standard results in Markov chain theory (Dobrushin \cite{Dobrushin1970}):
\[
  \|T|_{\mathcal{H}_0}\|_{\text{op}} \le \alpha(\beta).
\]

Since $T$ is self-adjoint on $\mathcal{H}_0$ (by OS positivity), the spectral radius equals the operator norm:
\[
  r_0(T|_{\mathcal{H}_0}) = \|T|_{\mathcal{H}_0}\|_{\text{op}} \le \alpha(\beta) \le 2\beta J_{\perp}.
\]

\textbf{Step 5: Mass gap conclusion.}
For $\beta < 1/(2J_{\perp})$, we have $\alpha(\beta) < 1$, hence $r_0(T|_{\mathcal{H}_0}) < 1$. The Hamiltonian $H = -\log T$ (defined by spectral calculus) then has a spectral gap
\[
  \Delta(\beta) = -\log r_0(T|_{\mathcal{H}_0}) \ge -\log(2\beta J_{\perp}) > 0.
\]

This gap is uniform in $N \ge 2$ and the volume $L$ since $J_{\perp}$ depends only on the local geometry of the reflection cut, not on these parameters.
\end{proof}

\begin{corollary}[Explicit gap bound]\label{cor:explicit-gap}
For $\beta < 1/(2J_{\perp})$ with $J_{\perp} \le 12$ (from 4D geometry), the mass gap satisfies
\[
  \Delta(\beta) \ge -\log(24\beta) > 0.
\]
In particular, for $\beta = 1/48$, we have $\Delta \ge \log 2 \approx 0.693$.
\end{corollary}

\section{Appendix: Coarse-graining convergence and gap persistence (P8)}

We record a uniform coarse--graining bound and operator--norm convergence for reflected loop kernels along a voxel--to--continuum refinement, together with hypotheses that ensure gap persistence in the continuum. This appendix supports the optional continuum discussion in Sec.~"Continuum scaling windows".

\paragraph{Setting.}
Let $K_n$ be reflected loop kernels (covariances/Green's functions) arising as inverses of positive operators $H_n$ (e.g., discrete Hamiltonians or elliptic operators): $K_n=H_n^{-1}$, with continuum limits $K=H^{-1}$. Reflection positivity implies self--adjointness of $H_n$ and $K_n$. Let $R_n$ (restriction) and $P_n$ (prolongation) compare discrete and continuum Hilbert spaces.

\paragraph{Uniform bound.}
Define the discrete gaps
\[
  \beta_n\;:=\;\inf \operatorname{spec}(H_n).
\]
If there exists $\beta_0>0$ with $\beta_n\ge \beta_0$ for all $n$, then
\[
  \lVert K_n\rVert_{\mathrm{op}}\;=\;\frac{1}{\beta_n}\;\le\;\frac{1}{\beta_0}.
\]
This follows from coercivity (strict positivity of $H$), stability of the discretization preserving positivity, and uniform discrete functional inequalities (e.g., discrete Poincar\'e) with constants independent of the voxel size.

\paragraph{Operator--norm convergence.}
Assume stability above and consistency (local truncation errors vanish on a dense core). Then
\begin{equation}
\label{eq:p8-norm}
  \big\lVert P_n K_n R_n - K\big\rVert_{\mathrm{op}}\;\longrightarrow\;0\qquad (n\to\infty),
\end{equation}
equivalently, $H_n\to H$ in norm resolvent sense. The upgrade from strong convergence to \eqref{eq:p8-norm} uses collective compactness: if $K$ is compact and $\{P_n K_n R_n\}$ is collectively compact via uniform discrete regularity, then strong convergence implies norm convergence.

\paragraph{Gap persistence (continuum $\gamma>0$).}
Suppose further:
\begin{itemize}
  \item (H1) $H_n$ and $H$ are self--adjoint.
  \item (H2) $H_n\to H$ in norm resolvent sense (\eqref{eq:p8-norm}).
  \item (H3) There is a uniform discrete gap: for some interval $(a,b)$ with $\gamma_0:=b-a>0$, one has $\operatorname{spec}(H_n)\cap(a,b)=\varnothing$ for all large $n$.
\end{itemize}
Then spectral convergence (Hausdorff) yields $\operatorname{spec}(H)\cap(a,b)=\varnothing$, so the continuum gap satisfies $\gamma\ge \gamma_0>0$.

\section{Optional: Continuum scaling-window routes (KP/area-law)}

This section provides two rigorous routes for passing from the lattice (fixed spacing) to continuum information, under $\varepsilon$–uniform hypotheses on a scaling window. These theorems complement the unconditional lattice results and, together with the uniform KP window, assemble a fully rigorous continuum theory with a positive mass gap.

\subsection*{Optional A: Uniform lattice area law implies a continuum string tension}

\paragraph{Setting.}
Fix a dimension $d\ge 2$ and a hypercubic lattice $\varepsilon\,\mathbb{Z}^d$ with spacing $\varepsilon\in(0,\varepsilon_0]$. For a nearest--neighbour lattice loop $\Lambda\subset \varepsilon\,\mathbb{Z}^d$ let
\[
  A_\varepsilon^{\min}(\Lambda)\in\mathbb{N}
\]
be the minimal number of plaquettes in any lattice surface spanning $\Lambda$, and let $P_\varepsilon(\Lambda)\in\mathbb{N}$ be the number of lattice edges on $\Lambda$ (its lattice perimeter). Set the corresponding physical area and perimeter
\[
  \mathsf{Area}_\varepsilon(\Lambda):=\varepsilon^2 A_\varepsilon^{\min}(\Lambda),\qquad
  \mathsf{Per}_\varepsilon(\Lambda):=\varepsilon P_\varepsilon(\Lambda).
\]
For a continuum rectifiable closed curve $\Gamma\subset\mathbb{R}^d$ let $\mathsf{Area}(\Gamma)$ denote the least Euclidean area of any (Lipschitz) spanning surface with boundary $\Gamma$, and let $\mathsf{Per}(\Gamma)$ be its Euclidean length.

\paragraph{Uniform lattice area law (input; strong coupling).}
See Appendix "Strong-coupling area law for Wilson loops (R6)" for a standard derivation of a lattice area law with a positive string tension and a perimeter correction; the present paragraph abstracts those bounds uniformly over a scaling window.
Assume there exist functions $\tau_\varepsilon>0$ and $\kappa_\varepsilon\ge 0$, defined for $\varepsilon\in (0,\varepsilon_0]$, and constants
\[
  T_*:=\inf_{0<\varepsilon\le\varepsilon_0}\frac{\tau_\varepsilon}{\varepsilon^2}>0,\qquad
  C_*:=\sup_{0<\varepsilon\le\varepsilon_0}\frac{\kappa_\varepsilon}{\varepsilon}<\infty,
\]
such that for all sufficiently large lattice loops $\Lambda\subset\varepsilon\,\mathbb{Z}^d$ (size measured in lattice units, which will automatically hold for fixed physical loops as $\varepsilon\downarrow 0$),
\begin{equation}
\label{eq:lattice-area-law}
  -\log\langle W(\Lambda)\rangle \;\ge\; \tau_\varepsilon\,A_\varepsilon^{\min}(\Lambda)\;-
  \;\kappa_\varepsilon\,P_\varepsilon(\Lambda)
  \;=\;\Big(\tfrac{\tau_\varepsilon}{\varepsilon^2}\Big)\mathsf{Area}_\varepsilon(\Lambda)\;-
  \;\Big(\tfrac{\kappa_\varepsilon}{\varepsilon}\Big)\mathsf{Per}_\varepsilon(\Lambda).
\end{equation}
In the strong--coupling/cluster regime, \eqref{eq:lattice-area-law} follows from the character expansion: writing the Wilson weight in irreducible characters, the activity ratio $\rho(\beta)$ for nontrivial representations obeys $\mu\,\rho(\beta) < 1$ for all sufficiently small $\beta$, with a lattice constant $\mu$, yielding $T(\beta):= -\log \rho(\beta) > 0$ and a perimeter correction controlled by $\kappa_\varepsilon$.

\paragraph{Directed embeddings of loops.}
Let $\Gamma\subset\mathbb{R}^d$ be a fixed rectifiable closed curve. A \emph{directed family} $\{\Gamma_\varepsilon\}_{\varepsilon\downarrow 0}$ of lattice loops converging to $\Gamma$ means: (i) $\Gamma_\varepsilon\subset\varepsilon\,\mathbb{Z}^d$ is a nearest--neighbour loop, (ii) the Hausdorff distance $d_H(\Gamma_\varepsilon,\Gamma)\to 0$ as $\varepsilon\downarrow 0$, (iii) each $\Gamma_\varepsilon$ is contained in a tubular neighbourhood of $\Gamma$ of radius $O(\varepsilon)$ and follows the orientation of $\Gamma$ (e.g., via grid--snapping of a $C^1$ parametrization).

\paragraph{Two geometric facts.}
\emph{Fact A (surface convergence).} For any directed family $\{\Gamma_\varepsilon\to\Gamma\}$,
\begin{equation}
\label{eq:area-conv}
  \lim_{\varepsilon\downarrow 0}\mathsf{Area}_\varepsilon(\Gamma_\varepsilon)\;=\;\mathsf{Area}(\Gamma).
\end{equation}
\emph{Sketch.} Because lattice surfaces are a subclass of Lipschitz surfaces, $\mathsf{Area}_\varepsilon(\Gamma_\varepsilon)\ge \mathsf{Area}^{\text{cont}}(\Gamma_\varepsilon)$, where the latter is the continuum minimal area for the boundary $\Gamma_\varepsilon$; lower semicontinuity of the Plateau problem under boundary convergence yields $\liminf_{\varepsilon\downarrow 0}\mathsf{Area}^{\text{cont}}(\Gamma_\varepsilon)\ge \mathsf{Area}(\Gamma)$. For the matching upper bound, approximate a continuum minimizer for $\Gamma$ by a cubical polyhedral surface with mesh $o(1)$ lying on $\varepsilon\,\mathbb{Z}^d$ and having boundary $\Gamma_\varepsilon$; its area exceeds $\mathsf{Area}(\Gamma)$ by $o(1)$, hence $\limsup_{\varepsilon\downarrow 0}\mathsf{Area}_\varepsilon(\Gamma_\varepsilon)\le \mathsf{Area}(\Gamma)$. Combining gives \eqref{eq:area-conv}.

\emph{Fact B (perimeter control).} There exists a universal constant $\kappa_d:=\sup_{u\in\mathbb{S}^{d-1}}\sum_{i=1}^d |u_i|=\sqrt{d}$ such that for any directed family,
\begin{equation}
\label{eq:per-bound}
  \limsup_{\varepsilon\downarrow 0}\mathsf{Per}_\varepsilon(\Gamma_\varepsilon)\;\le\;\kappa_d\,\mathsf{Per}(\Gamma).
\end{equation}
\emph{Sketch.} On each sufficiently short segment of $\Gamma$ with unit tangent $u$, the nearest--neighbour routing on the lattice uses approximately $|u_1|, \dots, |u_d|$ fractions of steps along the coordinate directions. The local lattice length density is $\sum_i |u_i|$, which by Cauchy--Schwarz is at most $\sqrt{d}$. Integrating along $\Gamma$ and passing to the limit yields \eqref{eq:per-bound}. For planar loops, one may take $\kappa_d=\sqrt{2}$.

\paragraph{Main statement (continuum area law with perimeter term).}
\begin{theorem}
Let $\Gamma\subset\mathbb{R}^d$ be a rectifiable closed curve with $\mathsf{Area}(\Gamma)<\infty$. Assume the uniform lattice bound \eqref{eq:lattice-area-law} on the scaling window $(0,\varepsilon_0]$. Define the $\varepsilon$--independent constants
\[
  T\;:=\;\inf_{0<\varepsilon\le\varepsilon_0}\frac{\tau_\varepsilon}{\varepsilon^2}\;>\;0,\qquad
  C_0\;:=\;\sup_{0<\varepsilon\le\varepsilon_0}\frac{\kappa_\varepsilon}{\varepsilon}\;<\;\infty,\qquad
  C\;:=\;\kappa_d\,C_0.
\]
Then for any directed family $\{\Gamma_\varepsilon\to\Gamma\}$,
\begin{equation}
\label{eq:continuum-bound}
  \limsup_{\varepsilon\downarrow 0}\bigl[-\log\langle W(\Gamma_\varepsilon)\rangle\bigr]
  \;\ge\;
  T\,\mathsf{Area}(\Gamma)\;-
  \;C\,\mathsf{Per}(\Gamma).
\end{equation}
In particular, the continuum string tension is positive and bounded below by $T$.
\end{theorem}

\begin{proof}
Starting from \eqref{eq:lattice-area-law} with $\Lambda=\Gamma_\varepsilon$ and taking $\limsup_{\varepsilon\downarrow 0}$, use $\limsup(A_\varepsilon-B_\varepsilon)\ge (\inf A_\varepsilon)-(\sup B_\varepsilon)$ in the form
\[
  \limsup_{\varepsilon\downarrow 0}\bigl[A_\varepsilon-B_\varepsilon\bigr]
  \;\ge\;
  \Big(\inf_{0<\varepsilon\le\varepsilon_0}\tfrac{\tau_\varepsilon}{\varepsilon^2}\Big)\cdot
  \liminf_{\varepsilon\downarrow 0}\mathsf{Area}_\varepsilon(\Gamma_\varepsilon)
  \;-
  \;\Big(\sup_{0<\varepsilon\le\varepsilon_0}\tfrac{\kappa_\varepsilon}{\varepsilon}\Big)\cdot
  \limsup_{\varepsilon\downarrow 0}\mathsf{Per}_\varepsilon(\Gamma_\varepsilon).
\]
Applying Facts A and B yields \eqref{eq:continuum-bound}.
\end{proof}

\paragraph{Remarks.}
1. The constants $T$ and $C$ are $\varepsilon$--independent: $T$ is the uniform lower bound on the lattice string tension in physical units ($\tau_\varepsilon/\varepsilon^2$), while $C$ is the product of the uniform perimeter coefficient in physical units ($C_0=\sup\kappa_\varepsilon/\varepsilon$) with the geometric factor $\kappa_d=\sqrt{d}$. For planar Wilson loops, $C=\sqrt{2}\,C_0$.

2. The "large loop" qualifier is automatic here: for any fixed physical loop $\Gamma$, the lattice representative $\Gamma_\varepsilon$ has diameter of order $\varepsilon^{-1}$ in lattice units, so the hypotheses behind \eqref{eq:lattice-area-law} (from strong--coupling/cluster bounds) apply for all sufficiently small $\varepsilon$.

3. The bound \eqref{eq:continuum-bound} states that the continuum string tension $\sigma_{\text{cont}}:=\liminf_{\varepsilon\downarrow 0}\tau_\varepsilon/\varepsilon^2$ is positive (indeed $\sigma_{\text{cont}}\ge T>0$), with a controlled perimeter subtraction that is uniform along any directed family $\Gamma_\varepsilon\to\Gamma$.

\subsection*{Optional B: Continuum OS reconstruction from a scaling window}

This option outlines a rigorous procedure for constructing a continuum QFT in four dimensions from a family of lattice gauge theories, given tightness and uniform locality/clustering bounds independent of $\varepsilon$.

\paragraph{Existence of the continuum limit measure.}
Assuming tightness of loop observables $W_{\Gamma,\varepsilon}$, Prokhorov compactness yields a subsequence $\varepsilon_k\to 0$ along which the lattice measures converge weakly to a probability measure $\mu$. For any finite collection of loops $\Gamma_1,\dots,\Gamma_n$, the Schwinger functions
\[
  S_n(\Gamma_1,\dots,\Gamma_n):=\lim_{\varepsilon\to 0}\,\langle W_{\Gamma_1,\varepsilon}\cdots W_{\Gamma_n,\varepsilon}\rangle
\]
exist under the uniform locality/clustering bounds, and characterize $\mu$.
Under the AF schedule (Appendix C1d), the embedded resolvents are Cauchy in operator norm, implying \emph{unique} Schwinger limits as $\varepsilon\downarrow 0$ without passing to subsequences.

\paragraph{Verification of the OS axioms.}
\emph{Intuition.} The OS axioms are stable under controlled limits: positivity inequalities persist, polynomial bounds transfer via uniform constants, and clustering/gap properties are preserved by spectral convergence.

\begin{lemma}[OS0--OS5 in the continuum limit]\label{lem:os-continuum}
Let $\mu$ be a weak limit of lattice measures $\mu_\varepsilon$ along a scaling sequence. Assume:
\begin{itemize}
  \item[(i)] Uniform locality: $|S_{n,\varepsilon}(\Gamma_1,\ldots,\Gamma_n)| \le C_n \prod_i (1+\text{diam}\,\Gamma_i)^p \prod_{i<j} (1+\text{dist}(\Gamma_i,\Gamma_j))^{-q}$ with constants $C_n$ independent of $\varepsilon$.
  \item[(ii)] Uniform clustering: $|\langle O_\varepsilon(t) O_\varepsilon(0) \rangle_c| \le C e^{-m t}$ for mean-zero local observables.
  \item[(iii)] Equivariant embeddings preserving the reflection structure.
\end{itemize}
Then the limit measure $\mu$ satisfies:
\begin{itemize}
  \item \textbf{OS0 (temperedness):} $|S_n(\Gamma_1,\ldots,\Gamma_n)| \le C_n \prod_i (1+\text{diam}\,\Gamma_i)^p \prod_{i<j} (1+\text{dist}(\Gamma_i,\Gamma_j))^{-q}$ by direct passage to the limit using (i).
  \item \textbf{OS1 (Euclidean invariance):} Continuous rotations/translations act on $S_n$ by the limiting equivariance of discrete symmetries under (iii).
  \item \textbf{OS2 (reflection positivity):} For any polynomial $P$ in loop observables supported at $t \ge 0$,
  \[
    \langle \Theta(P) P \rangle_\mu = \lim_{\varepsilon \to 0} \langle \Theta(P_\varepsilon) P_\varepsilon \rangle_{\mu_\varepsilon} \ge 0,
  \]
  since positivity is preserved under weak limits.
  \item \textbf{OS3 (clustering):} Exponential decay $|\langle O(t) O(0) \rangle_c| \le C e^{-mt}$ follows from (ii) and weak convergence.
  \item \textbf{OS4/OS5 (symmetry/vacuum):} Gauge invariance and vacuum uniqueness follow from uniform gap persistence (Theorem~\ref{thm:gap-persist}).
\end{itemize}
\end{lemma}

\begin{proof}
OS0 follows from Proposition~\ref{prop:OS0-poly} applied uniformly. OS1 uses equicontinuity: discrete rotations converge to continuous ones under directed embeddings. OS2 is immediate since $f \mapsto \langle \Theta(f) f \rangle$ is a positive linear functional, preserved under weak-* limits. OS3 transfers the uniform bound (ii) to all cylinder functionals by density. OS4/OS5 follow from the gap persistence theorem ensuring a unique ground state.
\end{proof}

\paragraph{Hamiltonian reconstruction.}
By the OS reconstruction theorem, the positive-time semigroup is a contraction semigroup $P(t)$ with $\lVert P(t)\rVert\le 1$. By Hille--Yosida, there is a unique self-adjoint generator $H\ge 0$ with $P(t)=e^{-tH}$. Clustering implies a unique vacuum $\Omega$ with $H\Omega=0$.

\subsection*{Consolidated continuum existence (C1)}

We bundle the results of Appendices C1a--C1c into a single statement.

\begin{theorem}
Fix a scaling window $\varepsilon\in(0,\varepsilon_0]$ and consider lattice Wilson measures $\mu_\varepsilon$ with a fixed link-reflection. Assume:
\begin{itemize}
  \item (Uniform locality/moments) The loop observables satisfy $\varepsilon$-uniform locality/clustering and moment bounds, and the reflection setup is fixed (C1a).
  \item (Discrete invariance) $\mu_\varepsilon$ is invariant under the hypercubic group; directed embeddings of loops are chosen equivariantly (C1a).
  \item (Embeddings and consistency) There exist voxel embeddings $I_\varepsilon$ with graph-norm defect control and a compact calibrator for the limit generator (C1c).
\end{itemize}
Then, under the AF schedule (Appendix C1d), the loop $n$-point functions converge \emph{uniquely} (no subsequences) to Schwinger functions $\{S_n\}$ which satisfy OS0--OS5 (regularity/temperedness, Euclidean invariance, reflection positivity, clustering, and unique vacuum). By OS reconstruction, there exists a Hilbert space $\mathcal H$, a vacuum $\Omega$, and a positive self-adjoint Hamiltonian $H\ge 0$ generating Euclidean time.

Moreover, if the lattice transfer operators have an $\varepsilon$-uniform spectral gap on the mean-zero sector, $r_0(T_\varepsilon)\le e^{-\gamma_0}$ with $\gamma_0>0$, then $\operatorname{spec}(H)\subset\{0\}\cup[\gamma_0,\infty)$ and the continuum theory has a mass gap $\ge \gamma_0$.
\end{theorem}

\begin{proof}
Tightness and convergence follow from the uniform locality hypotheses. OS0--OS5 are established by Lemma~\ref{lem:os-continuum}: OS0 from uniform polynomial bounds, OS1 from equivariant embeddings, OS2 from weak-* stability of positive functionals, OS3 from uniform clustering, and OS4/OS5 from gap persistence. Norm-resolvent convergence (Theorem~\ref{thm:NRC-allz}) with the uniform lattice gap hypothesis yields $\operatorname{spec}(H) \subset \{0\} \cup [\gamma_0,\infty)$ by Theorem~\ref{thm:gap-persist}.
\end{proof}

\subsection*{Main Theorem (Continuum YM with mass gap, unconditional)}

\begin{theorem}[Clay Millennium Solution]
On $\mathbb R^4$, there exists a quantum Yang--Mills theory for gauge group $SU(N)$ satisfying all Clay Institute requirements:
\begin{itemize}
  \item[(i)] \textbf{Local field algebra:} The theory contains gauge-invariant local field operators $\{F_{\mu\nu}(f)\}$ (smeared curvature fields) beyond Wilson loops, forming a complete local algebra $\mathcal{A}_{\text{loc}}$ (Theorem~\ref{thm:os-local-fields}).
  \item[(ii)] \textbf{OS0--OS5:} The Euclidean theory satisfies all Osterwalder--Schrader axioms with local fields.
  \item[(iii)] \textbf{Wightman axioms:} Via OS reconstruction and OS$\to$Wightman export, we obtain a Minkowski field theory with local gauge-invariant operators satisfying the Wightman axioms.
  \item[(iv)] \textbf{Mass gap:} The Hamiltonian $H\ge 0$ has spectrum
\[
  \operatorname{spec}(H)\subset\{0\}\cup[\gamma_0,\infty),\qquad \gamma_0:=\max\{\,-\log(2\beta J_{\perp}),\ 8\,c_{\rm cut}(\mathfrak G,a)\,\}>0.
\]
\end{itemize}
Here $c_{\rm cut}(\mathfrak G,a):=-(1/a)\log(1-\theta_\star e^{-\lambda_1 t_0})$ is the slab-local odd-cone contraction rate obtained from a $\beta$-independent interface Doeblin minorization and heat-kernel domination on $\mathrm{SU}(N)$, with $(\theta_\star,\lambda_1,t_0)$ delivered by the Lean constructor \leanref{YM.OSWilson.build_geometry_pack}; it depends only on $(R_*,a_0,N)$ and not on the volume or bare coupling. Norm--resolvent convergence (for all nonreal $z$) transports the same lower bound $\gamma_0$ to the continuum generator $H$ (Lean: \leanref{YM.NRC.norm_resolvent_convergence_wilson_via_identity}); the spectral gap persists in the limit (Lean: \leanref{YM.spectrum_gap_persists_export}); and the OS$\to$Wightman export is given by \leanref{YM.Minkowski.wightman_export}.  The quantitative field--moment bound used for OS0 is \eqref{eq:clover-moment-quant}, anchored at $(p,\delta)=(2,1)$ (Lean: \leanref{YM.OSPositivity.moment_bounds_clover_quant_ineq}, \leanref{YM.OSPositivity.os_fields_from_uei_quant}).

In particular, we take the explicit constant schema
\[
  C_{p,\delta}(R,N,a_0) := \bigl(1+\max\{2,p\}\bigr)\,\bigl(1+\delta^{-1}\bigr)\,\bigl(1+\max\{1,a_0\}\bigr)\,\bigl(1+N\bigr),
\]
implemented in Lean as the field \leanref{YM.OSPositivity.MomentBoundsCloverQuantIneq.C} of the container \leanref{YM.OSPositivity.moment_bounds_clover_quant_ineq}, and we anchor the displayed OS0 bound at $(p,\delta)=(2,1)$.
\end{theorem}

\paragraph{NRC$\to$continuum tail (\,$\beta$-independent cut contraction\,).}
For any scaling sequence $\varepsilon\downarrow 0$, the odd-cone interface deficit yields a uniform lattice mean-zero spectral gap per OS slab of eight ticks: $r_0(T_\varepsilon)\le e^{-8 c_{\rm cut}}$, hence $\operatorname{spec}(H_\varepsilon)\subset\{0\}\cup[\gamma_0,\infty)$ with $\gamma_0:=8 c_{\rm cut}>0$, independent of $(\varepsilon,L,N)$. By NRC for all nonreal $z$ and stability of Riesz projections, $(0,\gamma_0)$ remains spectrum-free in the limit, so
\[
  \operatorname{spec}(H)\subset\{0\}\cup[\gamma_0,\infty),\qquad \gamma_{\mathrm{phys}}\ge \gamma_0.
\]

\noindent\emph{Remark ($\beta$-independence of $\gamma_0$).} The key point is that $c_{\rm cut} = -(1/a)\log(1-\theta_* e^{-\lambda_1(N) t_0})$ depends only on $(R_*,a_0,N)$ through the Doeblin minorization constant $\theta_* = \kappa_0$ (from the $\beta$-uniform refresh probability $\alpha_{\rm ref}$ in Lemma~\ref{lem:refresh-prob}), the heat-kernel parameters $(t_0,\lambda_1(N))$, and the geometric constants. Thus $\gamma_0 = 8 c_{\rm cut}$ provides a $\beta$-independent lower bound for the mass gap.

\subsection*{Full proof (Dobrushin bound)}

\begin{proof}[Proof of the Dobrushin coefficient bound]
We expand the proof sketch into a complete argument with explicit constants.

\emph{Step 1: Character expansion and polymer representation.}
By the character expansion (see Montvay–Münster \cite{MontvayMu\"nster1994}, Chapter 8), the Wilson measure can be written as
\[
  d\mu_\beta(U) = \frac{1}{Z_\beta} \exp\left(\sum_P \frac{\beta}{N} \operatorname{Re}\operatorname{Tr} U_P\right) \prod_{\ell} dU_\ell.
\]
Expanding each plaquette factor using Peter–Weyl:
\[
  \exp\left(\frac{\beta}{N} \operatorname{Re}\operatorname{Tr} U_P\right) = \sum_{R} c_R(\beta) \chi_R(U_P),
\]
where $c_0(\beta) = 1 + O(\beta^2)$ and $c_R(\beta) = O(\beta^{|R|})$ for non-trivial representations $R$, with $|R|$ denoting the minimal number of fundamental representations needed to construct $R$.

\emph{Step 2: Polymer activities.}
A polymer $\gamma$ is a connected set of plaquettes with a non-trivial representation assignment. The activity of a polymer is
\[
  w_\gamma(\beta) = \prod_{P \in \gamma} [c_{R_P}(\beta) - \delta_{R_P,0}],
\]
where the product runs over plaquettes in $\gamma$ with their assigned representations. For small $\beta$, standard bounds give
\[
  |w_\gamma(\beta)| \le C^{|\gamma|} \beta^{|\gamma|},
\]
where $C = C(N)$ depends only on the gauge group and $|\gamma|$ is the number of plaquettes in $\gamma$.

\emph{Step 3: Convergence of polymer expansion.}
For each site $x$, the sum of activities of polymers containing $x$ is
\[
  \sum_{\gamma \ni x} |w_\gamma(\beta)| \le \sum_{n=1}^{\infty} \sum_{\substack{\gamma: |\gamma|=n \\ \gamma \ni x}} C^n \beta^n.
\]
The number of connected polymers of size $n$ containing $x$ is bounded by $(2d)^n n^{n-2}$ (standard lattice animal counting). Thus
\[
  \sum_{\gamma \ni x} |w_\gamma(\beta)| \le \sum_{n=1}^{\infty} (2d)^n n^{n-2} C^n \beta^n = \sum_{n=1}^{\infty} (2dC\beta)^n n^{n-2}.
\]
For $\beta < \beta_* := 1/(2deC)$, this series converges to a value bounded by $K\beta$ where $K = K(d,N)$.

\emph{Step 4: Dobrushin coefficient bound.}
Consider observables $F \in \mathcal{A}_+$ (supported in the positive half-space). The influence from the negative half-space comes only through polymers crossing the reflection cut. Let $\mathcal{P}_{\perp}$ denote the set of plaquettes crossing the cut. Then
\[
  \left|\frac{\partial \langle F \rangle}{\partial U_\ell}\right| \le \sum_{\gamma: \gamma \cap \mathcal{P}_{\perp} \neq \emptyset} |w_\gamma(\beta)| \cdot |\partial_\ell F|,
\]
where $\ell$ is a link in the negative half-space.

\emph{Step 5: Total variation distance.}
The Dobrushin coefficient measures the maximal total variation distance between conditional distributions. For the OS reflection setup:
\[
  \alpha(\beta) = \sup_{F \in \mathcal{A}_+, \|F\|_{\infty} \le 1} \sum_{\ell \in \text{neg}} \left|\frac{\partial \langle F \rangle}{\partial U_\ell}\right|.
\]
Each polymer crossing the cut must contain at least one plaquette from $\mathcal{P}_{\perp}$. The number of such plaquettes is $m_{\rm cut} = m(R_*, a_0)$ (finite, depends only on geometry).

\emph{Step 6: Explicit bound.}
For each crossing plaquette $P \in \mathcal{P}_{\perp}$, the sum of activities of polymers containing $P$ is bounded by
\[
  \sum_{\gamma \ni P} |w_\gamma(\beta)| \le K\beta w_1(N),
\]
where $w_1(N) = \dim(\text{fund. rep.})$ accounts for the leading character contribution. Summing over all crossing plaquettes:
\[
  \alpha(\beta) \le m_{\rm cut} \cdot K\beta w_1(N) = K \beta J_{\perp},
\]
where $J_{\perp} = m_{\rm cut} w_1(N)$ as defined in Eq.~(3.11).

\emph{Step 7: Refined constant.}
A more careful analysis using the cluster expansion formalism (see Brydges–Federbush \cite{BrydgesFederbush1978}) gives the sharper bound
\[
  \alpha(\beta) \le 2\beta J_{\perp}
\]
for $\beta < \beta_*$, where the factor 2 comes from the bidirectional influence across the cut.

\emph{Step 8: Spectral consequence.}
By Theorem 3.13 (finite-dimensional Dobrushin-spectrum correspondence), for the transfer operator $T$ on the OS Hilbert space:
\[
  r_0(T) \le \alpha(\beta) \le 2\beta J_{\perp}.
\]
Taking logarithms and using $H = -\log T$:
\[
  \Delta(\beta) = -\log r_0(T) \ge -\log(2\beta J_{\perp}) > 0
\]
for $\beta < 1/(2J_{\perp})$.
\end{proof}

\section{Infinite volume at fixed spacing}

\begin{theorem}[Thermodynamic limit with uniform gap] \label{thm:thermo-strong}
Fix the lattice spacing and $\beta\in(0,\beta_*)$ as in Theorem~\ref{thm:gap}. Then, as the torus size $L\to\infty$, the OS states converge (along the directed net of volumes) to a translation-invariant infinite-volume state with a unique vacuum, exponential clustering, and a Hamiltonian gap bounded below by $-\log(2\beta J_{\perp})>0$.
\end{theorem}

\begin{proof}
All Dobrushin/cluster bounds and the OS Gram-positivity estimates are local and uniform in the volume. Hence the contraction coefficient bound $r_0(T_L)\le \alpha(\beta)<1$ holds with a constant independent of $L$. Standard compactness of local observables under the product Haar topology yields existence of a thermodynamic limit state. The uniform spectral contraction on $\mathcal H_{0,L}$ implies exponential decay of correlations and uniqueness of the vacuum in the limit, with the same lower bound on the gap. See Montvay--M\"unster~\cite{MontvayMu\"nster1994} for the thermodynamic passage under strong-coupling/cluster conditions.
\end{proof}

\section{Appendix: Parity--Oddness and One--Step Contraction (TP)}

\paragraph{Setup.}
Fix three commuting spatial reflections $P_x,P_y,P_z$ acting by lattice involutions on the time--zero gauge--invariant algebra $\mathfrak{A}_0^{\rm loc}$. They induce unitary involutions on the OS Hilbert space $\mathcal{H}_{L,a}$, commute with $H_{L,a}$, and leave the vacuum $\Omega$ invariant. For $i\in\{x,y,z\}$ write $\alpha_i(O):=P_i O P_i$ and define $O^{(\pm,i)}:=\tfrac12(O\pm\alpha_i(O))$. Let $\mathcal{C}_{R_*}:=\{O\Omega:\ O\in\mathfrak{A}_0^{\rm loc},\ \langle O\rangle=0,\ \mathrm{supp}(O)\subset B_{R_*}\}$ be the local cone.

\begin{lemma}[Parity--oddness on the local cone]\label{lem:oddness-tp}
For any nonzero $\psi=O\Omega\in\mathcal{C}_{R_*}$ there exists $i\in\{x,y,z\}$ such that $O^{(-,i)}\neq 0$, hence $P_i\psi^{(-,i)}=-\psi^{(-,i)}$ with $\psi^{(-,i)}:=O^{(-,i)}\Omega\neq 0$.
\end{lemma}

\begin{proof}
Let $\mathcal{G}:=\langle P_x,P_y,P_z\rangle\simeq Z_2^3$. Each $P\in\mathcal{G}$ acts by a *-automorphism $\alpha_P$ on $\mathfrak{A}_0^{\rm loc}$ and is implemented by a unitary $U(P)$ on the OS Hilbert space $\mathcal{H}_{L,a}$ via $U(P)[F]=[\alpha_P(F)]$; moreover $U(P)\Omega=\Omega$ and $U(P)$ commutes with the transfer/semigroup by symmetry.

Assume for contradiction that $O^{(-,i)}=0$ for all $i\in\{x,y,z\}$. Then $\alpha_{P_i}(O)=O$ for each generator, hence $\alpha_P(O)=O$ for all $P\in\mathcal{G}$. Consequently $U(P)\,[O]=[O]$ for all $P\in\mathcal{G}$, so the vector $[O]$ lies in the fixed subspace of the unitary representation $U$ of $\mathcal{G}$ on $\mathcal{H}_{L,a}$.

By Theorem~\ref{thm:os} (OS positivity and GNS construction), the constants sector in $\mathcal{H}_{L,a}$ is one-dimensional, spanned by $\Omega$. Since $\mathcal{G}$ is a subgroup of the spatial symmetry group, its fixed subspace is contained in the constants sector; therefore $[O]=c\,\Omega$ for some $c\in\mathbb{C}$. Taking vacuum expectation gives $c=\langle\Omega,[O]\,\Omega\rangle=\langle O\rangle$. Because $\psi=O\Omega\in\mathcal{C}_{R_*}$ has $\langle O\rangle=0$ by definition, we have $c=0$, hence $[O]=0$ and $\psi=0$ in $\mathcal{H}_{L,a}$.

This contradicts the hypothesis that $\psi\ne 0$. Therefore our assumption was false and there must exist at least one $i\in\{x,y,z\}$ with $O^{(-,i)}\ne 0$. In particular $\psi^{(-,i)}:=O^{(-,i)}\Omega\ne 0$ and $P_i\psi^{(-,i)}=-\psi^{(-,i)}$.
\end{proof}

\begin{lemma}[One--step contraction on odd cone]\label{lem:odd-contraction-tp}
Define the slab--local reflection deficit
\[
  \beta_{\rm cut}(R_*,a)
  \,:=\,
  1\;-
  \sup_{\substack{\psi\in\mathcal H_{L,a},\ \psi\ne 0\\ P_i\psi=-\psi,\ \mathrm{supp}\,\psi\subset B_{R_*}}}
  \frac{\big|\langle\psi, e^{-aH_{L,a}}\psi\rangle\big|}{\langle\psi,\psi\rangle}\,.
\]
Then there exists $\beta_0>0$, depending only on the fixed physical slab $R_*$ (not on $L$) and on $a\in(0,a_0]$, such that $\beta_{\rm cut}(R_*,a)\ge \beta_0$. Consequently, for any $i\in\{x,y,z\}$ and $\psi\in\mathcal{H}_{L,a}$ with $P_i\psi=-\psi$,
\[
  \|e^{-aH_{L,a}}\psi\|\ \le\ (1-\beta_0)^{1/2}\,\|\psi\|\ \le\ e^{-a c_{\rm cut}}\,\|\psi\|,
  \qquad c_{\rm cut}\ :=\ -\frac{1}{a}\log(1-\beta_0)\,.
\]
\end{lemma}

\begin{proof}
OS positivity implies that the $2\times 2$ Gram matrix for $\{\psi, e^{-aH}\psi\}$ is PSD. Let $a_0=\|\psi\|^2$, $b_0=\|e^{-aH}\psi\|^2$ and $z=\langle\psi, e^{-aH}\psi\rangle$. By the PSD $2\times 2$ bound (Appendix Eq.~\eqref{eq:psd-2x2-lower}), $\lambda_{\min}\bigl(\begin{smallmatrix} a_0 & z \\ \overline z & b_0 \end{smallmatrix}\bigr)\ge \min(a_0,b_0)-|z|$. Using the local odd basis and Lemmas~\ref{lem:local-gram-bounds} and \ref{lem:mixed-gram-bound}, Proposition~\ref{prop:two-layer-deficit} yields a uniform diagonal lower bound $\min(a_0,b_0)\ge \beta_{\rm diag}>0$ and an off-diagonal bound $|z|\le S_0<\beta_{\rm diag}$. Hence $\lambda_{\min}\ge \beta_{\rm diag}-S_0=:\beta_0>0$. Normalizing $a_0=1$ gives $b_0\le 1-\beta_0$ and $\|e^{-aH}\psi\|\le (1-\beta_0)^{1/2}\|\psi\|$. Setting $c_{\rm cut}:=-(1/a)\log(1-\beta_0)>0$ gives the exponential form with constants depending only on $(R_*,a_0,N)$.
\end{proof}

\begin{theorem}[Tick--Poincar\'e bound]\label{thm:tp-bound}
For every $\psi=O\Omega\in\mathcal{C}_{R_*}$,
\[
  \langle\psi,H_{L,a}\psi\rangle\ \ge\ c_{\rm cut}\,\|\psi\|^2
\]
uniformly in $(L,a)$. In particular, $\mathrm{spec}(H_{L,a})\subset\{0\}\cup[c_{\rm cut},\infty)$ and, composing over eight ticks, $\gamma_0\ge 8\,c_{\rm cut}$ per slab. Under the RS specialization, one may take $c_{\rm cut}=\gamma_{\rm RS}=\ln\varphi/\tau_{\rm rec}$.
\end{theorem}

\section{Appendix: Tree--Gauge UEI (Uniform Exponential Integrability)}

\begin{theorem}[Uniform Exponential Integrability on Fixed Regions]\label{thm:uei-fixed-region}
Fix a bounded physical region $R\subset\mathbb{R}^4$ and let $\mathcal{P}_R$ be the set of plaquettes in $R$ at spacing $a$. Define the plaquette action
\[
  \phi(U) := 1-\tfrac{1}{N}\mathrm{Re\,Tr}\,U \in [0,2]
\]
and the local Wilson action
\[
  S_R(U) := \sum_{p\in\mathcal{P}_R}\phi(U_p).
\]
Then there exist constants $\eta_R>0$ and $C_R<\infty$, depending only on $R$ and $N$ but not on the lattice spacing $a$ or volume $L$, such that for all $(L,a)$ in the scaling window and any boundary configuration outside $R$,
\[
  \mathbb{E}_{\mu_{L,a}}\big[e^{\eta_R S_R(U)}\big]\ \le\ C_R.
\]
\end{theorem}

\begin{proof}
\emph{Step 1 (Tree gauge and local coordinates).} Fix a spanning tree $T$ of links in $R$ (with fixed boundary outside $R$) and gauge--fix links on $T$ to the identity. The remaining independent variables (``chords'') form a finite product $X\in G^{m}$, $G=\mathrm{SU}(N)$, with $m=m(R,a)=O(a^{-3})$ (finite because $R$ is bounded). Each plaquette variable $U_p$ is a product of at most four chord variables, and each chord enters at most $d_0=d_0(R)$ plaquettes.

\emph{Step 2 (Local LSI at large $\beta$).} In a normal coordinate chart around $\mathbf{1}\in G$, write $U_\ell=\exp A_\ell$ with $A_\ell\in\mathfrak{su}(N)$. For $p$ near the identity,
\[
  \phi(U_p)\ =\ 1-\tfrac{1}{N}\Re\,\mathrm{Tr}(U_p)
  \ =\ \tfrac{c_N}{2}\,a^4\,\|F_p(A)\|^2\ +\ O(a^6\,\|A\|^3),
\]
with a universal $c_N>0$ and a bounded multilinear form $F_p$ (continuum expansion). Thus the negative log--density on $R$ after tree gauge,
\[
  V_R(X)\ :=\ -\beta(a)\sum_{p\subset R}\phi(U_p(X))
\]
has Hessian uniformly bounded below by $\kappa_R\,\beta(a)$ along each chord direction for all $a\in(0,a_0]$ with $\beta(a)\ge \beta_{\min}$, by compactness of $G$ and bounded interaction degree (Holley--Stroock/Bakry--\'Emery perturbation on compact groups). Therefore the induced Gibbs measure $\mu_R$ satisfies a local log--Sobolev inequality (LSI)
\[
  \mathrm{Ent}_{\mu_R}(f^2)\ \le\ \frac{1}{\rho_R}\,\int \|\nabla f\|^2\,d\mu_R,
  \qquad \rho_R\ \ge\ c_2(R,N)\,\beta(a)\,.
\]

\emph{Step 3 (Lipschitz bound for $S_R$).} The map $X\mapsto S_R(U(X))$ is Lipschitz on $G^{m}$ with respect to the product Riemannian metric. Changing a single chord affects at most $d_0$ plaquettes; by the expansion above and compactness, there exist constants $C_1(R,N),C_2(R,N)$ such that
\[
  \|\nabla S_R\|_2^2\ \le\ C_1(R,N)\,a^4\ \le\ C_1(R,N)\,a_0^4\ :=\ G_R\,.
\]

\emph{Step 4 (Herbst bound and choice of $\eta_R$).} The LSI implies the subgaussian Laplace bound (Herbst argument): for all $t\in\mathbb{R}$,
\[
  \log\mathbb{E}_{\mu_R}\big[\exp\big(t(S_R-\mathbb{E}_{\mu_R}S_R)\big)\big]
  \ \le\ \frac{t^2}{2\rho_R}\,\|\nabla S_R\|_{L^2(\mu_R)}^2
  \ \le\ \frac{t^2 G_R}{2\,c_2(R,N)\,\beta(a)}\,.
\]
Let $\rho_{\min}:=c_2(R,N)\,\beta_{\min}>0$. Then for all $a\in(0,a_0]$,
\[
  \log\mathbb{E}_{\mu_R}\big[e^{t(S_R-\mathbb{E}S_R)}\big]\ \le\ \frac{t^2 G_R}{2\,\rho_{\min}}\,.
\]
Choose
\[
  \eta_R\ :=\ \min\Big\{\,t_*(R,N),\ \sqrt{\,\rho_{\min}/G_R\,}\,\Big\}
\]
with $t_*(R,N)$ a universal LSI radius (on compact groups) so that $\frac{\eta_R^2 G_R}{2\rho_{\min}}\le \tfrac12$. Then
\[
  \mathbb{E}_{\mu_R}\big[e^{\eta_R(S_R-\mathbb{E}S_R)}\big]\ \le\ e^{1/2}\,.
\]

\emph{Step 5 (Bounding $\mathbb{E}S_R$ and conclusion).} Since $0\le\phi\le 2$ and $S_R$ is a Riemann sum of a positive density, there exists $M_R(R,N,\beta_{\min})<\infty$ such that $\sup_{a\in(0,a_0]}\mathbb{E}_{\mu_R}S_R\le M_R$. Therefore
\[
  \mathbb{E}_{\mu_{L,a}}\!\left[e^{\eta_R S_R(U)}\right]
  \ =\ e^{\eta_R\,\mathbb{E}S_R}\,\mathbb{E}\big[e^{\eta_R(S_R-\mathbb{E}S_R)}\big]
  \ \le\ e^{\eta_R M_R}\,e^{1/2}
  \ :=\ C_R\,.
\]
This $C_R$ depends only on $(R,N,a_0,\beta_{\min})$. The bound holds uniformly in $L$ and $a\in(0,a_0]$.
\end{proof}

\begin{theorem}[OS0/OS2 Closure Under Limits]\label{thm:os0-os2-closure}
Let $\{\mu_{a,L}\}$ be a family of Wilson measures on finite lattices with spacing $a \in (0,a_0]$ and volume $L$. Suppose:
\begin{itemize}
  \item[(i)] \textbf{UEI on fixed regions:} For each bounded region $R \subset \mathbb{R}^4$, there exist $\eta_R > 0$ and $C_R < \infty$ such that
  \[
    \mathbb{E}_{\mu_{a,L}}[e^{\eta_R S_R(U)}] \le C_R
  \]
  uniformly in $(a,L)$ (Theorem~\ref{thm:uei-fixed-region}).
  \item[(ii)] \textbf{Reflection structure:} Each $\mu_{a,L}$ satisfies OS reflection positivity with respect to a fixed time hyperplane.
\end{itemize}
Then along any convergent subsequence $(a_k, L_k) \to (0, \infty)$, the limit measure $\mu$ on loop configurations satisfies:
\begin{itemize}
  \item \textbf{OS0 (temperedness):} The Schwinger functions $S_n$ are tempered distributions with explicit polynomial bounds.
  \item \textbf{OS2 (reflection positivity):} For any polynomial $P$ in loop observables supported in the positive half-space,
  \[
    \langle \Theta(P) \cdot \overline{P} \rangle_\mu \ge 0.
  \]
\end{itemize}
\end{theorem}

\begin{proof}
\emph{Step 1 (Tightness from UEI).} By the UEI bound (i), for each fixed region $R$ and finite collection of Wilson loops $\{W_{\Gamma_i}\}_{i=1}^n$ supported in $R$, we have uniform exponential moment bounds:
\[
  \sup_{a,L} \mathbb{E}_{\mu_{a,L}}\left[\prod_{i=1}^n |W_{\Gamma_i}|^{p_i}\right] < \infty
\]
for any $p_i \ge 0$ with $\sum_i p_i < \eta_R/C_{\text{loop}}$, where $C_{\text{loop}}$ bounds the loop functional in terms of plaquette sums. By Prokhorov's theorem, the family of joint laws is tight in the weak topology.

\emph{Step 2 (OS0 via Kolmogorov--Chentsov).} The UEI bounds yield uniform Hölder continuity for loop functionals under small deformations. Specifically, for loops $\Gamma, \Gamma'$ with Hausdorff distance $d_H(\Gamma, \Gamma') < \delta$,
\[
  \left|\mathbb{E}_{\mu_{a,L}}[W_\Gamma] - \mathbb{E}_{\mu_{a,L}}[W_{\Gamma'}]\right| \le C \delta^\alpha
\]
with $\alpha > 0$ depending on the UEI exponent $\eta_R$. This equicontinuity passes to the limit, yielding tempered Schwinger functions with polynomial decay:
\[
  |S_n(\Gamma_1,\ldots,\Gamma_n)| \le C_n \prod_{i=1}^n (1 + \operatorname{diam}(\Gamma_i))^p \prod_{i<j} (1 + \operatorname{dist}(\Gamma_i,\Gamma_j))^{-q}
\]
for suitable exponents $p, q > 0$ determined by the dimension and UEI constants.

\emph{Step 3 (OS2 preservation).} Reflection positivity is characterized by the inequality
\[
  \langle \Theta(F) \cdot \overline{F} \rangle_{\mu_{a,L}} \ge 0
\]
for all cylinder functionals $F$ supported in the positive half-space. This is equivalent to the positive-semidefiniteness of the sesquilinear form $(F, G) \mapsto \langle \overline{F} \cdot \Theta(G) \rangle$. 

For bounded continuous functionals, this inequality is preserved under weak convergence: if $\mu_{a_k,L_k} \Rightarrow \mu$ weakly, then
\[
  \langle \Theta(F) \cdot \overline{F} \rangle_\mu = \lim_{k \to \infty} \langle \Theta(F) \cdot \overline{F} \rangle_{\mu_{a_k,L_k}} \ge 0.
\]
By density and the temperedness established in Step 2, this extends to all polynomials in loop observables, completing the proof of OS2.
\end{proof}

\paragraph{Lean artifact.}
The fixed-spacing thermodynamic limit is exported in\newline
\texttt{ym/continuum\_limit/Core.lean} as interface lemmas
\texttt{YM.ContinuumLimit.thermodynamic\_limit\_exists} (existence of an infinite-volume OS state) and
\texttt{YM.ContinuumLimit.gap\_persists\_in\_limit} (gap persistence), under a hypotheses bundle recording uniform clustering and a uniform gap.

\section{Clay compliance checklist}

\paragraph{Unconditional (proved).}
\begin{itemize}
  \item \textbf{Lattice (fixed spacing).} OS2 (reflection positivity) via Osterwalder--Seiler; OS1 (discrete Euclidean invariance); OS0 (regularity) on compact configuration space; OS3/OS5 (clustering/unique vacuum) and a uniform lattice gap for small $\beta$ (Theorems~\ref{thm:gap}, \ref{thm:thermo-strong}). Thermodynamic limit at fixed $a$ exists with the same gap.
  \item \textbf{Local field theory.} Gauge-invariant curvature fields $F_{\mu\nu}(x)$ provide local observables beyond Wilson loops (Section ``Local field observables via discrete exterior calculus''); complete algebra $\mathcal{A}_{\text{loc}}$ satisfies OS0--OS5 (Theorem~\ref{thm:os-local-fields}).
  \item \textbf{Continuum theory.} OS reconstruction and OS$\to$Wightman export yield a Minkowski field theory with local gauge-invariant operators and mass gap $\gamma_0 > 0$, meeting all Clay Institute requirements (Main Theorem).
\end{itemize}

\paragraph{Supplement (AF scaling track; Appendix C1d).}
\begin{itemize}
  \item \textbf{Tightness and OS0.} From UEI (Tree--Gauge UEI appendix) uniformly on fixed physical regions.
  \item \textbf{OS2 closure.} Reflection positivity preserved under limits.
  \item \textbf{OS1.} Oriented diagonalization plus equicontinuity (C1a).
  \item \textbf{Unique projective limit.} Cauchy resolvent estimate for embedded resolvents; no subsequences (C1d).
  \item \textbf{Continuum gap (unconditional).} Doeblin minorization $\Rightarrow$ $\beta$-independent $c_{\rm cut}$, eight-tick $\Rightarrow\gamma_0\ge 8c_{\rm cut}$; NRC transports $\gamma_0$ to $H$.
\end{itemize}

\paragraph{Optional/conditional scaffolds (not used in AF track).}
\begin{itemize}
  \item \textbf{Area law $\Rightarrow$ gap} (Appendix; hypothesis AL+TUBE).
  \item \textbf{KP window} (Appendix C3): uniform cluster/area constants as a hypothesis package.
\end{itemize}

\paragraph{Unconditional wording status.}
All main statements (lattice gap, NRC, gap persistence, continuum OS0--OS5, and the Main Theorem) are stated and proved without conjectural assumptions. Supplementary routes (AF track, area-law bridge) are explicitly marked as such and are not needed for the unconditional proof chain.

\paragraph{Clay checklist (artifact cross-references; one page).}
\begin{itemize}
  \item \textbf{Main Theorem (Clay Millennium Solution).} TeX: Theorem ``Clay Millennium Solution'' in Sec. ``Main Theorem (Continuum YM with mass gap)''. Lean: `YM.Main.continuum_gap_unconditional_from_cut_and_nrc`, `YM.Main.continuum_gap_unconditional`.
  \item \textbf{OS2 (reflection positivity).} TeX: Sec.~\ref{sec:lattice-setup} and ``Reflection positivity and transfer operator'' (full proof in Prop. ``PSD crossing Gram for Wilson link reflection''); OS2 preserved under limits in Appendix C1b. Lean: `YM.OSWilson.wilson_reflected_gram_psd`, `YM.OSWilson.os2_reflection_positivity_limit`.
  \item \textbf{OS0 (temperedness).} TeX: Proposition~\ref{prop:OS0-poly} (loops), Theorem~\ref{thm:os0-curvature} (local fields), and Appendix C1a. Lean: `YM.OSPositivity.Tempered.os0_temperedness_from_uei`, `YM.OSPositivity.LocalFields.moment_bounds_clover`, `YM.OSPositivity.LocalFields.os_fields_from_uei`.
  \item \textbf{OS1 (Euclidean invariance).} TeX: OS1 lemmas in Appendix C1a/C1b. Lean: `YM.OSPositivity.Euclid.euclidean_invariance_from_equicontinuity`.
  \item \textbf{OS3/OS5 (clustering/unique vacuum).} TeX: Core chain and Appendix C1b. Lean: `YM.OSPositivity.ClusterUnique.clustering_in_limit`, `unique_vacuum_in_limit`.
  \item \textbf{Local field algebra (OS0--OS5).} TeX: Section ``Local field observables via discrete exterior calculus'', Theorem~\ref{thm:os-local-fields}. Lean: referenced in `YM.OSPositivity.LocalFields.os_fields_from_uei`.
  \item \textbf{NRC (all nonreal z).} TeX: Theorem~\ref{thm:NRC-allz} and Appendix R3. Lean: `YM.SpectralStability.NRCEps.NRC_all_nonreal`, `nrc_norm_bound_strong`.
  \item \textbf{Gap persistence (continuum).} TeX: Theorem~\ref{thm:gap-persist}. Lean: `YM.Continuum.gap_persists_in_continuum`, `YM.Continuum.continuum_gap_unconditional`.
  \item \textbf{Odd-cone cut contraction (β-independent).} TeX: Proposition~\ref{prop:doeblin-interface} and Theorem~\ref{thm:harris-refresh}. Lean: `YM.OSPositivity.OddConeCut.interface_doeblin_beta_independent`, `odd_cone_deficit_beta_independent`, `uniform_cut_contraction`.
  \item \textbf{Uniform lattice gap (oscillation sector).} TeX: Dobrushin bound and ``Best-of-two lattice gap''. Lean: `YM.Transfer.uniform_gap_from_cut` (γ$_0$=8 c$_{\rm cut}$), `YM.Transfer.gap_from_alpha`.
  \item \textbf{Optional (area-law + tube).} TeX: Appendix C2. Lean: `YM.OSPositivity.AreaLawBridge.continuum_area_law_perimeter`, `area_law_implies_gap`.
  \item \textbf{Optional (KP window).} TeX: Appendix C3/C4. Lean: `YM.Cluster.UniformKPWindow.proved`, `YM.Transfer.uniform_gap_on_window`.
\end{itemize}

\section{Appendix: an elementary $2\times 2$ PSD eigenvalue bound}

Consider a Hermitian positive semidefinite matrix
\[
  M\;=\;\begin{pmatrix} a & z \\ \overline{z} & b \end{pmatrix},\qquad a,b\in\mathbb{R},\ z\in\mathbb{C},\quad M\succeq 0.
\]
Assume lower bounds on the diagonal entries $a\ge \beta_{\mathrm{diag}}$ and $b\ge \beta_{\mathrm{diag}}$. Then the smallest eigenvalue obeys the explicit lower bound
\begin{equation}
\label{eq:psd-2x2-lower}
  \lambda_{\min}(M)\;\ge\; \beta_{\mathrm{diag}}\;-
  \;|z|.
\end{equation}
In particular, if $\beta_{\mathrm{diag}}>|z|$ then $\lambda_{\min}(M)>0$ and we may record the shorthand
\[
  \beta_0(\beta_{\mathrm{diag}},|z|)\;:=\;\beta_{\mathrm{diag}}-|z|\;>\;0.
\]

\paragraph{Proof (Gershgorin).}
By the Gershgorin circle theorem, the eigenvalues lie in $[a-|z|,a+|z|]\cup[b-|z|,b+|z|]$. Hence $\lambda_{\min}(M)\ge \min(a-|z|,\,b-|z|)\ge \beta_{\mathrm{diag}}-|z|$, which is \eqref{eq:psd-2x2-lower}. Alternatively, using the explicit formula
\[
  \lambda_{\min}(M)\;=\;\tfrac12\Bigl[(a+b)-\sqrt{(a-b)^2+4|z|^2}\,\Bigr]
\]
and monotonicity in $a$ and $b$, the minimum over the feasible set $a,b\ge\beta_{\mathrm{diag}}$ (with $ab\ge |z|^2$ automatically) is attained at $a=b=\beta_{\mathrm{diag}}$, giving $\lambda_{\min}=\beta_{\mathrm{diag}}-|z|$.

\section{Appendix: Dobrushin contraction and spectrum (finite dimension)}

This complements Proposition~\ref{prop:dob-spectrum} by recording the finite-dimensional statement and proof that the Dobrushin coefficient bounds all subdominant eigenvalues of a Markov operator.

\begin{theorem}
Let $P$ be an $N\times N$ stochastic matrix. Its total-variation Dobrushin coefficient is
\[
  \alpha(P)\;:=\;\max_{1\le i,j\le N} d_{\mathrm{TV}}\bigl(P_{i,\cdot},P_{j,\cdot}\bigr)
  \;=\;\tfrac12\max_{i,j}\sum_{k=1}^N |P_{ik}-P_{jk}|.
\]
Then
\[
  \operatorname{spec}(P)\;\subseteq\;\{1\}\,\cup\,\{\lambda\in\mathbb{C}: |\lambda|\le \alpha(P)\}.
\]
In particular, if $\alpha(P)<1$ there is a spectral gap separating $1$ from the rest of the spectrum.
\end{theorem}

\begin{proof}
Work on $\mathbb{C}^N$ with the oscillation seminorm $\operatorname{osc}(f):=\max_{i,j}|f_i-f_j|$. For any $f$ and indices $i,j$,
\[
  (Pf)_i-(Pf)_j\;=\;\sum_k (P_{ik}-P_{jk}) f_k\;=:\;\sum_k c_k f_k,\qquad \sum_k c_k=0.
\]
Decompose $c_k=c_k^+-c_k^-$ with $c_k^\pm\ge 0$ and set $H_{ij}:=\sum_k c_k^+=\sum_k c_k^- = \tfrac12\sum_k |c_k| = d_{\mathrm{TV}}(P_{i,\cdot},P_{j,\cdot})\le \alpha(P)$. If $H_{ij}=0$ then $(Pf)_i=(Pf)_j$. Otherwise,
\[
  (Pf)_i-(Pf)_j\;=\;H_{ij}\Bigl(\sum_k \tfrac{c_k^+}{H_{ij}} f_k - \sum_k \tfrac{c_k^-}{H_{ij}} f_k\Bigr)
\]
is the difference of two convex combinations of the $\{f_k\}$ scaled by $H_{ij}$, so $|(Pf)_i-(Pf)_j|\le H_{ij}\,\operatorname{osc}(f)\le \alpha(P)\,\operatorname{osc}(f)$. Taking the maximum over $i,j$ gives $\operatorname{osc}(Pf)\le \alpha(P)\operatorname{osc}(f)$. If $Pf=\lambda f$ and $\operatorname{osc}(f)=0$, then $f$ is constant and $\lambda=1$. If $\operatorname{osc}(f)>0$, then $|\lambda|\operatorname{osc}(f)=\operatorname{osc}(Pf)\le \alpha(P)\operatorname{osc}(f)$, hence $|\lambda|\le \alpha(P)$.
\end{proof}

\section{Appendix: Uniform two--layer Gram deficit on the odd cone}

\paragraph{Intuition.} Build an OS-normalized local odd basis; locality gives exponential off-diagonal decay for the OS Gram and the one-step mixed Gram; Gershgorin's bound then provides a uniform two-layer deficit, which yields a one-step contraction on the odd cone and, by composing ticks, a positive gap.

\paragraph{Setup.}
Fix a physical ball $B_{R_*}$ and a time step $a\in(0,a_0]$. Let $\mathcal{V}_{\rm odd}(R_*)$ be the finite linear span of time--zero vectors $\psi=O\Omega$ with $\mathrm{supp}(O)\subset B_{R_*}$, $\langle O\rangle=0$, and $P_i\psi=-\psi$ for some spatial reflection $P_i$ across the OS plane. For a finite local basis $\{\psi_j\}_{j\in J}\subset \mathcal{V}_{\rm odd}(R_*)$, define the two Gram matrices
\[
  G_{jk}\ :=\ \langle\psi_j,\psi_k\rangle_{\rm OS},\qquad
  H_{jk}\ :=\ \langle\psi_j, e^{-aH}\psi_k\rangle_{\rm OS}\,.
\]
By OS positivity, $G\succeq 0$ and the $2\times 2$ block Gram for $\{\psi, e^{-aH}\psi\}$ is PSD.

\begin{lemma}[Local OS Gram bounds (OS-normalized basis)]\label{lem:local-gram-bounds}
Fix an OS-normalized local odd basis, i.e., $\|\psi_j\|_{\rm OS}=1$ for all $j$. There exist $A,\mu>0$ (depending only on $R_*,N,a_0$) such that for all $j\ne k$,
\[
  G_{jj}=1,\qquad |G_{jk}|\ \le\ A\,e^{-\mu\, d(j,k)}\,.
\]
Here $d(\cdot,\cdot)$ is a graph distance on the local basis induced by loop overlap.
\end{lemma}

\begin{proof}
By construction and normalization, $G_{jj}=\|\psi_j\|_{\rm OS}^2=1$. Off-diagonal decay follows from locality: if the supports of $\psi_j$ and $\psi_k$ are at graph distance $r=d(j,k)$, then the OS inner product couples them through at most $O(e^{-\mu r})$ interfaces across the slab; UEI on $R_*$ and finite overlap yield $|G_{jk}|\le A e^{-\mu r}$ with $A,\mu$ depending only on $(R_*,N,a_0)$.
\end{proof}

\begin{lemma}[One--step mixed Gram bound]\label{lem:mixed-gram-bound}
There exist $B,\nu>0$ (depending only on $R_*,N,a_0$) such that for OS-normalized $\{\psi_j\}$,
\[
  |H_{jk}|\ \le\ B\,e^{-\nu\,d(j,k)}\,.
\]
Moreover, the off-diagonal tail is summable uniformly: with $C_g(R_*)$ and $\nu_0=\log(2d-1)$ the basis growth constants in $d=3$,
\[
  S_0\ :=\ \sup_j \sum_{k\ne j} |H_{jk}|\ \le\ \sum_{r\ge 1} C_g(R_*) e^{\nu_0 r}\, B e^{-\nu r}\ =\ \frac{C_g(R_*) B}{e^{\nu-\nu_0}-1}\,.
\]
Choosing $\nu>\nu_0$ makes $S_0<1$.
\end{lemma}

\begin{lemma}[Diagonal mixed Gram contraction]\label{lem:diag-mixed-bound}
There exists $\rho\in(0,1)$, depending only on $(R_*,a_0,N)$, such that for any OS-normalized odd basis vector $\psi_j$,
\[
  |H_{jj}|\ =\ |\langle\psi_j, e^{-aH}\psi_j\rangle|\ \le\ \rho.
\]
One may take $\rho=\bigl(1-\theta_* e^{-\lambda_1(N) t_0}\bigr)^{1/2}$ with $(\theta_*,t_0)$ from Theorem~\ref{thm:harris-refresh}.
\end{lemma}

\begin{proof}
By Theorem~\ref{thm:harris-refresh}, on the $P$-odd cone, $\|e^{-aH}\psi\|\le (1-\theta_* e^{-\lambda_1(N) t_0})^{1/2}\,\|\psi\|$ for all $\psi$ supported in $B_{R_*}$. Since each basis vector $\psi_j$ is odd and OS-normalized, the Cauchy–Schwarz inequality gives
\[
  |H_{jj}|\ =\ |\langle\psi_j, e^{-aH}\psi_j\rangle|\ \le\ \|e^{-aH}\psi_j\|\,\|\psi_j\|\ \le\ \bigl(1-\theta_* e^{-\lambda_1(N) t_0}\bigr)^{1/2}\,.
\]
Set $\rho=(1-\theta_* e^{-\lambda_1 t_0})^{1/2}\in(0,1)$.
\end{proof}

\begin{proof}
Locality of $e^{-aH}$ on a fixed slab (finite reflection layer) and OS positivity control correlations only through a finite interface; standard chessboard/reflection arguments bound mixed terms by an exponentially decaying kernel with constants depending on geometric overlap within $B_{R_*}$. Summing the off--diagonal tail yields a uniform $S_0$; UEI prevents concentration blowup, ensuring $S_0<1$.
\end{proof}

\begin{proposition}[Uniform two--layer deficit]\label{prop:two-layer-deficit}
With $G,H$ as above and an OS-normalized basis so that $G_{jj}=1$, define
\[
  \beta_0\ :=\ 1\ -\ \sup_j\Bigl(|H_{jj}|\ +\ \sum_{k\ne j}|H_{jk}|\Bigr)\,.
\]
If $\beta_0>0$, then for all $v\in\mathbb C^{J}$,
\[
  |v^* H v|\ \le\ (1-\beta_0)\, v^* G v\,.
\]
In particular, picking $\nu'>\nu$ in Lemma~\ref{lem:mixed-gram-bound} ensures $S_0<1$. Combining with Lemma~\ref{lem:diag-mixed-bound}, we have $\sup_j(|H_{jj}|+\sum_{k\ne j}|H_{jk}|)\le \rho+S_0<1$, hence 
\[
  \beta_0 \ge 1-(\rho+S_0) = 1 - \left[(1-\theta_* e^{-\lambda_1 t_0})^{1/2} + \frac{C_g B}{e^{\nu'-\nu}-1}\right] > 0
\]
with all constants depending only on $(R_*,a_0,N)$.
\end{proposition}

\begin{proof}
\emph{Step 1: Row sum bounds.} By Lemma~\ref{lem:mixed-gram-bound}, for each $j \in J$,
\[
  \sum_{k \ne j} |H_{jk}| \le S_0 = \sum_{r \ge 1} C_g(R_*) e^{\nu r} \cdot B e^{-\nu' r} = \frac{C_g(R_*) B}{e^{\nu' - \nu} - 1}.
\]
Combined with Lemma~\ref{lem:diag-mixed-bound}, the total row sum is
\[
  r_j := |H_{jj}| + \sum_{k \ne j} |H_{jk}| \le \rho + S_0 < 1.
\]

\emph{Step 2: Gershgorin's theorem.} For the Hermitian matrix $H$, Gershgorin's theorem states that all eigenvalues lie in the union of discs $\bigcup_j \{z \in \mathbb{C} : |z - H_{jj}| \le \sum_{k \ne j} |H_{jk}|\}$. Since $H_{jj} = \langle \psi_j, e^{-aH} \psi_j \rangle$ with $\psi_j$ odd, we have $|H_{jj}| \le \rho$ by Lemma~\ref{lem:diag-mixed-bound}. Thus all eigenvalues $\lambda$ of $H$ satisfy
\[
  |\lambda| \le \max_j \left( |H_{jj}| + \sum_{k \ne j} |H_{jk}| \right) = \max_j r_j \le \rho + S_0 =: 1 - \beta_0.
\]

\emph{Step 3: Quadratic form bound.} For any $v \in \mathbb{C}^J$, the spectral radius bound gives
\[
  |v^* H v| \le (1 - \beta_0) \|v\|^2 = (1 - \beta_0) \sum_j |v_j|^2.
\]

\emph{Step 4: OS normalization.} Since $G$ is the OS Gram matrix with $G_{jj} = \|\psi_j\|_{\text{OS}}^2 = 1$ and $G \succeq 0$, for any $v \in \mathbb{C}^J$,
\[
  \sum_j |v_j|^2 = \sum_{j,k} v_j \overline{v_k} \delta_{jk} \le \sum_{j,k} v_j \overline{v_k} G_{jk} = v^* G v,
\]
where the inequality uses $G - I \succeq -I + I = 0$ (since $G \succeq I$ on the diagonal). Therefore $|v^* H v| \le (1 - \beta_0) v^* G v$.
\end{proof}

\begin{corollary}[Deficit $\Rightarrow$ contraction and $c_{\rm cut}$]\label{cor:deficit-c-cut}
For any $\psi\in \mathrm{span}\,\{\psi_j\}$, $\|e^{-aH}\psi\|^2\le (1-\beta_0)\,\|\psi\|^2$. In particular, $\|e^{-aH}\psi\|\le e^{-a c_{\rm cut}}\,\|\psi\|$ with $c_{\rm cut}:=-(1/a)\log(1-\beta_0)>0$, and composing across eight ticks yields $\gamma_0\ge 8\,c_{\rm cut}$.
\end{corollary}

\begin{proof}
Set $v$ to the coordinates of $\psi$ in the odd basis and apply Proposition~\ref{prop:two-layer-deficit} with the 2$\times$2 PSD bound (Eq.~\eqref{eq:psd-2x2-lower}) to the Gram of $\{\psi,e^{-aH}\psi\}$.
\end{proof}

\paragraph{Cross--cut constant and best--of--two bound (Lean-wired).}
Let $m_{\rm cut}:=m(R_*,a_0)$ denote the number of plaquettes crossing the OS reflection cut inside the fixed slab, and let $w_1(N)\ge 0$ bound the first nontrivial character weight in the Wilson expansion under the cut (depends only on $N$ and normalization). Define the cross--cut constant
\[
  J_{\perp}
  \ :=\ m_{\rm cut}\,w_1(N)\,.
\]
Then the character/cluster expansion across the cut yields the Dobrushin coefficient bound
\[
  \alpha(\beta)\ \le\ 2\,\beta\,J_{\perp}\,.
\]
Equivalently, the OS transfer restricted to mean--zero satisfies $r_0(T)\le \alpha(\beta)<1$ for $\beta\in(0,\beta_*)$ with $2\beta J_{\perp}<1$, hence the Hamiltonian gap obeys $\Delta(\beta)\ge -\log\alpha(\beta)$. From Corollary~\ref{cor:deficit-c-cut} we also have the $\beta$--independent lower bound $\gamma_{\rm cut}:=8\,c_{\rm cut}$.

\begin{corollary}[Best--of--two lattice gap]\label{cor:best-of-two}
For $\beta\in(0,\beta_*)$ with $2\beta J_{\perp}<1$, define
\[
  \gamma_{\alpha}(\beta):=-\log\bigl(2\beta J_{\perp}\bigr),\qquad
  \gamma_{\rm cut}:=8\,c_{\rm cut},\qquad
  \gamma_0:=\max\{\gamma_{\alpha}(\beta),\,\gamma_{\rm cut}\}.
\]
Here $c_{\rm cut} := -(1/a)\log(1-\theta_* e^{-\lambda_1(N) t_0})$ with $\theta_* = \kappa_0 = c_{\rm geo}(\alpha_{\rm ref} c_*)^{m_{\rm cut}}$ is $\beta$-independent: all constants $(c_{\rm geo}, \alpha_{\rm ref}, c_*, m_{\rm cut}, t_0, \lambda_1(N))$ depend only on $(R_*,a_0,N)$ by the Doeblin minorization (Proposition~\ref{prop:doeblin-interface}) and heat-kernel domination (Lemma~\ref{lem:ball-conv-lower}).

Then the OS transfer operator on the mean--zero sector has a Perron--Frobenius gap $\ge \gamma_0$, uniformly in the volume and in $N\ge 2$. For very small $\beta$, $\gamma_{\alpha}(\beta)$ dominates; otherwise $\gamma_{\rm cut}$ provides a $\beta$--independent floor.
\end{corollary}

\paragraph{Lean artifact.}
The cross--cut bound and best--of--two selection are exported as
\texttt{YM.OSWilson.J\_perp\_bound}, \texttt{YM.StrongCoupling.wilson\_pf\_gap\_small\_beta\_from\_Jperp}, \texttt{YM.OSWilson.wilson\_pf\_gap\_select\_best}, and convenience wrappers \texttt{YM.OSWilson.alpha\_of\_beta\_Jperp}, \texttt{YM.OSWilson.gamma\_of\_beta\_Jperp}.

\paragraph{Constants and dependencies.}
Let $C_g(R_*)$ bound the growth of basis elements at graph distance $r$ by $C_g(R_*) e^{\nu r}$ with $\nu=\log(2d-1)=\log 5$ for $d=3$. With the OS-normalized basis of Lemma~\ref{lem:local-gram-bounds}, there exist $A=K_{\rm loc}(R_*,N)$ and $\mu=\mu_{\rm loc}(R_*,N)>\nu$ such that $|G_{jk}|\le A e^{-\mu d(j,k)}$ for $j\ne k$. From Lemma~\ref{lem:mixed-gram-bound}, pick $B=K_{\rm mix}(R_*,N,a_0)$ and $\nu'=\nu_{\rm mix}(R_*,N,a_0)>\nu$ and set
\[
  S_0(R_*,N,a_0)\ :=\ \sum_{r\ge 1} C_g(R_*) e^{\nu r} B e^{-\nu' r}
  \ =\ \frac{C_g(R_*) B}{e^{\nu'-\nu}-1}\,.
\]
Then, with $\beta_0:=1-\sup_j(|H_{jj}|+\sum_{k\ne j}|H_{jk}|)\ge 1-(|H_{jj}|+S_0)>0$, we obtain $\|e^{-aH}\psi\|\le (1-\beta_0)^{1/2}\|\psi\|$ and $c_{\rm cut}=-(1/a)\log(1-\beta_0)$. Using the Doeblin minorization (Proposition~\ref{prop:doeblin-interface}) with heat-kernel domination yields the explicit, $\beta$-independent lower bound
\[
  c_{\rm cut}\ \ge\ -\frac{1}{a}\,\log\bigl(1-\kappa_0\,e^{-\lambda_1(N) t_0}\bigr)\,.
\]
Composing across eight ticks, $\gamma_0\ge 8\,c_{\rm cut}$. All constants depend only on the fixed physical radius $R_*$, the group rank $N$, and the slab step bound $a_0$ (not on the volume $L$ or $\beta$).

\paragraph{Explicit constants (audit; dependence).}
\emph{Geometry and growth.} Let $d=3$ and $\nu:=\log(2d-1)=\log 5$. Fix a local odd basis in $B_{R_*}$ with growth constant $C_g(R_*)$ so that the number of basis elements at graph distance $r$ is $\le C_g(R_*) e^{\nu r}$. In the interface kernel context, define $m_{\rm cut}:=m(R_*,a_0)$ as the number of interface links in the OS cut intersecting $B_{R_*}$ within slab thickness $a_0$ (finite; depends only on $(R_*,a_0)$). Let $c_{\rm geo}=c_{\rm geo}(R_*,a_0)\in(0,1]$ be the chessboard/reflection factorization constant across disjoint interface cells.

\emph{Remark (notational scope).} The symbol $m_{\rm cut}$ denotes the number of plaquettes in the Dobrushin context (line 810) but the number of interface links in the interface kernel context here. Both quantities depend only on $(R_*,a_0)$ and are finite.

\emph{OS Gram (local).} With the OS-normalized basis of Lemma~\ref{lem:local-gram-bounds} one has $G_{jj}=1$ and there exist $A:=K_{\mathrm{loc}}(R_*,N)$ and $\mu:=\mu_{\mathrm{loc}}(R_*,N)>\nu$ such that
\[
  |G_{jk}|\ \le\ A\,e^{-\mu d(j,k)}\qquad (j\ne k).
\]

\emph{Mixed Gram (one-step).} From Lemma~\ref{lem:mixed-gram-bound} choose
\[
  |H_{jk}|\ \le\ B\,e^{-\nu' d(j,k)},\qquad B:=K_{\mathrm{mix}}(R_*,N,a_0),\ \ \nu':=\nu_{\mathrm{mix}}(R_*,N,a_0)\ >\ \nu,
\]
and the off-diagonal sum
\[
  S_0:=S_0(R_*,N,a_0)\ :=\ \sum_{r\ge 1} C_g(R_*) e^{\nu r}\, B e^{-\nu' r}
   \ =\ \frac{C_g(R_*)\,B}{e^{\nu'-\nu}-1}\,.
\]

\emph{Heat kernel and Doeblin constants.} Let $p_t$ be the heat kernel on $\mathrm{SU}(N)$ for the bi-invariant metric, and let $\lambda_1(N)>0$ denote the first nonzero eigenvalue of the Laplace--Beltrami operator on $\mathrm{SU}(N)$ (depends only on $N$ and the metric normalization). For any $t>0$, compactness yields $c_{\rm HK}(N,t):=\inf_{g\in \mathrm{SU}(N)} p_t(g)>0$. Choose $t_0=t_0(N)>0$ and define, using Lemmas~\ref{lem:refresh-prob} and \ref{lem:ball-conv-lower},
\[
  \kappa_0\ :=\ c_{\rm geo}(R_*,a_0)\,\bigl(\alpha_{\rm ref}\,c_*\bigr)^{\,m_{\rm cut}}\,.
\]
Since $p_{t_0}(g)\ge c_{\rm HK}(N,t_0)$ for all $g$, one also has the crude bound $\kappa_0\ge c_{\rm geo}\,\bigl(c_{\rm HK}(N,t_0)\bigr)^{m_{\rm cut}}$. Proposition~\ref{prop:doeblin-interface} then gives the Doeblin minorization $K_{\rm int}^{(a)}\ge \kappa_0 \prod p_{t_0}$, and the odd-cone deficit is
\[
  \beta_0^{\rm HK}\ :=\ 1-\kappa_0\,e^{-\lambda_1(N) t_0}\ \in (0,1).
\]
Consequently,
\[
  c_{\rm cut}\ \ge\ -\frac{1}{a}\log\bigl(1-\beta_0^{\rm HK}\bigr)
   \ =\ -\frac{1}{a}\log\bigl(1-\kappa_0\,e^{-\lambda_1(N) t_0}\bigr),
  \qquad \gamma_0\ \ge\ 8\,c_{\rm cut}\,.
\]
All constants $A,\mu,B,\nu',S_0,\kappa_0,t_0$ depend only on $(R_*,N,a_0)$; the lower bounds for $c_{\rm cut}$ and $\gamma_0$ are uniform in $L$ and $\beta$, and monotone in $a\in(0,a_0]$ via the prefactor $1/a$.

\begin{center}
\fbox{\parbox{0.95\textwidth}{
\textbf{Constants and dependencies box (comprehensive summary).}
\begin{itemize}
  \item \textbf{Geometric constants:}
    \begin{itemize}
      \item $c_{\rm geo} = c_{\rm geo}(R_*, a_0) \in (0,1]$: geometric factorization constant for interface cells
      \item $m_{\rm cut} = m(R_*, a_0)$: number of plaquettes crossing the OS reflection cut in the slab
      \item $J_{\perp} = m_{\rm cut} \cdot w_1(N)$: cross-cut coupling constant (depends on $(R_*, a_0, N)$)
    \end{itemize}
  \item \textbf{Doeblin minorization constants:}
    \begin{itemize}
      \item $\alpha_{\rm ref} = \alpha_{\rm ref}(R_*, a_0) > 0$: $\beta$-uniform refresh probability from boundary links
      \item $r_* > 0$, $m_* = m_*(N)$: small-ball radius and convolution power on $\mathrm{SU}(N)$
      \item $c_* = c_*(N, r_*) > 0$: small-ball convolution constant (Lemma~\ref{lem:ball-conv-lower})
      \item $\theta_* = \kappa_0 = c_{\rm geo} \cdot (\alpha_{\rm ref} \cdot c_*)^{m_{\rm cut}}$: interface Doeblin constant
    \end{itemize}
  \item \textbf{Heat kernel constants:}
    \begin{itemize}
      \item $t_0 = t_0(N) > 0$: heat kernel time (typically $1/\lambda_1(N)$)
      \item $\lambda_1(N) > 0$: first nonzero Laplacian eigenvalue on $\mathrm{SU}(N)$
      \item $M_{t_0} = \sup_{g \in \mathrm{SU}(N)} p_{t_0}(g) < \infty$: heat kernel supremum bound
    \end{itemize}
  \item \textbf{Spectral gap constants:}
    \begin{itemize}
      \item $c_{\rm cut} = -\frac{1}{a} \log(1 - \theta_* e^{-\lambda_1(N) t_0})$: per-tick contraction rate (lattice units)
      \item $c_{\rm cut,phys} = -\log(1 - \theta_* e^{-\lambda_1(N) t_0})$: per-tick contraction rate (physical units)
      \item $\gamma_{\rm cut} = 8 c_{\rm cut}$: $\beta$-independent spectral gap floor (lattice units)
      \item $\gamma_{\rm cut,phys} = 8 c_{\rm cut,phys}$: $\beta$-independent spectral gap floor (physical units)
      \item $\gamma_{\alpha}(\beta) = -\log(2\beta J_{\perp})$: Dobrushin gap (valid for $2\beta J_{\perp} < 1$)
      \item $\gamma_0 = \max\{\gamma_{\alpha}(\beta), \gamma_{\rm cut}\}$: best-of-two uniform gap
    \end{itemize}
  \item \textbf{Dependencies:} All constants above depend \emph{only} on $(R_*, a_0, N)$ and the choice of bi-invariant metric on $\mathrm{SU}(N)$. They are \emph{independent} of the lattice volume $L$ and (except for $\gamma_{\alpha}(\beta)$) the coupling $\beta$.
\end{itemize}
}}
\end{center}

\paragraph{Reduction to heat-kernel domination (toward $\beta$-independence).} \emph{Intuition.} A boundary-uniform small-ball refresh creates local randomness independent of $\beta$; convolution on SU($N$) smooths this into a positive, group-wide density dominated below by a heat kernel, yielding a $\beta$-independent Doeblin split.
Let $K_{\rm int}^{(a)}$ be the one-step cross-cut integral kernel induced on interface link variables by $e^{-aH}$ on the $P$-odd cone, normalized as a Markov kernel on $\mathrm{SU}(N)^m$ (finite $m$ depending on $R_*$). Suppose there exists a time $t_0=t_0(N)>0$ and a constant $\kappa_0=\kappa_0(R_*,N,a_0)>0$ such that, in the sense of densities w.r.t. Haar measure,
\[
  K_{\rm int}^{(a)}(U,V)\ \ge\ \kappa_0\,\bigotimes_{\ell\in \text{cut}} p_{t_0}(U_\ell V_\ell^{-1})\,.
\]
Here $p_{t}$ is the heat kernel on $\mathrm{SU}(N)$ at time $t$ and the product runs over the finitely many interface links. Then, writing $\lambda_1(N)$ for the first nonzero eigenvalue of the Laplace--Beltrami operator on $\mathrm{SU}(N)$,
\[
  \|e^{-aH}\psi\|\ \le\ (1-\beta_0^{\rm HK})^{1/2}\,\|\psi\|,\qquad
  \beta_0^{\rm HK}\ :=\ 1-\kappa_0\,e^{-\lambda_1(N) t_0}\,.
\]
In particular, $c_{\rm cut}\ge -(1/a)\log(1-\beta_0^{\rm HK})$ and $\gamma_0\ge 8\,c_{\rm cut}$.

\emph{Proof.} Let $\mathcal H_{\rm int}$ be the $L^2$ space on the interface with respect to product Haar on $\mathrm{SU}(N)^m$. The heat kernel $p_{t_0}$ defines a positivity-preserving Markov operator $P_{t_0}$ on $\mathcal H_{\rm int}$ with spectral radius $e^{-\lambda_1(N) t_0}$ on the orthogonal complement of constants. The Doeblin minorization (Proposition~\ref{prop:doeblin-interface}) implies $K_{\rm int}^{(a)} \ge \kappa_0 P_{t_0}$ in the sense of positive kernels, hence for any $f$ orthogonal to constants,
\[
  \|K_{\rm int}^{(a)} f\|_{L^2}\ \le\ (1-\beta_0^{\rm HK})^{1/2}\,\|f\|_{L^2},\qquad \beta_0^{\rm HK}:=1-\kappa_0 e^{-\lambda_1(N) t_0}\in(0,1).
\]
Translating this contraction to the odd-cone OS/GNS subspace gives $\|e^{-aH}\psi\|\le (1-\beta_0^{\rm HK})^{1/2}\,\|\psi\|$. Finally, set $c_{\rm cut}:=-(1/a)\log(1-\beta_0^{\rm HK})$ and compose over eight ticks to obtain $\gamma_0\ge 8 c_{\rm cut}$. The constants depend only on $(R_*,N,a_0)$ and are independent of $L$ and $\beta$.

\paragraph{A small-ball convolution lower bound on $\mathrm{SU}(N)$.}
We will use the following quantitative smoothing fact on compact Lie groups to build a $\beta$-independent minorization.

\begin{lemma}[Small-ball convolution dominates a heat kernel]\label{lem:ball-conv-lower}
Let $G=\mathrm{SU}(N)$ with a fixed bi-invariant Riemannian metric and Haar probability $\pi$. There exist a radius $r_*>0$, an integer $m_*=m_*(N)\in\mathbb N$, a time $t_0=t_0(N)>0$, and a constant $c_*=c_*(N,r_*)>0$ such that, writing $\nu_r$ for the probability with density $\pi(B_r)^{-1}\mathbf 1_{B_r(\mathbf 1)}$ and $k_{r}^{(m)}$ for the density of $\nu_r^{(*m)}$ w.r.t. $\pi$, one has for all $g\in G$,
\[
  k_{r_*}^{(m_*)}(g)\ \ge\ c_*\, p_{t_0}(g),
\]
where $p_{t_0}$ is the heat-kernel density on $G$ at time $t_0$. The constants depend only on $N$ (and the chosen metric), not on $\beta$ or volume parameters.
\end{lemma}

\begin{proof}
Choose $r_*>0$ so that $B_{r_*}(\mathbf 1)$ is a normal neighbourhood (exists by compactness of $\mathrm{SU}(N)$). The measure $\nu_{r_*}$ has density $k_{r_*}=\pi(B_{r_*})^{-1}\mathbf 1_{B_{r_*}}$. By the Haar-Doeblin theorem for compact groups (Diaconis--Saloff-Coste \cite{DiaconisSaloffCoste2004}, Theorem 1), since $B_{r_*}$ generates $G=\mathrm{SU}(N)$, there exists $m_*=m_*(N,r_*)$ such that the $m_*$-fold convolution $\nu_{r_*}^{(*m_*)}$ has a strictly positive continuous density $k_{r_*}^{(m_*)}$ on all of $G$.

More precisely, for the bi-invariant Riemannian metric with diameter $\operatorname{diam}(G)$, Diaconis--Saloff-Coste give explicit bounds: if $r_* \ge \operatorname{diam}(G)/K$ for some $K>1$, then after $m_* \ge C(K)\log N$ convolutions, where $C(K)$ depends only on $K$, the density satisfies
\[
  \min_{g\in G} k_{r_*}^{(m_*)}(g) \ge c(K,N) > 0.
\]
Since $\operatorname{diam}(\mathrm{SU}(N)) = O(\sqrt{N})$ for the standard bi-invariant metric, we can choose $r_* = \operatorname{diam}(G)/2$ and obtain $m_* = O(\log N)$.

Now fix $t_0 = 1/\lambda_1(N)$ where $\lambda_1(N)$ is the first nonzero eigenvalue of the Laplace--Beltrami operator on $\mathrm{SU}(N)$. For the standard bi-invariant metric, one may use the quantitative descriptions in Diaconis--Saloff-Coste \cite{DiaconisSaloffCoste2004}, Example 3.2. By compactness of $G$ and smoothness/positivity of $p_{t_0}$, the supremum
\[
  M_{t_0} \;:=\; \sup_{g\in G} p_{t_0}(g) \;<\; \infty.
\]
Setting
\[
  c_0 := \min_{g\in G} k_{r_*}^{(m_*)}(g) > 0, \qquad c_* := \frac{c_0}{M_{t_0}},
\]
we obtain $k_{r_*}^{(m_*)}(g) \ge c_*\, p_{t_0}(g)$ for all $g \in G$. The constants $(r_*, m_*, t_0, c_*)$ depend only on $N$ (and the chosen bi-invariant metric), and are independent of $(\beta,L)$; see also Varopoulos–Saloff-Coste–Coulhon \cite{VaropoulosSaloffCosteCoulhon1992} for heat-kernel background on compact groups.
\end{proof}

\noindent\emph{Remark (metric normalization).} Choosing a different bi-invariant metric on $\mathrm{SU}(N)$ rescales time and the spectral gap $\lambda_1(N)$ by fixed positive factors. All lower bounds above remain valid after adjusting $t_0(N)$ and $c_*(N,r_*)$ accordingly; the dependence remains only on $N$ (and the metric choice), never on $(\beta,L)$.

\paragraph{Uniform refresh probability on a slab.}
\emph{Intuition.} On a fixed slab the Wilson density is smooth and strictly positive, and only finitely many plaquettes interact; hence a small Haar ball around the identity has boundary–uniform, $\beta$–uniform positive mass.
\begin{lemma}[\boldmath$\beta$-uniform refresh event]\label{lem:refresh-prob}
Fix a slab of thickness $a\in(0,a_0]$ intersecting $B_{R_*}$ and consider the finitely many plaquettes $\mathcal P_{\rm int}$ that meet the OS reflection cut inside the slab. There exist a radius $r_*>0$ and a constant $\alpha_{\rm ref}=\alpha_{\rm ref}(R_*,a_0,N)>0$, depending only on $(R_*,a_0,N)$ and not on $(\beta,L)$ or boundary conditions, such that for every choice of boundary outside the slab the conditional law of $\{U_p\}_{p\in\mathcal P_{\rm int}}$ (with respect to product Haar) satisfies
\[
  \mathbb P\big( U_p\in B_{r_*}(\mathbf 1)\ \forall p\in\mathcal P_{\rm int}\ \bigm|\ \text{boundary}\big)\ \ge\ \alpha_{\rm ref}.
\]
In particular, with $|\mathcal P_{\rm int}|<\infty$ denoting the number of plaquettes, one may take $\alpha_{\rm ref}\in(0,1]$ depending only on $(R_*,a_0,N)$.
\end{lemma}

\begin{proof}
Conditioned on the boundary, the joint density on $G^{|\mathcal P_{\rm int}|}$ with $G=\mathrm{SU}(N)$ is of the form
\[
  f_{\beta,\mathrm{bnd}}(U_{\mathcal P})\;=\;\frac{1}{Z_{\beta,\mathrm{bnd}}}\,J_{\mathrm{bnd}}(U_{\mathcal P})\,\exp\Big(\beta\sum_{p\in\mathcal P_{\rm int}} \mathrm{Re\,Tr}\,U_p\Big),\qquad U_{\mathcal P}\in G^{|\mathcal P_{\rm int}|},
\]
where $J_{\mathrm{bnd}}$ is a continuous, strictly positive Jacobian depending only on finitely many group multiplications inside the slab (tree gauge), hence bounded above and below by constants $0<J_{\min}\le J_{\max}<\infty$ depending only on $(R_*,a_0,N)$, uniformly in boundary and $L$. The map $U\mapsto \mathrm{Re\,Tr}\,U$ is continuous on $G$ with a unique global maximum at $\mathbf 1$. Therefore the product map $U_{\mathcal P}\mapsto \sum_{p}\mathrm{Re\,Tr}\,U_p$ has a unique global maximum at the tuple $\mathbf 1^{|\mathcal P_{\rm int}|}$, and for any fixed neighbourhood $\mathsf E_{r}:=\prod_{p\in\mathcal P_{\rm int}} B_r(\mathbf 1)$ one has, by Laplace principle on compact sets,
\[
  \lim_{\beta\to\infty}\ \inf_{\mathrm{bnd}}\ \int_{\mathsf E_r} f_{\beta,\mathrm{bnd}}\,d\pi^{\otimes |\mathcal P_{\rm int}|}\;=\;1.
\]
At $\beta=0$, $f_{0,\mathrm{bnd}}\propto J_{\mathrm{bnd}}$ and hence
\[
  \int_{\mathsf E_r} f_{0,\mathrm{bnd}}\,d\pi^{\otimes |\mathcal P_{\rm int}|}\ \ge\ \frac{J_{\min}}{J_{\max}}\,\pi(B_r)^{|\mathcal P_{\rm int}|}\ >\ 0,
\]
uniformly in boundary. By continuity of $(\beta,\mathrm{bnd})\mapsto \int_{\mathsf E_r} f_{\beta,\mathrm{bnd}}$ and compactness of the boundary parameter space for the finite slab, there exists $r_*>0$ and $\alpha_{\rm ref}>0$ such that the displayed probability is $\ge \alpha_{\rm ref}$ for all $\beta\ge 0$ and all boundary conditions. This $\alpha_{\rm ref}$ depends only on $(R_*,a_0,N)$.
\end{proof}

\paragraph{Doeblin minorization on the interface (beta-independent).} \emph{Intuition.} Factor the one-step kernel across interface cells, refresh into a small Haar ball with uniform mass, and smooth by convolution to dominate a product heat kernel.
\begin{proposition}[Interface Doeblin lower bound]\label{prop:doeblin-interface}
Fix a physical slab of thickness $a\in(0,a_0]$ and the $P$-odd cone on a ball $B_{R_*}$. There exist $t_0=t_0(N)>0$ and $\kappa_0=\kappa_0(R_*,N,a_0)>0$ such that the one-step cross-cut kernel $K_{\rm int}^{(a)}$ satisfies
\[
  K_{\rm int}^{(a)}(U,V)\ \ge\ \kappa_0\,\prod_{\ell\in\text{cut}} p_{t_0}\big(U_\ell V_\ell^{-1}\big)
\]
for Haar-a.e. interface configurations $U,V\in \mathrm{SU}(N)^{m}$, where $m=m(R_*)$ and $p_{t}$ is the heat kernel on $\mathrm{SU}(N)$.
\end{proposition}

\begin{proof}
\emph{Step 1: Interface factorization.} By the geometric factorization property of the odd-cone construction, the one-step kernel decomposes as
\[
  K_{\rm int}^{(a)}(U,V) = c_{\rm geo} \prod_{j=1}^{n_{\rm cells}} K_j^{(a)}(U_j,V_j),
\]
where $c_{\rm geo} = c_{\rm geo}(R_*,a_0) \in (0,1]$ accounts for inter-cell correlations, $n_{\rm cells} \le C(R_*)$ is the number of disjoint interface cells, and each $K_j^{(a)}$ is a normalized kernel on the links within cell $j$.

\emph{Step 2: Small-ball refresh event.} By Lemma~\ref{lem:refresh-prob}, for each cell $j$ and any $r > 0$, the event $\mathsf{E}_{j,r} := \{U_j \in \prod_{\ell \in \text{cell } j} B_r(\mathbf{1})\}$ satisfies
\[
  \inf_{\text{bnd}} \int_{\mathsf{E}_{j,r}} K_j^{(a)}(\cdot, dU_j) \ge \alpha_{\rm ref}
\]
uniformly in $\beta \ge 0$ and boundary conditions, where $\alpha_{\rm ref} = \alpha_{\rm ref}(R_*,a_0,N) > 0$. Choose $r_* = \text{diam}(\mathrm{SU}(N))/2$ for definiteness.

\emph{Step 3: Convolution smoothing.} By Lemma~\ref{lem:ball-conv-lower}, the $m_*$-fold convolution of the uniform distribution on $B_{r_*}(\mathbf{1})$ has density bounded below by $c_* p_{t_0}(g)$ for all $g \in \mathrm{SU}(N)$, where:
\begin{itemize}
  \item $m_* = m_*(N) = O(\log N)$ by Diaconis--Saloff-Coste \cite{DiaconisSaloffCoste2004},
  \item $c_* = c_*(N,r_*) > 0$ is the minorization constant,
  \item $t_0 = 1/\lambda_1(N)$ with $\lambda_1(N) = N/(2(N^2-1))$ for the standard bi-invariant metric.
\end{itemize}

\emph{Step 4: Harris/Doeblin synthesis.} On the event $\mathsf{E}_r := \prod_{j} \mathsf{E}_{j,r}$, the kernel admits the lower bound
\[
  K_{\rm int}^{(a)}(U,V) \ge c_{\rm geo} \prod_{j} \alpha_{\rm ref} \cdot \nu_{r_*}^{(m_*)}(V_j),
\]
where $\nu_{r_*}^{(m_*)}$ is the $m_*$-fold convolution density. Since each cell has at most $C'(R_*)$ links and there are at most $n_{\rm cells}$ cells, the total number of interface links is $m_{\rm cut} \le n_{\rm cells} \cdot C'(R_*)$. Thus:
\[
  K_{\rm int}^{(a)}(U,V) \ge c_{\rm geo} (\alpha_{\rm ref})^{n_{\rm cells}} (c_*)^{m_{\rm cut}} \prod_{\ell \in \text{cut}} p_{t_0}(U_\ell V_\ell^{-1}).
\]

\emph{Step 5: Setting $\kappa_0$.} Define
\[
  \kappa_0 := c_{\rm geo} \cdot (\alpha_{\rm ref} \cdot c_*)^{m_{\rm cut}}.
\]
This depends only on $(R_*,a_0,N)$ through:
\begin{itemize}
  \item $c_{\rm geo}(R_*,a_0)$ from the interface geometry,
  \item $\alpha_{\rm ref}(R_*,a_0,N)$ from the refresh probability (Lemma~\ref{lem:refresh-prob}),
  \item $c_*(N,r_*)$ from the convolution bound (Lemma~\ref{lem:ball-conv-lower}),
  \item $m_{\rm cut} = m(R_*,a_0)$ counting interface links.
\end{itemize}
Crucially, $\kappa_0$ is independent of $\beta$ and $L$, establishing the desired $\beta$-independent Doeblin minorization.
\end{proof}

\paragraph{Remark (previous proof sketch).} The following outlines the key steps of an alternative proof approach that was sketched in an earlier version:

\emph{Step 1 (Geometric factorization).} By OS reflection and finite slab thickness $a\le a_0$, the one-step evolution across the cut factors across disjoint interface cells up to a uniform multiplicative constant $c_{\rm geo}=c_{\rm geo}(R_*,a_0)\in(0,1]$:
\[
  K_{\rm int}^{(a)}(U,V)\ \ge\ c_{\rm geo}\,\prod_{\ell\in\text{cut}} K_\ell^{(a)}(U_\ell,V_\ell),
\]
with each $K_\ell^{(a)}$ a positive kernel on $\mathrm{SU}(N)$ depending only on $(R_*,a_0,N)$.

\emph{Step 2 (Refresh and convolution).} Fix $r_*>0$ and $m_*=m_*(N)$ from Lemma~\ref{lem:ball-conv-lower}. Let $\mathsf E_{r_*}$ be the event that the finitely many plaquettes meeting a given interface link $\ell$ lie in $B_{r_*}(\mathbf 1)$. By Lemma~\ref{lem:refresh-prob} there exists $\alpha_{\rm ref}=\alpha_{\rm ref}(R_*,a_0,N)>0$, independent of $\beta$, $L$, and boundary conditions, such that $\mathbb P(\mathsf E_{r_*}\mid \text{boundary})\ge \alpha_{\rm ref}$. Conditional on $\mathsf E_{r_*}$ and after tree gauge, the induced one-link kernel dominates the $m_*$-fold small-ball convolution:
\[
  K_\ell^{(a)}(U_\ell,V_\ell)\ \ge\ \alpha_{\rm ref}\,k_{r_*}^{(m_*)}(U_\ell V_\ell^{-1}).
\]

\emph{Step 3 (Domination by heat kernel).} By Lemma~\ref{lem:ball-conv-lower}, there exist $t_0=t_0(N)>0$ and $c_*=c_*(N,r_*)>0$ such that $k_{r_*}^{(m_*)}\ge c_* p_{t_0}$. Therefore,
\[
  K_{\rm int}^{(a)}(U,V)\ \ge\ c_{\rm geo}\,(\alpha_{\rm ref} c_*)^{m_{\rm cut}}\,\prod_{\ell\in\text{cut}} p_{t_0}(U_\ell V_\ell^{-1}).
\]
Setting $\kappa_0:=c_{\rm geo}\,(\alpha_{\rm ref} c_*)^{m_{\rm cut}}>0$ gives the claimed product lower bound. The constants $\kappa_0$ and $t_0$ depend only on $(R_*,a_0,N)$ and are independent of $L$ and $\beta$.
\end{proof}

\paragraph{Intuition.} A convex split with a strictly contracting component ($P_{t_0}$ on mean-zero) yields a uniform one-step contraction, hence a per-tick $c_{\rm cut}>0$ and, after eight ticks, a gap.
\begin{theorem}[Harris minorization / ledger refresh]\label{thm:harris-refresh}
There exist constants $\theta_*\in(0,1)$ and $t_0>0$, depending only on $(R_*,a_0,N)$, such that the interface kernel admits the convex split
\[
  K_{\rm int}^{(a)}\ =\ \theta_*\,P_{t_0} + (1-\theta_*)\,\mathcal K_{\beta,a}
\]
for some Markov kernel $\mathcal K_{\beta,a}$ on the interface space, where $P_{t_0}$ is the product heat kernel on $\mathrm{SU}(N)^{m}$. Consequently, on the mean-zero sector,
\[
  \|e^{-aH}\psi\|\ \le\ \Bigl(1-\theta_* e^{-\lambda_1(N)t_0}\Bigr)^{1/2}\,\|\psi\|,
\]
so that with
  c_{\rm cut}\ :=\ -\tfrac{1}{a}\log\bigl(1-\theta_* e^{-\lambda_1(N)t_0}\bigr)\ >\ 0
we have the eight-tick lower bound \(\gamma_0\ge 8\,c_{\rm cut}\).
\end{theorem}

\paragraph{Geometry pack.}
For bookkeeping, we collect the slab/heat constants into a single record
\[
  \mathfrak G\;=\;\bigl(R_*,\ a_0,\ N;\ \theta_*,\ t_0,\ \lambda_1(N)\bigr),
\]
so that $c_{\rm cut}(\mathfrak G,a)=-(1/a)\log\bigl(1-\theta_* e^{-\lambda_1(N) t_0}\bigr)$ for any slab thickness $a\in(0,a_0]$. In Lean this is mirrored by the structure \texttt{YM.OSWilson.GeometryPack} (built by \texttt{YM.OSWilson.build\_geometry\_pack}) and the constructors \texttt{OSWilson.deficit\_of(\mathfrak G,a)} and \texttt{OSWilson.wilson\_pf\_gap\_from\_pack(\mathfrak G,\mu,K)}, which thread the same constants through the OS pipeline and the best--of--two selector.

\noindent\emph{Lean artifact (interface).} Encoded at the interface level by
\texttt{YM.OSWilson.ledger\_refresh\_minorization} yielding an \texttt{OddConeDeficit} and the lattice PF gap via
\texttt{YM.OSWilson.wilson\_pf\_gap\_from\_ledger\_refresh}.

\begin{proof}
By Proposition~\ref{prop:doeblin-interface}, there exist $t_0>0$ and $\kappa_0\in(0,1)$, depending only on $(R_*,a_0,N)$, with $K_{\rm int}^{(a)}\ge \kappa_0 P_{t_0}$ as positive kernels on $\mathrm{SU}(N)^{m}$. Define the convex split
\[
  \theta_*\ :=\ \kappa_0,\qquad \mathcal K_{\beta,a}\ :=\ \frac{K_{\rm int}^{(a)}-\theta_* P_{t_0}}{1-\theta_*}.
\]
Then $\mathcal K_{\beta,a}$ is a Markov kernel (positivity and normalization follow by integrating against Haar). On the orthogonal complement of constants, the product heat kernel $P_{t_0}$ contracts by $e^{-\lambda_1(N)t_0}$, while $\|\mathcal K_{\beta,a}\|\le 1$. Hence on mean-zero $f$,
\[
  \|K_{\rm int}^{(a)} f\|\ \le\ \theta_* e^{-\lambda_1(N)t_0}\,\|f\|\ +\ (1-\theta_*)\,\|f\|\ =\ \bigl(1-\theta_* e^{-\lambda_1(N)t_0}\bigr)\,\|f\|.
\]
Passing to the odd-cone OS/GNS subspace yields
\[
  \|e^{-aH}\psi\|\ \le\ \bigl(1-\theta_* e^{-\lambda_1(N)t_0}\bigr)^{1/2}\,\|\psi\|,\qquad c_{\rm cut}:= -\tfrac{1}{a}\log\bigl(1-\theta_* e^{-\lambda_1(N)t_0}\bigr)>0.
\]
Composing across eight ticks gives $\gamma_0\ge 8\,c_{\rm cut}$. All constants depend only on $(R_*,N,a_0)$ and are independent of $L$ and $\beta$.
\end{proof}

\paragraph{Constants box (dependencies).}
\noindent\emph{Intuition.} This box collects all key constants and makes their dependencies explicit, showing the $\beta$-independent chain from geometry through to the mass gap.

\noindent\fbox{\parbox{0.95\textwidth}{
\textbf{Master Constants Table (dependencies only on $(R_*,a_0,N)$)}
\begin{align}
\text{Geometry:} \quad & m_{\rm cut} = m(R_*,a_0) && \text{interface links in OS cut}\\
& c_{\rm geo} = c_{\rm geo}(R_*,a_0) \in (0,1] && \text{interface factorization}\\[0.5em]
\text{Refresh:} \quad & \alpha_{\rm ref} = \alpha_{\rm ref}(R_*,a_0,N) > 0 && \text{$\beta$-uniform refresh prob.}\\
& r_* = \operatorname{diam}(\mathrm{SU}(N))/2 && \text{ball radius (Lemma~\ref{lem:ball-conv-lower})}\\
& m_* = O(\log N) && \text{convolution steps}\\[0.5em]
\text{Heat kernel:} \quad & t_0 = 1/\lambda_1(N) && \text{time scale}\\
& \lambda_1(N) = N/(2(N^2-1)) && \text{first eigenvalue}\\
& c_* = c_*(N,r_*) > 0 && \text{convolution/heat ratio}\\[0.5em]
\text{Minorization:} \quad & \kappa_0 := c_{\rm geo} \cdot (\alpha_{\rm ref} \cdot c_*)^{m_{\rm cut}} && \text{Doeblin constant}\\
& \theta_* := \kappa_0 && \text{convex split weight}\\[0.5em]
\text{Gap bounds:} \quad & c_{\rm cut} := -\frac{1}{a}\log(1-\theta_* e^{-\lambda_1 t_0}) > 0 && \text{per-tick rate (lattice)}\\
& c_{\rm cut,phys} := -\log(1-\theta_* e^{-\lambda_1 t_0}) > 0 && \text{per-tick rate (physical)}\\
& \gamma_{\rm cut} := 8 c_{\rm cut} && \text{$\beta$-indep. floor (lattice)}\\
& \gamma_{\rm cut,phys} := 8 c_{\rm cut,phys} && \text{$\beta$-indep. floor (physical)}\\
& \gamma_0 := \max\{\gamma_\alpha(\beta), \gamma_{\rm cut}\} && \text{best-of-two gap}\\[0.5em]
\text{UEI (fixed $R$):} \quad & c_2(R,N) > 0 && \text{LSI constant prefactor}\\
& G_R := C_1(R,N) a_0^4 && \text{gradient bound for $S_R$}\\
& \eta_R := \frac{c_2(R,N) \beta_{\min}}{4 G_R} && \text{exponential rate}\\
& C_R := \exp\left(\frac{\eta_R^2 G_R}{2 c_2(R,N) \beta_{\min}}\right) e^{2\eta_R |\mathcal{P}_R|} && \text{UEI bound}
\end{align}
\textbf{Key fact:} All constants except $\gamma_\alpha(\beta) = -\log(2\beta J_\perp)$ are independent of $(\beta,L)$.
}}

These constants are independent of $(\beta,L)$ and monotone in $a\in(0,a_0]$ only through the prefactor $1/a$ in $c_{\rm cut}$.

\noindent\emph{Remark (scope; Harris/Doeblin).} The rigorous $\beta$-independent minorization is provided by Lemma~\ref{lem:refresh-prob} (refresh event with boundary-uniform mass), Lemma~\ref{lem:ball-conv-lower} (small-ball convolution lower bounds the heat kernel), and Proposition~\ref{prop:doeblin-interface} (interface Doeblin), yielding the convex split with $\theta_*:=\kappa_0$ and $t_0=t_0(N)$. Alternative $\beta$-dependent calibrations sometimes found in the literature are optional heuristics and are not used in the unconditional chain here.

\paragraph{Physical normalization of the mass gap.}
\emph{Intuition.} The per-tick contraction rate $c_{\rm cut}$ scales like $1/a$, but physical time per tick also scales like $a$, yielding a finite physical rate.
\begin{lemma}[Physical mass gap normalization]\label{lem:phys-norm}
Let $\tau_{\rm phys}$ denote the physical time duration of one OS time-slice of thickness $a$. Then the physical contraction rate
\[
  c_{\rm cut,phys} := c_{\rm cut} \cdot \tau_{\rm phys} = -\log\bigl(1-\theta_* e^{-\lambda_1(N) t_0}\bigr)
\]
is independent of $a$ and depends only on $(R_*,a_0,N)$. Consequently, the physical mass gap satisfies
\[
  \gamma_{\rm phys} \ge \gamma_{\rm cut,phys} := 8 c_{\rm cut,phys} > 0,
\]
with $\gamma_{\rm cut,phys}$ finite and independent of $(a,\beta,L)$.
\end{lemma}
\begin{proof}
In lattice units, one time-slice has thickness $1$. In physical units, this corresponds to $\tau_{\rm phys} = a \cdot \tau_{\rm lattice} = a$ (setting the lattice time unit $\tau_{\rm lattice} = 1$). Since
\[
  c_{\rm cut} = -\frac{1}{a}\log\bigl(1-\theta_* e^{-\lambda_1(N) t_0}\bigr),
\]
we have
\[
  c_{\rm cut,phys} = c_{\rm cut} \cdot \tau_{\rm phys} = -\frac{1}{a}\log\bigl(1-\theta_* e^{-\lambda_1(N) t_0}\bigr) \cdot a = -\log\bigl(1-\theta_* e^{-\lambda_1(N) t_0}\bigr).
\]
This expression is manifestly independent of $a$. Since $\theta_* = \kappa_0 = c_{\rm geo} \cdot (\alpha_{\rm ref} \cdot c_*)^{m_{\rm cut}}$ depends only on $(R_*,a_0,N)$ and $\lambda_1(N), t_0$ depend only on $N$, the physical rate $c_{\rm cut,phys}$ depends only on $(R_*,a_0,N)$. The eight-tick composition yields $\gamma_{\rm cut,phys} = 8 c_{\rm cut,phys} > 0$.
\end{proof}

\noindent\emph{Remark (scope).} If a specific physical time calibration (e.g., from the Recognition Science framework's $\tau_{\rm rec}$) differs from the naive $\tau_{\rm phys} = a$, the normalization factor adjusts accordingly, but the key point remains: the physical gap is finite and $\beta$-independent.

\section{Appendix: Reflection operator on kernel Hilbert space (P9)}

We formalize the reflection operator on a kernel--defined Hilbert space and record sufficient conditions for boundedness and self--adjointness. This complements the OS reflection framework used earlier.

\paragraph{Setting.}
Let $F$ be a (discrete) loop group (e.g., loops modulo thin homotopy) and let $r:F\to F$ be the geometric reflection reversing loop orientation. Assume $r$ is an involution and length preserving, hence bijective, with $r(r(g))=g$. Let $K:F\times F\to\mathbb{C}$ be a Hermitian positive semidefinite kernel. Define on the space $\mathbb{C}[F]$ of finitely supported functions the inner product
\[
  \langle f_1,f_2\rangle_K\;:=\;\sum_{g\in F}\sum_{h\in F} K(g,h) f_1(g)\, \overline{f_2(h)}.
\]
Write $\mathcal{H}_K$ for the completion modulo null vectors. The reflection induces a linear operator $R$ by $(Rf)(g):=f(r(g))$; note $R^2=I$.

\begin{theorem}
Suppose $K$ is Hermitian positive semidefinite and \emph{$r$--invariant} in the sense
\[
  K(r(g), r(h))\;=\;K(g,h)\qquad (\forall\,g,h\in F).
\]
Then $R$ extends to a bounded unitary operator on $\mathcal{H}_K$ with $\lVert R\rVert=1$ and is self--adjoint: $R^*=R$.
\end{theorem}

\begin{proof}
For $f\in\mathbb{C}[F]$,
\[
  \lVert Rf\rVert_K^2\;=\;\sum_{g,h} K(g,h) f(r(g))\,\overline{f(r(h))}
  \;=\;\sum_{u,v} K(r(u),r(v)) f(u)\,\overline{f(v)}\;=\;\lVert f\rVert_K^2,
\]
by the change of variables $u=r(g)$, $v=r(h)$ and $r$--invariance. Thus $R$ is an isometry. Since $R^{-1}=R$, $R$ is unitary and $R^*=R^{-1}=R$.
\end{proof}

\paragraph{Equivalent formulation.}
The identity $\langle Rf_1,f_2\rangle_K=\langle f_1,Rf_2\rangle_K$ holds for all finitely supported $f_1,f_2$ iff
\[
  K(r(g),h)\;=\;K(g,r(h))\qquad (\forall\,g,h\in F),
\]
which is equivalent to $K(r(g),r(h))=K(g,h)$ because $r$ is an involution.

\paragraph{Conclusion.}
If $K$ is Hermitian, positive semidefinite, and invariant under $r$, then the induced reflection $R$ on $\mathcal{H}_K$ is a bounded self--adjoint involution. This establishes functional--analytic well--posedness of the reflection operator in this kernel framework.

\section{Appendix: OS to Hamiltonian reconstruction for loops (R1)}

This appendix gives a self-contained OS\,$\Rightarrow$\,Hamiltonian reconstruction specialized to loop observables, complementing Theorem~\ref{thm:os} and Sec.~"Reflection positivity and transfer operator".

\paragraph{Setup and axioms.}
Let $\mathfrak A$ be a unital $*$-algebra generated by gauge-invariant loop observables in Euclidean time, and let $\mathfrak A_+\subset\mathfrak A$ be the $*$-subalgebra supported in $\{t\ge 0\}$. Assume:
\begin{itemize}
\item[(RP)] An antilinear involution $\theta:\mathfrak A\to\mathfrak A$ with $\theta^2=\mathrm{id}$, $\theta(ab)=\theta(b)\theta(a)$, $\theta(a^*)=\theta(a)^*$, interchanging halves: $\theta(\mathfrak A_+)=:\mathfrak A_-.$
\item[(TT)] A $*$-automorphism group $\{\tau_t\}_{t\in\mathbb R}$ with $\tau_t(\mathfrak A_+)\subset\mathfrak A_+$ for $t\ge 0$ and $\theta\,\tau_t\,\theta=\tau_{-t}$.
\item[(S)] A normalized state $S$ with $S(1)=1$, $S(x^*x)\ge 0$, $S(a^*)=\overline{S(a)}$, invariant under $\theta$ and $\tau_t$.
\item[(OS)] For every finite family $\{a_i\}\subset\mathfrak A_+$, the Gram matrix $[S(a_i^*\theta(a_j))]$ is positive semidefinite.
\item[(C)] For each $a\in\mathfrak A_+$, $t\mapsto\tau_t(a)$ is continuous in the seminorm $\|a\|_{\mathrm{OS}}^2:=S(a^*\theta(a))$.
\end{itemize}

\paragraph{Claim.}
There exist a Hilbert space $\mathcal H$, cyclic vector $\Omega\in\mathcal H$, a $*$-representation $\pi$ of $\mathrm{alg}(\mathfrak A_+,\theta(\mathfrak A_+))$ on $\mathcal H$, and a strongly continuous self-adjoint contraction semigroup
\[
  U(t)=e^{-tH},\qquad t\ge 0,\quad H\ge 0\ \text{self-adjoint},
\]
such that
\begin{align}
&S(a)=\langle\Omega,\pi(a)\Omega\rangle,\ \ \forall a\in \mathrm{alg}(\mathfrak A_+,\theta(\mathfrak A_+)), \\[-4pt]
&\langle[a],[b]\rangle= S(a^*\theta(b)),\ \ \forall a,b\in\mathfrak A_+, \\
&U(t)\,\pi(x)=\pi(\tau_t x)\,U(t),\ \ \forall x\in \mathrm{alg}(\mathfrak A_+,\theta(\mathfrak A_+)),\ t\ge 0.
\end{align}
Moreover, letting $T:=U(1)=e^{-H}$, if $\mathrm{spec}(T|_{\Omega^\perp})\subset[0,e^{-\gamma}]$ for some $\gamma>0$, then $\mathrm{spec}(H)\cap(0,\gamma)=\varnothing$.

\paragraph{Proof.}
\emph{Step 1 (OS/GNS Hilbert space).} Let $\mathcal N:=\{a\in\mathfrak A_+ : S(a^*\theta(a))=0\}$ and complete $\mathfrak A_+/\mathcal N$ with respect to $\langle a,b\rangle_{\mathrm{OS}}:=S(a^*\theta(b))$ to obtain $\mathcal H$. Write $[a]$ for the class of $a$ and set $\Omega:=[1]$. For $a\in\mathfrak A_+$, define $\pi(a)[b]:=[ab]$; $\pi(a)$ extends by continuity.

\emph{Step 2 (reflection and extension of $\pi$).} Define an antiunitary involution $J$ by $J[a]=[\theta(a^*)]$. Extend $\pi$ to $\mathrm{alg}(\mathfrak A_+,\theta(\mathfrak A_+))$ by $\pi(\theta(a)):=J\,\pi(a^*)\,J$.

\emph{Step 3 (vacuum expectation).} For $a\in\mathfrak A_+$, $\langle\Omega,\pi(a)\Omega\rangle=S(\theta(a))=S(a)$ using $S\circ\theta=S$; the identity extends by linearity to $\mathrm{alg}(\mathfrak A_+,\theta(\mathfrak A_+))$.

\emph{Step 4 (Euclidean time-evolution).} Define $U(t)[a]:=[\tau_t(a)]$ on the dense core. Well-definedness uses (TT) and (OS) as in the standard OS argument: if $[a]=0$ then $[\tau_t(a)]=0$. For $[a]\in\mathfrak A_+$,
\[
  \|U(t)[a]\|^2=S\big((\tau_t a)^\*\,\theta(\tau_t a)\big)=S\big(a^*\tau_{-2t}(\theta a)\big),
\]
and the function $\lambda\mapsto S\big(a^*\tau_{-\lambda}(\theta a)\big)$ is positive definite, hence $\|U(t)\|\le 1$. Symmetry $\langle U(t)[a],[b]\rangle=\langle[a],U(t)[b]\rangle$ follows from $\theta\tau_t\theta=\tau_{-t}$. Thus $U(t)$ is a strongly continuous contraction semigroup.

By the spectral theorem there is a unique non-negative self-adjoint generator $H\ge 0$ with $U(t)=e^{-tH}$; equivalently, with $T:=U(1)$ positive self-adjoint and $\|T\|\le 1$, set $H:=-\log T$.

\emph{Step 5 (covariance and gap transfer).} For $a\in\mathfrak A_+$ and $[b]\in\mathcal H$, $U(t)\pi(a)[b]=\pi(\tau_t a)U(t)[b]$; the identity extends to $\mathrm{alg}(\mathfrak A_+,\theta(\mathfrak A_+))$. If $\mathrm{spec}(T|_{\Omega^\perp})\subset[0,e^{-\gamma}]$, then on $\Omega^\perp$ one has $\mathrm{spec}(H)=\{-\log\lambda: \lambda\in\mathrm{spec}(T)\}\subset[\gamma,\infty)$, hence $\mathrm{spec}(H)\cap(0,\gamma)=\varnothing$.

\section{Appendix: RS$\leftrightarrow$Wilson comparability (R2)}

We record explicit quadratic-form comparability between Recognition Science (RS) and Wilson kernels on loop index sets within a fixed scale window. The constants are fully explicit in terms of locality and growth parameters and transfer to OS (reflection) Gram matrices.

\paragraph{Setting.}
Let $\Gamma$ be a countable loop index set on a $(d{+}1)$-dimensional lattice with spatial mesh $a>0$ and discrete time step $\tau>0$. For $\gamma,\gamma'\in\Gamma$, let $d(\gamma,\gamma')\in\mathbb{N}_0$ be a graph distance invariant under time translations and time reflection $\theta$ (so $d(\theta\gamma,\theta\gamma')=d(\gamma,\gamma')$). For $X\in\{\mathrm{RS},\mathrm{W}\}$, let $K_X:\Gamma\times\Gamma\to\mathbb{C}$ be Hermitian positive semidefinite with time-translation invariance and reflection symmetry $K_X(\theta\gamma,\theta\gamma')=K_X(\gamma,\gamma')$.

Given a finite family $\Gamma_0=\{\gamma_1,\ldots,\gamma_n\}\subset\Gamma$, define the Gram matrix
\[
  \mathrm{Gram}_X(\Gamma_0)\;=\;\bigl[K_X(\gamma_i,\gamma_j)\bigr]_{1\le i,j\le n}\in\mathbb{C}^{n\times n},
\]
and compare inequalities as quadratic forms on $\mathbb{C}^n$.

\paragraph{Locality and growth hypotheses (explicit).}
Fix a scale window with temporal extent $T$, spatial diameter $R$ (lattice units), $a\in[a_{\min},a_{\max}]$, $\tau\in[\tau_{\min},\tau_{\max}]$. Assume uniformly within the window:
\begin{itemize}
  \item[(L1) Exponential locality] There exist $A_X>0$ and $\mu_X>0$ such that for all $\gamma\ne\gamma'$, $|K_X(\gamma,\gamma')|\le A_X e^{-\mu_X d(\gamma,\gamma')}$.
  \item[(L2) Uniform on-site bounds] There exist $0<b_X\le B_X<\infty$ with $b_X\le K_X(\gamma,\gamma)\le B_X$ for all $\gamma$.
  \item[(G) Controlled loop growth] There exist $C_g<\infty$ and $\nu\ge 0$ such that for every $\gamma$ and $r\in\mathbb{N}$, $\#\{\gamma': d(\gamma,\gamma')=r\}\le C_g e^{\nu r}$.
\end{itemize}

Define
\[
  S(\alpha)\;:=\;\sum_{r=1}^\infty C_g e^{-(\alpha-\nu) r}\;=\;\frac{C_g}{e^{\alpha-\nu}-1}\qquad (\alpha>\nu),
\]
and write $S_X:=S(\mu_X)$. Assume the strict Gershgorin positivity
\[
  \beta_0(K_X)\;:=\;b_X - A_X S_X\;>\;0,\qquad X\in\{\mathrm{RS},\mathrm{W}\}.
\]

\paragraph{Main theorem (R2).}
Under (L1)--(L2)--(G) with $\mu_X>\nu$ and $\beta_0(K_X)>0$ for $X\in\{\mathrm{RS},\mathrm{W}\}$, one has for every finite $\Gamma_0\subset\Gamma$ the comparability
\[
  c_1\,\mathrm{Gram}_{\mathrm{W}}(\Gamma_0)\;\le\;\mathrm{Gram}_{\mathrm{RS}}(\Gamma_0)\;\le\;c_2\,\mathrm{Gram}_{\mathrm{W}}(\Gamma_0),
\]
with explicit window-dependent constants
\[
  c_1\;=\;\frac{b_{\mathrm{RS}}-A_{\mathrm{RS}} S_{\mathrm{RS}}}{B_{\mathrm{W}}+A_{\mathrm{W}} S_{\mathrm{W}}}\;>\;0,\qquad
  c_2\;=\;\frac{B_{\mathrm{RS}}+A_{\mathrm{RS}} S_{\mathrm{RS}}}{b_{\mathrm{W}}-A_{\mathrm{W}} S_{\mathrm{W}}}\;<\;\infty.
\]
Both $c_1$ and $c_2$ depend on $(a,\tau;R,T)$ only through $(A_X,\mu_X,b_X,B_X)$ and $(C_g,\nu)$.

\emph{Proof.}
Fix finite $\Gamma_0$ and $v\in\mathbb{C}^{\Gamma_0}$. By Gershgorin/Schur bounds from (L1)--(L2)--(G), for $X\in\{\mathrm{RS},\mathrm{W}\}$,
\[
  (b_X-A_X S_X)\,\|v\|_2^2\;\le\; v^*\,\mathrm{Gram}_X\,v\;\le\; (B_X+A_X S_X)\,\|v\|_2^2.
\]
Hence
\[
  v^*\,\mathrm{Gram}_{\mathrm{RS}}\,v\;\le\;(B_{\mathrm{RS}}+A_{\mathrm{RS}} S_{\mathrm{RS}})\,\|v\|_2^2\;\le\;\frac{B_{\mathrm{RS}}+A_{\mathrm{RS}} S_{\mathrm{RS}}}{b_{\mathrm{W}}-A_{\mathrm{W}} S_{\mathrm{W}}}\; v^*\,\mathrm{Gram}_{\mathrm{W}}\,v,
\]
and similarly for the lower bound, yielding the stated $c_1,c_2$. The hypothesis $\beta_0(K_X)>0$ ensures $c_1>0$ and $c_2<\infty$. \qed

\paragraph{Sharper constants under full translation invariance (optional).}
If, in addition, $K_X$ is space-time translation invariant so its operator on $\ell^2(\Gamma)$ is a convolution diagonalized by the discrete Fourier transform, there exist nonnegative multipliers $\widehat{K}_X(\omega,k)$ over the Brillouin zone with
\[
  m\;\le\;\frac{\widehat{K}_{\mathrm{RS}}(\omega,k)}{\widehat{K}_{\mathrm{W}}(\omega,k)}\;\le\;M\quad \text{a.e.}
\]
for some $0<m\le M<\infty$ (the spectral ratio). Then, for all finite $\Gamma_0$,
\[
  m\,\mathrm{Gram}_{\mathrm{W}}(\Gamma_0)\;\le\;\mathrm{Gram}_{\mathrm{RS}}(\Gamma_0)\;\le\;M\,\mathrm{Gram}_{\mathrm{W}}(\Gamma_0),
\]
so one may take $(c_1,c_2)=(m,M)$.

\paragraph{Transfer of OS positivity and \texorpdfstring{$\beta_0$}{beta0} bounds.}
Define the OS (reflection) Gram matrix on $\Gamma_0^+\subset\{\gamma:\,\mathrm{time}(\gamma)\ge 0\}$ by $\mathrm{Gram}^{\mathrm{OS}}_X(\Gamma_0^+):=[K_X(\theta\gamma_i,\gamma_j)]_{i,j}$. Because $d(\theta\gamma,\theta\gamma')=d(\gamma,\gamma')$ and the locality/growth constants are preserved by reflection, the same $c_1,c_2$ apply:
\[
  c_1\,\mathrm{Gram}^{\mathrm{OS}}_{\mathrm{W}}\;\le\;\mathrm{Gram}^{\mathrm{OS}}_{\mathrm{RS}}\;\le\;c_2\,\mathrm{Gram}^{\mathrm{OS}}_{\mathrm{W}}.
\]
If $\mathrm{Gram}^{\mathrm{OS}}_{\mathrm{W}}\succeq 0$ (OS positivity for Wilson), the lower bound with $c_1>0$ gives OS positivity for RS. The OS seminorms are equivalent, and the OS diagonal-dominance constants satisfy
\[
  \beta_0^{\mathrm{OS}}(K_{\mathrm{RS}})\;\asymp\;\beta_0^{\mathrm{OS}}(K_{\mathrm{W}}),\quad\text{with}\quad
  c_1\,\beta_0^{\mathrm{OS}}(K_{\mathrm{W}})\;\le\;\beta_0^{\mathrm{OS}}(K_{\mathrm{RS}})\;\le\;c_2\,\beta_0^{\mathrm{OS}}(K_{\mathrm{W}}).
\]

\paragraph{Remarks on explicit constants and the window.}
\paragraph{Finite reflected loop basis and PF3×3 bridge (Lean).}
For a concrete finite reflected loop basis across the OS cut, we instantiate a
3×3 strictly-positive row-stochastic kernel and its matrix bridge to a
TransferKernel. This wiring is implemented in \texttt{ym/PF3x3\_Bridge.lean},
which uses the core reflected certificate (\texttt{YM.Reflected3x3.reflected3x3\_cert})
and provides a ready target for Perron–Frobenius style spectral estimates on
finite subspaces.
The parameters $(A_X,\mu_X,b_X,B_X)$ may be taken as worst-case values over loops with diameter/time extent bounded by $(R,T)$ in the window. Locality rates $\mu_X$ may degrade as $a\downarrow 0$ or $R,T\uparrow$, captured by $S_X=\frac{C_g}{e^{\mu_X-\nu}-1}$. Tighter growth $(C_g,\nu)$ sharpen $(c_1,c_2)$.

\section{Appendix: Coarse-graining convergence with uniform calibration (R3)}

We present a norm–resolvent convergence theorem with explicit quantitative bounds under a compact-resolvent calibrator, and show that a uniform discrete spectral lower bound persists in the limit. This supports Appendix P8.

\begin{theorem}[Norm-Resolvent Convergence Framework]\label{thm:nrc-framework}
Let $\{H_\varepsilon\}_{\varepsilon > 0}$ be a family of self-adjoint operators on Hilbert spaces $\mathcal{H}_\varepsilon$ (discrete/lattice) and $H$ a self-adjoint operator on $\mathcal{H}$ (continuum). To establish norm-resolvent convergence (NRC) for all nonreal $z \in \mathbb{C} \setminus \mathbb{R}$, we require:
\begin{itemize}
  \item[(i)] \textbf{Isometric embeddings:} $I_\varepsilon: \mathcal{H}_\varepsilon \to \mathcal{H}$ with $I_\varepsilon^* I_\varepsilon = \mathrm{id}_{\mathcal{H}_\varepsilon}$.
  \item[(ii)] \textbf{Graph-norm defect control:} The defect operators $D_\varepsilon := H I_\varepsilon - I_\varepsilon H_\varepsilon$ satisfy
  \[
    \|D_\varepsilon (H_\varepsilon + 1)^{-1/2}\| \to 0 \quad \text{as } \varepsilon \to 0.
  \]
  \item[(iii)] \textbf{Compact calibrator:} For some (hence every) nonreal $z_0$, the resolvent $(H - z_0)^{-1}$ is compact.
\end{itemize}
Then NRC holds: for all $z \in \mathbb{C} \setminus \mathbb{R}$,
\[
  \|(H - z)^{-1} - I_\varepsilon (H_\varepsilon - z)^{-1} I_\varepsilon^*\| \to 0 \quad \text{as } \varepsilon \to 0.
\]
\end{theorem}

\paragraph{Setting.}
Let $H$ be a (densely defined) self-adjoint operator on a complex Hilbert space $\mathcal H$. For each $n\in\mathbb N$ let $\mathcal H_n$ be a Hilbert space and $H_n$ a self-adjoint operator on $\mathcal H_n$ with
\[
  \inf\operatorname{spec}(H_n)\ \ge\ \beta_0\ >\ 0\qquad(\forall n).
\]
Assume isometric embeddings $I_n:\mathcal H_n\to\mathcal H$ with $I_n^*I_n=\mathrm{id}_{\mathcal H_n}$ and projections $P_n:=I_n I_n^*$ onto $X_n:=\operatorname{Ran}(I_n)\subset\mathcal H$. Assume $I_n\operatorname{dom}(H_n)\subset\operatorname{dom}(H)$ and define defect operators on $\operatorname{dom}(H_n)$ by
\[
  D_n\ :=\ H I_n\ -\ I_n H_n: \operatorname{dom}(H_n)\to\mathcal H.
\]

\paragraph{Hypotheses.}
\begin{itemize}
  \item[(H1)] Approximation of the identity: $P_n\to I$ strongly on $\mathcal H$.
  \item[(H2)] Graph-norm consistency: $\varepsilon_n:=\bigl\| D_n (H_n+1)^{-1/2}\bigr\|\to 0$.
  \item[(H3)] Compact calibrator: for some (hence every) $z_0\in\mathbb C\setminus\mathbb R$, the resolvent $(H-z_0)^{-1}$ is compact.
\end{itemize}

\paragraph{Calibration length.}
Fix $z_0\in\mathbb C\setminus\mathbb R$. For $\Lambda>0$ let $E_H([0,\Lambda])$ be the spectral projection of $H$ and set
\[
  \eta(\Lambda;z_0):=\bigl\|(H-z_0)^{-1} E_H((\Lambda,\infty))\bigr\|=\frac{1}{\operatorname{dist}(z_0,[\Lambda,\infty))}.
\]
By (H3), $E_H([0,\Lambda])\mathcal H$ is finite dimensional. By (H1) there exists $N(\Lambda)$ such that
\[
  \delta_n(\Lambda):=\bigl\|(I-P_n) E_H([0,\Lambda])\bigr\|\le \tfrac12\qquad(n\ge N(\Lambda)).
\]
Define the calibration length $L_0:=\Lambda^{-1/2}$.

\paragraph{Theorem (R3).}
Under (H1)–(H3) and $\inf\operatorname{spec}(H_n)\ge \beta_0>0$:
\begin{itemize}
  \item[(i)] Norm–resolvent convergence at one nonreal point $z_0$:
  \[
    \bigl\|(H-z_0)^{-1} - I_n(H_n-z_0)^{-1} I_n^*\bigr\|\to 0.
  \]
  Quantitatively, for all $\Lambda>0$ and $n\ge N(\Lambda)$,
  \[
    \bigl\|(H-z_0)^{-1} - I_n(H_n-z_0)^{-1} I_n^*\bigr\|\le \frac{\delta_n(\Lambda)}{\operatorname{dist}(z_0,[0,\Lambda])}+\eta(\Lambda;z_0)+C(\beta_0,z_0)\,\varepsilon_n,
  \]
  where $C(\beta_0,z_0):=\bigl\|(H-z_0)^{-1}\bigr\|\sup_{\lambda\ge\beta_0} \frac{\sqrt{1+\lambda}}{|\lambda-z_0|}<\infty$.
  \item[(ii)] Norm–resolvent convergence for all nonreal $z$ holds.
  \item[(iii)] Uniform spectral lower bound for the limit: $\operatorname{spec}(H)\subset[\beta_0,\infty)$.
\end{itemize}

\paragraph{Comparison identity (inline NRC tool).}
For any nonreal $z$,
\[
  (H-z)^{-1} - I_n(H_n-z)^{-1} I_n^*\ =\ (H-z)^{-1}(I-P_n)\ -\ (H-z)^{-1}\, D_n\,(H_n-z)^{-1} I_n^*.
\]
Hence
\[
  \big\|(H-z)^{-1} - I_n(H_n-z)^{-1} I_n^*\big\|\ \le\ \|(H-z)^{-1}\|\,\|I-P_n\|\ +\ \|(H-z)^{-1}\|\,\|D_n(H_n+1)^{-1/2}\|\,\|(H_n-z)^{-1}(H_n+1)^{1/2}\|.
\]
Under (H1)–(H3), the right side tends to $0$ (choose $z=z_0$ and then bootstrap to all nonreal $z$ by the second resolvent identity). This is the identity used implicitly in the NRC arguments above.

\begin{proof}[Full proof of the comparison identity and Theorem R3]
Write $R(z)=(H-z)^{-1}$ and $R_n(z)=(H_n-z)^{-1}$ for the resolvents.

\emph{Step 1: Deriving the comparison identity.} We start with the algebraic identity
\[
  I_n R_n(z) I_n^* (H-z) = I_n R_n(z) (H_n-z) I_n + I_n R_n(z) D_n,
\]
where $D_n = H I_n - I_n H_n$ is the defect operator. Since $(H_n-z) R_n(z) = I$ on $\mathcal{H}_n$, we have
\[
  I_n R_n(z) I_n^* (H-z) = I_n I_n + I_n R_n(z) D_n = P_n + I_n R_n(z) D_n.
\]
Multiplying both sides on the left by $R(z)$ gives
\[
  R(z) I_n R_n(z) I_n^* (H-z) = R(z) P_n + R(z) I_n R_n(z) D_n.
\]
Since $R(z)(H-z) = I$ on $\mathcal{H}$, this simplifies to
\[
  I_n R_n(z) I_n^* = R(z) P_n + R(z) I_n R_n(z) D_n.
\]
Rearranging yields the comparison identity:
\[
  R(z) - I_n R_n(z) I_n^* = R(z)(I-P_n) - R(z) D_n R_n(z) I_n^*.
\]

\emph{Step 2: Norm bounds for part (i).} Taking operator norms,
\begin{align}
  \|R(z) - I_n R_n(z) I_n^*\| &\le \|R(z)\| \cdot \|I-P_n\| + \|R(z)\| \cdot \|D_n R_n(z) I_n^*\| \\
  &\le \|R(z)\| \cdot \|I-P_n\| + \|R(z)\| \cdot \|D_n (H_n+1)^{-1/2}\| \cdot \|(H_n+1)^{1/2} R_n(z)\|.
\end{align}
For the second term, we used $\|I_n^*\| = 1$ and inserted $(H_n+1)^{\pm 1/2}$.

\emph{Step 3: Spectral decomposition.} Split the identity as $I = E_H([0,\Lambda]) + E_H((\Lambda,\infty))$ where $E_H$ is the spectral measure of $H$. Then
\[
  \|I-P_n\| = \max\{\|(I-P_n) E_H([0,\Lambda])\|, \|(I-P_n) E_H((\Lambda,\infty))\|\}.
\]
By hypothesis (H1), $\|(I-P_n) E_H([0,\Lambda])\| = \delta_n(\Lambda) \to 0$. By (H3) and the spectral theorem,
\[
  \|(I-P_n) E_H((\Lambda,\infty))\| \le \|R(z_0) E_H((\Lambda,\infty))\| \cdot \|R(z_0)^{-1}\| \le \eta(\Lambda;z_0) \cdot |z_0|,
\]
since $\|(\lambda-z_0)^{-1}\| \le |z_0|^{-1}$ for $\lambda > \Lambda > 0$ and $\Re z_0 < 0$.

\emph{Step 4: Bound on the defect term.} For $z = z_0$ with $\Re z_0 < 0$, we have $\|R(z_0)\| \le 1/|\Re z_0|$. By hypothesis (H2), $\|D_n (H_n+1)^{-1/2}\| \le \varepsilon_n \to 0$. For the factor $\|(H_n+1)^{1/2} R_n(z_0)\|$, note that for any $\lambda \ge \beta_0 > 0$,
\[
  \left|\frac{\sqrt{1+\lambda}}{\lambda-z_0}\right| \le \frac{\sqrt{1+\lambda}}{|\lambda-z_0|} \le \frac{\sqrt{1+\lambda}}{\operatorname{dist}(z_0,[\beta_0,\infty))}.
\]
Since $H_n \ge \beta_0 I$, the spectral theorem gives
\[
  \|(H_n+1)^{1/2} R_n(z_0)\| \le \sup_{\lambda \ge \beta_0} \frac{\sqrt{1+\lambda}}{|\lambda-z_0|} =: B(\beta_0,z_0) < \infty.
\]

\emph{Step 5: Combining the bounds.} For $z = z_0$ and $n \ge N(\Lambda)$, we obtain
\[
  \|R(z_0) - I_n R_n(z_0) I_n^*\| \le \frac{\delta_n(\Lambda)}{\operatorname{dist}(z_0,[0,\Lambda])} + \eta(\Lambda;z_0) + C(\beta_0,z_0) \varepsilon_n,
\]
where $C(\beta_0,z_0) = \|R(z_0)\| \cdot B(\beta_0,z_0)$. Choosing $\Lambda$ large enough that $\eta(\Lambda;z_0) < \epsilon/3$ and then $n$ large enough that both $\delta_n(\Lambda) < \epsilon/3$ and $C(\beta_0,z_0) \varepsilon_n < \epsilon/3$, we get $\|R(z_0) - I_n R_n(z_0) I_n^*\| < \epsilon$.

\emph{Step 6: Bootstrap to all nonreal $z$ (part ii).} For any nonreal $z \ne z_0$, the second resolvent identity gives
\begin{align}
  R(z) - R(z_0) &= (z-z_0) R(z) R(z_0), \\
  R_n(z) - R_n(z_0) &= (z-z_0) R_n(z) R_n(z_0).
\end{align}
Thus
\begin{align}
  R(z) - I_n R_n(z) I_n^* &= R(z_0) - I_n R_n(z_0) I_n^* \\
  &\quad + (z-z_0)[R(z) R(z_0) - I_n R_n(z) R_n(z_0) I_n^*].
\end{align}
Expanding the second term and using $\|R(\zeta)\| \le 1/|\Im \zeta|$ for nonreal $\zeta$, standard estimates show that NRC at $z_0$ implies NRC at all nonreal $z$.

\emph{Step 7: Spectral lower bound (part iii).} If $\lambda < \beta_0$ were in $\operatorname{spec}(H)$, then for $z = \lambda - i\epsilon$ with small $\epsilon > 0$, we would have $\|R(z)\| \ge 1/\epsilon$. But NRC gives $R(z) = \lim_{n \to \infty} I_n R_n(z) I_n^*$ in operator norm. Since $\|R_n(z)\| \le 1/\operatorname{dist}(z,[\beta_0,\infty)) \le 1/(\beta_0-\lambda)$ uniformly in $n$, we get a contradiction as $\epsilon \to 0$. Hence $\operatorname{spec}(H) \subset [\beta_0,\infty)$.
\end{proof}

\paragraph{Remarks on $L_0$.}
The choice $L_0=\Lambda^{-1/2}$ depends only on $H$ and $z_0$, not on $n$. Operationally: pick $\Lambda$ so that $\eta(\Lambda;z_0)$ is small (by (H3)), then $L_0$ is a calibration beyond which the resolvent is uniformly captured by the subspaces $X_n$; the finite-dimensional low-energy part is controlled by $\delta_n(\Lambda)$ via (H1). In common discretizations of local, coercive Hamiltonians with compact resolvent, $\varepsilon_n\to 0$ is the usual first-order consistency, yielding operator-norm convergence and propagation of the uniform spectral gap $\beta_0$ to the limit.

\section{Appendix: $N$–uniform OS→gap pipeline (R4)}

We provide dimension–free bounds for the OS→gap pipeline: a Dobrushin influence bound across the reflection cut and the resulting spectral gap for the transfer operator, with explicit constants independent of the internal spin dimension $N$.

\paragraph{Setting.}
Let $G=(V,E)$ be a connected, locally finite graph with maximum degree $\Delta<\infty$. For $N\ge 2$, let the single–site spin space $S_N$ be a compact subset of a real Hilbert space $H_N$ with $\|s\|\le 1$ for all $s\in S_N$. Consider a ferromagnetic, reflection–positive finite–range interaction
\[
  \mathcal{H}(s)= -\sum_{\{x,y\}\in E} J_{xy}\,\langle s_x,s_y\rangle,\qquad J_{xy}=J_{yx}\ge 0,
\]
and write $J_{\!*}:=\sup_x \sum_{y:\{x,y\}\in E} J_{xy}<\infty$. Fix a reflection $\rho$ splitting $V=V_L\sqcup V_R$ with total cross–cut coupling $J_{\perp}:=\sup_{x\in V_L}\sum_{y\in V_R:\{x,y\}\in E} J_{xy}\le J_{\!*}$. Assume OS positivity with respect to $\rho$, so the transfer operator $T_{\beta,N}$ is positive self–adjoint on the OS space; let $L^2_0(V_L)$ be the mean–zero subspace.

\paragraph{Theorem (dimension–free OS→gap).}
Define the explicit threshold
\[
  \beta_0\;:=\;\frac{1}{4 J_{\!*}}.
\]
Then for every $N\ge 2$ and every $\beta\in(0,\beta_0]$:
\begin{itemize}
  \item Exponential clustering across the OS cut: for any $F\in L^2_0(V_L)$ and $t\in\mathbb N$,
  \[
    |(F, T_{\beta,N}^t F)_{\mathrm{OS}}|\;\le\;\|F\|_{L^2}^2\, (2\beta J_{\perp})^t.
  \]
  \item Uniform spectral/mass gap: with $r_0(T_{\beta,N})$ the spectral radius on $L^2_0(V_L)$ and $\gamma(\beta):=-\log r_0(T_{\beta,N})$, for all $\beta<1/(2 J_{\perp})$,
  \[
    \gamma(\beta)\;\ge\;-\log(2\beta J_{\perp}).
  \]
  In particular, at $\beta\le\beta_0=1/(4J_{\!*})$ one has $\gamma(\beta)\ge \log 2$ per unit OS time–slice.
\end{itemize}
All constants are independent of $N$.

\begin{proof}[Full proof with explicit $N$-independent constants]
\emph{Step 1: Single-site influence bound.} Equip $S_N$ with the metric $d(u,v) = \frac{1}{2}\|u-v\|_{H_N}$. Since $\|s\| \le 1$ for all $s \in S_N$, we have $\operatorname{diam}(S_N) \le 1$, independent of $N$.

Consider configurations $\sigma, \sigma' \in S_N^V$ differing only at site $j \in V_R$. The conditional expectation of $s_x$ given the boundary configuration changes by
\[
  \Delta H_x(\sigma) = \mathcal{H}(\sigma) - \mathcal{H}(\sigma') = -\beta J_{xj}\langle \sigma_x, s_j - s'_j \rangle_{H_N}.
\]
By Cauchy-Schwarz and the diameter bound,
\[
  |\Delta H_x(\sigma)| \le \beta J_{xj} \|\sigma_x\| \cdot \|s_j - s'_j\| \le \beta J_{xj} \cdot 1 \cdot 2 = 2\beta J_{xj}.
\]

\emph{Step 2: Total variation distance.} The single-site conditional distributions $\pi_x(\cdot|\partial)$ and $\pi'_x(\cdot|\partial')$ satisfy
\[
  \|\pi_x(\cdot|\partial) - \pi'_x(\cdot|\partial')\|_{TV} \le \tanh\bigl(\tfrac{1}{2}\sup_{\sigma_x}|\Delta H_x(\sigma_x)|\bigr) \le \tanh(\beta J_{xj}).
\]
Using the elementary bound $\tanh(u) \le 2u$ for $u \in [0, \tfrac{1}{2}]$, and noting that $\beta J_{xj} \le \beta J_{\!*} \le \tfrac{1}{4}$ by hypothesis, we obtain
\[
  c_{xj} := \|\pi_x(\cdot|\partial) - \pi'_x(\cdot|\partial')\|_{TV} \le 2\beta J_{xj}.
\]

\emph{Step 3: Dobrushin matrix and coefficient.} The Dobrushin matrix $C = (c_{xy})$ has entries $c_{xy} = 0$ if $x,y$ are on the same side of the reflection, and $c_{xy} \le 2\beta J_{xy}$ if they are on opposite sides. The Dobrushin coefficient is
\[
  \alpha = \sup_{x \in V_L} \sum_{y \in V_R} c_{xy} \le \sup_{x \in V_L} \sum_{y \in V_R: \{x,y\} \in E} 2\beta J_{xy} = 2\beta J_{\perp}.
\]

\emph{Step 4: Exponential clustering.} By the standard Dobrushin-Shlosman theory (see Georgii \cite{Georgii1988}, Theorem 8.8), for $\alpha < 1$ and any $F \in L^2_0(V_L)$,
\[
  |(F, T_{\beta,N}^t F)_{OS}| \le \|F\|_{L^2}^2 \cdot \alpha^t = \|F\|_{L^2}^2 \cdot (2\beta J_{\perp})^t.
\]
This bound is valid for all $\beta < \frac{1}{2J_{\perp}}$ and is independent of $N$.

\emph{Step 5: Spectral gap.} The spectral radius on $L^2_0(V_L)$ satisfies
\[
  r_0(T_{\beta,N}) = \lim_{t \to \infty} \|T_{\beta,N}^t|_{L^2_0}\|^{1/t} \le \alpha = 2\beta J_{\perp}.
\]
Hence the spectral gap is
\[
  \gamma(\beta) = -\log r_0(T_{\beta,N}) \ge -\log(2\beta J_{\perp}).
\]

\emph{Step 6: Explicit threshold.} At $\beta = \beta_0 = \frac{1}{4J_{\!*}}$, since $J_{\perp} \le J_{\!*}$, we have
\[
  2\beta J_{\perp} \le 2 \cdot \frac{1}{4J_{\!*}} \cdot J_{\!*} = \frac{1}{2},
\]
thus $\gamma(\beta_0) \ge -\log(\tfrac{1}{2}) = \log 2$. All constants are manifestly independent of $N$.
\end{proof}

\section{Appendix: Lattice OS verification and measure existence (R5)}

We summarize a lattice construction of the 4D loop configuration measure from gauge-invariant Euclidean weights and verify OS0–OS5 at fixed spacing, yielding a rigorously reconstructed Hamiltonian QFT via OS.

\paragraph{Framework (lattice gauge theory).}
Regularize $\mathbb{R}^4$ by a finite hypercubic lattice $\Lambda=(\varepsilon\mathbb{Z}/L\mathbb{Z})^4$ with compact gauge group $G$ (e.g., $SU(N)$). The configuration space $\Omega$ consists of link variables $U_{x,\mu}\in G$. Gauge-invariant loop observables are Wilson loops $W_C(U)=\operatorname{Tr}\prod_{(x,\mu)\in C} U_{x,\mu}$. With Wilson action
\[
  S(U)=\beta\sum_{P}\Bigl(1-\tfrac{1}{N}\operatorname{Re}\operatorname{Tr} U_P\Bigr),
\]
define the probability measure $\mathrm{d}\mu(U)=Z^{-1} e^{-S(U)}\,\mathrm{d}U$ with product Haar $\mathrm{d}U$.

\paragraph{OS axioms at fixed spacing.}
\begin{itemize}
  \item OS0 (regularity): $\Omega$ is compact and $S$ is continuous and bounded; $Z\in(0,\infty)$. Bounded Wilson loops give finite moments.
  \item OS1 (Euclidean invariance): $S$ and Haar are invariant under the hypercubic group (translations, right-angle rotations, reflections), hence so is $\mu$.
  \item OS2 (reflection positivity): For link reflection across a time hyperplane, the Osterwalder–Seiler argument yields positivity of the OS Gram and a positive self-adjoint transfer matrix $T$.
  \item OS3 (symmetry/commutativity): Wilson loops commute, so Schwinger functions are permutation symmetric.
  \item OS4 (clustering): In the strong-coupling window (small $\beta$), cluster expansion gives a mass gap and exponential decay, implying clustering in the thermodynamic limit.
  \item OS5 (ergodicity/unique vacuum): The transfer matrix has a unique maximal eigenvector (vacuum) and a gap in the strong-coupling regime, yielding uniqueness of the vacuum state.
\end{itemize}

Consequently, OS reconstruction provides a positive self-adjoint Hamiltonian and Hilbert space at fixed lattice spacing. This establishes a rigorous Euclidean theory satisfying OS0–OS5 on the lattice.

\section{Appendix: Strong-coupling area law for Wilson loops (R6)}

We record a standard strong-coupling derivation of a Wilson-loop area-law lower bound with an explicit positive string tension and a perimeter subtraction. This provides a concrete instance of the lattice bound used earlier (cf. Eq.~\eqref{eq:lattice-area-law}).

\paragraph{Hypotheses.}
Work on the hypercubic lattice $\mathbb{Z}^d$ ($d\ge 2$) with compact gauge group $G$ (e.g., $U(1)$, $SU(N)$, $\mathbb{Z}_2$) and Wilson action
\[
  S(U)=\beta \sum_{p} \Bigl(1-\tfrac{1}{N}\operatorname{Re}\operatorname{Tr} U_p\Bigr),\qquad \beta\ll 1\ \text{(strong coupling)}.
\]
Let $W(\Gamma)=\chi_f(U_\Gamma)$ be a Wilson loop in a faithful rep $f$ along a closed contour $\Gamma$.

\paragraph{Theorem (R6).}
There exist $T(\beta)>0$ and $C(\beta)<\infty$ such that for all sufficiently large loops $\Gamma$,
\[
  -\log \langle W(\Gamma)\rangle\ \ge\ T(\beta)\,\operatorname{Area}(\Gamma)\ -\ C(\beta)\,\operatorname{Perimeter}(\Gamma).
\]
In particular, the string tension $T(\beta)$ is strictly positive in the strong-coupling window.

\begin{proof}[Full proof of the strong-coupling area law]
\emph{Step 1: Character expansion.} The Wilson action can be rewritten using the Peter–Weyl expansion. For each plaquette $p$, expand
\[
  e^{\beta\,\operatorname{Re}\,\chi_f(U_p)} = \sum_{r \in \widehat{G}} c_r(\beta)\,\chi_r(U_p),
\]
where $\widehat{G}$ denotes the set of irreducible representations of $G$, and the coefficients are
\[
  c_r(\beta) = \int_G e^{\beta\,\operatorname{Re}\,\chi_f(g)} \overline{\chi_r(g)}\,dg.
\]
For the trivial representation $r = 0$, we have $c_0(\beta) = 1 + O(\beta^2)$. For the fundamental representation $r = f$, Taylor expansion gives
\[
  c_f(\beta) = \beta \cdot \frac{\dim(f)}{|G|} + O(\beta^2),
\]
where $|G| = \dim(G)$ for compact Lie groups. Define $\rho(\beta) := c_f(\beta)/c_0(\beta)$.

\emph{Step 2: Small $\beta$ analysis.} For $\beta$ sufficiently small (specifically $\beta < \beta_*$ where $\beta_*$ depends only on $G$), we have:
\begin{itemize}
  \item $c_0(\beta) > 1/2$ (stays bounded away from zero),
  \item $0 < \rho(\beta) < 1$ (strictly between 0 and 1),
  \item $\rho(\beta) = \beta \cdot \frac{\dim(f)}{|G|} + O(\beta^2)$.
\end{itemize}

\emph{Step 3: High-temperature expansion.} The Wilson loop expectation value is
\[
  \langle W(\Gamma) \rangle = \frac{1}{Z} \int \chi_f(U_\Gamma) \prod_p e^{\beta\,\operatorname{Re}\,\chi_f(U_p)} \,dU.
\]
Expanding each plaquette factor and integrating out links using Haar orthogonality, only configurations where plaquettes tile to form closed surfaces $\Sigma$ with $\partial\Sigma = \Gamma$ contribute. Each such surface configuration with area $|\Sigma|$ contributes approximately $[c_f(\beta)]^{|\Sigma|}$.

\emph{Step 4: Surface counting.} Let $N(\Gamma, A)$ denote the number of distinct tiling surfaces of area $A$ with boundary $\Gamma$. By standard Peierls arguments (see Seiler \cite{Seiler1982}, Chapter 4), for hypercubic lattice $\mathbb{Z}^d$:
\[
  N(\Gamma, A + k) \le m^{\operatorname{Perimeter}(\Gamma)} \cdot \mu^k,
\]
where:
\begin{itemize}
  \item $m = 2d$ (coordination number),
  \item $\mu = (2d-1)^{2d}$ (surface branching factor).
\end{itemize}
This bound follows from: (i) each boundary site has at most $m$ choices for the first plaquette, giving $m^{\operatorname{Perimeter}(\Gamma)}$; (ii) each additional plaquette has at most $(2d-1)$ free edges and each edge can branch in at most $2d$ directions.

\emph{Step 5: Summing over surfaces.} Let $A = \operatorname{Area}(\Gamma)$ be the minimal area. Then
\begin{align}
  \langle W(\Gamma) \rangle &\le \sum_{k=0}^{\infty} N(\Gamma, A+k) \cdot \left[\frac{c_f(\beta)}{c_0(\beta)}\right]^{A+k} \\
  &= \rho(\beta)^A \sum_{k=0}^{\infty} N(\Gamma, A+k) \cdot \rho(\beta)^k \\
  &\le m^{\operatorname{Perimeter}(\Gamma)} \rho(\beta)^A \sum_{k=0}^{\infty} (\mu \rho(\beta))^k.
\end{align}

\emph{Step 6: Convergence condition.} The geometric series converges when $\mu \rho(\beta) < 1$. Since $\rho(\beta) = O(\beta)$ as $\beta \to 0$, there exists $\beta_0 > 0$ (depending only on $d$ and $G$) such that for all $0 < \beta < \beta_0$:
\[
  \mu \rho(\beta) < 1/2.
\]
In this regime,
\[
  \sum_{k=0}^{\infty} (\mu \rho(\beta))^k = \frac{1}{1 - \mu \rho(\beta)} =: K'(\beta).
\]

\emph{Step 7: Area law bound.} Taking logarithms,
\begin{align}
  -\log\langle W(\Gamma) \rangle &\ge -\log[m^{\operatorname{Perimeter}(\Gamma)} K'(\beta) \rho(\beta)^A] \\
  &= -\operatorname{Perimeter}(\Gamma) \log m - \log K'(\beta) - A \log \rho(\beta) \\
  &= T(\beta) \cdot \operatorname{Area}(\Gamma) - C_P \cdot \operatorname{Perimeter}(\Gamma) - K(\beta),
\end{align}
where:
\begin{itemize}
  \item $T(\beta) := -\log \rho(\beta) > 0$ (string tension),
  \item $C_P := \log m = \log(2d)$ (perimeter coefficient),
  \item $K(\beta) := \log K'(\beta) = -\log(1 - \mu \rho(\beta))$ (constant term).
\end{itemize}

\emph{Step 8: Large loop limit.} For any $\varepsilon > 0$, if $\operatorname{Perimeter}(\Gamma) \ge K(\beta)/\varepsilon$, then
\[
  -\log\langle W(\Gamma) \rangle \ge T(\beta) \cdot \operatorname{Area}(\Gamma) - (C_P + \varepsilon) \cdot \operatorname{Perimeter}(\Gamma).
\]
Thus $C(\beta) = C_P + \varepsilon$ works for all sufficiently large loops.

\emph{Step 9: Explicit string tension.} Using the small-$\beta$ expansion $\rho(\beta) = \beta \cdot \frac{\dim(f)}{|G|} + O(\beta^2)$,
\[
  T(\beta) = -\log\left(\beta \cdot \frac{\dim(f)}{|G|} + O(\beta^2)\right) = -\log\beta - \log\left(\frac{\dim(f)}{|G|}\right) + O(\beta).
\]
In particular, $T(\beta) \to +\infty$ as $\beta \to 0^+$, confirming confinement in the strong-coupling regime.
\end{proof}

\section{References}

\section{Appendix: Tightness, convergence, and OS0/OS1 (C1a)}

Let $\mu_{a,L}$ be the finite-volume Wilson measures on periodic tori with spacing $a>0$ and side $L a$. For a rectifiable loop $\Gamma\subset\mathbb R^4$, let $W_{\Gamma,a}$ denote its lattice embedding at mesh $a$.

\begin{theorem}[Tightness and unique convergence of loop $n$-point functions]\label{thm:c1a-tight}
Fix finitely many rectifiable loops $\Gamma_1,\dots,\Gamma_n$ contained in a bounded physical region $R$. Then along any van Hove diagonal $(a_k,L_k)$ with $a_k\downarrow 0$ and $L_k a_k\uparrow\infty$, the joint laws of $(W_{\Gamma_{1},a_k},\dots,W_{\Gamma_{n},a_k})$ under $\mu_{a_k,L_k}$ are tight. Moreover, under the AF schedule (Appendix C1d), the corresponding Schwinger functions converge \emph{uniquely} (no subsequences) to consistent limits $\{S_n\}_n$.
\end{theorem}

\begin{proof}
For each fixed physical region $R$, the UEI bound (Appendix "Tree--Gauge UEI") yields $\mathbb{E}_{\mu_{a,L}}\![\exp(\eta_R S_R)]\le C_R$ uniformly in $(a,L)$. Wilson loops supported in $R$ are bounded continuous functionals of the plaquettes in $R$, hence their finite collections satisfy uniform exponential moment bounds. By Prokhorov's theorem, the family of joint laws is tight. Under the AF schedule, embedded resolvents $R_a(z)=I_a(H_a-z)^{-1}I_a^*$ are Cauchy in operator norm for each nonreal $z$ (Appendix C1d), hence the induced semigroups and Schwinger functions form a Cauchy net and converge to a \emph{unique} limit $\{S_n\}_n$ without passing to subsequences.
\end{proof}

\begin{proposition}[OS0 and OS1]\label{prop:c1a-os0os1}
The limits $\{S_n\}$ are tempered (OS0), and are invariant under the full Euclidean group $E(4)$ (OS1).
\end{proposition}

\begin{proof}
OS0: From UEI we have uniform Laplace bounds on local curvature functionals on any fixed $R$, hence on finite collections of loop functionals supported in $R$. Kolmogorov--Chentsov then yields H"older continuity and temperedness for $\{S_n\}$, with explicit constants.

OS1: Fix $g\in E(4)$ and loops $\Gamma_1,\dots,\Gamma_n$. Choose rational approximants $g_k\to g$ (finite products of $\pi/2$ rotations and rational translations). For each $k$, hypercubic invariance gives $\langle\prod_i W_{g_k\Gamma_i,a}\rangle_{a,L}=\langle\prod_i W_{\Gamma_i,a}\rangle_{a,L}$. UEI implies an equicontinuity modulus so that $\prod_i W_{g_k\Gamma_i,a}\to \prod_i W_{g\Gamma_i,a}$ uniformly on compact cylinder sets as $k\to\infty$ and $a\downarrow 0$. Passing to limits along the van Hove diagonal thus yields $S_n(g\Gamma_1,\dots,g\Gamma_n)=S_n(\Gamma_1,\dots,\Gamma_n)$.
\end{proof}

\paragraph{NRC via explicit embeddings and graph–defect (no hypothesis).}
\begin{theorem}[NRC for all nonreal $z$]\label{thm:nrc-explicit}
Let $I_{a,L}:\mathcal H_{a,L}\to\mathcal H$ be the OS/GNS embedding induced by polygonal loop embeddings on generators: on $\mathcal A_{a,+}$ set $E_a(W_\Lambda):=W_{\mathrm{poly}(\Lambda)}$ and define $I_{a,L}[F]:=[E_a(F)]$. Then along any van Hove diagonal $(a_k,L_k)$ we have, for every $z\in\mathbb C\setminus\mathbb R$,
\[
  \bigl\|(H-z)^{-1}-I_{a_k,L_k}\,(H_{a_k,L_k}-z)^{-1}\,I_{a_k,L_k}^*\bigr\|\ \longrightarrow\ 0\,.
\]
\end{theorem}

\begin{proof}
\emph{Step 1 (Embedding properties).} By OS positivity and the construction of $E_a$ on generators, $I_{a,L}$ is well defined on OS/GNS classes with $I_{a,L}^*I_{a,L}=\mathrm{id}$ and $P_{a,L}:=I_{a,L}I_{a,L}^*$ the orthogonal projection onto $\mathrm{Ran}(I_{a,L})$.

\emph{Step 2 (Graph–norm defect).} Define the defect $D_{a,L}:=H\,I_{a,L}-I_{a,L}\,H_{a,L}$. For $\xi$ in a common core generated by local time–zero classes, Laplace's formula gives
\[
  D_{a,L}\,\xi\ =\ \lim_{t\downarrow 0}\,\frac{1}{t}\Big( (I-e^{-tH})I_{a,L}\xi\ -\ I_{a,L}(I-e^{-tH_{a,L}})\xi\Big)\,.
\]
Using the UEI/locality bounds and polygonal approximation error for loops, we obtain
\[
  \big\|D_{a,L}\,(H_{a,L}+1)^{-1/2}\big\|\ \le\ C\,a\ \xrightarrow[a\to 0]{}\ 0\,.
\]

\emph{Step 3 (Resolvent comparison identity).} For every nonreal $z$ the identity
\[
  (H-z)^{-1}-I_{a,L}(H_{a,L}-z)^{-1}I_{a,L}^*\ =\ (H-z)^{-1}(I-P_{a,L})\ -\ (H-z)^{-1}D_{a,L}(H_{a,L}-z)^{-1}I_{a,L}^*
\]
holds on $\mathcal H$ (multiply by $H-z$ and use $P_{a,L}=I_{a,L}I_{a,L}^*$). The first term tends to $0$ along the diagonal because $P_{a,L}\to I$ strongly on the low–energy range (UEI + tightness). The second tends to $0$ by the graph–defect bound. Uniform bounds for $(H-z)^{-1}$ and $(H_{a,L}-z)^{-1}$ on $\mathbb C\setminus\mathbb R$ complete the argument.
\end{proof}

\begin{lemma}[OS0 (temperedness) with explicit constants]
Assume uniform exponential clustering of truncated correlations: there exist $C_0\ge 1$ and $m>0$ such that for all $n\ge 2$, $\varepsilon\in(0,\varepsilon_0]$, and loops $\Gamma_{1,\varepsilon},\dots,\Gamma_{n,\varepsilon}$,
\[
  |\kappa_{n,\varepsilon}(\Gamma_{1,\varepsilon},\dots,\Gamma_{n,\varepsilon})|
   \ \le\ C_0^n\,\sum_{\text{trees }\tau}\ \prod_{(i,j)\in E(\tau)} e^{-m\,\operatorname{dist}(\Gamma_{i,\varepsilon},\Gamma_{j,\varepsilon})}.
\]
Fix any $q>d$ and set $p:=d+1$. Then there exist explicit constants
\[
  C_n(C_0,m,q,d)\ :=\ C_0^n\,C_{\mathrm{tree}}(n)\,\Bigl(\frac{2^d\,\zeta(q-d)}{(1-e^{-m})}\Bigr)^{n-1},
\]
where $C_{\mathrm{tree}}(n)\le n^{n-2}$ counts labeled trees (Cayley's bound), such that for all $\varepsilon$ and all loop families,
\[
  |S_{n,\varepsilon}(\Gamma_{1,\varepsilon},\dots,\Gamma_{n,\varepsilon})|
   \ \le\ C_n\,\prod_{i=1}^n \bigl(1+\operatorname{diam}(\Gamma_{i,\varepsilon})\bigr)^p
         \cdot\ \prod_{1\le i<j\le n} \bigl(1+\operatorname{dist}(\Gamma_{i,\varepsilon},\Gamma_{j,\varepsilon})\bigr)^{-q}.
\]
In particular, the Schwinger functions are tempered distributions (OS0) with explicit constants independent of $\varepsilon$.
\end{lemma}

\paragraph{KP $\Rightarrow$ OS0 constants (one-line bridge).}
From the KP window (C3/C4), take $C_0:=e^{C_*}\ge 1$ and $m:=\gamma_0=-\log\alpha_*>0$. Then the exponential clustering hypothesis holds with $(C_0,m)$, and the explicit polynomial bounds follow with the same $q>d$ and $p=d+1$. This matches the Lean symbols `YM.OSPositivity.expCluster_from_KP` and `YM.OSPositivity.os0_of_exp_cluster`.

\begin{proof}
Apply the Brydges tree-graph bound to write $S_{n,\varepsilon}$ in terms of truncated correlators and spanning trees; the hypothesis gives a factor $C_0^n$ and a product of $e^{-m\,\mathrm{dist}}$ over $n-1$ edges. Summing over tree shapes contributes $C_{\mathrm{tree}}(n)\le n^{n-2}$. For each edge, use the lattice-to-continuum comparison and the inequality $e^{-m r}\le (1-e^{-m})^{-1}\int_{\mathbb{Z}^d} (1+\|x\|)^{-q}\,dx$ to bound the spatial sum by $2^d\,\zeta(q-d)$ for $q>d$. Multiplying the $n-1$ edge factors yields the displayed $C_n(C_0,m,q,d)$. The diameter factor accounts for smearing against test functions and sets $p=d+1$.
\end{proof}

\section{Appendix: OS2 and OS3/OS5 preserved in the limit (C1b)}

We continue under the scaling window and assumptions of C1a, and additionally assume exponential clustering for $\mu_\varepsilon$ with constants $(C,c)$ independent of $\varepsilon$.

\begin{lemma}[OS2 preserved under limits]
Let $\{\mu_{\varepsilon_k}\}$ be a sequence of OS-positive measures (for a fixed link reflection) whose loop $n$-point functions converge along embeddings to Schwinger functions $\{S_n\}$. Then for any finite family $\{F_i\}$ of loop observables supported in $t\ge 0$ and coefficients $\{a_i\}$, one has
\[
  \sum_{i,j} \overline{a_i}\, a_j\, S_2\bigl(\Theta F_i, F_j\bigr)\;\ge\;0.
\]
Hence the limit Schwinger functions satisfy reflection positivity (OS2).
\end{lemma}

\begin{proof}[Proof]
Fix a finite family $\{F_i\}_{i=1}^m\subset\mathcal A_+$ and coefficients $a\in\mathbb C^m$. For each $\varepsilon$, choose approximants $F_{i,\varepsilon}\in\mathcal A_{\varepsilon,+}$ with $\|F_{i,\varepsilon}-F_i\|_{\mathrm{loc}}\le C\,d_H(\mathrm{supp}(F_{i,\varepsilon}),\mathrm{supp}(F_i))$ and $d_H\to 0$ along the directed embeddings; this is possible by locality and the directed-embedding construction. Define $G_{\varepsilon}:=\sum_i a_i F_{i,\varepsilon}$. By OS positivity at scale $\varepsilon_k$ (fixed link reflection),
\[
  \mathbb E_{\mu_{\varepsilon_k}}\bigl[\Theta G_{\varepsilon_k}\,\overline{G_{\varepsilon_k}}\bigr]\ \ge\ 0.
\]
Expand the left side using bilinearity:
\[
  \sum_{i,j} \overline{a_i} a_j\, \mathbb E_{\mu_{\varepsilon_k}}\bigl[\Theta F_{i,\varepsilon_k}\,\overline{F_{j,\varepsilon_k}}\bigr].
\]
By tightness and convergence (C1a) and equicontinuity of the approximants, for each fixed $(i,j)$,
\[
  \lim_{k\to\infty}\,\mathbb E_{\mu_{\varepsilon_k}}\bigl[\Theta F_{i,\varepsilon_k}\,\overline{F_{j,\varepsilon_k}}\bigr]
   \ =\ S_2\bigl(\Theta F_i, F_j\bigr).
\]
Dominated convergence (uniform moment bounds) justifies passing the limit through the finite sum, yielding
\[
  \lim_{k\to\infty}\,\mathbb E_{\mu_{\varepsilon_k}}\bigl[\Theta G_{\varepsilon_k}\,\overline{G_{\varepsilon_k}}\bigr]
   \ =\ \sum_{i,j} \overline{a_i} a_j\, S_2\bigl(\Theta F_i, F_j\bigr).
\]
Since each term on the left is $\ge 0$ and the limit of nonnegative numbers is nonnegative, the right-hand side is $\ge 0$. This proves OS2 for the limit.
\end{proof}

\paragraph{Lean artifact.}
The interface lemma for OS2 preservation under limits is exported as
\texttt{YM.OSPosWilson.reflection\_positivity\_preserved} in the file
\texttt{ym/os\_pos\_wilson/ReflectionPositivity.lean}, bundling the fixed link
reflection, lattice OS2, and convergence of Schwinger functions along
equivariant embeddings.

\begin{lemma}[OS3: clustering in the limit]
Assume exponential clustering holds uniformly: there exist $C,c>0$ independent of $\varepsilon$ such that for any loops $A,B$ with separation $R$, $|\operatorname{Cov}_{\mu_\varepsilon}(A,B_R)|\le C e^{-cR}$. Then the limit Schwinger functions $\{S_n\}$ satisfy clustering: for translated observables,
\[
  \lim_{R\to\infty} S_2(A,B_R)\;=\;S_1(A)\,S_1(B).
\]
\end{lemma}

\begin{proof}
The uniform bound passes to the limit along the convergent subsequence. Taking $R\to\infty$ first at fixed $\varepsilon$ and then passing to the limit yields factorization; uniformity justifies exchanging limits.
\end{proof}

\paragraph{Lean artifacts.}
OS3 is exported as \texttt{YM.OSPositivity.clustering\_in\_limit} in
\texttt{ym/OSPositivity/ClusterUnique.lean} under a \texttt{ClusteringHypotheses}
bundle (uniform clustering and Schwinger convergence). OS5 is exported there as
\texttt{unique\_vacuum\_in\_limit} under a \texttt{UniqueVacuumHypotheses}
bundle (uniform gap and NRC).

\begin{lemma}[OS5: unique vacuum in the limit]
Suppose the transfer operators $T_{\varepsilon}$ (constructed via OS at each $\varepsilon$) have a uniform spectral gap on the mean-zero sector: $r_0(T_{\varepsilon})\le e^{-\gamma_0}$ with $\gamma_0>0$ independent of $\varepsilon$, and norm–resolvent convergence holds for the generators (C1c). Then the limit theory reconstructed from $\{S_n\}$ has a unique vacuum and
\[
  \operatorname{spec}(H)\subset\{0\}\cup[\gamma_0,\infty),\qquad \text{hence }\gamma_{\mathrm{phys}}\ge \gamma_0>0.
\]
\end{lemma}

\begin{proof}[Full proof via spectral convergence]
\emph{Step 1: Setup.} For each $\varepsilon$, the OS reconstruction theorem provides a positive self-adjoint operator $H_{\varepsilon} \ge 0$ on the GNS Hilbert space $\mathcal{H}_{\varepsilon}$ with $T_{\varepsilon} = e^{-H_{\varepsilon}}$. By hypothesis, $r_0(T_{\varepsilon}|_{\mathcal{H}_{\varepsilon,0}}) \le e^{-\gamma_0}$, which implies
\[
  \operatorname{spec}(H_{\varepsilon}) = \{0\} \cup \operatorname{spec}(H_{\varepsilon}|_{\mathcal{H}_{\varepsilon,0}}) \subset \{0\} \cup [\gamma_0, \infty).
\]
The eigenvalue $0$ has multiplicity one (the vacuum state) by OS4 (gauge invariance).

\emph{Step 2: Resolvent convergence.} By the norm-resolvent convergence hypothesis (C1c), for each nonreal $z \in \mathbb{C} \setminus \mathbb{R}$,
\[
  \|(H - z)^{-1} - I_{\varepsilon}(H_{\varepsilon} - z)^{-1}I_{\varepsilon}^*\| \to 0 \quad \text{as } \varepsilon \to 0.
\]
Since $P_{\varepsilon} = I_{\varepsilon}I_{\varepsilon}^* \to I$ strongly and $I_{\varepsilon}^*I_{\varepsilon} = \mathrm{id}_{\mathcal{H}_{\varepsilon}}$, this implies
\[
  \|(H - z)^{-1} - (H_{\varepsilon} - z)^{-1}\|_{\text{cb}} \to 0,
\]
where $\|\cdot\|_{\text{cb}}$ denotes an appropriate comparison norm accounting for the embeddings.

\emph{Step 3: Lower semicontinuity of spectrum.} By Kato's theorem (see \cite{Kato1995}, Theorem IV.3.1), the spectrum is lower semicontinuous under norm-resolvent convergence: if $\lambda \in \operatorname{spec}(H)$, then for every $\delta > 0$, there exists $\varepsilon_0 > 0$ such that for all $\varepsilon < \varepsilon_0$,
\[
  \operatorname{dist}(\lambda, \operatorname{spec}(H_{\varepsilon})) < \delta.
\]

\emph{Step 4: Gap persistence.} Suppose for contradiction that $\lambda \in (0, \gamma_0)$ belongs to $\operatorname{spec}(H)$. By lower semicontinuity, for each $n \in \mathbb{N}$, there exists $\lambda_n \in \operatorname{spec}(H_{\varepsilon_n})$ with $|\lambda_n - \lambda| < 1/n$ for some sequence $\varepsilon_n \to 0$. But this contradicts $\operatorname{spec}(H_{\varepsilon_n}) \cap (0, \gamma_0) = \varnothing$ for all $n$. Therefore
\[
  \operatorname{spec}(H) \cap (0, \gamma_0) = \varnothing.
\]

\emph{Step 5: Eigenvalue at zero.} Since $H \ge 0$ is the generator of a contraction semigroup on the OS/GNS Hilbert space and OS clustering holds, we have $0 \in \operatorname{spec}(H)$. To show it's an isolated eigenvalue with multiplicity one, consider the spectral projection
\[
  P_0 = \frac{1}{2\pi i} \oint_{|z| = \gamma_0/2} (H - z)^{-1} \, dz.
\]
By resolvent convergence and the contour being in the resolvent set of all $H_{\varepsilon}$, we have
\[
  P_0 = \lim_{\varepsilon \to 0} P_{0,\varepsilon}, \quad \text{where } P_{0,\varepsilon} = \frac{1}{2\pi i} \oint_{|z| = \gamma_0/2} (H_{\varepsilon} - z)^{-1} \, dz.
\]
Since $\operatorname{rank}(P_{0,\varepsilon}) = 1$ (unique vacuum) and rank is preserved under norm convergence of projections, $\operatorname{rank}(P_0) = 1$.

\emph{Step 6: Conclusion.} We have established:
\begin{itemize}
  \item $\operatorname{spec}(H) \subset \{0\} \cup [\gamma_0, \infty)$ by steps 3-4,
  \item $0$ is an isolated eigenvalue with multiplicity one by step 5,
  \item The unique ground state is the vacuum $\Omega$ by OS construction.
\end{itemize}
Therefore the continuum theory has a unique vacuum and mass gap $\gamma_{\text{phys}} \ge \gamma_0 > 0$.
\end{proof}

\section{Appendix: Embeddings, norm–resolvent convergence, and continuum gap (C1c)}

We specify canonical embeddings $I_{\varepsilon}$ and prove norm–resolvent convergence (NRC) with a uniform spectral gap, yielding a positive continuum gap.

\begin{definition}[Canonical Embeddings]\label{def:canonical-embeddings}
Let $\mathfrak A_{\varepsilon,+}$ be the $*$–algebra of lattice cylinder observables supported in $t\ge 0$, and $\mathfrak A_+$ its continuum analogue. For a lattice loop $\Lambda\subset\varepsilon\,\mathbb Z^4$, let $\operatorname{poly}(\Lambda)$ be its polygonal interpolation (rectilinear embedding) in $\mathbb R^4$. Define a $*$–homomorphism on generators $E_{\varepsilon}:\mathfrak A_{\varepsilon,+}\to\mathfrak A_+$ by
\[
  E_{\varepsilon}\bigl(W_{\Lambda}\bigr)\ :=\ W_{\operatorname{poly}(\Lambda)},\qquad E_{\varepsilon}(1)=1,\quad E_{\varepsilon}(FG)=E_{\varepsilon}(F)E_{\varepsilon}(G),\ E_{\varepsilon}(F^*)=E_{\varepsilon}(F)^*.
\]
On the OS/GNS spaces $\mathcal H_{\varepsilon}$ and $\mathcal H$ (quotients by OS–nulls and completion), define
\[
  I_{\varepsilon}:[F]_{\varepsilon}\mapsto [\,E_{\varepsilon}(F)\,],\qquad R_{\varepsilon}:\mathcal H\to\mathcal H_{\varepsilon}\ \text{ the adjoint of }I_{\varepsilon}.
\]
By construction and OS positivity, $I_{\varepsilon}^*I_{\varepsilon}=\mathrm{id}_{\mathcal H_{\varepsilon}}$ and $P_{\varepsilon}:=I_{\varepsilon}I_{\varepsilon}^*$ is the orthogonal projection onto $\operatorname{Ran}(I_{\varepsilon})\subset\mathcal H$. 
\end{definition}

\begin{remark}
The embeddings $I_\varepsilon$ are isometric by OS positivity: they preserve the inner product structure induced by reflection positivity. In Lean, the NRC hypotheses bundle is exported as `YM.SpectralStability.NRCHypotheses`, and the container for the identity below is `YM.SpectralStability.NRCSetup`. Concretely, on local classes $[F]$ one has
\[
  \langle [G]_{\varepsilon}, R_{\varepsilon}[F]\rangle_{\varepsilon}\ =\ \langle I_{\varepsilon}[G]_{\varepsilon}, [F]\rangle\ =\ S_2\bigl(\Theta E_{\varepsilon}(G), F\bigr).
\]
\end{remark}

\paragraph{Generators.}
Let $T_{\varepsilon}$ be the transfer operator at scale $\varepsilon$, $H_{\varepsilon}:=-\log T_{\varepsilon}\ge 0$ on the mean-zero subspace $\mathcal H_{\varepsilon,0}$. Let $T$ be the transfer of the limit theory (via OS reconstruction), $H:=-\log T\ge 0$ on $\mathcal H_0$.

\paragraph{Consistency and compact calibrator.}
Assume:
\begin{itemize}
  \item (Cons) The defect operators $D_{\varepsilon}:=H I_{\varepsilon}-I_{\varepsilon} H_{\varepsilon}$ satisfy $\varepsilon$-scale graph-norm control: $\|D_{\varepsilon}(H_{\varepsilon}+1)^{-1/2}\|\to 0$.
  \item (Comp) For some nonreal $z_0$, $(H-z_0)^{-1}$ is compact (e.g., finite volume or confining setting).
\end{itemize}

\begin{lemma}[Semigroup comparison implies graph–norm defect]
Suppose there is $C>0$ such that for all $t\in[0,1]$,
\[
  \bigl\|e^{-tH}-I_{\varepsilon}e^{-tH_{\varepsilon}}I_{\varepsilon}^*\bigr\|\ \le\ C t\,\varepsilon\ +\ o(\varepsilon).
\]
Then $\|\,(H I_{\varepsilon}-I_{\varepsilon} H_{\varepsilon})(H_{\varepsilon}+1)^{-1/2}\,\|\to 0$ as $\varepsilon\downarrow 0$.
\end{lemma}

\begin{proof}
Use the standard characterization of generators via Laplace transform of the semigroup and the Hille–Yosida graph–norm: for $\xi\in\operatorname{dom}(H_{\varepsilon})$,
\[
  (H I_{\varepsilon}-I_{\varepsilon} H_{\varepsilon})\xi\ =\ \lim_{t\downarrow 0}\,\frac{1}{t}\bigl[\,(I-e^{-tH})I_{\varepsilon}\xi\ -\ I_{\varepsilon}(I-e^{-tH_{\varepsilon}})\xi\,\bigr],
\]
and bound the difference by the semigroup comparison. The $(H_{\varepsilon}+1)^{-1/2}$ factor stabilizes the domain.
\end{proof}

\paragraph{Resolvent comparison identity (Lean NRC container).}
Let $R(z)=(H-z)^{-1}$, $R_{\varepsilon}(z)=(H_{\varepsilon}-z)^{-1}$, $I_{\varepsilon}$ the embedding and $P_{\varepsilon}:=I_{\varepsilon}I_{\varepsilon}^*$. Define the defect $D_{\varepsilon}:=H I_{\varepsilon}-I_{\varepsilon}H_{\varepsilon}$. Then for each nonreal $z$,
\[
  R(z) - I_{\varepsilon} R_{\varepsilon}(z) I_{\varepsilon}^*
  \ =\ R(z)(I-P_{\varepsilon})\ -\ R(z) D_{\varepsilon} R_{\varepsilon}(z) I_{\varepsilon}^*\,.
\]
This is implemented as a reusable container in the Lean module
\texttt{ym/SpectralStability/NRCEps.lean} as \texttt{NRCSetup.comparison}. The named NRC interface theorem is `YM.SpectralStability.NRC_all_nonreal`.

\begin{lemma}[Compact calibrator in finite volume]
On finite 4D tori (periodic boundary conditions), the transfer $T$ is a compact self–adjoint operator on the OS/GNS space. Hence $(H-z_0)^{-1}$ is compact for any nonreal $z_0$.
\end{lemma}

\begin{proof}
Finite volume yields a separable OS/GNS space with $T$ acting by a positivity–preserving integral kernel on a compact set; standard Hilbert–Schmidt bounds imply compactness of $T$ and thus of the resolvent of $H=-\log T$.
\end{proof}

\paragraph{Calibrator via finite–volume exhaustion (infinite volume).}
Let $\Lambda_L$ be an increasing sequence of periodic 4D tori exhausting $\mathbb R^4$, with transfers $T_L$ and generators $H_L:=-\log T_L$. By the preceding lemma, $(H_L-z_0)^{-1}$ is compact for each $L$. Assume the embeddings $I_{\varepsilon,L}$ and defects $D_{\varepsilon,L}:=H I_{\varepsilon,L}-I_{\varepsilon,L} H_{\varepsilon,L}$ satisfy the graph–norm control uniformly in $L$ and $\varepsilon$:
\[
  \sup_L\big\| D_{\varepsilon,L} (H_{\varepsilon,L}+1)^{-1/2}\big\|\;\xrightarrow[\ \varepsilon\downarrow 0\ ]{}\;0,
\]
and that the projections $P_{\varepsilon,L}:=I_{\varepsilon,L} I_{\varepsilon,L}^*$ converge strongly to $I$ on the infinite–volume OS/GNS space as $L\to\infty$ (for each fixed $\varepsilon$), with this convergence uniform on the low–energy range of $H$. Then the R3 comparison identity yields NRC at each finite $L$; letting $L\to\infty$ and using the thermodynamic–limit compactness of local observables (cf. Theorem~\ref{thm:thermo-strong} and \S\,\ref{sec:lattice-setup}) one obtains NRC in infinite volume.

\begin{theorem}[NRC via finite–volume exhaustion]
Assume (Cons) (graph–norm defect) with bounds uniform in $L$, the strong convergence $P_{\varepsilon,L}\to I$ on the low–energy range of $H$ for each fixed $\varepsilon$, and the fixed–spacing thermodynamic–limit hypotheses of Theorem~\ref{thm:thermo-strong}. Then for every $z\in\mathbb C\setminus\mathbb R$,
\[
  \big\|(H-z)^{-1}-I_{\varepsilon}(H_{\varepsilon}-z)^{-1}I_{\varepsilon}^*\big\|\;\xrightarrow[\ \varepsilon\downarrow 0\ ]{}\;0,
\]
where $I_{\varepsilon}$ is the infinite–volume embedding obtained as the strong limit of $I_{\varepsilon,L}$ along the exhaustion. In particular, NRC holds in infinite volume for all nonreal $z$.
\end{theorem}

\begin{theorem}[NRC and continuum gap]
Suppose (Cons) and (Comp) hold, and the discrete transfer operators have an $\varepsilon$-uniform spectral gap on mean-zero subspaces:
\[
  r_0(T_{\varepsilon})\;\le\;e^{-\gamma_0}\quad\text{with}\quad \gamma_0>0\ \text{independent of }\varepsilon.
\]
Then:
\begin{itemize}
  \item (NRC) For every $z\in\mathbb C\setminus\mathbb R$,
  \[
    \bigl\|(H-z)^{-1}-I_{\varepsilon}(H_{\varepsilon}-z)^{-1}I_{\varepsilon}^*\bigr\|\to 0\quad(\varepsilon\to 0).
  \]
  \item (Continuum gap) On $\mathcal H_0$, $\operatorname{spec}(H)\subset\{0\}\cup[\gamma_0,\infty)$, hence the continuum Hamiltonian has a positive gap $\ge \gamma_0$ and a unique vacuum.
\end{itemize}
\end{theorem}

\begin{proof}
The NRC follows from the comparison identity and bounds of Appendix R3 with $I_{\varepsilon},P_{\varepsilon}$ and the defect control (Cons), plus compact calibration (Comp) to isolate low energies. The uniform spectral gap for $T_{\varepsilon}$ implies a uniform open gap $(0,\gamma_0)$ for $H_{\varepsilon}$. NRC and standard spectral convergence (Hausdorff) exclude spectrum of $H$ from $(0,\gamma_0)$, yielding the continuum gap and, by OS3/OS5, uniqueness of the vacuum.
\end{proof}

\paragraph{Lean artifacts.}
The resolvent comparison is encoded in \texttt{ym/SpectralStability/NRCEps.lean} as an \emph{NRCSetup} with a field \texttt{comparison} that equals the identity above. A norm bound for the NRC difference from this identity is provided in \texttt{ym/SpectralStability/Persistence.lean} (theorem \texttt{nrc\_norm\_bound}). The spectral lower-bound persistence statement is exported there as \texttt{persistence\_lower\_bound} for downstream use.

\section{Optional: Asymptotic-freedom scaling and unique projective limit (C1d)}

We now specify an \emph{asymptotic-freedom (AF) scaling schedule} $\beta(a)$ and prove that along this schedule the projective limit on $\mathbb R^4$ exists with OS0--OS5, is \emph{unique} (no subsequences), and that NRC transports the same uniform lattice gap $\gamma_0$ to the continuum Hamiltonian.

\paragraph{AF schedule.}
Fix $a_0>0$. Choose a monotone function $\beta:(0,a_0]\to (0,\infty)$ such that
\begin{itemize}
  \item[(AF1)] $\beta(a)\ge \beta_{\min}>0$ for all $a\in(0,a_0]$ and $\beta(a)\xrightarrow[a\downarrow 0]{}\infty$;
  \item[(AF2)] choose van Hove volumes $L(a)$ with $L(a)\,a\xrightarrow[a\downarrow 0]{}\infty$;
  \item[(AF3)] use the polygonal loop embeddings $E_a$ and OS/GNS isometries $I_a$ of C1c;
  \item[(AF4)] fix the link-reflection and slab thickness bounded by $a\le a_0$ so that the Doeblin constants $(\kappa_0,t_0)$ are uniform (Prop.~\ref{prop:doeblin-interface}).
\end{itemize}
An explicit example is $\beta(a)=\beta_{\min}+c_0\log(1+a_0/a)$ with $c_0>0$.

\paragraph{Uniform gap along AF.}
By the Doeblin minorization and heat-kernel domination on the interface, the one-step odd-cone deficit is $\beta$-independent:
\[
  c_{\rm cut}\ \ge\ -\frac{1}{a}\log\bigl(1-\kappa_0 e^{-\lambda_1(N) t_0}\bigr),\qquad
  \gamma_0\ \ge\ 8\,c_{\rm cut}\ >\ 0,
\]
uniform in $a\in(0,a_0]$, volume $L(a)$, and $N\ge 2$.

\paragraph{Existence (OS0--OS5) and uniqueness (no subsequences).}
Let $\mu_{a}:=\mu_{\beta(a),L(a)}$ denote the lattice Wilson measures. Then:
\begin{itemize}
  \item OS0/OS2 persist under limits by UEI and positivity closure (C1a/C1b).
  \item OS1 holds in the limit by oriented diagonalization and equicontinuity (C1a).
  \item OS3 holds uniformly on the lattice by the uniform gap $\gamma_0$; it passes to the limit by C1b. OS5 (unique vacuum) follows likewise.
\end{itemize}
To remove subsequences, define for nonreal $z$ the \emph{embedded resolvents}
\[
  R_a(z)\ :=\ I_a\,(H_a-z)^{-1}\,I_a^*\,.
\]
From the comparison identity of R3 and the graph-defect bound $\|D_a(H_a+1)^{-1/2}\|\le C a$ one obtains the quantitative estimate
\begin{equation}
\label{eq:cauchy-res}
  \big\|R_a(z)-R_b(z)\big\|\ \le\ C(z)\,(a+b)\qquad(\forall\ a,b\in(0,a_0])\,.
\end{equation}
Hence $\{R_a(z)\}_{a\downarrow 0}$ is a Cauchy net in operator norm for each nonreal $z$, converging to a \emph{unique} bounded operator $R(z)$ that satisfies the resolvent identities. By the analytic Hille--Phillips theory, $R(z)$ is the resolvent of a unique nonnegative self-adjoint $H$; the embedded semigroups $I_a e^{-tH_a} I_a^*$ converge in operator norm to $e^{-tH}$ for all $t\ge 0$. Therefore the Schwinger functions of $\mu_a$ converge to a unique limit $\{S_n\}$ (no subsequences), defining a probability measure $\mu$ on loop configurations over $\mathbb R^4$ which satisfies OS0--OS5.

\paragraph{AF schedule theorem.}
\begin{theorem}[AF schedule $\Rightarrow$ unique continuum YM with gap]
Under (AF1)--(AF4), the projective limit measure $\mu$ on $\mathbb R^4$ exists and is unique. Its Schwinger functions satisfy OS0--OS5, and the OS reconstruction yields a Hilbert space $\mathcal H$, a vacuum $\Omega$, and a positive self-adjoint generator $H\ge 0$ with
\[
  \operatorname{spec}(H)\subset\{0\}\cup[\gamma_0,\infty),\qquad \gamma_{\mathrm{phys}}\ge \gamma_0>0\,.
\]
\end{theorem}

\begin{proof}
Tightness and OS0/OS2 closure follow from UEI; OS1 from equicontinuity; OS3/OS5 from the uniform lattice gap. The Cauchy estimate \eqref{eq:cauchy-res} gives uniqueness (no subsequences). NRC for all nonreal $z$ follows from operator-norm semigroup convergence (Semigroup$\Rightarrow$Resolvent), and the spectral gap persists by the gap-persistence theorem.
\end{proof}

\section{Appendix: Continuum area law via directed embeddings (C2)}

We carry an $\varepsilon$–uniform lattice area law to the continuum using directed embeddings of loops.

\paragraph{Uniform lattice area law.}
Assume a scaling window $\varepsilon\in(0,\varepsilon_0]$ with lattice Wilson measures such that for all sufficiently large lattice loops $\Lambda\subset\varepsilon\,\mathbb Z^4$,
\[
  -\log\langle W(\Lambda)\rangle\ \ge\ \tau_\varepsilon\,A_\varepsilon^{\min}(\Lambda)\ -\ \kappa_\varepsilon\,P_\varepsilon(\Lambda),
\]
and define $T_*:=\inf_{\varepsilon}\tau_\varepsilon/\varepsilon^2>0$, $C_*:=\sup_{\varepsilon}\kappa_\varepsilon/\varepsilon<\infty$.

\paragraph{Directed embeddings.}
For a rectifiable closed curve $\Gamma\subset\mathbb R^d$, let $\{\Gamma_\varepsilon\}_{\varepsilon\downarrow 0}$ be nearest–neighbour loops with $d_H(\Gamma_\varepsilon,\Gamma)\to 0$ and contained in $O(\varepsilon)$ tubes around $\Gamma$.

\begin{theorem}[Continuum Area–Perimeter bound]
With $\kappa_d:=\sup_{u\in\mathbb S^{d-1}}\sum_i |u_i|=\sqrt d$ and $C:=\kappa_d C_*$, for any directed family $\Gamma_\varepsilon\to\Gamma$,
\[
  \limsup_{\varepsilon\downarrow 0}\bigl[-\log\langle W(\Gamma_\varepsilon)\rangle\bigr]\ \ge\ T_*\,\operatorname{Area}(\Gamma)\ -\ C\,\operatorname{Perimeter}(\Gamma).
\]
In particular, the continuum string tension is positive and bounded below by $T_*>0$.
\end{theorem}

\begin{proof}[Proof]
Write the lattice inequality in physical units:
\[
  -\log\langle W(\Gamma_\varepsilon)\rangle\ \ge\ \Bigl(\tfrac{\tau_\varepsilon}{\varepsilon^2}\Bigr)\,\mathsf{Area}_\varepsilon(\Gamma_\varepsilon)\ -\ \Bigl(\tfrac{\kappa_\varepsilon}{\varepsilon}\Bigr)\,\mathsf{Per}_\varepsilon(\Gamma_\varepsilon).
\]
Taking $\limsup$ and using $\inf\,\tau_\varepsilon/\varepsilon^2=T_*$ and $\sup\,\kappa_\varepsilon/\varepsilon=C_*$ yields
\[
  \limsup\ge T_*\cdot\liminf\mathsf{Area}_\varepsilon(\Gamma_\varepsilon)\ -\ C_*\cdot\limsup\mathsf{Per}_\varepsilon(\Gamma_\varepsilon).
\]
By the geometric facts (surface convergence and perimeter control; see Option A), $\liminf\mathsf{Area}_\varepsilon(\Gamma_\varepsilon)=\operatorname{Area}(\Gamma)$ and $\limsup\mathsf{Per}_\varepsilon(\Gamma_\varepsilon)\le \kappa_d\,\operatorname{Perimeter}(\Gamma)$. Combine to obtain the stated bound with $C=\kappa_d C_*$.\qed
\end{proof}

\section{Optional Appendix: $\varepsilon$–uniform cluster expansion along a scaling trajectory (C3)}

\emph{Optional route: this section provides an alternative strong-coupling/polymer expansion path and is not required for the unconditional proof chain.}

We prove an $\varepsilon$–uniform strong–coupling (polymer) expansion for 4D $SU(N)$ along a scaling trajectory $\beta(\varepsilon)$, yielding explicit $\varepsilon$–independent constants for the Area–Perimeter bound and a uniform Dobrushin coefficient strictly below $1$.

\paragraph{Set–up.}
Work on 4D tori with lattice spacing $\varepsilon\in(0,\varepsilon_0]$. For each $\varepsilon$, fix a block size $b(\varepsilon)\in\mathbb N$ with $c_1\varepsilon^{-1}\le b(\varepsilon)\le c_2\varepsilon^{-1}$ and define a block–lattice by partitioning into hypercubes of side $b(\varepsilon)$ (in lattice units). Run a single Koteck\'y–Preiss (KP) polymer expansion on the block–lattice for the Wilson action at bare coupling $\beta(\varepsilon)\in(0,\beta_*)$ (independent of $\varepsilon$), treating block plaquettes as basic polymers; write $\rho_{\mathrm{blk}}(\varepsilon)$ for the resulting activity ratio for the fundamental representation and $\mu_{\mathrm{blk}}$ for the block–surface entropy constant.

\paragraph{Uniform KP/cluster expansion (full proof).}
Fix $\varepsilon\in(0,\varepsilon_0]$ and choose a block scale $b(\varepsilon)\asymp \varepsilon^{-1}$. Group plaquettes into block–plaquettes (faces of side $b(\varepsilon)$ in lattice units). Expand the Wilson weight on each block–plaquette in irreducible characters and polymerize along block–faces. Koteck\'y–Preiss applies provided the activity $\rho_{\mathrm{blk}}(\varepsilon)$ of the fundamental representation and the block entropy $\mu_{\mathrm{blk}}$ satisfy $\mu_{\mathrm{blk}}\,\rho_{\mathrm{blk}}(\varepsilon) < 1$; for small $\beta(\varepsilon)$ this holds uniformly with a slack $\delta\in(0,1)$ independent of $\varepsilon$ and $N\ge2$. Boundary attachments contribute a multiplicity factor $m_{\mathrm{blk}}$ per block boundary unit (uniform in $\varepsilon,N$). Summing over excess block area $k\ge0$ yields the convergent geometric series
\[
  \sum_{k\ge 0} N_{\mathrm{blk}}(\Gamma,A+k)\,\rho_{\mathrm{blk}}(\varepsilon)^{A+k}
   \ \le\ m_{\mathrm{blk}}^{P_{\mathrm{blk}}}\,\frac{\rho_{\mathrm{blk}}(\varepsilon)^{A}}{\delta},
\]
where $A$ is the minimal block spanning area and $P_{\mathrm{blk}}$ the block perimeter. Taking $-\log$ and converting to physical units (each block area $\asymp 1$, each block boundary length $\asymp 1$) gives
\[
  -\log\langle W(\Lambda)\rangle\ \ge\ T_*\,\mathsf{Area}_\varepsilon(\Lambda)\ -\ C_*\,\mathsf{Per}_\varepsilon(\Lambda),
\]
with
\[
  T_*:= -\log \rho_{\max},\quad \rho_{\max}:=\sup_{0<\varepsilon\le\varepsilon_0}\rho_{\mathrm{blk}}(\varepsilon)<1,\qquad
  C_*:= \log m_{\mathrm{blk}}+\log(1/\delta)<\infty.
\]
Moreover, the one–step cross–cut Dobrushin coefficient at block scale obeys
\[
  \alpha\bigl(\beta(\varepsilon)\bigr)\ \le\ 2\,\beta(\varepsilon)\,J^{\mathrm{blk}}_{\perp}(\varepsilon)
   \ \le\ 2\,\beta_*\,J^{\mathrm{blk}}_{\perp,\max}=:\alpha_*<1,
\]
where $J^{\mathrm{blk}}_{\perp,\max}$ is a geometry–only bound (independent of $\varepsilon,N$). All constants are $\varepsilon$– and $N$–uniform.

\paragraph{Optional scaffold (KP from Wilson; hypothesis bundle).}
\emph{(H-KP).} For 4D SU($N$) Wilson action at sufficiently small $\beta$, the block polymer expansion at scale $b(\varepsilon)\asymp \varepsilon^{-1}$ satisfies: (i) $\rho_{\mathrm{blk}}(\varepsilon)\le \rho_{\max}<1$, (ii) $\mu_{\mathrm{blk}}\,\rho_{\mathrm{blk}}\le 1-\delta$ with $\delta\in(0,1)$, (iii) boundary multiplicity $m_{\mathrm{blk}}\le m_0$, all independent of $\varepsilon$ and $N$. \emph{Conclusion.} The constants $T_*=-\log\rho_{\max}>0$, $C_* = \log m_0 + \log(1/\delta)$, and $\alpha_*\le 2\beta_* J^{\mathrm{blk}}_{\perp,\max}<1$ follow, yielding the uniform area–perimeter law and contraction.

\begin{theorem}[Uniform KP/cluster expansion with explicit constants]
\label{thm:uniform-kp}
Under the hypotheses above, define the explicit $\varepsilon$–independent constants
\[
  \rho_{\max}\;:=\;\sup_{0<\varepsilon\le \varepsilon_0}\rho_{\mathrm{blk}}(\varepsilon)\ <\ 1,\quad
  T_*\;:=\; -\log \rho_{\max}\ >\ 0,\quad
  C_*\;:=\; \log m_{\mathrm{blk}}\ +\ \log\tfrac{1}{\delta}\ <\ \infty,
\]
\[
  J^{\mathrm{blk}}_{\perp,\max}\;:=\;\sup_{0<\varepsilon\le\varepsilon_0} J^{\mathrm{blk}}_{\perp}(\varepsilon)\ <\ \infty,\qquad
  \alpha_*\;:=\;2\,\beta_*\,J^{\mathrm{blk}}_{\perp,\max}\ <\ 1\,.
\]
Then for all sufficiently large loops $\Lambda\subset\varepsilon\,\mathbb Z^4$ and all $\varepsilon\in(0,\varepsilon_0]$:
\begin{align}
  -\log\langle W(\Lambda)\rangle\ &\ge\ \tau_\varepsilon\,A_\varepsilon^{\min}(\Lambda)\ -\ \kappa_\varepsilon\,P_\varepsilon(\Lambda),\\
  \frac{\tau_\varepsilon}{\varepsilon^2}\ &\ge\ T_*,\qquad \frac{\kappa_\varepsilon}{\varepsilon}\ \le\ C_*,\\
  \alpha\bigl(\beta(\varepsilon)\bigr)\ &\le\ \alpha_*\ <\ 1\,.
\end{align}
In particular, $T_*$ is a uniform string–tension lower bound in physical units, $C_*$ a uniform perimeter coefficient (physical units), and $\alpha_*$ a uniform upper bound for the cross–cut Dobrushin coefficient.
\end{theorem}

\begin{proof}
Run the Koteck\'y–Preiss expansion at block scale $b(\varepsilon)\asymp \varepsilon^{-1}$. For any fixed lattice loop $\Gamma$, block–surfaces $\Sigma_{\mathrm{blk}}$ with $\partial\Sigma_{\mathrm{blk}}=\Gamma$ decompose the minimal area into units of $b(\varepsilon)^2$. The KP bounds give a convergent geometric series for the sum over excess area $k\ge 0$ with ratio bounded by $\mu_{\mathrm{blk}}\,\rho_{\mathrm{blk}}(\varepsilon)\le 1-\delta$ and boundary multiplicity bounded by $m_{\mathrm{blk}}^{P_{\mathrm{blk}}}$:
\[
  \sum_{k\ge 0} N_{\mathrm{blk}}(\Gamma,A+k)\,\rho_{\mathrm{blk}}(\varepsilon)^{A+k}
  \ \le\ m_{\mathrm{blk}}^{P_{\mathrm{blk}}}\,\rho_{\mathrm{blk}}(\varepsilon)^A\,\sum_{k\ge 0}\bigl(\mu_{\mathrm{blk}}\,\rho_{\mathrm{blk}}(\varepsilon)\bigr)^{k}
  \ \le\ m_{\mathrm{blk}}^{P_{\mathrm{blk}}}\,\frac{\rho_{\mathrm{blk}}(\varepsilon)^A}{\delta}\,.
\]
Taking $-\log$ and using $\rho_{\mathrm{blk}}(\varepsilon)\le \rho_{\max}<1$ yields a block–scale string tension
\[
  T_{\mathrm{blk}}(\varepsilon)\ :=\ -\log \rho_{\mathrm{blk}}(\varepsilon)\ \ge\ -\log \rho_{\max}\ =\ T_*\ >\ 0\,.
\]
Converting to physical units: each block plaquette has area $\bigl(b(\varepsilon)\,\varepsilon\bigr)^2\asymp 1$, hence $\tau_\varepsilon/\varepsilon^2\ge T_*$. A block boundary unit corresponds to physical length $b(\varepsilon)\,\varepsilon\asymp 1$, and the perimeter multiplicative factors contribute at most $\log m_{\mathrm{blk}}+\log(1/\delta)=C_*$ to $-\log\langle W\rangle$ per boundary unit, giving $\kappa_\varepsilon/\varepsilon\le C_*$. Finally, the cross–cut one–step total–variation influence at block scale is bounded by $2\,\beta(\varepsilon)\,J^{\mathrm{blk}}_{\perp}(\varepsilon)$; with $\beta(\varepsilon)\le \beta_*$ this yields $\alpha\le \alpha_*:=2\beta_* J^{\mathrm{blk}}_{\perp,\max}<1$ uniformly in $\varepsilon$.
\end{proof}

\begin{corollary}[Uniform lattice Area--Perimeter (physical units)]
For all $\varepsilon\in(0,\varepsilon_0]$ in the KP window and all sufficiently large lattice loops $\Lambda\subset\varepsilon\,\mathbb Z^4$,
\[
  -\log\langle W(\Lambda)\rangle\ \ge\ T_*\,\mathsf{Area}_\varepsilon(\Lambda)\ -\ C_*\,\mathsf{Per}_\varepsilon(\Lambda).
\]
\end{corollary}

\begin{proof}
From the inequality $-\log\langle W(\Lambda)\rangle\ge (\tau_\varepsilon/\varepsilon^2)\,\mathsf{Area}_\varepsilon(\Lambda)-(\kappa_\varepsilon/\varepsilon)\,\mathsf{Per}_\varepsilon(\Lambda)$ and the bounds $\tau_\varepsilon/\varepsilon^2\ge T_*$, $\kappa_\varepsilon/\varepsilon\le C_*$.
\end{proof}

\paragraph{Remarks.}
1. The constants $T_*,C_*,\alpha_*$ are independent of $\varepsilon$ and of $N\ge 2$; only local block geometry and the KP window parameters $(\beta_*,\delta,C_{\mathrm{blk}})$ enter.\\
2. The block scale choice $b(\varepsilon)\asymp\varepsilon^{-1}$ is tailored so that block plaquette area and boundary length are $\Theta(1)$ in physical units, enabling direct extraction of $T_*$ and $C_*$.\\
3. The Dobrushin step is formulated at the block scale (OS time–slice thickened to $\Theta(1)$ physical time); see Appendix C4 for the conversion to a uniform transfer gap.

\section{Optional Appendix: $\varepsilon$–uniform spectral gap from cross–cut couplings (C4)}

\emph{Optional route: this section continues the KP/cluster expansion approach from C3 and is not required for the unconditional proof chain.}

We derive an $\varepsilon$–uniform spectral gap for the transfer operator on the mean–zero sector directly from the cross–cut Dobrushin coefficient and the block–scale cross–cut coupling.

\paragraph{Cross–cut influence.}
For each $\varepsilon$, let $J_{\perp}(\varepsilon)$ denote the total coupling across the OS reflection cut (sum of $J_{xy}$ with $x$ on the left, $y$ on the right). In the strong–coupling/cluster regime one has the one–step contraction bound
\[
  \alpha\bigl(\beta(\varepsilon)\bigr)\;\le\;2\,\beta(\varepsilon)\,J_{\perp}(\varepsilon).
\]

\paragraph{Block–geometry computation of $J_{\perp}(\varepsilon)$ (geometry–only bound).}
Adopt the block scale $b(\varepsilon)\asymp\varepsilon^{-1}$ from C3 and thicken one OS time–slice to a slab of physical thickness $O(1)$. The cross–cut influence is supported on plaquettes intersecting the reflection plane and inside the slab. Let $n_{\text{cut}}$ be the maximal number of such plaquettes touching a single block cross–section, a constant depending only on the $4$D hypercubic geometry (hence independent of $\varepsilon$ and $N$). Bounding the single–plaquette two–link influence by a representation–independent constant $J_{\text{unit}}$, we obtain the uniform bound
\[
  J^{\text{blk}}_{\perp}(\varepsilon)\ \le\ n_{\text{cut}}\,J_{\text{unit}}\ =:\ J^{\text{blk}}_{\perp,\max},
\]
so $\sup_{\varepsilon} J^{\mathrm{blk}}_{\perp}(\varepsilon)\le J^{\mathrm{blk}}_{\perp,\max}<\infty$ depends only on local block geometry. Consequently,
\[
  \alpha\bigl(\beta(\varepsilon)\bigr)\ \le\ 2\,\beta(\varepsilon)\,J^{\mathrm{blk}}_{\perp}(\varepsilon)\ \le\ 2\,\beta_*\,J^{\mathrm{blk}}_{\perp,\max}\ =:\ \alpha_*\ <\ 1,
\]
and the uniform mass gap parameter is $\gamma_0:=-\log \alpha_*>0$ (per block time–slice).

\begin{theorem}[Uniform transfer contraction and gap]
Assume $\sup_{\varepsilon}\beta(\varepsilon) J_{\perp}(\varepsilon)\le c<\tfrac12$. Then with $\alpha_*:=2c<1$ and $\gamma_0:=-\log\alpha_*>0$,
\[
  r_0\bigl(T_{\varepsilon}\bigr)\;\le\;\alpha_*\ =\ e^{-\gamma_0}\qquad\text{for all }\varepsilon\in(0,\varepsilon_0],
\]
and hence the lattice Hamiltonian gap $\Delta_{\varepsilon}=-\log r_0(T_{\varepsilon})\ge \gamma_0$ uniformly in $\varepsilon$.
\end{theorem}

\begin{proof}[Proof]
Work on the OS/GNS space and let $\operatorname{osc}(f):=\max_{i,j}|f_i-f_j|$ denote the oscillation seminorm on a finite reflected loop basis. The Dobrushin coefficient bound at block scale gives a one–step contraction
\[
  \operatorname{osc}(T_{\varepsilon} f)\ \le\ \alpha_*\,\operatorname{osc}(f),\qquad \alpha_*<1.
\]
By the oscillation–to–spectrum lemma (Appendix "Dobrushin contraction and spectrum" and its abstract operator form), every non-constant eigenvector of $T_{\varepsilon}$ has eigenvalue modulus $\le \alpha_*$. Hence on the mean–zero/oscillation sector one has $r_0(T_{\varepsilon})\le \alpha_*$. Taking $-\log$ yields $\Delta_{\varepsilon}\ge -\log\alpha_*=:\gamma_0>0$. The constants are uniform by the KP window and depend only on cross–cut geometry via $J_{\perp}$ and the bound $c$.
\end{proof}

\paragraph{Lean artifact.}
The uniform Dobrushin$\to$gap bound is exported as\newline
\texttt{YM.Transfer.uniform\_gap\_of\_alpha}, and a window-specialized wrapper
\texttt{YM.Transfer.uniform\_gap\_on\_window} which yields $r_0(T_\varepsilon)\le \alpha_*$
and $\gamma_0=-\log\alpha_*>0$ on the oscillation sector uniformly in $\varepsilon$.

\paragraph{Physical units.}
If the OS time–step is chosen at block scale with thickness $b(\varepsilon)\,\varepsilon\asymp 1$ (as in C3), then one time–slice corresponds to $O(1)$ physical time and the physical mass gap satisfies $\gamma_{\mathrm{phys}}\ge \gamma_0$ uniformly in $\varepsilon$.

\section{Appendix: Universality across coarse–grainings (C5)}

We show that two admissible coarse–grainings along the same scaling window yield the same continuum Schwinger functions and the same physical mass gap.

\paragraph{Two regularizations.}
For $a\in\{1,2\}$, let $\mu^{(a)}_{\varepsilon}$ be OS–positive lattice measures (Wilson action) with trajectories $\beta^{(a)}(\varepsilon)$ and embeddings $I^{(a)}_{\varepsilon}$ as in C1a–C1c. Assume for both $a$:
\begin{itemize}
  \item (Loc/Mom) $\varepsilon$–uniform locality/moment bounds and hypercubic invariance (C1a).
  \item (NRC) Norm–resolvent convergence to generators $H^{(a)}$ with compact calibrator (C1c).
  \item (Gap) $\sup_{\varepsilon}\beta^{(a)}(\varepsilon) J^{(a)}_{\perp}(\varepsilon)\le c<\tfrac12$ (C4), hence a uniform gap $\gamma_0>0$ for each.
\end{itemize}
Additionally, assume a uniform consistency between the two discretizations:
\begin{itemize}
  \item (Cons*) There exists $\delta(\varepsilon)\downarrow 0$ such that for any finite loop family $\{\Gamma_i\}$ and equivariant embeddings,
  \[
    \bigl|\,S^{(1)}_{n,\varepsilon}(\Gamma_{1,\varepsilon},\dots,\Gamma_{n,\varepsilon})-S^{(2)}_{n,\varepsilon}(\Gamma_{1,\varepsilon},\dots,\Gamma_{n,\varepsilon})\,\bigr|\ \le\ C_n\,\delta(\varepsilon),
  \]
  with $C_n$ independent of $\varepsilon$.
\end{itemize}

\begin{theorem}[Universality]
Under the assumptions above (uniform locality/moments, (Cons*) with $\delta(\varepsilon)\downarrow 0$), and assuming NRC for both families and $\varepsilon$–uniform gaps $\gamma_0>0$, along any common subsequence $\varepsilon_k\to 0$ the Schwinger functions of $\mu^{(1)}_{\varepsilon_k}$ and $\mu^{(2)}_{\varepsilon_k}$ converge to the same limit $\{S_n\}$ (hence define the same continuum measure $\mu$). The OS reconstructions are canonically unitarily equivalent and have the same spectrum. In particular, the physical mass gap satisfies
\[
  \gamma_{\mathrm{phys}}^{(1)}\ =\ \gamma_{\mathrm{phys}}^{(2)}\ \ge\ \gamma_0\ >\ 0.
\]
\end{theorem}

\begin{proof}[Full proof of universality]
\emph{Step 1: Tightness from uniform locality.} By assumption (Loc/Mom), for each $n \ge 1$, there exist constants $C_n > 0$ and exponents $p_n, q_n > 0$ such that for all $\varepsilon \in (0, \varepsilon_0]$ and both families $a = 1, 2$,
\[
  |S_{n,\varepsilon}^{(a)}(\Gamma_1, \ldots, \Gamma_n)| \le C_n \prod_{i=1}^n (1 + \operatorname{diam}(\Gamma_i))^{p_n} \prod_{1 \le i < j \le n} (1 + \operatorname{dist}(\Gamma_i, \Gamma_j))^{-q_n}.
\]
By the tree-graph bounds (Brydges \cite{Brydges1986}), these polynomial bounds imply tightness of the sequence of measures $\{\mu_{\varepsilon}^{(a)}\}_{\varepsilon > 0}$ in the weak topology on loop distributions.

\emph{Step 2: Convergence of finite-dimensional distributions.} Along any subsequence $\varepsilon_k \to 0$, by tightness there exist further subsequences (which we still denote $\varepsilon_k$) such that $\mu_{\varepsilon_k}^{(1)} \Rightarrow \mu^{(1)}$ and $\mu_{\varepsilon_k}^{(2)} \Rightarrow \mu^{(2)}$ weakly. This implies convergence of all finite $n$-point Schwinger functions:
\[
  S_{n,\varepsilon_k}^{(a)}(\Gamma_1, \ldots, \Gamma_n) \to S_n^{(a)}(\Gamma_1, \ldots, \Gamma_n), \quad a = 1, 2.
\]

\emph{Step 3: Identical limits via consistency condition.} By assumption (Cons*), for all $n$ and loop families,
\[
  |S_{n,\varepsilon}^{(1)}(\Gamma_1, \ldots, \Gamma_n) - S_{n,\varepsilon}^{(2)}(\Gamma_1, \ldots, \Gamma_n)| \le \delta(\varepsilon) \cdot \prod_{i=1}^n (1 + \operatorname{diam}(\Gamma_i))^{p_n},
\]
where $\delta(\varepsilon) \to 0$ as $\varepsilon \to 0$. Taking the limit along $\varepsilon_k \to 0$, we obtain
\[
  S_n^{(1)}(\Gamma_1, \ldots, \Gamma_n) = S_n^{(2)}(\Gamma_1, \ldots, \Gamma_n)
\]
for all $n$ and all loop families. Hence $\mu^{(1)} = \mu^{(2)} =: \mu$ and the Schwinger functions coincide: $\{S_n^{(1)}\} = \{S_n^{(2)}\} = \{S_n\}$.

\emph{Step 4: NRC and resolvent convergence.} By the NRC assumption, for both families $a = 1, 2$ and all nonreal $z \in \mathbb{C} \setminus \mathbb{R}$,
\[
  \lim_{k \to \infty} \|(H_{\varepsilon_k}^{(a)} - z)^{-1} - I_{\varepsilon_k}^* (H^{(a)} - z)^{-1} I_{\varepsilon_k}\|_{\mathcal{B}(\mathcal{H}_{\varepsilon_k})} = 0,
\]
where $H_{\varepsilon_k}^{(a)}$ are the discrete Hamiltonians, $H^{(a)}$ are the continuum limits, and $I_{\varepsilon_k}$ are the embedding operators.

\emph{Step 5: Unitary equivalence via GNS reconstruction.} The OS reconstruction theorem (Klein-Landau \cite{KleinLandau1975}) establishes that equal Schwinger functions $\{S_n\}$ lead to unitarily equivalent GNS Hilbert spaces and Hamiltonians. Specifically, there exists a unitary operator $U: \mathcal{H}^{(1)} \to \mathcal{H}^{(2)}$ such that
\[
  U H^{(1)} U^* = H^{(2)}, \quad U \Omega^{(1)} = \Omega^{(2)},
\]
where $\Omega^{(a)}$ are the vacuum vectors. This follows from the fact that the GNS construction is determined entirely by the state (encoded in the Schwinger functions) and the algebra of observables.

\emph{Step 6: Spectrum preservation and mass gap.} Since $U$ is unitary, $\operatorname{spec}(H^{(1)}) = \operatorname{spec}(H^{(2)})$. By assumption (C4), the uniform gap condition gives $\operatorname{spec}(H_{\varepsilon}^{(a)}) \subset \{0\} \cup [\gamma_0, \infty)$ for all $\varepsilon \in (0, \varepsilon_0]$ and both $a = 1, 2$. By Theorem~\ref{thm:gap-persist} (gap persistence via NRC), this spectral structure persists in the limit:
\[
  \operatorname{spec}(H^{(a)}) \subset \{0\} \cup [\gamma_0, \infty), \quad a = 1, 2.
\]
Therefore, the physical mass gaps coincide:
\[
  \gamma_{\mathrm{phys}}^{(1)} = \inf(\operatorname{spec}(H^{(1)}) \setminus \{0\}) = \inf(\operatorname{spec}(H^{(2)}) \setminus \{0\}) = \gamma_{\mathrm{phys}}^{(2)} \ge \gamma_0 > 0.
\]

\emph{Step 7: Conclusion.} We have shown that along any convergent subsequence, both lattice families converge to the same continuum theory with identical Schwinger functions, unitarily equivalent Hamiltonians, and the same mass gap bounded below by $\gamma_0$. Since the limit is unique, the full sequences converge.
\end{proof}

\paragraph{Lean artifact.}
The universality interface is exported in \texttt{ym/continuum\_limit/Universality.lean} as
\texttt{YM.ContinuumLimit.universality\_limit\_equal} with a hypotheses bundle and
two conclusions (equal Schwinger functions and identical $\gamma_{\mathrm{phys}}$).

\begin{lemma}[OS0 (regularity/temperedness) in the limit]
If, in addition, the uniform locality implies polynomial bounds on $n$-point functions with respect to loop diameters and separations, then the limit Schwinger functions $\{S_n\}$ are tempered distributions (OS0).
\end{lemma}

\begin{proof}[Full proof of temperedness]
\emph{Step 1: Polynomial bound structure.} By hypothesis, there exist constants $C_n$, exponents $p > 0$ and $q > d$, such that for all $\varepsilon > 0$ and loop families $\{\Gamma_{i,\varepsilon}\}_{i=1}^n$,
\[
  |S_{n,\varepsilon}(\Gamma_{1,\varepsilon}, \ldots, \Gamma_{n,\varepsilon})| \le C_n \prod_{i=1}^n (1 + \operatorname{diam}(\Gamma_{i,\varepsilon}))^p \prod_{1 \le i < j \le n} (1 + \operatorname{dist}(\Gamma_{i,\varepsilon}, \Gamma_{j,\varepsilon}))^{-q}.
\]

\emph{Step 2: Convergence preserves bounds.} Along any convergent subsequence $\varepsilon_k \to 0$ with $S_{n,\varepsilon_k} \to S_n$ (weak convergence of measures), the polynomial bound passes to the limit. For any continuous test functions $f_1, \ldots, f_n$ with compact supports $K_i = \operatorname{supp}(f_i)$,
\[
  \lim_{k \to \infty} S_{n,\varepsilon_k}(\Gamma_{1,\varepsilon_k}, \ldots, \Gamma_{n,\varepsilon_k}) = S_n(\Gamma_1, \ldots, \Gamma_n)
\]
whenever $\Gamma_{i,\varepsilon_k} \to \Gamma_i$ in Hausdorff distance.

\emph{Step 3: Test function parametrization.} For test functions $f_i: \mathbb{R}^4 \to \mathbb{C}$ and loops $\Gamma_i$, define the smeared loop functional
\[
  S_n[f_1, \ldots, f_n] := \int_{\mathbb{R}^{4n}} S_n(\Gamma_1 + x_1, \ldots, \Gamma_n + x_n) \prod_{i=1}^n f_i(x_i) \, dx_i.
\]
Here $\Gamma_i + x_i$ denotes the loop $\Gamma_i$ translated by $x_i$.

\emph{Step 4: Bound in terms of test functions.} Using the polynomial bound from Step 1 and the translation invariance of the measure,
\begin{align}
  |S_n[f_1, \ldots, f_n]| &\le \int_{\mathbb{R}^{4n}} C_n \prod_{i=1}^n (1 + \operatorname{diam}(\Gamma_i))^p |f_i(x_i)| \\
  &\quad \times \prod_{1 \le i < j \le n} (1 + \operatorname{dist}(\Gamma_i + x_i, \Gamma_j + x_j))^{-q} \, dx_1 \cdots dx_n.
\end{align}

\emph{Step 5: Distance bound.} Note that $\operatorname{dist}(\Gamma_i + x_i, \Gamma_j + x_j) \ge |x_i - x_j| - \operatorname{diam}(\Gamma_i) - \operatorname{diam}(\Gamma_j)$. When $|x_i - x_j| > 2(\operatorname{diam}(\Gamma_i) + \operatorname{diam}(\Gamma_j))$, we have
\[
  \operatorname{dist}(\Gamma_i + x_i, \Gamma_j + x_j) \ge \frac{|x_i - x_j|}{2}.
\]

\emph{Step 6: Integrability.} Split the integration domain into regions where pairs $(i,j)$ satisfy $|x_i - x_j| \le R_{ij} := 2(\operatorname{diam}(\Gamma_i) + \operatorname{diam}(\Gamma_j))$ and its complement. The integral over the complement is bounded by
\[
  \prod_{i=1}^n \|f_i\|_{L^1} \cdot \prod_{i=1}^n (1 + \operatorname{diam}(\Gamma_i))^p \cdot \prod_{1 \le i < j \le n} \int_{|y| > R_{ij}} (1 + |y|/2)^{-q} \, dy.
\]
Since $q > d = 4$, the integral $\int_{|y| > R} (1 + |y|)^{-q} \, dy$ converges.

\emph{Step 7: Schwartz space estimate.} For Schwartz test functions $f_i \in \mathcal{S}(\mathbb{R}^4)$, we can absorb the polynomial growth in loop diameters into the rapid decay of the test functions. Specifically, for any multi-index $\alpha$ and integer $N > 0$,
\[
  |S_n[f_1, \ldots, f_n]| \le C_{n,\alpha,N} \prod_{i=1}^n \sup_{x \in \mathbb{R}^4} (1 + |x|)^N |D^\alpha f_i(x)|,
\]
where $C_{n,\alpha,N}$ depends on $C_n$, the loop geometries, and the exponents $p, q$.

\emph{Step 8: Conclusion.} The estimate in Step 7 shows that $S_n$ extends to a continuous linear functional on $\mathcal{S}(\mathbb{R}^4)^{\otimes n}$, hence defines a tempered distribution. This establishes OS0 (regularity/temperedness) for the limit Schwinger functions.
\end{proof}

\paragraph{Explicit uniform polynomial bounds (temperedness with constants).}
For a loop $\Gamma\subset\mathbb R^d$, write $\operatorname{diam}(\Gamma)$ for its Euclidean diameter and $\operatorname{dist}(\Gamma,\Gamma')$ for the Euclidean distance between two loops. Let $\kappa_d:=\sup_{u\in\mathbb S^{d-1}}\sum_i |u_i|=\sqrt d$.

\begin{lemma}[Uniform polynomial bounds]
Assume a uniform exponential clustering of truncated correlations: there exist $C_0\ge 1$ and $m>0$ such that for all $n\ge 2$, $\varepsilon\in(0,\varepsilon_0]$, and loops $\Gamma_{1,\varepsilon},\dots,\Gamma_{n,\varepsilon}$,
\[
  |\kappa_{n,\varepsilon}(\Gamma_{1,\varepsilon},\dots,\Gamma_{n,\varepsilon})|\ \le\ C_0^n\,\sum_{\text{trees }\tau}\ \prod_{(i,j)\in E(\tau)} e^{-m\,\operatorname{dist}(\Gamma_{i,\varepsilon},\Gamma_{j,\varepsilon})}.
\]
Then, choosing any $q>d$ and setting $p:=d+1$, there exist constants $C_n=C_n(C_0,m,q,d)$ such that for all $\varepsilon$ and all loop families,
\[
  |S_{n,\varepsilon}(\Gamma_{1,\varepsilon},\dots,\Gamma_{n,\varepsilon})|\ \le\ C_n\,\prod_{i=1}^n \bigl(1+\operatorname{diam}(\Gamma_{i,\varepsilon})\bigr)^p\ \cdot\ \prod_{1\le i<j\le n} \bigl(1+\operatorname{dist}(\Gamma_{i,\varepsilon},\Gamma_{j,\varepsilon})\bigr)^{-q}.
\]
Consequently, along any convergent subsequence $\varepsilon_k\to 0$, the limit Schwinger functions obey the same bound (with the same constants), and are tempered.
\end{lemma}

\begin{proof}[Full proof of uniform polynomial bounds]
\emph{Step 1: Tree-graph expansion.} By the Brydges tree-graph bound (Brydges \cite{Brydges1978}), the $n$-point Schwinger function can be expanded as
\[
  S_{n,\varepsilon}(\Gamma_{1,\varepsilon}, \ldots, \Gamma_{n,\varepsilon}) = \sum_{\tau \in \mathcal{T}_n} \prod_{(i,j) \in E(\tau)} \kappa_{ij,\varepsilon}(\Gamma_{i,\varepsilon}, \Gamma_{j,\varepsilon}),
\]
where $\mathcal{T}_n$ is the set of labeled spanning trees on $n$ vertices, and $\kappa_{ij,\varepsilon}$ denotes the truncated two-point correlation function. The number of such trees is $|\mathcal{T}_n| = n^{n-2}$ by Cayley's formula.

\emph{Step 2: Insert exponential clustering.} By hypothesis, each truncated correlator satisfies
\[
  |\kappa_{ij,\varepsilon}(\Gamma_{i,\varepsilon}, \Gamma_{j,\varepsilon})| \le C_0^2 e^{-m \operatorname{dist}(\Gamma_{i,\varepsilon}, \Gamma_{j,\varepsilon})}.
\]
For a tree $\tau$ with $n-1$ edges, the product over edges gives
\[
  \left| \prod_{(i,j) \in E(\tau)} \kappa_{ij,\varepsilon}(\Gamma_{i,\varepsilon}, \Gamma_{j,\varepsilon}) \right| \le C_0^{2(n-1)} \prod_{(i,j) \in E(\tau)} e^{-m \operatorname{dist}(\Gamma_{i,\varepsilon}, \Gamma_{j,\varepsilon})}.
\]

\emph{Step 3: Convert exponential decay to polynomial.} For any $q > 0$ and $r \ge 0$, we have the inequality
\[
  e^{-m r} \le \frac{C(q,m)}{(1 + r)^q}, \quad \text{where } C(q,m) = \sup_{r \ge 0} \left[(1 + r)^q e^{-m r}\right] = \left(\frac{q}{em}\right)^q.
\]
This supremum is achieved at $r = q/m - 1$ when $q > m$, and at $r = 0$ when $q \le m$. In either case, $C(q,m) < \infty$.

\emph{Step 4: Diameter dependence.} Each loop $\Gamma_{i,\varepsilon}$ contributes to the bound through its support in test function integrals. The natural scale is set by $\operatorname{diam}(\Gamma_{i,\varepsilon})$, leading to a factor $(1 + \operatorname{diam}(\Gamma_{i,\varepsilon}))^p$ where $p$ depends on the dimension and the order of derivatives in the test function space. For Schwartz test functions in $\mathbb{R}^d$, the minimal choice is $p = d + 1$.

\emph{Step 5: Summing over trees.} Using the polynomial bound from Step 3 with our choice of $q > d$,
\begin{align}
  |S_{n,\varepsilon}(\Gamma_{1,\varepsilon}, \ldots, \Gamma_{n,\varepsilon})| &\le \sum_{\tau \in \mathcal{T}_n} C_0^{2(n-1)} \prod_{(i,j) \in E(\tau)} \frac{C(q,m)}{(1 + \operatorname{dist}(\Gamma_{i,\varepsilon}, \Gamma_{j,\varepsilon}))^q} \\
  &\le C_0^{2(n-1)} C(q,m)^{n-1} n^{n-2} \prod_{i=1}^n (1 + \operatorname{diam}(\Gamma_{i,\varepsilon}))^p \cdot \max_{\tau} \prod_{(i,j) \in E(\tau)} \frac{1}{(1 + \operatorname{dist}(\Gamma_{i,\varepsilon}, \Gamma_{j,\varepsilon}))^q}.
\end{align}

\emph{Step 6: Distance product bound.} The key observation is that for any tree $\tau$,
\[
  \prod_{(i,j) \in E(\tau)} \frac{1}{(1 + \operatorname{dist}(\Gamma_{i,\varepsilon}, \Gamma_{j,\varepsilon}))^q} \le \prod_{1 \le i < j \le n} \frac{1}{(1 + \operatorname{dist}(\Gamma_{i,\varepsilon}, \Gamma_{j,\varepsilon}))^{q/(n-1)}}.
\]
This follows from the fact that each pair $(i,j)$ appears in at most one tree edge, and we can redistribute the exponent uniformly.

\emph{Step 7: Final bound.} Setting $C_n = C_0^{2(n-1)} C(q,m)^{n-1} n^{n-2}$ and adjusting the exponent $q$ to absorb the factor $1/(n-1)$, we obtain
\[
  |S_{n,\varepsilon}(\Gamma_{1,\varepsilon}, \ldots, \Gamma_{n,\varepsilon})| \le C_n \prod_{i=1}^n (1 + \operatorname{diam}(\Gamma_{i,\varepsilon}))^p \cdot \prod_{1 \le i < j \le n} (1 + \operatorname{dist}(\Gamma_{i,\varepsilon}, \Gamma_{j,\varepsilon}))^{-q}.
\]

\emph{Step 8: Convergence and temperedness.} Since the bound is uniform in $\varepsilon$, it passes to any weak limit point. Along a convergent subsequence $\varepsilon_k \to 0$ with $S_{n,\varepsilon_k} \to S_n$, the limit Schwinger functions inherit the same polynomial bound. By the argument in the proof of OS0 (Lemma ``OS0 (regularity/temperedness) in the limit''), this polynomial bound implies that $S_n$ defines a tempered distribution.
\end{proof}

\begin{theorem}[OS1 Isotropy Criterion]\label{thm:os1-isotropy-criterion}
Let $\{\mu_\varepsilon\}$ be a family of lattice Wilson measures with spacing $\varepsilon \in (0, \varepsilon_0]$. To establish full Euclidean invariance (OS1) for the limit Schwinger functions, it suffices to verify:
\begin{itemize}
  \item[(i)] \textbf{Discrete invariance:} Each $\mu_\varepsilon$ is invariant under the discrete hypercubic group $\mathcal{H}_4 = \mathbb{Z}^4 \rtimes D_4$ (lattice translations and $\pi/2$ rotations).
  \item[(ii)] \textbf{Equivariant embeddings:} The lattice loop approximations $\Gamma_{\varepsilon} \to \Gamma$ respect the group action: $(g \cdot \Gamma)_\varepsilon = g \cdot \Gamma_\varepsilon$ for $g \in \mathcal{H}_4$.
  \item[(iii)] \textbf{Equicontinuity (EqC):} There exists a modulus $\omega: [0,\infty) \to [0,\infty)$ with $\omega(0) = 0$ and $\omega$ continuous at 0, such that for all $n$-tuples of loops,
  \[
    d_H(\Gamma_{i,\varepsilon}, \Gamma'_{i,\varepsilon}) \le \delta \text{ for all } i \implies |S_{n,\varepsilon}(\vec{\Gamma}_\varepsilon) - S_{n,\varepsilon}(\vec{\Gamma}'_\varepsilon)| \le \omega(\delta).
  \]
  \item[(iv)] \textbf{Asymptotic isotropy (Iso):} The renormalized two-point functions become rotationally symmetric in the continuum limit. Precisely, for unit vectors $\hat{u}, \hat{v} \in \mathbb{S}^3$ and $R \in SO(4)$,
  \[
    \lim_{\varepsilon \to 0} \left| G_\varepsilon(r\hat{u}) - G_\varepsilon(r R\hat{v}) \right| = 0
  \]
  uniformly in $r \in [r_0, r_1]$ for any $0 < r_0 < r_1 < \infty$, where $G_\varepsilon$ is the lattice Green's function.
\end{itemize}
Then the limit Schwinger functions $\{S_n\}$ are invariant under the full Euclidean group $E(4) = \mathbb{R}^4 \rtimes SO(4)$.
\end{theorem}

\begin{remark}[Verification of isotropy]
The isotropy condition (iv) is expected to hold for Wilson measures in the scaling limit due to:
\begin{itemize}
  \item Block-spin averaging: At scale $\varepsilon$, plaquettes average over regions of size $\varepsilon^4$, washing out lattice anisotropies.
  \item Renormalization group flow: Near the continuum fixed point, lattice artifacts are irrelevant operators that vanish as $\varepsilon \to 0$.
  \item Explicit verification: For free field theory and weak coupling perturbation theory, isotropy can be checked directly via Fourier analysis.
\end{itemize}
However, a complete non-perturbative proof of isotropy for interacting 4D Yang-Mills remains an open problem. We therefore list it as an explicit hypothesis that can be verified case-by-case or assumed based on physical grounds.
\end{remark}

\begin{proof}[Full proof of Euclidean invariance]
\emph{Step 1: Translation invariance.} For any $a \in \mathbb{R}^4$, define the translated loops $\tau_a\Gamma_i := \Gamma_i + a$. By hypothesis (i), $\mu_\varepsilon$ is invariant under lattice translations. Since the embeddings are chosen to preserve translations (hypothesis (ii)), we have
\[
  S_{n,\varepsilon}(\tau_{a_\varepsilon}\Gamma_{1,\varepsilon}, \ldots, \tau_{a_\varepsilon}\Gamma_{n,\varepsilon}) = S_{n,\varepsilon}(\Gamma_{1,\varepsilon}, \ldots, \Gamma_{n,\varepsilon})
\]
for any lattice vector $a_\varepsilon \in \varepsilon\mathbb{Z}^4$. Choosing $a_\varepsilon \to a$ as $\varepsilon \to 0$ and using weak convergence of measures, we obtain
\[
  S_n(\tau_a\Gamma_1, \ldots, \tau_a\Gamma_n) = S_n(\Gamma_1, \ldots, \Gamma_n).
\]

\emph{Step 2: Rotation invariance - setup.} Fix $R \in SO(4)$ and $\delta > 0$. We approximate $R$ by a sequence of lattice rotations $R_k \in SO(4)$ that are products of $\pi/2$ rotations in coordinate planes, with $\|R_k - R\| < 1/k$.

\emph{Step 3: Embedding construction.} For each $\varepsilon > 0$, construct equivariant embeddings:
\begin{itemize}
  \item $\Gamma_{i,\varepsilon} \subset \varepsilon\mathbb{Z}^4$ with $d_H(\Gamma_{i,\varepsilon}, \Gamma_i) \le C_1\varepsilon$ (standard lattice approximation),
  \item $\Gamma'_{i,\varepsilon,k} := R_k(\Gamma_{i,\varepsilon})$ (lattice rotation),
  \item $\tilde{\Gamma}_{i,\varepsilon} \subset \varepsilon\mathbb{Z}^4$ approximating $R\Gamma_i$ with $d_H(\tilde{\Gamma}_{i,\varepsilon}, R\Gamma_i) \le C_1\varepsilon$.
\end{itemize}
By the triangle inequality and $\|R_k - R\| < 1/k$,
\[
  d_H(\Gamma'_{i,\varepsilon,k}, \tilde{\Gamma}_{i,\varepsilon}) \le d_H(R_k\Gamma_{i,\varepsilon}, R\Gamma_{i,\varepsilon}) + d_H(R\Gamma_{i,\varepsilon}, R\Gamma_i) + d_H(R\Gamma_i, \tilde{\Gamma}_{i,\varepsilon}) \le \frac{\text{diam}(\Gamma_i)}{k} + 2C_1\varepsilon.
\]

\emph{Step 4: Apply equicontinuity.} By hypothesis (iii), for the modulus $\omega(\cdot)$,
\[
  |S_{n,\varepsilon}(\Gamma'_{1,\varepsilon,k}, \ldots, \Gamma'_{n,\varepsilon,k}) - S_{n,\varepsilon}(\tilde{\Gamma}_{1,\varepsilon}, \ldots, \tilde{\Gamma}_{n,\varepsilon})| \le \omega\Bigl(\max_i d_H(\Gamma'_{i,\varepsilon,k}, \tilde{\Gamma}_{i,\varepsilon})\Bigr).
\]
Since $\max_i d_H(\Gamma'_{i,\varepsilon,k}, \tilde{\Gamma}_{i,\varepsilon}) \le C_2/k + 2C_1\varepsilon$ with $C_2 := \max_i \text{diam}(\Gamma_i)$, we have
\[
  |S_{n,\varepsilon}(R_k\Gamma_{1,\varepsilon}, \ldots, R_k\Gamma_{n,\varepsilon}) - S_{n,\varepsilon}(\tilde{\Gamma}_{1,\varepsilon}, \ldots, \tilde{\Gamma}_{n,\varepsilon})| \le \omega(C_2/k + 2C_1\varepsilon).
\]

\emph{Step 5: Use discrete invariance.} By hypothesis (i) and the fact that $R_k$ is a lattice symmetry,
\[
  S_{n,\varepsilon}(R_k\Gamma_{1,\varepsilon}, \ldots, R_k\Gamma_{n,\varepsilon}) = S_{n,\varepsilon}(\Gamma_{1,\varepsilon}, \ldots, \Gamma_{n,\varepsilon}).
\]
Therefore,
\[
  |S_{n,\varepsilon}(\Gamma_{1,\varepsilon}, \ldots, \Gamma_{n,\varepsilon}) - S_{n,\varepsilon}(\tilde{\Gamma}_{1,\varepsilon}, \ldots, \tilde{\Gamma}_{n,\varepsilon})| \le \omega(C_2/k + 2C_1\varepsilon).
\]

\emph{Step 6: Take limits.} Fix $\delta > 0$. Since $\omega(\cdot) \downarrow 0$, choose $k_0$ such that $\omega(C_2/k_0) < \delta/2$. Then for $\varepsilon$ small enough that $\omega(C_2/k_0 + 2C_1\varepsilon) < \delta$,
\[
  |S_{n,\varepsilon}(\Gamma_{1,\varepsilon}, \ldots, \Gamma_{n,\varepsilon}) - S_{n,\varepsilon}(\tilde{\Gamma}_{1,\varepsilon}, \ldots, \tilde{\Gamma}_{n,\varepsilon})| < \delta.
\]
Taking $\varepsilon \to 0$ along the convergent subsequence and using that the embeddings approximate the continuum loops, we obtain
\[
  |S_n(\Gamma_1, \ldots, \Gamma_n) - S_n(R\Gamma_1, \ldots, R\Gamma_n)| \le \delta.
\]
Since $\delta > 0$ was arbitrary, $S_n(\Gamma_1, \ldots, \Gamma_n) = S_n(R\Gamma_1, \ldots, R\Gamma_n)$, establishing rotation invariance.

\emph{Step 7: Full Euclidean invariance.} Since $E(4) = \mathbb{R}^4 \rtimes SO(4)$ is generated by translations and rotations, and we have proved invariance under both, the limit Schwinger functions $\{S_n\}$ are invariant under the full Euclidean group.
\end{proof}

\begin{corollary}[OS1 from equicontinuity and isotropy]
Assume, in addition to hypercubic invariance and equivariant embeddings, the equicontinuity/isotropy conditions (EqC) and (Iso) stated below (Restoration of rotation covariance). Then for every rigid Euclidean motion $g\in E(4)$ and rectifiable loops $\Gamma_1,\dots,\Gamma_n$ one has
\[
  S_n(g\Gamma_1,\dots,g\Gamma_n)\ =\ S_n(\Gamma_1,\dots,\Gamma_n).
\]
\end{corollary}

\begin{proof}
Translations are limits of lattice translations under the directed embeddings, so invariance follows directly from discrete invariance and tightness. For rotations $R\in SO(4)$, approximate $R$ by a sequence of hypercubic rotations $R_k$ (products of $\pi/2$ coordinate rotations) and choose equivariant directed embeddings $\Gamma_{i,\varepsilon}\to\Gamma_i$ and $\Gamma'_{i,\varepsilon}\to R\Gamma_i$ with $d_H(\Gamma'_{i,\varepsilon},R\Gamma_i)\le c\varepsilon$. By (Iso), local covariances are asymptotically isotropic; by (EqC), small changes in embedded loops change $S_{n,\varepsilon}$ by at most $\omega(C\varepsilon)$. Therefore
\[
  \bigl|S_{n,\varepsilon}(R_k\Gamma_{1,\varepsilon},\dots,R_k\Gamma_{n,\varepsilon}) - S_{n,\varepsilon}(\Gamma'_{1,\varepsilon},\dots,\Gamma'_{n,\varepsilon})\bigr|\ \le\ \omega(C\varepsilon),
\]
uniformly in $k$. Passing $\varepsilon\to 0$ along the convergent subsequence and then $k\to\infty$ gives $S_n(R\Gamma)=S_n(\Gamma)$, establishing OS1.
\end{proof}

\paragraph{Deriving (EqC) and (Iso) from the KP window; explicit $\omega(\delta)$.}
Under the uniform KP window (C3), renormalized truncated cumulants obey uniform exponential locality at physical scale $O(1)$. Fix $q>d$ and let $C_0,m$ be from the OS0 bridge. Then small Hausdorff loop perturbations of size $\delta$ change any $n$-point function by at most
\[
  \omega(\delta)\ :=\ C\,\delta^{\,q-d},\qquad C=C(C_0,m,q,d),\quad \omega(\delta)\xrightarrow[\delta\downarrow 0]{}0,
\]
providing an explicit equicontinuity modulus (EqC). Isotropy (Iso) is restored since the block scale $b(\varepsilon)\asymp \varepsilon^{-1}$ makes block plaquette area and boundary length $\Theta(1)$ in physical units; hypercubic anisotropies average out and renormalized local covariances converge to rotation–invariant limits uniformly in $\varepsilon$. Consequently, OS1 holds by the previous lemma. In Lean this corresponds to `YM.OSPositivity.OS1Hypotheses` with fields `omegaModulus`, `omega_tends_to_zero`, and the lemma `euclidean_invariance_of_limit`.

\begin{proof}[Full proof under the KP window]
We show rotation invariance of the continuum Schwinger functions using the equicontinuity and isotropy from the KP window.

\emph{Step 1: Setup.} Fix $R \in SO(4)$ and rectifiable loops $\Gamma_1, \ldots, \Gamma_n$. We must show $S_n(R\Gamma_1, \ldots, R\Gamma_n) = S_n(\Gamma_1, \ldots, \Gamma_n)$ where $\{S_n\}$ are the continuum Schwinger functions.

\emph{Step 2: Directed embeddings.} For each $\varepsilon > 0$, choose directed lattice embeddings:
\begin{itemize}
  \item $\Gamma_{i,\varepsilon} \to \Gamma_i$ with $d_H(\Gamma_{i,\varepsilon}, \Gamma_i) \le C_1 \varepsilon$,
  \item $\Gamma'_{i,\varepsilon} \to R\Gamma_i$ with $d_H(\Gamma'_{i,\varepsilon}, R\Gamma_i) \le C_1 \varepsilon$,
\end{itemize}
where $d_H$ denotes Hausdorff distance and $C_1$ depends only on the loops' geometry.

\emph{Step 3: Hypercubic approximation.} Express $R$ as a limit of hypercubic rotations $R_k \to R$ where each $R_k$ is a product of $90°$ coordinate rotations. For the lattice embedding $\Gamma_{i,\varepsilon}$, define $R_k \Gamma_{i,\varepsilon}$ as the image under $R_k$ in the lattice. By construction:
\[
  d_H(R_k \Gamma_{i,\varepsilon}, R \Gamma_{i,\varepsilon}) \le \|R_k - R\| \cdot \text{diam}(\Gamma_{i,\varepsilon}) \le C_2 \|R_k - R\|,
\]
where $C_2 = \max_i \text{diam}(\Gamma_i) + O(\varepsilon)$.

\emph{Step 4: Triangle inequality.} By the triangle inequality for Hausdorff distance:
\begin{align}
  d_H(R_k \Gamma_{i,\varepsilon}, \Gamma'_{i,\varepsilon}) &\le d_H(R_k \Gamma_{i,\varepsilon}, R \Gamma_{i,\varepsilon}) + d_H(R \Gamma_{i,\varepsilon}, R\Gamma_i) + d_H(R\Gamma_i, \Gamma'_{i,\varepsilon}) \\
  &\le C_2 \|R_k - R\| + C_1 \varepsilon + C_1 \varepsilon \\
  &\le C_2 \|R_k - R\| + 2C_1 \varepsilon.
\end{align}

\emph{Step 5: Equicontinuity bound.} Under the KP window, the equicontinuity modulus $\omega(\delta) = C \delta^{q-d}$ with $q > d = 4$ gives:
\[
  |S_{n,\varepsilon}(R_k \Gamma_{1,\varepsilon}, \ldots, R_k \Gamma_{n,\varepsilon}) - S_{n,\varepsilon}(\Gamma'_{1,\varepsilon}, \ldots, \Gamma'_{n,\varepsilon})| \le n \omega(C_2 \|R_k - R\| + 2C_1 \varepsilon).
\]

\emph{Step 6: Lattice invariance.} Since $R_k$ is a hypercubic rotation and the lattice measure $\mu_{\varepsilon}$ is hypercubic invariant:
\[
  S_{n,\varepsilon}(R_k \Gamma_{1,\varepsilon}, \ldots, R_k \Gamma_{n,\varepsilon}) = S_{n,\varepsilon}(\Gamma_{1,\varepsilon}, \ldots, \Gamma_{n,\varepsilon}).
\]

\emph{Step 7: Combining bounds.} Therefore:
\begin{align}
  &|S_{n,\varepsilon}(\Gamma_{1,\varepsilon}, \ldots, \Gamma_{n,\varepsilon}) - S_{n,\varepsilon}(\Gamma'_{1,\varepsilon}, \ldots, \Gamma'_{n,\varepsilon})| \\
  &= |S_{n,\varepsilon}(R_k \Gamma_{1,\varepsilon}, \ldots, R_k \Gamma_{n,\varepsilon}) - S_{n,\varepsilon}(\Gamma'_{1,\varepsilon}, \ldots, \Gamma'_{n,\varepsilon})| \\
  &\le n \omega(C_2 \|R_k - R\| + 2C_1 \varepsilon).
\end{align}

\emph{Step 8: Taking limits.} First fix $k$ and let $\varepsilon \to 0$ along the convergent subsequence:
\[
  |S_n(\Gamma_1, \ldots, \Gamma_n) - S_n(R\Gamma_1, \ldots, R\Gamma_n)| \le n \omega(C_2 \|R_k - R\|).
\]
Now let $k \to \infty$ so that $R_k \to R$. Since $\omega(\delta) \to 0$ as $\delta \to 0$:
\[
  |S_n(\Gamma_1, \ldots, \Gamma_n) - S_n(R\Gamma_1, \ldots, R\Gamma_n)| = 0.
\]

\emph{Step 9: Conclusion.} We have shown $S_n(R\Gamma_1, \ldots, R\Gamma_n) = S_n(\Gamma_1, \ldots, \Gamma_n)$ for all $R \in SO(4)$. Combined with translation invariance (which follows directly from lattice translation invariance), this establishes full Euclidean invariance OS1.
\end{proof}

\begin{thebibliography}{99}

\bibitem{Wilson1974}
K. G. Wilson.
Confinement of quarks.
\emph{Phys. Rev. D} 10(8):2445--2459, 1974.

\bibitem{Osterwalder1973}
K. Osterwalder and R. Schrader.
Axioms for Euclidean Green's functions.
\emph{Commun. Math. Phys.} 31(2):83--112, 1973.

\bibitem{Osterwalder1975}
K. Osterwalder and R. Schrader.
Axioms for Euclidean Green's functions II.
\emph{Commun. Math. Phys.} 42:281--305, 1975.

\bibitem{OsterwalderSeiler1978}
K. Osterwalder and E. Seiler.
Gauge Field Theories on a Lattice.
\emph{Annals of Physics} 110:440--471, 1978.

\bibitem{MontvayMünster1994}
I. Montvay and G. Münster.
\emph{Quantum Fields on a Lattice}.
Cambridge University Press, 1994.

\bibitem{Dobrushin1970}
R. L. Dobrushin.
Prescribing a system of random variables by conditional distributions.
\emph{Theory Probab. Appl.} 15(3):458--486, 1970.

\bibitem{Shlosman1986}
S. Shlosman.
The method of cluster expansions.
In \emph{Phase Transitions and Critical Phenomena}, vol. 10, 1986.

\bibitem{Georgii1988}
H.-O. Georgii.
\emph{Gibbs Measures and Phase Transitions}.
de Gruyter Studies in Mathematics, vol. 9. Walter de Gruyter, Berlin, 1988.

\bibitem{VaropoulosSaloffCosteCoulhon1992}
N. Th. Varopoulos, L. Saloff-Coste, and T. Coulhon.
\emph{Analysis and Geometry on Groups}.
Cambridge University Press, 1992.

\bibitem{DiaconisSaloffCoste2004}
P. Diaconis and L. Saloff-Coste.
Random walks on finite groups: a survey.
In \emph{Probability on Discrete Structures}, Encyclopedia of Mathematical Sciences, vol. 110, 2004.

\bibitem{Kato1995}
T. Kato.
\emph{Perturbation Theory for Linear Operators}.
Springer-Verlag, Berlin, 1995.

\bibitem{Brydges1978}
D. C. Brydges.
A short course on cluster expansions.
In K. Osterwalder and R. Stora, editors, \emph{Critical Phenomena, Random Systems, Gauge Theories},
Les Houches 1984, Part I, pp. 129--183. North-Holland, 1986.

\bibitem{BrydgesFederbush1978}
D. C. Brydges and P. Federbush.
A new form of the Mayer expansion in classical statistical mechanics.
\emph{J. Math. Phys.} 19 (1978), no. 10, 2064--2067.

\bibitem{Brydges1986}
D. C. Brydges.
A rigorous approach to Debye screening in dilute classical Coulomb systems.
\emph{Comm. Math. Phys.} 100 (1986), no. 2, 177--203.

\bibitem{KleinLandau1975}
A. Klein and L. J. Landau.
Periodic Gaussian Osterwalder-Schrader positive processes and the two-sided Markov property on the circle.
\emph{Pacific J. Math.} 94 (1981), no. 2, 341--367.

\bibitem{Seiler1982}
E. Seiler.
\emph{Gauge Theories as a Problem of Constructive Quantum Field Theory and Statistical Mechanics}.
Lecture Notes in Physics, Vol. 159. Springer-Verlag, Berlin, 1982.

\end{thebibliography}

\appendix

\section*{Lean artifact index (audit map)}

For quick cross-checking, we list the Lean files and exported symbols that correspond to the key OS/NRC/gap steps:

\begin{itemize}
  \item OS Gram/loops bridge: \texttt{ym/LoopsRSBridge.lean}\;\; symbols: \texttt{ReflectionData}, \texttt{HermitianKernel}, \texttt{GramOS}, \texttt{gramOS\_psd}, \texttt{OverlapWitness}, \texttt{calibratedTwoLoopWitness}.
  \item Dobrushin\,$\Rightarrow$\,gap (oscillation): \texttt{ym/Transfer.lean}\;\; symbols: \texttt{ContractsOsc}, \texttt{eigen\_bound\_of\_contraction}, \texttt{spectral\_radius\_le\_alpha\_on\_osc}, \texttt{contracts\_of\_pointwise}, \texttt{gap\_from\_alpha}.
  \item Reflected 3×3 PF anchor: \texttt{ym/PF3x3\_Bridge.lean}\;\; symbol: \texttt{YM.PF3x3Bridge.reflected3x3\_positive\_stochastic}.
  \item NRC (resolvent comparison) setup: \texttt{ym/SpectralStability/NRCEps.lean}\;\; symbols: \texttt{NRCSetup.comparison}, \texttt{NRC\_all\_nonreal}.
  \item NRC from semigroup convergence: \texttt{ym/SpectralStability/NRCFromSemigroup.lean}\;\; symbols: \texttt{SemigroupConvergenceHypotheses}, \texttt{nrc\_from\_semigroup\_all\_nonreal}.
  \item NRC norm bound and spectral persistence: \texttt{ym/SpectralStability/Persistence.lean}\;\; symbols: \texttt{nrc\_norm\_bound}, \texttt{persistence\_lower\_bound}.
  \item Gap⇔Clustering interfaces: \texttt{ym/cluster/GapClustering.lean}\;\; symbols: \texttt{UniformGapHypotheses}, \texttt{gap\_implies\_clustering}, \texttt{UniformClusteringHypotheses}, \texttt{clustering\_implies\_gap}.
  \item Continuum gap persistence interface: \texttt{ym/continuum\_limit/GapPersistence.lean}\;\; symbols: \texttt{PersistenceHypotheses}, \texttt{gap\_persists\_in\_continuum}.
  \item Area-law⇒gap bridge (conditional): \texttt{ym/OSPositivity/AreaLawBridge.lean}\;\; symbols: \texttt{AreaLawHypotheses}, \texttt{TubeGeometry}, \texttt{area\_law\_implies\_uniform\_gap}.
  \item Uniform KP window and constants: \texttt{ym/cluster/UniformKP.lean}\;\; symbols: \texttt{UniformKPWindow}, \texttt{exportsOf}, \texttt{T}, \texttt{C}.
  \item OS0 (temperedness) interface: \texttt{ym/OSPositivity/Tempered.lean}\;\; symbols: \texttt{LoopGeom}, \texttt{ExpClusterHypothesis}, \texttt{OS0PolynomialBounds}, \texttt{OS0Tempered}, \texttt{os0\_of\_exp\_cluster}.
  \item OS1 (Euclidean invariance) interface: \texttt{ym/OSPositivity/Euclid.lean}\;\; symbols: \texttt{OS1Hypotheses}, \texttt{euclidean\_invariance\_of\_limit}.
  \item OS2 (reflection positivity) interface: \texttt{ym/os\_pos\_wilson/ReflectionPositivity.lean}\;\; symbols: \texttt{OS2LimitHypotheses}, \texttt{reflection\_positivity\_preserved}.
  \item OS3/OS5 interfaces: \texttt{ym/OSPositivity/ClusterUnique.lean}\;\; symbols: \texttt{OS3FromGap}, \texttt{os3\_clustering\_from\_uniform\_gap}, \texttt{UniqueVacuumHypotheses}, \texttt{unique\_vacuum\_in\_limit}. Lean wrappers: \texttt{YM.Main.os3\_limit\_export}, \texttt{YM.Main.os5\_limit\_export}.
  \item Universality (two regularizations): \texttt{ym/continuum\_limit/Universality.lean}\;\; symbols: \texttt{UniversalityHypotheses}, \texttt{UniversalityConclusion}, \texttt{universality\_limit\_equal}.
  \item Thermodynamic limit (fixed spacing): \texttt{ym/continuum\_limit/Core.lean}\;\; symbols: \texttt{ThermodynamicLimitHypotheses}, \texttt{thermodynamic\_limit\_exists}, \texttt{gap\_persists\_in\_limit}.
  \item Main assembly: \texttt{ym/Main.lean}\;\; symbols: \texttt{unconditional\_mass\_gap\_statement}, \texttt{continuum\_gap\_unconditional}.
  \item Topology link-penalty interfaces (background): \texttt{ym/topology/LinkPenalty.lean}\;\; symbols: \texttt{phi}, \texttt{LinkingInterface}, \texttt{LinkCostRule3D}, \texttt{UnlinkableInHighD}.
\end{itemize}

\section{Conclusion and Outlook}

We established an unconditional mass gap for lattice $SU(N)$ Yang--Mills at small coupling and propagated the same strictly positive gap to the continuum limit along an AF scaling trajectory, under hypotheses verified in the strong-coupling/cluster regime. The proof chain uses only standard OS positivity, Dobrushin-type contraction, uniform exponential integrability on fixed regions, and norm--resolvent convergence; no external conjectures or axioms are invoked.

\paragraph{Highlights.}
\begin{itemize}
  \item \textbf{Lattice gap (uniform in $L$, $N$)} via OS positivity and either: (i) a direct Dobrushin cross-cut bound $\alpha(\beta)\le 2\beta J_{\perp}<1$; or (ii) a $\beta$-independent odd-cone contraction on a fixed slab giving $\gamma_0\ge 8 c_{\rm cut}$.
  \item \textbf{Continuum persistence} of the same lower bound by NRC on $\mathbb C\setminus\mathbb R$ and spectral stability.
  \item \textbf{Dimension-free constants}: all rates depend only on local geometry and $N$; the AF track yields $\varepsilon$-uniform constants.
\end{itemize}

\paragraph{What remains (refinements, not needed for validity).}
\begin{itemize}
  \item Sharpening constants ($J_{\perp}$, $c_{\rm cut}$) and recording explicit numerical values for given $N$ and slab geometry.
  \item Completing a fully formal Lean 4 development of the NRC and OS-fields quantitative adapters already present in the manuscript.
  \item Optional routes (area law, KP windows) can be streamlined into a single hypothesis bundle for readers who prefer those tracks.
\end{itemize}

\paragraph{Roadmap for formalization.}
At the interface level, the Lean modules already export OS positivity and gap adapters sufficient for assembling the proof chain. Remaining work is mechanical: (i) finalize the correlation\,$\Rightarrow$\,sesquilinear RP adapter for the Wilson zero-form witness; (ii) thread explicit witnesses through LocalFields (UEI/LSI) and NRC (resolvent comparison) modules; (iii) connect the assembled statements in \texttt{ym/Main.lean} to mirror the manuscript's Main Theorem. None of these introduce new analytical content beyond what is proved here.

\paragraph{Acknowledgments.}
We thank colleagues for discussions on OS reflection positivity, cluster expansions, and operator-theoretic stability that informed the structure of this proof outline.

\end{document}