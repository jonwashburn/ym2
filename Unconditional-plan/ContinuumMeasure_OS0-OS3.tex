\documentclass[11pt]{article}

\title{Non\-perturbative Continuum Measure for 4D SU($N$) Yang--Mills: OS0--OS3 via Tightness, Reflection Positivity, and Clustering}
\author{Jonathan Washburn \\ Recognition Science, Recognition Physics Institute \\ Austin, Texas, USA \\ \texttt{jon@recognitionphysics.org}}
\date{\today}

\begin{document}
\maketitle

\begin{abstract}
We construct a continuum Euclidean probability measure for 4D SU($N$) Yang--Mills (YM) as a projective limit of finite\-volume Wilson measures. The construction yields OS0 (temperedness/equicontinuity), OS1 (Euclidean invariance), and OS2 (reflection positivity). OS3 (clustering) follows under an explicit decay criterion, and in particular from a uniform area law. The arguments are measure\-theoretic and require no parameter tuning.
\end{abstract}

\section*{Setup and hypotheses}

Let $a>0$ be lattice spacing and $L\in\mathbb N$ a side length. Denote by $\mu_{a,L}$ the periodic finite\-volume Wilson measures with central plaquette weight $w$ having a nonnegative character expansion (reflection positivity on the lattice). Consider gauge\-invariant cylindrical observables generated by Wilson loops on a fixed countable dense set of loops.

We assume: (i) uniform local moment/equicontinuity bounds (OS0 kernel) on loop holonomies; (ii) a van Hove diagonal sequence $(a_n,L_n)$ along which joint loop laws converge for every finite loop family; (iii) an orientation\-interleaved diagonalization to pass discrete symmetries to the limit; and (iv) an explicit decay criterion (either a two\-point mass gap for a local observable or a uniform area law), to obtain clustering (OS3).

\section*{Projective construction and OS2}

Compactness of SU($N$) yields tightness of finite\-dimensional loop laws. Consistency of marginals passes to limits; Kolmogorov extension produces a probability measure on loop assignments, hence on generalized connections modulo gauge (closure of holonomy relations). Reflection positivity (OS2) is closed under weak limits because $\langle \Theta F, F\rangle$ is a continuous cylindrical functional and is nonnegative at each finite lattice.

\section*{OS0 and OS1}

Uniform moment/equicontinuity bounds give temperedness (OS0) of Schwinger functionals and Kolmogorov\-Chentsov H"older continuity for loop holonomies almost surely. Discrete translation/rotation invariance and the interleaved diagonalization transfer Euclidean symmetry to the limit (OS1) by density of rational motions and continuity.

\section*{OS3 (clustering)}

If a two\-point mass gap holds for a local observable, the reconstructed Hamiltonian has a spectral gap, which implies exponential clustering in Euclidean time; OS1 rotates large spacelike separations into time separations up to controlled errors. Alternatively, a uniform area law plus a geometric tube bound yields exponential decay of connected loop correlators with rate proportional to the product of the string tension and the tube constant, giving OS3.

\end{document}


